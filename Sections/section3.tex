\section{Прямое произведение групп}
\subsection{Внешнее прямое произведение}
    Пусть $G_1,...,G_k$ - группы.\\
    $G = G_1 \times ... \times G_k = \{(g_1,...,g_k)| g_i \in G_i\}$.\\
    $(g_1,...,g_k)\cdot(\tilde{g}_1,...,\tilde{g}_k) = (g_1\tilde{g}_1,...,g_k\tilde{g}_k)$\\
    ($g_i\tilde{g}_i$ перемножаются по правилу бинарной операции на $G_i$).
\begin{subtheorem}
    $(G, \cdot)$ - группа.
\end{subtheorem}
\begin{proof}\tab
    \begin{enumerate}
        \item $(a_1,...,a_k)((b_1,...,b_k)(c_1,...,c_k)) = (a_1(b_1c_1),...,a_k(b_kc_k)) =\\= ((a_1b_1)c_1,...,(a_kb_k)c_k) = ((a_1,...,a_k)(b_1,...,b_k))(c_1,...,c_k)$ 
        \item Нейтральный элемент - $(e_1,...,e_k)$ ($e_i$ - нейтральный в $G_i$)
        \item $(g_1,...,g_k)^{-1} = (g_1^{-1},...,g_k^{-1})$ 
    \end{enumerate}
\end{proof}
\begin{definition}
    Данная группа $(G, \cdot)$ называется прямым произведением групп $G_1,...,G_k$. Обозначается $G = G_1 \times ... \times G_k$; $G_i$ называются множителями.\\
    В аддитивной терминологии те же рассуждения определяют прямую сумму $G = G_1 \oplus ... \oplus G_k$, где $G_i$ - слагаемые.
\end{definition}
\begin{examples}\tab
    \begin{enumerate}
        \item $G_1 = \Z_3, G_2 = S_3, G = G_1 \times G_2$.\\
        $(1, (12))\cdot (2, (13)) = (1+2, (12)(13)) = (0, (132))$.
        \item $D_n(\F) \simeq \undermat{n}{\F^* \times ... \times \F^*} \ (D_n(\F)$ - группа диагональных матриц порядка $n$). 
    \end{enumerate}
\end{examples}
\begin{subtheorem}\tab
    \begin{enumerate}
        \item Если $(m, n) = 1$, то $\Z_m \times \Z_n \simeq Z_{nm}$ - циклическая группа;
        \item Если $(m, n) \neq 1$, то $\Z_m \times \Z_n$ - не циклическая.
    \end{enumerate}
\end{subtheorem}
\begin{proof}\tab
    \begin{enumerate}
        \item Обозначим за $[a]_s \in \Z_s$ класс вычетов по модудю $s$, содержащий $a$.\\
        Рассмотрим отображение $\phi: \Z_{mn} \rightarrow \Z_m \times \Z_n$ такое, что $\phi: [a]_{mn} \mapsto ([a]_m, [a]_n)$. Очевидно, что это гомоморфизм: 
        \[\phi([a]_{mn} \cdot [b]_{mn}) = ([ab]_m, [ab]_n) = ([a]_m, [a]_n)([b]_m, [b]_n) = \phi([a]_{mn})\phi([b]_{mn})\]
        Найдём $\textup{Ker }\phi:$
        \[\phi([a]_{mn}) = ([0]_m, [0]_n) \Longleftrightarrow \begin{cases}m \mid a\\ n\mid a \end{cases} \underset{(m,n)=1}{\Longrightarrow} mn \mid a \Longrightarrow \textup{Ker }\phi = \{[0]_{mn}\}\]
        По теореме о гомоморфизме $\textup{Im }\phi = \Z_{mn} / \textup{Ker }\phi = \Z_{mn} \Longrightarrow |\textup{Im } \phi| = mn$. Так как $|\Z_m \times \Z_n| = mn$ и $\textup{Im }\phi \leq \Z_m \times \Z_n, \ \textup{Im }\phi = \Z_m \times \Z_n$.\\
        Отсюда $\phi$ - биекция (инъекция из $\textup{Ker }\phi = \{e\}$), т.е. $\phi$  -изоморфизм.
        \item Пусть $(m,n) = d \neq 1$ ($m = dk_1, n = dk_2$). Тогда $\forall g = (g_1, g_2) \in \Z_m \times \Z_n$:
        \[(g_1, g_2)^{dk_1k_2} = (g_1^{dk_1k_2}, g_2^{dk_1k_2}) = (0^{k_2}, 0^{k_1}) = (0, 0)\]
        Отсюда $\textup{ord } (g_1, g_2) = dk_1k_2 = \frac{mn}{d} < mn = |\Z_m \times \Z_n|$. Значит, $\Z_m \times \Z_n$ не является циклической.
    \end{enumerate}
\end{proof}
\begin{consequense}
    Пусть $n = p_1^{s_1}...p_k^{s_k}$ - разложение на простые множители. Тогда $\Z_n = \Z_{p_1^{s_1}}\times ...\times \Z_{p_k^{s_k}}$.
\end{consequense}
\begin{proof}
    Очевидно следует из теоремы.
\end{proof}
\begin{consequense}(Китайская теорема об остатках)
    Если числа $a_1,...,a_n$ попарно взаимно просты, то для любых целых $r_1,...,r_n \ (0 \leq r_i < n) \ \exists ! N \ (0 \leq N < a_1\cdot...\cdot a_n)$ такой, что $N \equiv r_i (\textup{mod } a_i)$ 
\end{consequense}
\begin{proof}
    Из теоремы следует, что $\Z_{a_1}\times...\times \Z_{a_n}\simeq \Z_a$ ($a = a_1\cdot...\cdot a_n$).\\
    Это означает, что набор остатков $(r_1,...,r_n) \in \Z_{a_1}\times...\times \Z_{a_n}$ изоморфизм из теоремы однозначно переводит в элемент $N \in \Z_a$ такой, что $r_i = [N]_{a_i}$, что и требовалось.
\end{proof}
\subsection{Внутреннее прямое произведение}
\begin{definition}
    Пусть $G$ - группа, $H_1,...,H_k \leq G$.\\
    $G$ раскладывается в прямое произведение подгрупп $H_1,...,H_k$, если:
    \begin{enumerate}
        \item $\forall g \in G \ \exists! \ h_i \in H_i : g = h_1...h_k$;
        \item $\forall i \neq j: \forall h_i \in H_i, h_j \in H_j \ h_ih_j = h_jh_i$.
    \end{enumerate}
    Обозначается $G = H_1 \times ... \times H_k$ ($G = H_1 \oplus ... \oplus H_k$ в аддитивной терминологии).
\end{definition}
\begin{remark}
    Из определения следует, что $(h_1...h_k)(\tilde{h}_1...\tilde{h}_k) = (h_1\tilde{h}_1)...(h_k\tilde{h}_k)$. 
\end{remark}
\begin{definition}
    Пусть $H, N \leq G$. Обозначим $NH = \{nh | n \in N, h \in H\}$
\end{definition}
\begin{subtheorem}
    Пусть $N\unlhd G, H \leq G$. Тогда $NH$ - подгруппа в $G$, причём $NH = HN$.  
\end{subtheorem}
\begin{proof}
    Рассмотрим $(n_1h_1)(n_2h_2) = \undermat{= \tilde{n}}{n_1(h_1n_2h_1^{-1})}\undermat{= \tilde{h}}{h_1h_2} = \tilde{n}\tilde{h} \in NH$.\\
    $e \in N \cap H \Longrightarrow e \cdot e = e \in NH$.\\
    $(nh)^{-1} = h^{-1}n^{-1} = (h^{-1}n^{-1}h)h^{-1} \in NH$.\\
    Отсюда $NH$ - подгруппа. Покажем, что $NH = HN$:
    \[\forall nh \in NH:\ nh = (hh^{-1})nh = h(h^{-1}nh) \in HN \Longrightarrow NH \subseteq HN\]
    \[\forall hn \in HN:\ hn = hn(h^{-1}h) = (hnh^{-1})h  \in NH \Longrightarrow HN \subseteq NH\]
    Отсюда $NH = HN$.
\end{proof}
\begin{lemmanum}
    Пусть $H, N \unlhd G, H \cap N = \{e\}$. Тогда $\forall h \in H, n \in N \ nh = hn$.
\end{lemmanum}
\begin{proof}
    Рассмотрим выражение $(hn)(nh)^{-1} = hnh^{-1}n^{-1}$:
    \[hnh^{-1}n^{-1} = h(nh^{-1}n^{-1}) \in H ; \ \ hnh^{-1}n^{-1} = (hnh^{-1})n^{-1} \in N\]
    Значит, $hnh^{-1}n^{-1} \in H \cap N = \{e\} \Longrightarrow (hn)(nh)^{-1} = e \Longrightarrow hn = nh$
\end{proof}
\begin{theoremnum}
    Пусть $H_1, H_2 \leq G$. Тогда $G = H_1\times H_2 \Longleftrightarrow \begin{cases}
        (1) \ H_1, H_2 \unlhd G\\
        (2) \ H_1 \cap H_2 = \{e\}\\
        (3) \ G = H_1H_2
    \end{cases}$
\end{theoremnum}
\begin{proof}
    $ \\$ $\Longrightarrow:$ Пусть $G = H_1 \times H_2$.\\
    (3) - очевидно из пункта 1 определения.\\
    (1): $\forall h_1 \in H_1, g \in G: \ g = \tilde{h}_1\tilde{h}_2 \ (\tilde{h}_1 \in H_1, \tilde{h}_2 \in H_2) \Longrightarrow$
    \[gh_1g^{-1} = \tilde{h}_1(\tilde{h}_2h_1\tilde{h}_2^{-1})\tilde{h}_1^{-1} \underset{(\text{2 из опр})}{=} \tilde{h}_1h_1\tilde{h}_1^{-1} \in H_1\]
    Отсюда $H_1 \unlhd G$ (аналогично $H_2 \unlhd G$).\\
    (2): Пусть $\exists h \in H_1\cap H_2$. Тогда $h = he = eh$ - два разложения на произведение элементов подгрупп. Они совпадают только в случае $h = e$, т.е. $H_1 \cap H_2 = \{e\}$.\\
    $\Longleftarrow:$ Пусть даны условия (1) - (3).\\
    По лемме 1 из (1), (2) очевидно следует пункт 2 определения.\\
    Из (3): $\forall g \in G \ \exists h_i \in H_i: g = h_1h_2$.\\
    Допустим, что это разложение не единственно, т.е. $h_1h_2 = \tilde{h}_1\tilde{h}_2$.\\
    Тогда $\tilde{h}_1^{-1}h_1 = \tilde{h}_2h_2^{-1}$, а так как $H_1 \cap H_2 = \{e\}$, имеем $h_1 = \tilde{h}_1, h_2 = \tilde{h}_2$.
\end{proof}
\begin{theoremnum}
    Пусть $H_1,...,H_k \leq G$.\\
    Тогда $G = H_1\times...\times H_k \Longleftrightarrow \begin{cases}
        (1) \ H_1,...,H_k \unlhd G\\
        (2) \ \forall i \ H_i \cap \langle H_j \ | \ j \neq i \rangle = \{e\}\\
        (3) \ G = H_1...H_k
    \end{cases}$
\end{theoremnum}
\begin{proof}
    $ \\$ $\Longrightarrow:$ Пусть $G = H_1 \times ... \times H_k$.\\
    (3) - очевидно из пункта 1 определения.\\
    (1): $\forall h_i \in H_i, g \in G: \ g = \tilde{h}_1...\tilde{h}_k \ (\tilde{h}_i \in H_i) \Longrightarrow$
    \[gh_1g^{-1} = (\tilde{h}_1...\tilde{h}_k)h_i(\tilde{h}_k^{-1}...\tilde{h}_1^{-1}) \underset{(\text{2 из опр})}{=} \tilde{h}_ih_i\tilde{h}_i^{-1} \in H_i\]
    Отсюда $H_i \unlhd G$.\\
    (2): Пусть $\exists h \in H_i \cap \langle H_j \ | \ j \neq i \rangle$. Тогда $h = he = eh$ - два разложения на произведение элементов подгрупп. Они совпадают только в случае $h = e$, т.е. $H_i \cap \langle H_j \ | \ j \neq i \rangle = \{e\}$.\\
    $\Longleftarrow:$ Пусть даны условия (1) - (3).\\
    По лемме 1 из (1), (2) очевидно следует пункт 2 определения.\\
    Из (3): $\forall g \in G \ \exists h_i \in H_i: g = h_1...h_k$.\\
    Допустим, что это разложение не единственно, т.е. $h_1...h_k = \tilde{h}_1...\tilde{h}_k$.\\
    Тогда $\forall i: \tilde{h}_i^{-1}h_i = \prod \limits_{j \neq i}\tilde{h}_jh_j^{-1}$, а так как $H_i \cap \langle H_j \ | \ j \neq i \rangle = \{e\}$, имеем $h_i = \tilde{h}_i$.
\end{proof}
\begin{examples}\tab
    \begin{enumerate}
        \item $V_4 = \{e, a, b, c\} = \{e, a\} \times \{e, b\} \simeq \Z_2 \times \Z_2$;
        \item $\CC^* = \R_+ \times U \ (z = r \cdot e^{iy})$.
        \item $\Z$ не раскладывается в произведение нетривиальных подгрупп.\\
        Предположим противное, т.е. $\Z = H_1\times ...\times H_m$. Подгруппы $\Z$ имеют вид $k\Z$, т.е. $\Z = k_1\Z \times ...\times k_m\Z, k_i \neq 0$. Но тогда $k_1k_2 \in H_1 \cap H_2$ и $k_1k_2 \neq 0$, что противоречит теореме 2.
    \end{enumerate}
\end{examples}
\subsection{Связь между внутренним и внешним прямым произведением}
\begin{theoremnum}\tab
    \begin{enumerate}
        \item Если группа $G$ раскладывается в прямое произведение подгрупп $H_1,...,H_k$, то $G$ изоморфна прямому произведению групп $G_1,...,G_k$, где $\forall i \ G_i \simeq H_i$;
        \item Если группа $G$ изоморфна прямому произведению групп $G_1,...,G_k$, то $\exists H_i \leq G$ такие, что $G_i \simeq H_i$ и $G$ раскладывается в прямое произведение $H_1,...,H_k$.
    \end{enumerate}
\end{theoremnum}
\begin{proof}\tab
    \begin{enumerate}
        \item Имеем: $H_i \leq G, G = H_1 \times ... \times H_k$.\\
        Рассмотрим отображение $\phi: G \rightarrow G_1 \times ... \times G_k$, где $G_i = H_i$, такое, что $\forall g = h_1...h_k \in G \ \ \phi(h_1...h_k) \mapsto (h_1,...,h_k)$. Это изоморфизм:
        \begin{itemize}
            \item Биекция - очевидна;
            \item Гомоморфизм:
        \end{itemize}
        \[\phi((h_1...h_k)\cdot(h_1'...h_k')) = \phi(h_1h_1'...h_kh_k') = (h_1h_1',...,h_kh_k') =\]
        \[= (h_1,...,h_k)\cdot(h_1',...,h_k') = \phi(h_1...h_k) \cdot \phi(h_1'...h_k')\]
        \item Имеем: $G_1,...,G_k$ - группы, $G = \{(g_1,...,g_k) \ | \ g_i \in G_i\}$.\\
        Тогда $H_i = \{(e,...,e,g_i,e,...,e) \ | \ g_i \in G_i\}$ очевидно является подгруппой $G$, изоморфной $G_i$.\\
        Покажем, что $G = H_1 \times ... \times H_k$:
        \begin{itemize}
            \item $\forall g = (g_1,...,g_k) \in G \ \exists! \ h_i  = (e,...,e,g_i,e,...,e): g = h_1...h_k$;
            \item $\forall i \neq j, h_i = ((e,...,e,a_i,e,...,e)) \in H_i, h_j = (e,...,e,b_j,e,...,e) \in H_j:$
            \[h_ih_j = (e,...,e,a_i,e,...,e, b_j,e,...,e) = h_jh_i\]
        \end{itemize}
    \end{enumerate}
\end{proof}
\begin{theoremnum}
    \label{theoremnum:th1}
    Пусть $H_i \leq G, G = H_1\times ... \times H_k, N_i \unlhd H_i$. Тогда:
    \begin{enumerate}
        \item $N_1\times ... \times N_k \unlhd G$;
        \item $G /(N_1\times ... \times N_k) \simeq (H_1/N_1) \times ... \times (H_k/N_k)$.
    \end{enumerate}
\end{theoremnum}
\begin{proof}\tab
    \begin{enumerate}
        \item Очевидно, что $N_1\times ... \times N_k = N \leq G$.\\
        Покажем нормальность: $\forall g = h_1...h_k \in G, n = n_1...n_k \in N$
        \[gng^{-1} = (h_1...h_k)(n_1...n_k)(h_k^{-1}...h_1^{-1}) \underset{(n_i \in H_i)}{=} \overset{\in N_1}{(h_1n_1h_1^{-1})}...\overset{\in N_k}{(h_kn_kh_k^{-1})} \in N\] 
        \item Рассмотрим гомоморфизм $\phi: G \rightarrow (H_1/N_1) \times ... \times (H_k/N_k)$ такой, что $\phi: h_1...h_k \mapsto (h_1N_1,...,h_kN_k)$. Это сюръективный гомоморфизм, причём $\textup{Ker }\phi = N_1\times ... \times N_k$. Отсюда по теореме о гомоморфизме получаем необходимое утверждение.
    \end{enumerate}
\end{proof}
\begin{consequense}
    Если $G = H_1 \times H_2$, то $G/H_1 \simeq H_2, G/H_2 \simeq H_1$.
\end{consequense}
\setcounter{thcount}{0}
\setcounter{concount}{0}
\setcounter{subthcount}{0}
\setcounter{lemcount}{0}
\newpage