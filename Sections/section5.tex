\section{Действия группы на множестве}
\begin{definition}
    Пусть $X$ - произвольное множество. Биективное отображение $f: X \rightarrow X$ называется преобразованием множества $X$.\\
    Множество всех преобразований $X$ обозначается $S(X)$.
\end{definition}
\begin{subtheorem}
    $S(X)$ - группа относительно композиции.
\end{subtheorem}
\begin{proof}\tab
    \begin{enumerate}
        \item Ассоциативность - очевидно;
        \item Нейтральный элемент - тождественное преобразование;
        \item Обратный элемент - обратное преобразование (существует, т.к. биекция)
    \end{enumerate}
\end{proof}
\begin{definition}
    Группа $S(X)$ называется группой всех преобразований $X$.\\
    Произвольная $H \leq S(X)$ называется группой преобразований множества $X$.
\end{definition}
\begin{example}
    $GL(V)$ - группа невырожденных линейных операторов векторного пространства $V$ : $GL(V) \leq S(V)$.
\end{example}
\begin{definition}
    Пусть $G$ - произвольная группа, $X$ - произвольное множество. Действием группы $G$ на множестве $X$ называется гомоморфизм $\alpha: G \rightarrow S(X)$.
    Обозначается $G \acts X$ (или $G : H$)\\
    Элементы множества $X$ при этом называются точками.\\
    $\forall g \in G: g \mapsto \alpha(g)$ - преобразование множества $X$, т.е. биекция $X \rightarrow X$.\\
    Равенство $\alpha(g)(x) = y (\in X)$ записывают как $\alpha(g)x = y$ или $gx = y$.
\end{definition}
Так как $\alpha$ - гомоморфизм, имеем:
\[\forall g_1, g_2 \in G: \alpha(g_1g_2) = \alpha(g_1)\alpha(g_2) \Longrightarrow \alpha(g_1g_2)x = (\alpha(g_1)\alpha(g_2))x = \alpha(g_1)(\alpha(g_2)x)\]
Отсюда $(g_1g_2)x = g_1(g_2x)$. Аналогично:
\[\forall g \in G: \alpha(g^{-1}) = (\alpha(g))^{-1} \Longrightarrow \alpha(g^{-1})x = (\alpha(g)x)^{-1}\]
Отсюда $g^{-1}x = y \Longleftrightarrow gy = x$.\\
Если $H \leq S(X)$, то определено "тавтологическое"\  действие $H$ на $X: \alpha(h) = h$ - вложение $H \rightarrow S(X)$.
\begin{example}
    $GL(V) \acts V$: $\alpha(g)x = x \ \forall g \in G, x \in X$
\end{example}
В общем случае: $\alpha G \rightarrow S(X)$ - гомоморфизм, то есть $\textup{Im }\alpha \leq S(X), \textup{Ker }\alpha \unlhd G$.
\begin{definition}
    $\textup{Ker }\alpha$ называется ядром неэффективности действия группы $G$ на $X$.\\
    Если $\textup{Ker }\alpha = \{e\}$, то действие называется эффективным.
\end{definition}
\begin{remark}
    Всякое действие группы $G$ на множестве $X$ индуцирует и другие действия. Например:
    \begin{enumerate}
        \item $G \acts 2^X$;
        \item Если $Y \subset X$ - инвариантное подмножество относительно $G$, то $G \acts Y$. 
    \end{enumerate}
\end{remark}
\begin{example}
    Пусть $K$ - равносторонний треугольник, $G = \textup{Sym }K \leq S(X)$, где $X$ - множество точек треугольника.\\
    Тогда если $Y = \{v_1, v_2, v_3\}$ - вершины треугольника, а $Z = \{e_1, e_2, e_3\}$ - стороны треугольника, то действие $G \acts X$ индуцирует также и действия $G \acts Y, G \acts Z$
\end{example}
\begin{example}
    Пусть задано $G \acts X$, $\F$ - поле, $Y = \{f: X \rightarrow \F\}$ - алгебра всех функций $X \rightarrow \F$. Рассмотрим $\alpha: G \rightarrow S(Y): \forall g \in G \ \alpha(g)f = \tilde{f}$ такое, что $\tilde{f}(x) = f(g^{-1}x) \ \forall x \in X$. Покажем, что $\alpha$ - гомоморфизм:
    \[\forall g_1, g_2 \in G: \ (\alpha(g_1g_2)f)(x) = f((g_1g_2)^{-1}(x)) = f(g_2^{-1}(g_1^{-1}x)) = (\alpha(g_2)f)(g_1^{-1}x) =\]
    \[= \alpha(g_1)(\alpha(g_2)f)(x) = (\alpha(g_1)\alpha(g_2)f)(x)\]
\end{example}
\begin{remark}
    Если $G \acts X, H \leq G$, то определено также действие $H\acts X$ - ограничение действия на подгруппу.
\end{remark}
\begin{example}
    $G = S_3 \acts X$, где $X = \{1, 2, 3\}$ - действуют как подстановки.\\
    $H = \langle (1, 2, 3) \rangle \leq G$ - определено действие $H \acts X$ как ограничение $G \acts X$.
\end{example}
\subsection{Орбиты и стабилизаторы}
\begin{subtheorem}
    Отношение, заданное правилом $x \sim y \Longleftrightarrow \exists g \in G: gx = y$, является отношением эквивалентности.
\end{subtheorem}
\begin{proof}\tab
    \begin{itemize}
        \item Рефлексивность: $\forall x \in X: ex = x \Longrightarrow x \sim x$;
        \item Симметричность: 
        \[x \sim y \Longrightarrow \exists g \in G: gx = y \Longrightarrow g^{-1}gx = g^{-1}y \Longrightarrow g^{-1}y = x \Longrightarrow y \sim x\]
        \item Транзитивность:
    \end{itemize}
    \[\begin{cases} x \sim y \\ y \sim z \end{cases} \Longrightarrow \exists g_1, g_2 \in G: \begin{cases}y = g_1x \\ z = g_2y \end{cases} \Longrightarrow z = g_2(g_1x) = (g_2g_1)x \Longrightarrow x \sim z\]
\end{proof}
\begin{definition}
    Классы эквивалентности относительно этого отношения называются орбитами относительно действия $G \acts X$.\\
    Обозначается $\textup{Orb}(x) = \{y \in X \ | \ \exists g \in G: y = gx\}$
\end{definition}
\begin{example}
    Пусть $G$ - группа поворотов плоскости $\E^2$ вокруг точки $o$.\\
    Тогда при $G \acts E^2 \ \ \textup{Orb}(x)$ - окружность с центром в точке $o$ радиуса $|ox|$.
\end{example}
\begin{definition}
    Если $\textup{Orb}(x) = \{x\}$, то $x$ называется неподвижной точкой.
\end{definition}
\begin{definition}
    Если $\textup{Orb}(x) = X$, то действие называется транзитивным.
\end{definition}
\begin{remark}
    Это именно характеристка действия, так как $\exists x: \textup{Orb}(x) = X \Rightarrow \forall x \in X \ \textup{Orb}(x) = X$.
\end{remark}
\begin{example}
    $G$ - группа сдвигов (параллельных переносов) $\E^2$.\\
    Тогда $G \acts \E^2$ - транзитивное (из любой точки можно получить любую другую сдвигом на вектор, их соединяющий).
\end{example}
\begin{subtheorem}
    Если $y \in \textup{Orb}(x)$, то $\textup{Orb}(y) = \textup{Orb}(x)$.
\end{subtheorem}
\begin{proof}
    Напрямую следует из определения орбиты.
\end{proof}
\begin{definition}
    Стабилизатором (стационарной подгруппой) точки $x$ называется множество $\textup{St}(x) = \{g \in G \ | \ gx = x\}$.
\end{definition}
\begin{subtheorem}
    $\textup{St}(x) \leq G$.
\end{subtheorem}
\begin{proof}\tab
    \begin{itemize}
        \item $g_1, g_2 \in St(x) \Longrightarrow g_1x = g_2x = x$\\
        $(g_1g_2)x = g_1(g_2x) = g_1x = x \Longrightarrow g_1g_2 \in \textup{St}(x)$;
        \item $ex = x \Longrightarrow e \in \textup{St}(x)$;
        \item Пусть $g \in \textup{St}(x)$. Тогда $g(x) = x$, а также $g(g^{-1}x) = ex = x$. Так как образ $g$ при действии - биекция, имеем $x = g^{-1}x$, то есть $g^{-1} \in \textup{St}(x)$
    \end{itemize}    
\end{proof}
\begin{subtheorem}
    Если $y = gx$, то множество $M_y = \{h \in G \ | \ y = hx\}$ совпадает с множеством $g\textup{St}(x)$.
\end{subtheorem}
\begin{proof}
    Покажем оба включения:\\
    $\tab g \textup{St}(x) \subset M_y: \ \ \forall \tilde{g} \in \textup{St}(x): \ \tilde{g} = g\cdot g'$, где $g' \in \textup{St}(x)$. Тогда: $\tilde{g}x = (gg')x = g(g'x) = gx = y \Longrightarrow \tilde{g} \in M_y$. Отсюда $g \textup{St}(x) \subset M_y$.\\
    $\tab M_y \subset g \textup{St}(x): \ \ \forall h \in M_y: \ y = hx$. Также $y = gx \Longrightarrow gx = hx \Longrightarrow (g^{-1}h)x = g^{-1}(hx) = x \Longrightarrow g^{-1}h \in \textup{St}(x) \Longrightarrow h \in g \textup{St}(x)$. Отсюда $M_y \subset g \textup{St}(x)$.
\end{proof}
\begin{theorem}
    Отображение $\psi: \textup{Orb}(x) \rightarrow G/\textup{St}(x)$ (множество левых смежных классов, не факторгруппа!) такое, что $gx \mapsto g\textup{St}(x)$, является биекцией.
\end{theorem}
\begin{proof}\tab
    \begin{itemize}
        \item Корректность: \ Пусть $y = g_1x = g_2x$. Тогда:
        \[g_1x = g_2x \Longrightarrow g_2^{-1}(g_1x) = (g_2^{-1}g_1)x = x \Longrightarrow g_2^{-1}g_1 \in \textup{St}(x) \Longrightarrow\]
        \[\Longrightarrow g_1 \in g_2\textup{St}(x)\Longrightarrow g_1\textup{St}(x) = g_2\textup{St}(x) \Longrightarrow \psi(g_1x) = \psi(g_2x)\]
        \item Сюръективность - очевидна ($\forall g \in G$ $g\textup{St}(x)$ будет образом точки $gx$);
        \item Инъективность: Пусть $\psi(g_1) = \psi(g_2) $. Тогда:
        \[g_1\textup{St}(x) = g_2\textup{St}(x) \Longrightarrow g_2^{-1}g_1 \in \textup{St}(x) \Longrightarrow (g_2^{-1}g_1)x = x \Longrightarrow g_1x = g_2x\] 
    \end{itemize}
\end{proof}
\begin{consequensenum}
    $|\textup{Orb}(x)| = |G/\textup{St}(x)| = |G : \textup{St}(x)|$.
\end{consequensenum}
\begin{consequensenum}
    Если $G$ - конечная группа, то $|\textup{Orb}(x)| = \frac{|G|}{|\textup{St}(x)|}$.
\end{consequensenum}
\begin{example} 
    Пусть $K \in \E^3$ - куб, $G = \textup{Sym}^+(K) = \{g \in \textup{Isom}^+(\E^3) \ | \ gK = K\}$ - группа вращений $K$.\\
    Найдём $|G|$. Так как $G \leq S(X)$, где $X = \{v_1,...,v_8\}$ - множество вершин куба, $|G| < \infty$. Значит, если рассмотреть индуцированное действие $G \acts X$, то $|G| = |\textup{Orb}(v_1)|\cdot |\textup{St}(v_1)|$.\\
    $\textup{Orb}(v_1) = X$ (вершина может перейти в любую) $\Longrightarrow$ $|\textup{Orb}(v_1)| = 8$;\\
    $|\textup{St}(v_1)| = 3$ ($\textup{id}$ и два поворота вокруг большой диагонали, содержащей $v_1$);\\
    Отсюда $|G| = 8 \cdot 3 = 24$.\\
    Более того, покажем, что $G \simeq S_4$. Рассмотрим множество диагоналей куба $Y = \{d_1, d_2, d_3, d_4\}$. Так как при собственном движении диагонали переходят в диагонали, можем рассмотреть действие $G \acts Y \Longrightarrow \exists \alpha : G \rightarrow S(Y) \simeq S_4$ - гомоморфизм. Из $|G| = |S_4| = 24$ для доказательства того, что $\alpha$ - изоморфизм, достаточно показать сюръективность, а для этого достаточно показать, что все транспозиции диагоналей можно получить вращениями (достаточно, т.к. $S_4$ порождается транспозициями, а $\textup{Im } \alpha \leq S(Y)$).\\
    Такая транспозиция - это поворот на $\pi$ относительно прямой, проходящей через середины двух рёбер, соединяющих концы диагоналей.
\end{example}
\begin{exercise}
    Доказать, что если $L$ - правильный тетраэдр, то $\textup{Sym}(L) \simeq S_4$.
\end{exercise}
\begin{proof}
    Будем действовать аналогично - пусть $X = \{v_1,...,v_4\}$ - множество вершин тетраэдра, тогда действие $\textup{Sym}(L) \acts E^3$ индуцирует действие $\textup{Sym}(L) \acts X$, а отсюда $|\textup{Sym}(L)| = |\textup{Orb}(v_1)|\cdot |\textup{St}(v_1)|$.\\
    $\textup{Orb}(v_1) = X$ (вершина может перейти в любую) $\Longrightarrow$ $|\textup{Orb}(v_1)| = 4$;\\
    $|\textup{St}(v_1)| = 6$ (любые перестановки вершин на грани, не содержащей $v_1$);\\
    (проверка существования всех этих движений непосредственная)\\
    Отсюда $|G| = 4 \cdot 6 = 24$.\\
    Так как $S(X) \simeq S_4$, достаточно показать, что гомоморфизм действия - изоморфизм, а из равенства порядков достаточно сюръективности. Транспозиция любых двух вершин может быть получена симметрией относительно плоскости, проходящей через середину ребра, соединяющего вершины, и противоположное ребро.
\end{proof}
\begin{definition}
    Элементы $a, b \in G$ называются сопряжёнными, если $\exists g \in G$ такой, что $b = g^{-1}ag$. Обозначается $b = a^g$.
\end{definition}
\begin{remark}
    Такое обозначение не случайно: многие свойства возведения в степень присущи и оперции сопряжения. Однако в данном курсе эти свойства пока не понадобятся. 
\end{remark}
\begin{definition}
    Подгруппы $L, K \leq G$ называются сопряжёнными, если $\exists g \in G$ такой, что $K = g^{-1}Lg = \{g^{-1}lg \ | \ l \in L\}$.
\end{definition}
\begin{subtheorem}
    Пусть $y = gx$. Тогда $g\textup{St}(x)g^{-1} = \textup{St}(y)$.
\end{subtheorem}
\begin{proof}\tab
    \begin{itemize}
        \item $g\textup{St}(x)g^{-1} \overset{?}{\subseteq} \textup{St}(y)$:\\
        $\forall h \in \textup{St}(x): \ ghg^{-1}(y) = ghg^{-1}(gx) = gh(g^{-1}g)x = ghx = gx = y \Longrightarrow ghg^{-1} \in \textup{St}(y)$;
        \item $\textup{St}(y) \overset{?}{\subseteq} g\textup{St}(x)g^{-1}$: (аналогичные рассуждения, т.к. $y = gx \Longleftrightarrow x = g^{-1}y$)\\
        $\forall h \in \textup{St}(y): \ g^{-1}hg(x) = g^{-1}hg(g^{-1}y) = g^{-1}h(gg^{-1})y = g^{-1}hy = g^{-1}y = x \Longrightarrow g^{-1}hg \in \textup{St}(x) \Longrightarrow h \in g \textup{St}(x)g^{-1}$.
    \end{itemize}
\end{proof}
\subsection{Действия группы на себе}
Пусть $G$ - группа, $X = G$. Рассмотрим основные действия $G \acts G$ и покажем некоторые их свойства:
\begin{enumerate}
    \item Действие $G \acts G$ левыми сдвигами:\\
    $\alpha: G \rightarrow S(G)$ такое, что $\forall g \in G, h \in G: \ \alpha(g)h = gh$.\\
    Покажем, что $\alpha$ - гомоморфизм:
    \[\forall g_1, g_2 \in G: \ \alpha(g_1g_2)h = (g_1g_2)h = g_1(g_2h) = \alpha(g_1)(\alpha(g_2)h) = (\alpha(g_1)\alpha(g_2))h\]
    $g \in \textup{Ker } \alpha \Longrightarrow \forall h \in G: \ gh = h \Longrightarrow g = e \Longrightarrow \textup{Ker }\alpha = \{e\}$ - действие эффективно.\\
    Значит, по теореме о гомоморфизме $G \simeq \textup{Im }\alpha \leq S(G)$.
    \begin{consequense} (Теорема Кэли)\\
        Пусть $|G| = n$. Тогда $G$ изоморфна некоторой подгруппе $S_n$.
    \end{consequense}
    \begin{proof}
        Рассмотрим гомоморфизм $\alpha: G \rightarrow S(G)$, приведённый выше. Тогда $G \simeq \textup{Im }\alpha \leq S(G) \simeq S_n$, т.к. $|G| = n$.
    \end{proof}
    \item Действие $G \acts G$ правыми сдвигами:\\
    $\alpha: G \rightarrow S(G)$ такое, что $\forall g \in G, h \in G: \ \alpha(g)h = hg^{-1}$.\\
    Покажем, что $\alpha$ - гомоморфизм:
    \[\forall g_1, g_2 \in G: \ \alpha(g_1g_2)h = hg_2^{-1}g_1^{-1} = \alpha(g_1)(hg_2^{-1}) = (\alpha(g_1)\alpha(g_2))h\]
    $g \in \textup{Ker } \alpha \Longrightarrow \forall h \in G: \ hg^{-1} = h \Longrightarrow g = e \Longrightarrow \textup{Ker }\alpha = \{e\}$ - действие эффективно.
    \item Действие $G \acts G$ сопряжениями:\\
    $\alpha: G \rightarrow S(G)$ такое, что $\forall g \in G, h \in G: \ \alpha(g)h = ghg^{-1}$.\\
    Покажем, что $\alpha$ - гомоморфизм:
    \[\forall g_1, g_2 \in G: \ \alpha(g_1g_2)h = (g_1g_2)h(g_1g_2)^{-1} = g_1(g_2hg_2^{-1})g_1^{-1} = \alpha(g_1)(\alpha(g_2)h)\]
    \begin{subtheorem}
        $\forall g \in G: \ \alpha(g): G \rightarrow G$ - автоморфизм, т.е. изоморфизм $G$ на себя.
    \end{subtheorem}
    \begin{proof} 
        Биективность $\alpha(g)$ следует из $\alpha(g) \in S(G)$. Докажем, что $\alpha(g)$ - гомоморфизм:
        \[\alpha(g)(h_1h_2) = gh_1h_2g^{-1} = (gh_1g^{-1})(gh_2g^{-1}) = (\alpha(g)h_1)(\alpha(g)h_2)\]
        Значит, $\alpha(g)$ - автоморфизм $G$.
    \end{proof}
    \begin{definition}
        Такой автоморфизм называется внутренним автоморфизмом группы $G$ (относительно элемента $g$).
    \end{definition}
\end{enumerate}
\begin{subtheorem}\tab
    \begin{enumerate}
        \item Множество $\textup{Aut }G$ всех автоморфизмов группы $G$ - группа относительно композиции, причём $\textup{Aut }G \leq S(G)$.
        \item Множество $\textup{Int }G$ всех внутренних автоморфизмов группы $G$ - группа относительно композиции, причём $\textup{Int }G \unlhd \textup{Aut }G$.
    \end{enumerate}
\end{subtheorem}
\begin{proof}\tab
    \begin{enumerate}
        \item Достаточно проверить, что $\textup{Aut } G  \leq S(G)$:
        \begin{itemize}
            \item $\alpha_1, \alpha_2 \in \textup{Aut } G \Longrightarrow (\alpha_1\alpha_2) \in \textup{Aut } G$;
            \item $\textup{id} \in \textup{Aut } G$;
            \item $\alpha \in \textup{Aut } G \Longrightarrow \alpha^{-1} \in \textup{Aut } G$ (изоморфизм обратим).
        \end{itemize} 
        \item Для определения группы достаточно проверить, что $\textup{Int } G \leq \textup{Aut } G$:
        \begin{itemize}
            \item $\alpha_1, \alpha_2 \in \textup{Int } G \Longrightarrow \exists g_1, g_2 \in G: \alpha_i$ - сопряжение относительно $g_i$. Тогда $(\alpha_1\alpha_2)$ - сопряжение относительно $g_1g_2$, т.е. $(\alpha_1\alpha_2) \in \textup{Aut } G$;
            \item $\textup{id} \in \textup{Int } G$ - сопряжение относительно $e$;
            \item $\alpha \in \textup{Int } G \Longrightarrow \alpha$ - сопряжение относительно $g \in G \Longrightarrow \alpha^{-1}$ - сопряжение относительно $g^{-1} \Longrightarrow \alpha^{-1} \in \textup{Aut } G$.
        \end{itemize}  
        Проверим, что $\textup{Int }G \unlhd \textup{Aut }G$, т.е. $\forall \phi \in \textup{Aut }G,\ g \in G : \ \phi\alpha(g)\phi^{-1} \in \textup{Int } G$:
        \[(\phi\alpha(g)\phi^{-1})(h) = \phi(\alpha(g)(\phi^{-1}(h))) = \phi(g\phi^{-1}(h)g^{-1}) = \phi(g)\phi(\phi^{-1}(h))\phi(g^{-1})\]
        \[= \phi(g)h(\phi(g))^{-1} = \alpha(\phi(g))(h) \Longrightarrow \phi\alpha(g)\phi^{-1} = \alpha(\phi(g)) \in \textup{Int } G\]
    \end{enumerate}
\end{proof}
\begin{definition}
    $\textup{Aut }G$ называется группой аутизмов$_{\href{https://ru.wikipedia.org/wiki/\%D0\%90\%D1\%83\%D1\%82\%D0\%B8\%D0\%B7\%D0\%BC}{1}}$ группы $G$.\\
    $\textup{Int }G$ называется группой внутренних автоморфизмов группы $G$.
\end{definition}
Пусть $\alpha$ - действие $G\acts G$ сопряжениями. Тогда 
$\textup{Ker } \alpha = \{g \in G \ | \ \alpha(g)h = h \\ \forall h \in G\} = \{g \in G \ | \ ghg^{-1} = h \ \forall h \in G\} = \{g \in G \ | \ gh = hg \ \forall h \in G\}$, а $\textup{Im } \alpha = \textup{Int }G$.
\begin{definition}
    Множество $Z(G) = \{g \in G \ | \ gh = hg \ \forall h \in G\}$ называется центром группы $G$.
\end{definition}
\begin{properties}\tab
    \begin{enumerate}
        \item $Z(G) = \textup{Ker } \alpha$, где $\alpha$ - действие $G \acts G$ сопряжениями;
        \item $Z(G) \unlhd G$;
        \item $\forall H \leq Z(G): \ H \unlhd G$;
        \item $Z(G) = G \Longleftrightarrow G$ - абелева
    \end{enumerate}    
\end{properties}
\begin{proof}\tab
    \begin{enumerate}
        \item Доказано выше;
        \item Следует из (1) ($\textup{Ker }\alpha \unlhd G$ - свойство гомоморфизма);
        \item $\forall h \in H \leq Z(G),\ g \in G: ghg^{-1} = gg^{-1}h = h \in H \Longrightarrow H \unlhd G$;
        \item Очевидно из определения абелевой группы.
    \end{enumerate}
\end{proof}
\subsection{Классы сопряжённости и централизаторы}
\begin{definition}
    Пусть $\alpha$ - действие $G \acts G$ сопряжениями.\\
    Классом сопряжённости $x \in G$ называется орбита $x$ относительно $\alpha$.\\
    Централизатором элемента $x \in G$ называется стабилизатор $x$ относительно $\alpha$.\\
    Класс сопряжённости обозначается как $x^G = \{y \in G \ | \ \exists g \in G: y = gxg^{-1}\}$.
    Централизатор обозначается как $C(x) = \{g \in G \ |\ gxg^{-1} = x\}$.
\end{definition}
\begin{subtheoremnum}
    Если $|G| < \infty$, то $|x^{G}| = \frac{|G|}{|C(x)|}$.
\end{subtheoremnum}
\begin{proof}
    Очевидно следует из утверждения $|\textup{Orb}(x)| = \frac{|G|}{|\textup{St}(x)|}$.
\end{proof}
\begin{subtheoremnum}
    $x^G = \{x\} \Longleftrightarrow x \in Z(G)$.
\end{subtheoremnum}
\begin{proof}
    Очевидно следует из свойства 1 центра группы.
\end{proof}
\begin{definition}
    Группа $G$ называется тривиальной, если $G = \{e\}$.
\end{definition}
\begin{theorem}
    Центр любой нетривиальной $p$-группы нетривиален ($p$ - простое).
\end{theorem}
\begin{proof}
    Пусть $|G| = p^s$. Рассмотрим случаи:
    \begin{enumerate}
        \item $G$ - абелева $\Longrightarrow Z(G) = G$.
        \item $G$ - неабелева. Тогда $G$ разбивается на несколько непересекающихся классов сопряжённости: $G = \bigsqcup \limits_{i=1}^k x_i^G$.\\
        По утверждению 2 $|x_i^G| = 1 \Longleftrightarrow x_i \in Z(G)$, а по утверждению 1 $|x_i^G| = \frac{|G|}{C(x_i)}$\\
        Так как $G$ - $p$-группа, для $x_i \notin Z(G): \  |x_i^G| = p^{s_i}, s_i \geqslant 1$.\\
        Без ограничения общности пусть только $x_1,...,x_m \in Z(G)$ (всегда будет хотя бы один, так как $e \in (G)$). Тогда: 
        \[|G| = \undermat{|Z(G)|}{|x_1^G| + ... + |x_m^G|} + |x_{m+1}^G| + ... + |x_k^G| \Longrightarrow p^s = |Z(G)| + p^{s_{m+1}} + ... + p^{s_k}\]
        \\
        Отсюда $p \mid |Z(G)|$, а значит, $|Z(G)| \geqslant p > 1$ - центр нетривиален.
    \end{enumerate}
\end{proof}
\begin{consequense}
    Если $|G| = p^2$, где $p$ - простое, то $G$ - абелева.
\end{consequense}
\begin{proof}
    $G$ - $p$-группа $\Longrightarrow Z(G) \neq \{e\}$.\\
    Предположим, что $G$ неабелева, т.е. что $Z(G) \neq G$. \\
    Тогда, так как $|Z(G)| \mid |G| = p^2$ и $|Z(G)| \neq 1, p^2$, имеем $|Z(G)| = p$.\\
    Рассмотрим группу $G / Z(G)$. Её порядок равен $\frac{|G|}{|Z(G)|} = \frac{p^2}{p} = p \Longrightarrow G / Z(G)$ - циклическая, а значит, $G / Z(G) = \langle aZ(G) \rangle$. Тогда $\forall g \in G \ \exists t \in \Z: g \in a^tZ(G)$. Рассмотрим два произвольных элемента $g_1, g_2 \in G$ и докажем, что $g_1g_2 = g_2g_1$:
    \[\exists t_1, t_2 \in \Z: g_1 = a^{t_1}Z(G), g_2 = a^{t_2}Z(G) \Longrightarrow \exists z_1, z_2 \in Z(G):  g_1 = a^{t_1}z_1, g_2 = a^{t_2}z_2\]
    Так как элементы центра коммутируют со всеми элементами $G$, имеем:
    \[g_1g_2 = a^{t_1}z_1a^{t_2}z_2 = a^{t_1 + t_2}z_1z_2 = a^{t_2 + t_1}z_2z_1 = a^{t_2}z_2a^{t_1}z_1 = g_2g_1\]
    а значит, $G$ - абелева, что противоречит предположению.\\
    Отсюда $G$ не может быть неабелевой, т.е. $G$ - абелева. 
\end{proof}
\begin{lemmanum}
    Пусть $X$ - произвольное множество, $G \leq S(X)$. Тогда если $\phi \in G$ т.ч. $\phi: x \mapsto y$, то $\forall \psi \in G: \ \psi \circ \phi \circ \psi^{-1}: \psi(x) \mapsto \psi(y)$.
\end{lemmanum}
\begin{proof}
    Применим преобразование $\psi \circ \phi \circ \psi^{-1}$:
    \[(\psi \circ \phi \circ \psi^{-1})(\psi(x)) = \psi (\phi (\psi^{-1}(\psi(x)))) = \psi(\phi(x)) = \psi(y)\]
\end{proof}
\begin{subtheoremnum}
    Пусть $\sigma, \tilde{\sigma} \in S_n$. Тогда $\sigma, \tilde{\sigma}$ сопряжены в $S_n \Longleftrightarrow \sigma, \tilde{\sigma}$ имеют одинаковые цикловые структуры, т.е. наборы длин независимых циклов в разложении $\sigma, \tilde{\sigma}$ совпадают.
\end{subtheoremnum}
\begin{proof}
    $ \\ \Longrightarrow: \ \ $ Пусть $\sigma, \tilde{\sigma}$ сопряжены в $S_n \Longrightarrow \exists \tau \in S_n: \tilde{\sigma} = \tau \sigma \tau^{-1}$.\\
    Пусть $\sigma = (i_1i_2...i_s)(j_1j_2...j_t)...$ - разложение $\sigma$ в независимые циклы. Тогда $\sigma: i_1 \mapsto i_2, i_2 \mapsto i_3,..., i_s \mapsto i_1$, а тогда по лемме 1 $\tau\sigma\tau^{-1}: \tau(i_1) \mapsto \tau(i_2), \tau(i_2) \mapsto \tau(i_3),..., \tau(i_s) \mapsto \tau(i_1)$. Аналогичное рассуждение можно провести для всех независимых циклов $\sigma$, а значит, $\tau\sigma\tau^{-1} = (\tau(i_1)\tau(i_2)...\tau(i_s))(\tau(j_1)\tau(j_2)...\tau(j_t))...$ - длины циклов сохраняются.\\
    $\Longleftarrow: \ \ $ Пусть $\sigma, \tilde{\sigma}$ имеют одинаковые цикловые структуры. Можем поменять порядок циклов так, чтобы длины $i$-х циклов в $\sigma$ и $\tilde{\sigma}$ совпадали, т.е.
    \[\sigma = (i_1i_2...i_s)(j_1j_2...j_t)...; \ \ \tilde{\sigma} = (\tilde{i}_1\tilde{i}_2...\tilde{i}_s)(\tilde{j}_1\tilde{j}_2...\tilde{j}_t)...\]
    Тогда если $\tau = \begin{pmatrix} i_1&i_2&...&i_s&j_1&j_2&...&j_t&...\\ \tilde{i}_1&\tilde{i}_2&...&\tilde{i}_s&\tilde{j}_1&\tilde{j}_2&...&\tilde{j}_t&... \end{pmatrix}$, то по лемме 1 $\tilde{\sigma} = \tau \sigma \tau^{-1}$.  
\end{proof}
\begin{examples}
    $\sigma = (12)(345)(6)(7), \tilde{\sigma} = (15)(243)(6)(7)$ - сопряжены в $S_7$:\\$\tau = \begin{pmatrix} 1&2&3&4&5&6&7\\1&5&2&4&3&6&7  \end{pmatrix}$(из построения в теореме);\\
    $\sigma = (123)(45), \tau = (135) \Longrightarrow \tau\sigma\tau^{-1} = (325)(41)$.
\end{examples}
\begin{consequense}
    $Z(S_n) = \{\textup{id}\}$ при $n \leqslant 3$.
\end{consequense}
\begin{proof}
    Допустим, что в $Z(G)$ есть $\sigma \neq \textup{id}$. Разложим в независимые циклы: $\sigma = (ij...)...$. Так как $n \geqslant 3, \exists k \neq i, j$. Тогда при $\tau = (jk): \ \tau\sigma\tau^{-1} = (ik...)...$ - не совпадёт с $\sigma$ ($\tau\sigma\tau^{-1}(i) \neq \sigma(i)$) - противоречие.  
\end{proof}
\begin{exercise}
    Докажите, что $Z(A_n) = \{\textup{id}\}$ при $n \geqslant 4$.
\end{exercise}
\begin{proof}
    Допустим, что в $Z(G)$ есть $\sigma \neq \textup{id}$. Разложим в независимые циклы: $\sigma = (ij...)...$. Так как $n \geqslant 4, \exists k, l: k,l,i,j$ попарно различны. Тогда при $\tau = (jkl): \ \tau\sigma\tau^{-1} = (ik...)...$ - не совпадёт с $\sigma$ ($\tau\sigma\tau^{-1}(i) \neq \sigma(i)$)- противоречие.  
\end{proof}
\begin{subtheorem}
    $ \\H \unlhd G \Longleftrightarrow \begin{cases}
        H \leq G\\
        H \text{ - объединение нескольких классов сопряжённости } G
    \end{cases}$
\end{subtheorem}
\begin{proof}
    $ \\ \Longrightarrow: \ \ $ Пусть $H \unlhd G$. Очевидно, что $H \leq G$.\\
    Если $h \in H$, то $\forall g \in G  \ ghg^{-1} \in H$ - $H$ содержит классы сопряжённости всех её элементов $\Longrightarrow H = \bigcup \limits_{h \in H} h^G$.\\
    $\Longleftarrow: \ \ $ Пусть $H \leq G$ и $H$ - объединение классов сопряжённости. Тогда $\forall h \in H, g \in G: ghg^{-1} \in H$ ($H$ содержит весь класс сопряжённости $h^G$) $\Longrightarrow H \unlhd G$.   
\end{proof}
\setcounter{thcount}{0}
\setcounter{concount}{0}
\setcounter{subthcount}{0}
\setcounter{lemcount}{0}
\newpage
