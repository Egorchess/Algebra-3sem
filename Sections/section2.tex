\section{Свободные группы}
\begin{definition}
    Тривиальные (групповые) соотношения - соотношения, которые выводятся из аксиом группы (и, соответственно, есть в любой группе).
\end{definition}
Построим группу, в которой нет других соотношений.
\begin{definition}
    Пусть $A$ - множество символов (букв), $A^{-1}$ - множество символов (букв) $a^{-1}$, где $a \in A$.\\
    Условия на эти множества:
    \begin{enumerate}
        \item $\forall a^{-1} \in A^{-1} \Longrightarrow a^{-1} \notin A$;\\
        $\forall a \in A \Longrightarrow a \notin A^{-1}$;
        \item $(a^{-1})^{-1} = a$;\\
        Буквы $a, a^{-1}$ назовём взаимно обратными.
    \end{enumerate}
Множество $A^{\pm 1} = A \sqcup A^{-1}$ называется алфавитом.\\
Слово в алфавите $A^{\pm 1}$ - конечная последовательность букв $X = x_1...x_k$, где $x_i \in A^{\pm 1}$.\\
Длина слова $X$ (обозначается $|X|$) - количество букв в $X$.
\end{definition}
\begin{example}
    $A = \{a, b\}: X = abaab^{-1} \Rightarrow |X| = 5$.
\end{example} 
\begin{definition}
    Слово $X = x_1...x_k$ - сократимое, если $\exists i \in \overline{1,..,k-1}: x_i = x^{-1}_{i+1}$.\\
    Сокращением взаимно обратных букв назовём вычёркивание пары $x_i, x_{i+1}$ из $X$ (получим слово длины $|X| - 2$).\\
    За конечное число сокращений получим слово $\tilde{X}$, не являющееся сократимым - такое $\tilde{X}$ называется результатом полного сокращения слова $X$.
\end{definition}
\begin{definition}
    Рассмотрим множество $F(A)$ всех несократимых слов в $A^{\pm 1}$.\\
    Введём бинарную операцию на $F(A)$: пусть $X = x_1...x_k, Y = y_1...y_m$.\\
    Если $x_k \neq y_1^{-1}$, то $XY$ - конкатенация (приписывание) $X$ и $Y$:\\
    $XY = x_1...x_ky_1...y_m, |XY| = k+m$.\\
    Если $x_k = y_1^{-1}$, то $XY$ - результат полного сокращения слова $x_1...x_ky_1...y_m$.
\end{definition}
\begin{example}
    $(abcda^{-1}b)(b^{-1}ad^{-1}aab) = abcaab$.
\end{example}
\begin{definition}
    Если $|X| = 0$, то $X$ называется пустым словом (обозначим $\lambda$). Пустое слово по определению несократимо и лежит в $F(A)$.
\end{definition}
\begin{theorem}
    $F(A)$ с приведённой выше бинарной операцией - группа.
\end{theorem}
\begin{proof}\tab
    \begin{enumerate}
        \item Ассоциативность:\\
        Пусть $X = x_1...x_k, Z = z_1...z_m$.\\
        Случай $|Y| = 0 \Longrightarrow Y = \lambda$ очевиден ($XZ = XZ$);\\
        Индукция по длине слова $Y$:\\
        База индукции: $|Y| = 1 \Longrightarrow Y = a \in A^{\pm 1}$. Индукция по $|X| + |Z|$:\\
        \tab База внутренней индукции: \\
        \tab $|X| + |Z| = 0$ - очевидно ($a = a$);\\
        \tab $|X| + |Z| = 1$ - очевидно (одно из слов $X, Z$ пустое);\\
        \tab Шаг внутренней индукции ($k+m-2 \rightarrow k+m$) - рассмотрим случаи:
        \begin{itemize}
            \item $a^{-1} \neq x_k, a^{-1} \neq z_1: X(YZ) = x_1...x_kaz_1...z_m = (XY)Z$;
            \item $a^{-1} = x_k, a^{-1} \neq z_1: X(aZ) = X(az_1...z_m) =\\=$ результат полного сокращения $x_1...x_{k-1}a^{-1}az_1...z_m =\\=$ результат полного сокращения $x_1...x_{k-1}z_1...z_m = (Xa)Z$;
            \item $a^{-1} \neq x_k, a^{-1} = z_1$ - аналогично предыдущему;
            \item $a^{-1} = x_k, a^{-1} = z_1$: пусть $X = X'a^{-1}, Z = a^{-1}Z'$. Тогда:\\
            $X(aZ) = X(a(a^{-1}Z')) = XZ' = (X'a^{-1})Z'$\\
            $(Xa)Z = (X'a^{-1}a)Z = X'Z = X'(a^{-1}Z')$\\
            При этом $|X'| + |Y'| = k + m - 2$, то есть $X'(a^{-1}Z') = (X'a^{-1})Z'$ по предположению внутренней индукции.
        \end{itemize}
        Во всех случаях $X(aZ) = (Xa)Z \Longrightarrow$ база доказана.\\
        Шаг индукции: Пусть $Y = y_1...y_l$. Тогда:
        \[X(YZ) = X(y_1...y_l\cdot Z) = X((y_1...y_{l-1}\cdot y_l)Z) \overset{1}{=} X((y_1...y_{l-1})\cdot (y_lZ)) \overset{2}{=}\] 
        \[\overset{2}{=} (X\cdot y_1...y_{l-1}) (y_lZ )\overset{3}{=} (X\cdot y_1...y_l)Z = (XY)Z\]
        1, 3 - из утверждения базы индукции; 2 - по предположению индукции.
        \item $\lambda$ - нейтральный элемент;
        \item обратный элемент к $x_1...x_k$ - элемент $x_k^{-1}...x_1^{-1}$.
    \end{enumerate}
\end{proof}
\begin{definition}
    Построенная группа $F(A)$ называется свободной группой с базисом $A$. ($A$ также называется свободной порождающей системой группы).\\
    Любая группа, изоморфная $F(A)$, также называется свободной.
\end{definition}
\begin{subtheorem}
    Пусть $H \leq SL_2(\Z): H = \langle \begin{pmatrix} 1&m\\0&1\end{pmatrix}, \begin{pmatrix} 1&0\\m&1\end{pmatrix} \rangle$.\\
    Тогда $H \simeq F(A)$ с базисом $A = \{a, b\}$.
\end{subtheorem}
\begin{proof}
    Без доказательства.
\end{proof}
\begin{subtheorem}
    Все базисы свободной группы равномощны.
\end{subtheorem}
\begin{proof}
    Без доказательства.
\end{proof}
\begin{definition}
    Ранг свободной группы - мощность её базиса.
\end{definition}
\begin{remark}
    Заметим, что в $F(A)$ результат умножения определён однозначно $\Longrightarrow$ однозначно определён элемент $x_1\cdot ... \cdot x_k$, где $x_i \in A^{\pm 1}$.\\
    Тогда если считать слово $x_1...x_k$ результатом умножения $x_1\cdot ... \cdot x_k$, то можно опускать знак умножения, и в этом смысле работать и с сократимыми словами. 
\end{remark}
\begin{example}
    $abb^{-1}ba^{-1}a = a \cdot b \cdot b^{-1}\cdot b\cdot a^{-1}\cdot a = ab \in F(A)$.
\end{example}
\begin{theoremnum}[Универсальное свойство свободной группы]
    $ \\$Пусть $G$ - группа, $\{g_i \ | \ i \in I\} \subset G$ - произвольное множество её элементов.\\
    Рассмотрим свободную группу $F(A)$ с базисом $A = \{a_i \ | \ i \in I\}$.\\
    Тогда отображение $\phi: a_i \mapsto g_i$ продолжается до гомоморфизма $\phi: F(A) \rightarrow G$, причём единственным образом.
\end{theoremnum}
\begin{proof}
    Пусть $W = a_{i_1}^{\epsilon_1}...a_{i_k}^{\epsilon_k}$ - несократимое слово из $F(A)$, где $\epsilon_i = \pm 1, a_{i_j} \in A$. Зададим $\phi: F(A) \rightarrow G$ по правилу $\phi(W) = g_{i_1}^{\epsilon_1}...g_{i_k}^{\epsilon_k}$.\\
    Проверим, что $\phi$ - гомоморфизм ($W, \tilde{W} \in F(A), W = a_{i_1}^{\epsilon_1}...a_{i_k}^{\epsilon_k}, \tilde{W} = a_{j_1}^{\tau_1}...a_{j_m}^{\tau_m}$):
    \[\phi(W\tilde{W}) = \phi(a_{i_1}^{\epsilon_1}...a_{i_k}^{\epsilon_k}\cdot a_{j_1}^{\tau_1}...a_{j_m}^{\tau_m}) = g_{i_1}^{\epsilon_1}...g_{i_k}^{\epsilon_k}\cdot g_{j_1}^{\tau_1}...g_{j_m}^{\tau_m} =\]
    \[=(g_{i_1}^{\epsilon_1}...g_{i_k}^{\epsilon_k})\cdot (g_{j_1}^{\tau_1}...g_{j_m}^{\tau_m}) = \phi(W)\phi(\tilde{W})\]
    Единственность такого гомоморфизма очевидна:\\
    $\phi(a_{i_1}^{\epsilon_1}...a_{i_k}^{\epsilon_k}) = \phi(a_{i_1})^{\epsilon_1}...\phi(a_{i_k})^{\epsilon_k} = g_{i_1}^{\epsilon_1}...g_{i_k}^{\epsilon_k}$ - определено однозначно.
\end{proof}
\begin{example} 
    (несвободной группы)\\
    $S_3 = \langle (12), (123) \rangle: \forall g \in S_3 \ g^6 = id$. Попытаемся продолжить до гомоморфизма $S_3 \rightarrow Q_8$ отображение $\phi: (12) \mapsto i, (123) \mapsto j$:\\
    $-1 = i^2 = \phi((12))^2 = \phi((12)^2) = \phi(id) = 1$ - противоречие. 
\end{example}
\begin{consequensenum}
    Пусть $G$ - группа, $M = \{g_i \ | \ i \in I\}$ - порождающее множество $G$, $F(A)$ - свободная группа с базисом $A = \{a_i \ | \ i \in I\}$.\\
    Тогда $\exists!$ сюръективный гомоморфизм $\phi: F(A) \rightarrow G$ такой, что $\forall i \in I: \phi(a_i) = g_i$.
\end{consequensenum}
\begin{proof}
    Достаточно показать, что в этом случае гомоморфизм из доказательства теоремы сюръективен - это следует из того, что множество $\{g_i \ | \ i \in I\}$ порождает группу $G$ (каждый элемент представим как $g_{i_1}^{\epsilon_1}...g_{i_k}^{\epsilon_k} = \phi(a_{i_1}^{\epsilon_1}...a_{i_k}^{\epsilon_k})$).
\end{proof}
\begin{consequensenum}
    Любая группа $G$ изоморфна факторгруппе некоторой свободной группы по некоторой её нормальной подгруппе.
\end{consequensenum}
\begin{proof}
    Пусть $\phi: F(A) \rightarrow G$ - гомоморфизм из следствия 1.\\
    Так как $\textup{Ker }\phi \unlhd F(A)$, из теоремы о гомоморфизме $G = \textup{Im }\phi \simeq F(A)/\textup{Ker }\phi$.
\end{proof}
\begin{definition}
    Сюръективный гомоморфизм $\phi: F(A) \rightarrow G$ - из следствия 1 называется копредставлением группы $G$.
\end{definition}
\begin{remark}
    Копредставление зависит от выбора порождающего множества $M$.
\end{remark}
\subsection{Задание группы порождающими и определяющими соотношениями}
По следствию 2: $G \simeq F(A)/N$, где $N \unlhd F(A)$. Отсюда задание группы $G$ сводится к заданию $A$ и $N$.\\
$N$ - нормальная $\Longrightarrow \forall f \in F(A), \forall h \in N: fhf^{-1} \in N$.
\begin{definition}
    Пусть $\mathcal{R} \subseteq F(A)$. Нормальным замыканием множества $\mathcal{R}$ в группе $F(A)$ называется наименьшая (по включению) нормальная подгруппа, содержащая $\mathcal{R}$. Обозначается $\langle \langle \mathcal{R} \rangle \rangle^{F(A)}$
\end{definition}
\begin{subtheorem}
    $ \\\langle \langle \mathcal{R} \rangle \rangle^{F(A)} = \{(f_1r_1^{\epsilon_1}f_1^{-1})...(f_kr_k^{\epsilon_k}f_k^{-1}) \ | \ r_i \in \mathcal{R}, f_i \in F(A), \epsilon_i = \pm 1\}$
\end{subtheorem}
\begin{proof}
    $ \\$Пусть $\{(f_1r_1^{\epsilon_1}f_1^{-1})...(f_kr_k^{\epsilon_k}f_k^{-1}) \ | \ r_i \in \mathcal{R}, f_i \in F(A), \epsilon_i = \pm 1\} = H$. Тогда:\\
    $\langle \langle \mathcal{R} \rangle \rangle^{F(A)} \unlhd F(A) \Longrightarrow \forall r_i \in \mathcal{R}, f_i \in F(A), \epsilon_i \in \{\pm 1\}: f_ir_i^{\epsilon_i}f_i^{-1} \in \langle \langle \mathcal{R} \rangle \rangle^{F(A)} \Longrightarrow H \subseteq \langle \langle \mathcal{R} \rangle \rangle^{F(A)}$. Осталось показать, что $H \unlhd F(A)$:
    \[\forall h \in H, g \in F(A): ghg^{-1} = g(f_1r_1^{\epsilon_1}f_1^{-1})...(f_kr_k^{\epsilon_k}f_k^{-1})g^{-1} =\]
    \[=((gf_1)r_1^{\epsilon_1}(f_1^{-1}g^{-1}))...((gf_k)r_k^{\epsilon_k}(f_k^{-1}g^{-1})) = \]
    \[=((gf_1)r_1^{\epsilon_1}(gf_1)^{-1})...((gf_k)r_k^{\epsilon_k}(gf_k)^{-1}) \in H\]
    Отсюда минимальная группа, содержащая $\mathcal{R}$, в точности равна $H$.
\end{proof}
\begin{subtheorem}
    Любую нормальную подгруппу $N \unlhd F(A)$ можно задать как $N = \langle \langle \mathcal{R} \rangle \rangle^{F(A)}$ для подходящего $\mathcal{R} \subset F(A)$.
\end{subtheorem}
\begin{proof}
    Очевидно, подойдёт $\mathcal{R} = N$.
\end{proof}
\textbf{Элементарные преобразования над словами в $F(A)$:}\\
(под словами в $F(A)$ подразумеваются любые произведения букв, а не только элементы $F(A)$)
\begin{itemize}
    \item ЭП1: $W = W_1a^\epsilon a^{-\epsilon}W_2 \mapsto \tilde{W} = W_1W_2$, где $a \in A, \epsilon = \pm 1$;
    \item ЭП2: $W = W_1r^\epsilon W_2 \mapsto \tilde{W} = W_1W_2$, где $r \in \mathcal{R}, \epsilon = \pm 1$;
    \item ЭП$1'$ - обратное к ЭП1;
    \item ЭП$2'$ - обратное к ЭП2;
\end{itemize}
\begin{definition}
    Назовём слова $W$ и $\tilde{W}$ $\mathcal{R}$-эквивалентными, если от $W$ можно с помощью ЭП перейти к $\tilde{W}$.
\end{definition}
\begin{subtheorem}
    $\mathcal{R}$-эквивалентность - отношение эквивалентности.
\end{subtheorem}
\begin{proof}\tab
    \begin{itemize}
        \item Рефлексивность - очевидно;
        \item Симметричность - следует из обратимости каждого ЭП;
        \item Транзитивность - очевидно; 
    \end{itemize}
\end{proof}
\begin{theoremnum}
    Следующие условия эквивалентны:
    \begin{enumerate}
        \item $W \in \langle \langle \mathcal{R} \rangle \rangle^{F(A)}$;
        \item $W \ \mathcal{R}$-эквивалентно пустому слову $\lambda$;
        \item Если для произвольной группы $G$ с порождающим множеством $M = \{g_i \ | \ i \in I\}$ (т.е. заданным копредставлением $\phi: F(A) \rightarrow G$) верно, что $\forall r \in \mathcal{R}: \phi(r) = 1$ в $G$, то $\phi(W) = 1$ в $G$.
    \end{enumerate}
\end{theoremnum}
\begin{proof}\tab
    \begin{itemize}
        \item $1 \Longrightarrow 2: \ \ W \in \langle \langle \mathcal{R} \rangle \rangle^{F(A)} \Longrightarrow W = (f_1r_1^{\epsilon_1}f_1^{-1})...(f_kr_k^{\epsilon_k}f_k^{-1}) \underset{\text{ЭП2}}{\Longrightarrow} W \sim \tilde{W} = (f_1f_1^{-1})...(f_kf_k^{-1}) \underset{\text{ЭП1}}{\Longrightarrow} \lambda$;
        \item $2 \Longrightarrow 3$ Пусть $\phi: F(A) \rightarrow G$ взят из условия теоремы. Покажем, что при ЭП образ слова не меняется:
        \begin{enumerate}
            \item $\phi(W_1a^\epsilon a^{-\epsilon}W_2) =  \phi(W_1)\phi(a)^\epsilon\phi(a)^{-\epsilon}\phi(W_2) = \phi(W_1)\phi(W_2) = \phi(W_1W_2)$;
            \item $\phi(W_1r^\epsilon W_2) =  \phi(W_1)\phi(r)^\epsilon\phi(W_2) = \phi(W_1)\cdot 1^\epsilon \cdot \phi(W_2) = \phi(W_1W_2)$;
        \end{enumerate}
        При ЭП, обратных этим, образ слова аналогично не изменяется.\\
        Тогда если $W \underset{\text{ЭП}}{\sim} \lambda$, то $\phi(W) = \phi(\lambda) = 1$.  
        \item $3 \Longrightarrow 1: \ \ \forall r \in \mathcal{R}: \phi(r) = 1 \Longrightarrow r \in \textup{Ker }\phi; \ \phi(W) = 1 \Longrightarrow W \in \textup{Ker }\phi$.\\
        Рассмотрим в качестве $G$ группу $F(A)/N$, где $N = \langle \langle \mathcal{R} \rangle \rangle^{F(A)}$, а в качестве $\phi$ - $\pi$ (естественный гомоморфизм $F(A) \rightarrow F(A)/N$).\\
        $r \in N \Longrightarrow \pi(r) = 1$. Тогда по условию 3: $\pi(W) = 1 \Longrightarrow W \in \textup{Ker }\phi = N$.
    \end{itemize}
\end{proof}
\begin{definition}
    Если $W \in F(A)$ удовлетворяет любому из условий теоремы 2, то говорят, что соотношение $W = 1$ следует из соотношений $\{r = 1 \ | \ r \in \mathcal{R}\}$ или является следствием соотношений $\mathcal{R}$.
\end{definition}
\begin{definition}
    Рассмотрим копредставление произвольной группы $G$, т.е. $\phi: F(A) \rightarrow G$, где $A = \{a_i \ | \ i \in I\}$. Пусть слово $W \in F(A) (W = a_{i_1}^{\epsilon_1}...a_{i_k}^{\epsilon_k})$ такое, что $\phi(W) = g_{i_1}^{\epsilon_1}...g_{i_k}^{\epsilon_k} = 1$ в $G$.\\
    Тогда говорят о соотношении $W = 1$.\\
    (Для упрощения записи вместо $g_i$ пишут $a_i$).
\end{definition}
\begin{definition}
    Множество $\mathcal{R} \subset F(A)$ называется определяющим множеством соотношений группы $G$, если любое соотношение группы $G$ следует из $\mathcal{R}$.\\
    При этом элементы $\mathcal{R}$ называются определяющими соотношениями $G$.
    Обозначается $G = \langle A \ | \ \mathcal{R} \rangle$ (данная запись также называется копредставлением $G$).
\end{definition}
\begin{examples}\tab
    \begin{enumerate}
        \item $\Z_3 = \langle a | a^3 = 1 \rangle; a^{12} = 1$ - следствие;
        \item $V_4 = \langle a, b | a^2 = b^2 = 1, ab=ba \rangle; (ab)^2 = 1$ - следствие.
    \end{enumerate}
\end{examples}
\begin{theorem}[Теорема Дика]
    $ \\$Пусть $G$ - группа, заданная копредставлением $\langle A \ | \ R \rangle$, где $A = \{a_i \ | \ i \in I\}$.\\ Пусть $H$ - произвольная группа, $\{h_i \ | \ i \in I\} \subset H$ - произвольное множество её элементов.\\
    Тогда отображение $\phi$ на порождающих $\phi: a_i \mapsto h_i \ \forall i \in I$ продолжается до гомоморфизма $\phi: G \rightarrow H$ тогда и только тогда, когда $\forall r \in \mathcal{R}: \ \phi(r) = 1$ в $H$.
\end{theorem}
\begin{proof}
    Если $\phi: a_i \mapsto h_i$ и $\phi$ - гомоморфизм, то должно выполняться $\phi(a_{i_1}^{\epsilon_1}...a_{i_k}^{\epsilon_k}) = h_{i_1}^{\epsilon_1}...h_{i_k}^{\epsilon_k}$. Если это отображение корректно, то очевидно, что оно является искомым гомоморфизмом. Покажем корректность:\\
    Пусть $W = \tilde{W}$ в $G$. Тогда $\tilde{W}W^{-1} = 1$ в $G \Longrightarrow \tilde{W}W^{-1} \in \langle \langle \mathcal{R} \rangle \rangle^{F(A)}$ (так как по определению копредставления соотношение $\tilde{W}^{-1}W = 1$ следует из $R$).\\
    Отсюда $\tilde{W}W^{-1} \sim \lambda \Longrightarrow W \sim \tilde{W}W^{-1}W = \tilde{W}$. Из размышлений доказательства перехода $2 \Longrightarrow 3$ теоремы 2 видно, что из условия $\forall r \in \mathcal{R}: \ \phi(r) = 1$ в $H$ следует, что образ не изменяется при ЭП, то есть $\phi(W) = \phi(\tilde{W})$, т.е. отображение корректно.
\end{proof}
\setcounter{thcount}{0}
\setcounter{concount}{0}
\setcounter{subthcount}{0}
\newpage
