\section{Свободные группы}
\begin{definition}
    Тривиальные (групповые) соотношения - соотношения, которые выводятся из аксиом группы (и, соответственно, есть в любой группе).
\end{definition}
Построим группу, в которой нет других соотношений.
\begin{definition}
    Пусть $A$ - множество символов (букв), $A^{-1}$ - множество символов (букв) $a^{-1}$, где $a \in A$.\\
    Условия на эти множества:
    \begin{enumerate}
        \item $\forall a^{-1} \in A^{-1} \Longrightarrow a^{-1} \notin A$;\\
        $\forall a \in A \Longrightarrow a \notin A^{-1}$;
        \item $(a^{-1})^{-1} = a$;\\
        Буквы $a, a^{-1}$ назовём взаимно обратными.
    \end{enumerate}
Множество $A^{\pm 1} = A \sqcup A^{-1}$ называется алфавитом.\\
Слово в алфавите $A^{\pm 1}$ - конечная последовательность букв $X = x_1...x_k$, где $x_i \in A^{\pm 1}$.\\
Длина слова $X$ (обозначается $|X|$) - количество букв в $X$.
\end{definition}
\begin{example}
    $A = \{a, b\}: X = abaab^{-1} \Rightarrow |X| = 5$.
\end{example} 
\begin{definition}
    Слово $X = x_1...x_k$ - сократимое, если $\exists i \in \overline{1,..,k-1}: x_i = x^{-1}_{i+1}$.\\
    Сокращением взаимно обратных букв назовём вычёркиванием пары $x_i, x_{i+1}$ из $X$ (получим слово длины $|X| - 2$).\\
    За конечное число сокращений получим слово $\tilde{X}$, не являющееся сократимым - такое $\tilde{X}$ называется результатом полного сокращения слова $X$.
\end{definition}
\begin{definition}
    Рассмотрим множество $F(A)$ всех несократимых слов в $A^{\pm 1}$.\\
    Введём бинарную операцию на $F(A)$: пусть $X = x_1...x_k, Y = y_1...y_m$.\\
    Если $x_k \neq y_1^{-1}$, то $XY$ - конкатенация (приписывание) $X$ и $Y$:\\
    $XY = x_1...x_ky_1...y_m, |XY| = k+m$.\\
    Если $x_k = y_1^{-1}$, то $XY$ - результат полного сокращения слова $x_1...x_ky_1...y_m$.
\end{definition}
\begin{example}
    $(abcda^{-1}b)(b^{-1}ad^{-1}aab) = abcaab$.
\end{example}
\begin{definition}
    Если $|X| = 0$, то $X$ называется пустым словом (обозначим $\lambda$). Пустое слово по определению несократимо и лежит в $F(A)$.
\end{definition}
\begin{theorem}
    $F(A)$ с приведённой выше бинарной операцией - группа.
\end{theorem}
\begin{proof}\tab
    \begin{enumerate}
        \item Ассоциативность:\\
        Пусть $X = x_1...x_k, Z = z_1...z_m$.\\
        Случай $|Y| = 0 \Longrightarrow Y = \lambda$ очевиден ($XZ = XZ$);\\
        Индукция по длине слова $Y$:\\
        База индукции: $|Y| = 1 \Longrightarrow Y = a \in A^{\pm 1}$. Индукция по $|X| + |Z|$:\\
        \tab База внутренней индукции: \\
        \tab $|X| + |Z| = 0$ - очевидно ($a = a$);\\
        \tab $|X| + |Z| = 1$ - очевидно (одно из слов $X, Z$ пустое);\\
        \tab Шаг внутренней индукции ($k+m-2 \rightarrow k+m$) - рассмотрим случаи:
        \begin{itemize}
            \item $a^{-1} \neq x_k, a^{-1} \neq z_1: X(YZ) = x_1...x_kaz_1...z_m = (XY)Z$;
            \item $a^{-1} = x_k, a^{-1} \neq z_1: X(aZ) = X(az_1...z_m) =\\=$ результат полного сокращения $x_1...x_{k-1}a^{-1}az_1...z_m =\\=$ результат полного сокращения $x_1...x_{k-1}z_1...z_m = (Xa)Z$;
            \item $a^{-1} \neq x_k, a^{-1} = z_1$ - аналогично предыдущему;
            \item $a^{-1} = x_k, a^{-1} = z_1$: пусть $X = X'a^{-1}, Z = a^{-1}Z'$. Тогда:\\
            $X(aZ) = X(a(a^{-1}Z')) = XZ' = (X'a^{-1})Z'$\\
            $(Xa)Z = (X'a^{-1}a)Z = X'Z = X'(a^{-1}Z')$\\
            При этом $|X'| + |Y'| = k + m - 2$, то есть $X'(a^{-1}Z') = (X'a^{-1})Z'$ по предположению внутренней индукции.
        \end{itemize}
        Во всех случаях $X(aZ) = (Xa)Z \Longrightarrow$ база доказана.\\
        Шаг индукции: Пусть $Y = y_1...y_l$. Тогда:
        \[X(YZ) = X(y_1...y_l\cdot Z) = X((y_1...y_{l-1}\cdot y_l)Z) \overset{1}{=} X((y_1...y_{l-1})\cdot (y_lZ)) \overset{2}{=}\] 
        \[\overset{2}{=} ((X\cdot y_1...y_{l-1}) y_l)Z \overset{3}{=} (X\cdot y_1...y_l)Z = (XY)Z\]
        1, 3 - из утверждения базы индукции; 2 - по предположению индукции.
        \item $\lambda$ - нейтральный элемент;
        \item обратный элемент к $x_1...x_k$ - элемент $x_k^{-1}...x_1^{-1}$.
    \end{enumerate}
\end{proof}
\begin{definition}
    Построенная группа $F(A)$ называется свободной группой с базисом $A$. ($A$ также называется свободной порождающей системой группы).\\
    Любая группа, изоморфная $F(A)$, также называется свободной.
\end{definition}
\begin{subtheorem}
    Пусть $H \leq SL_2(\Z): H = \langle \begin{pmatrix} 1&m\\0&1\end{pmatrix}, \begin{pmatrix} 1&0\\m&1\end{pmatrix} \rangle$.\\
    Тогда $H \simeq F(A)$ с базисом $A = \{a, b\}$.
\end{subtheorem}
\begin{proof}
    Без доказательства.
\end{proof}
\begin{subtheorem}
    Все базисы свободной группы равномощны.
\end{subtheorem}
\begin{proof}
    Без доказательства.
\end{proof}
\begin{definition}
    Ранг свободной группы - мощность её базиса.
\end{definition}
\begin{remark}
    Заметим, что в $F(A)$ результат умножения определён однозначно $\Longrightarrow$ однозначно определён элемент $x_1\cdot ... \cdot x_k$, где $x_i \in A^{\pm 1}$.\\
    Тогда если считать слово $x_1...x_k$ результатом умножения $x_1\cdot ... \cdot x_k$, то можно опускать знак умножения, и в этом смысле работать и с сократимыми словами. 
\end{remark}
\begin{example}
    $abb^{-1}ba^{-1}a = a \cdot b \cdot b^{-1}\cdot b\cdot a^{-1}\cdot a = ab \in F(A)$.
\end{example}