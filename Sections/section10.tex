\section{Кольца и поля}
\begin{definition}
    Кольцо --- множество $K$, на котором введены две бинарные операции --- сложение и умножение --- удовлетворяющие следующим условиям:
    \begin{enumerate}
        \item $(K, +)$ --- абелева группа;
        \item $\forall a, b, c \in K: \ (b + c)a = ba + ca$ и $a(b+c) = ab + ac$ (дистрибутивность) 
    \end{enumerate}
\end{definition}
\begin{definition}
    Кольцо называется коммутативным (ассоциативным), если в нём умножение коммутативно (ассоциативно).
\end{definition}
\begin{definition}
    Кольцо $K$ называется кольцом с единицей, если \[\exists 1 \in K: \forall a \in K \ a\cdot 1 = 1 \cdot a = a\]
    Элемент 1 называется единицей.
\end{definition}
\begin{definition}
    Элемент кольца $K$ с единицей называется обратимым, если
    \[\exists b \in K: ab = ba = 1\]
    Элемент $b$ называется обратным к $a$ и обозначается $a^{-1}$.
\end{definition}
\begin{definition}
    Поле --- коммутативное ассоциативное кольцо с единицей, в котором любой ненулевой элемент обратим.
\end{definition}
\begin{examples}\tab
    \begin{enumerate}
        \item $\Z, \R[x]$ --- ассоциативное, коммутативное, с единицей;\\
        $M_n(\R)$ --- некоммутативное, ассоциативное, с единицей;\\
        $(V^3, +, \times)$ --- некоммутативное, неассоциативное, без единицы ($\times$ --- векторное произведение);\\
        $2\Z$ --- ассоциативное, коммутативное, без единицы;
        \item $\R, \Q, \CC$ --- поле;\\
        $\Z_n$ --- поле $\Longleftrightarrow n$ --- простое;\\
        $\R(x)$ (рациональные дроби над $\R$) --- поле.
    \end{enumerate}    
\end{examples}
\begin{definition}
    Если $a, b \in K$ такие, что $a, b \neq 0$ и $ab = 0$, то $a, b$ называются делителями нуля ($a$ --- левый делитель нуля, $b$ --- правый делитель нуля)
\end{definition}
\begin{examples} Делители нуля в кольцах:
    \begin{itemize}
        \item $\Z_6: 2, 3, 4$ --- все делители нуля;
        \item $M_2(\R)$: например, $\begin{pmatrix} 1&0 \\ 0&0 \end{pmatrix}\cdot \begin{pmatrix} 0&0 \\ 0&1 \end{pmatrix} = \begin{pmatrix} 0&0 \\ 0&0 \end{pmatrix}$
    \end{itemize}
\end{examples}
\begin{subtheorem}
    Если $a$ --- обратимый элемент ассоциативного кольца $K$ с единицей, то обратный элемент к $a$ единственный.
\end{subtheorem}
\begin{proof}
    Пусть $b$ и $c$ --- обратные к $a$. Тогда:
    \[b = b \cdot 1 = b(ac) = (ba)c = 1 \cdot c = c\]
\end{proof}
\begin{remark}
    В дальнейшем все рассматриваемые кольца --- ассоциативные (свойства неассоциативных колец не такие общие и не рассматриваются в данном курсе).
\end{remark}
\begin{subtheorem}
    Если элемент $a$ кольца $K$ обратим, то $a$ --- не делитель нуля. 
\end{subtheorem}
\begin{proof}
    От противного: пусть $a$ обратим и $a$ --- делитель нуля. Тогда:
    \[\exists b \neq 0 : ab = 0 \Longrightarrow a^{-1}ab = a^{-1}\cdot 0 \Longrightarrow b = 0 \text{ --- противоречие.}\]
\end{proof}
\begin{consequense}
    В поле нет делителей нуля.
\end{consequense}
\begin{definition}
    Подмножество $L$ кольца $K$ называется подкольцом, если
    \begin{enumerate}
        \item $(L, +)$ --- подгруппа аддитивной группы кольца $(K, +)$;
        \item $\forall a, b \in L: \ ab \in L$.
    \end{enumerate}
\end{definition}
\begin{subtheorem}
    Любое подкольцо $L$ кольца $K$ является кольцом относительно операций кольца $K$.  
\end{subtheorem}
\begin{example}
    $2\Z \subset \Z \subset \Q$.
\end{example}
\begin{definition}
    Подмножество $L$ поля $K$ называется подполем, если
    \begin{enumerate}
        \item $(L, +, \cdot)$ --- подкольцо кольца $(K, +, \cdot)$;
        \item $1 \in L$;
        \item $\forall a \in L: \ a^{-1} \in L$.
    \end{enumerate}
\end{definition}
\begin{subtheorem}
    Любое подполе $L$ поля $K$ является полем относительно операций поля $K$.  
\end{subtheorem}
\begin{examples}\tab
    \begin{enumerate}
        \item $\Z$ --- подкольцо $\Q$, но не подполе $\Q$;
        \item $\Q \subset \R \subset \CC$.
    \end{enumerate}    
\end{examples}
\subsection{Идеалы колец и факторкольца}
\begin{definition}
    Подмножество $L$ кольца $K$ называется левым (правым) идеалом, если:
    \begin{enumerate}
        \item $(L, +)$ --- подгруппа $(K, +)$;
        \item $\forall a \in K, x \in L: ax \in L \ (xa \in L)$
    \end{enumerate}
\end{definition}
\begin{remark}
    Левый идеал замкнут относительно умножения на элементы кольца слева, правый --- относительно умножения справа.
\end{remark}
\begin{definition}
    Подмножество $L$ кольца $K$ называется (двусторонним) идеалом, если:
    \begin{enumerate}
        \item $(L, +)$ --- подгруппа $(K, +)$;
        \item $\forall a \in K, x \in L: ax, xa \in L$
    \end{enumerate}
\end{definition}
\begin{subtheorem}
    (Левый, правый, двусторонний) идеал кольца $K$ --- подкольцо кольца $K$.
\end{subtheorem}
\begin{proof}
    Очевидно из определения.
\end{proof}
\begin{remark}
    Идеалы в кольцах можно считать аналогом нормальных подгрупп в группах.
\end{remark}
\begin{example}
    Пусть $K = \Z$. Тогда любое подкольцо $K$ имеет вид $H = m\Z$ --- любое такое подкольцо является идеалом в $\Z$
\end{example}
\begin{definition}
    В любом кольце $K$ есть идеалы $\{0\}, K$ --- они называются тривиальными идеалами.
\end{definition}
\begin{subtheorem}
    В поле нет нетривиальных идеалов.
\end{subtheorem}
\begin{proof}
    Пусть $F$ --- поле. Пусть $I \subset F$ --- идеал, причём $I \neq 0$. Тогда $\exists x \in I: x \neq 0$. Так как $F$ --- поле, $\exists x^{-1} \in F$, а отсюда $1 = x^{-1}x \in I$ (т.к. $x \in I$).\\
    Тогда для любого $a \in F$: $a = a\cdot 1 \in I \ (\text{т.к. } 1 \in F)\Longrightarrow I = F$.
\end{proof}

Пусть $K$ --- кольцо, $I$ --- идеал $K$. Рассмотрим множество
\[K/I = \{a + I \ | \ a \in K\}\]
и введём на нём операции:
\begin{enumerate}
    \item Сложение: $(a + I) + (b + I) = a + b + I$;
    \item Умножение: $(a + I) \cdot (b + I) = ab + I$.
\end{enumerate}
\begin{subtheorem}
    Данные операции корректны (не зависят от представителей классов).
\end{subtheorem}
\begin{proof}\tab
    \begin{enumerate}
        \item Сложение корректно из корректности сложения смежных классов в группах, так как $(I, +)$ --- нормальная подгруппа $(K, +)$;
        \item Пусть $a + I = \tilde{a} + I$, $b + I = \tilde{b} + I$. Докажем, что $ab + I = \tilde{a}\tilde{b} + I$:\\ $\tilde{a} = a + y_a$, $\tilde{b} = b + y_b$, где $y_a, y_b \in I$. Тогда:
        \[\forall x \in \tilde{a}\tilde{b} + I: \ x = \tilde{a}\tilde{b} + y, y \in I\]
        \[x =  \tilde{a}\tilde{b} + y = (a + y_a)(b + y_b) + y = ab + \undermat{\in I}{y_ab + ay_b + y_ay_b + y} = ab + I\]
    \end{enumerate}    
\end{proof}
\begin{subtheorem}
    Множество $K/I$ с введёнными операциями --- кольцо, причём если $K$ ассоциативно (коммутативно), то $K/I$ ассоциативно (коммутативно).
\end{subtheorem}
\begin{proof}
    Проверим определение кольца:
    \begin{itemize}
        \item $(K/I, +)$ --- группа по сложению (как факторгруппа $K/I$);
        \item $(a + I) + (b + I) = a + b + I = b + a + I = (b + I) + (a + I)$ --- коммутативность сложения;
        \item $(a + I)((b + I) + (c + I)) = (a + I)(b + c + I) = a(b + c) + I = ab + ac + I = \\ = (ab + I) + (ac + I) = (a+I)(b+I) + (a+I)(c+I)$;
        \item $((b + I) + (c + I))(a + I) = (b + c + I)(a + I) = (b + c)a + I = ba + ca + I = \\ = (ba + I) + (ca + I) = (b+I)(a+I) + (c+I)(a+I)$;
    \end{itemize}
    При этом:
    \begin{itemize}
        \item $K$ ассоциативно $\Longrightarrow (a+I)((b+I)(c+I)) = (a+I)(bc+I) = (a(bc) + I) = ((ab)c + I) = (ab+I)(c+I) = ((a+I)(b+I))(c+I) \Longrightarrow K/I$ ассоциативно;
        \item $K$ коммутативно $\Longrightarrow (a+I)(b+I) = ab+ I = ba + I = (b+I)(a+I) \Longrightarrow K/I$ коммутативно.
    \end{itemize} 
\end{proof}
\begin{definition}
    Кольцо $K/I$ с введёнными операциями называется факторкольцом $K$ по идеалу $I$.
\end{definition}
\subsection{Гомоморфизмы колец}
\begin{definition}
    Пусть $K, \tilde{K}$ --- кольца. Отображение $\phi: K \rightarrow \tilde{K}$ называется гомоморфизмом колец, если:
    \begin{enumerate}
        \item $\forall a, b \in K: \ \phi(a+b) = \phi(a)+\phi(b)$;
        \item $\forall a, b \in K: \ \phi(ab) = \phi(a)\phi(b)$;
    \end{enumerate}
\end{definition}
\begin{definition}
    Изоморфизм колец --- биективный гомоморфизм колец. 
\end{definition}
\begin{definition}
    Кольца $K, \tilde{K}$ называются изоморфными, если существует изоморфизм $\phi: K \rightarrow \tilde{K}$. Обозначается $K \simeq \tilde{K}$.
\end{definition}
\begin{definition}
    Пусть $\phi: K \rightarrow \tilde{K}$ --- гомоморфизм.\\
    Множество $\textup{Ker } \phi = \{a \in K \ | \ \phi(a) = 0\}$ называется ядром $\phi$.\\
    Множество $\textup{Im } \phi = \{b \in \tilde{K} \ | \ \exists a \in K: \phi(a) = b\}$ называется образом $\phi$.
\end{definition}
\begin{subtheorem}
    Пусть $\phi: K \rightarrow \tilde{K}$ --- гомоморфизм. Тогда:
    \begin{enumerate}
        \item $\textup{Ker } \phi$ --- идеал $K$;
        \item $\textup{Im } \phi$ --- подкольцо $\tilde{K}$.
    \end{enumerate}
\end{subtheorem}
\begin{proof}\tab
    \begin{enumerate}
        \item Проверим определение идеала:
        \begin{itemize}
            \item $\textup{Ker } \phi$ --- подгруппа $(K, +)$, так как $\phi$ --- гомоморфизм $(K, +) \rightarrow (\tilde{K}, +)$;
            \item $\forall a \in \textup{Ker }\phi, b \in K: \ \phi(ab) = \phi(a)\phi(b) = 0; \ \phi(ba) = \phi(b)\phi(a) = 0$. Значит, $ab, ba \in \textup{Ker } \phi$.
        \end{itemize}
        Отсюда $\textup{Ker } \phi$ --- идеал кольца $K$.
        \item Проверим определение подкольца:
        \begin{itemize}
            \item $\textup{Im } \phi$ --- подгруппа $(\tilde{K}, +)$, так как $\phi$ --- гомоморфизм $(K, +) \rightarrow (\tilde{K}, +)$;
            \item $\forall a, b \in \textup{Im } \phi: \exists x_a, x_b \in K: \phi(x_a) = a, \phi(x_b) = b$. Тогда $\phi(x_ax_b) = ab$, то есть $ab \in \textup{Im }\phi$.
        \end{itemize}
        Отсюда $\textup{Im } \phi$ --- подкольцо $\tilde{K}$.
    \end{enumerate}    
\end{proof}
\begin{subtheorem}
    Пусть $K$ --- кольцо, $I$ --- идеал $K$. Тогда $\exists$ гомоморфизм колец $\phi: K \rightarrow K/I$ такой, что $\textup{Ker } \phi = I, \textup{Im } \phi = K/I$.
\end{subtheorem}
\begin{proof}
    Подойдёт гомоморфизм $\pi: a \mapsto a + I$.
\end{proof}
\begin{definition}
    Приведённый выше гомоморфизм $\pi: K \rightarrow K/I$ называется каноническим (естественным, натуральным) гомоморфизмом колец $K$ и $K/I$.
\end{definition}
\begin{theorem} (О гомоморфизме колец)\\
    Пусть $K, \tilde{K}$ --- кольца, $\phi: K \rightarrow \tilde{K}$ --- гомоморфизм колец.\\
    Тогда $K / \textup{Ker }\phi \simeq \textup{Im }\phi$.
\end{theorem}
\begin{proof}
    $\textup{Ker }\phi$ --- идеал кольца $K$, то есть факторкольцо $K/\textup{Ker }\phi$ определено. Рассмотрим отображение 
    \[\psi: K/\textup{Ker }\phi \rightarrow \textup{Im }\phi \ \ a + \textup{Ker } \phi \mapsto \phi(a)\]
    В доказательстве теоремы о гомоморфизме групп было доказано, что это отображение корректно, биективно и сохраняет сложение. Проверим сохранение умножения:
    \[\psi((a+\textup{Ker }\phi)(b + \textup{Ker } \phi)) = \psi(ab+\textup{Ker }\phi) = \phi(ab) =\]
    \[= \phi(a)\phi(b) = \psi(a + \textup{Ker }\phi)\psi(b+\textup{Ker }\phi)\]
    Значит, $\psi$ --- изоморфизм колец. 
\end{proof}
\subsection{Главный идеал}
Пусть $K$ --- коммутативное ассоциативное кольцо с единицей, $S \subset K$ --- произвольное подмножество. Рассмотрим множество \[(S) = \{\sum \limits_{i=0}^ka_is_i \ |\ a_i \in K, s_i \in S\}\]
\begin{subtheorem}\tab
    \begin{enumerate}
        \item $(S)$ --- двусторонний идеал;
        \item $(S)$ --- наименьший двусторонний идеал, содержащий $S$.
    \end{enumerate}    
\end{subtheorem}
\begin{proof}\tab
    \begin{enumerate}
        \item Мы поверим, но проверим:
        \begin{itemize}
            \item $\forall x, y \in (S):\ x + y = \sum \limits_{i=0}^kx_is_i + \sum \limits_{i=0}^ky_is_i = \sum \limits_{i=0}^k(x_i + y_i)s_i \in (S)$;
            \item $0 = \sum \limits_{i=0}^k0\cdot s_i \in (S)$;
            \item $\forall x \in (S): \ -x = \sum \limits_{i=0}^kx_is_i = \sum \limits_{i=0}^k(-x_i)s_i \in (S)$;
            \item $\forall x \in (S), a \in K: \ xa = ax = a\sum \limits_{i=0}^kx_is_i = \sum \limits_{i=0}^kax_is_i \in (S)$.
        \end{itemize}
        Значит, $(S)$ --- двусторонний идеал.
        \item Пусть $I$ --- двусторонний идеал кольца $K$ такой, что $S \subset K$. Тогда из определения идеала $\forall a \in K, s \in S: \ as \in I$, а так как идеал --- подкольцо, любая сумма вида $\sum \limits_{i=0}^ka_is_i$, где $a_i \in K, s_i \in S$, лежит в $I$. Значит, $(S) \subseteq I$.
    \end{enumerate}
\end{proof}
\begin{definition}
    Если $I = (S)$, то говорят, что $I$ порождается множеством $S$.\\
    Если при этом $|S| = 1$, то $I$ называется главным идеалом.\\
    Иными словами, $I$ --- главный идеал кольца $K$, если $\exists u \in K: \ \forall x \in I: \ x = ua$ для некоторого $a \in K$.
\end{definition}
\begin{examples}\tab
    \begin{enumerate}
        \item $K = \Z$: $(m) = m\Z$;
        \item $K = F[x]$: $(x+1) = \{(x+1)f \ |\ f \in F[x]\}$.
    \end{enumerate}    
\end{examples}
\begin{definition}
    Коммутативное ассоциативное кольцо с единицей $1 \neq 0$, в котором любой идеал является главным, называется кольцом главных идеалов.
\end{definition}
\begin{example}
    $\Z$ --- кольцо главных идеалов.
\end{example}
\begin{definition}
    Коммутативное ассоциативное кольцо с единицей $1 \neq 0$, в котором нет делителей нуля, называется целостным кольцом.
\end{definition}
\begin{examples}
    $\Z, F[x]$ --- целостные кольца.
\end{examples}
Все напоминания данной главы из 1 семестра, их доказательства см. в \cite{algebra}.
\begin{reminder}
    Если $K$ --- целостное кольцо, то в $K$ определены понятия $a\mid b$, $\textup{НОД}(a, b)$.\\
    \textup{НОД} бывает не определён, но если он существует --- определён однозначно с точностью до ассоциированности (умножения на обратимые элементы). 
\end{reminder}
\begin{definition}
    Целостное кольцо $K$, не являющееся полем, называется евклидовым кольцом, если $\exists$ функция $N: K\setminus{0} \rightarrow \Z_+$ (она называется нормой) такая, что:
    \begin{enumerate}
        \item $\forall a, b \in K: N(ab) > N(a)$;
        \item $\forall a, b \in K, b \neq 0 \exists q, r \in K: \ a = bq + r$, где $\left[\begin{array}{l}
        r = 0\\
        N(r) < N(b)
    \end{array}\right.$
    \end{enumerate}
\end{definition}
\begin{examples}\tab
    \begin{enumerate}
        \item $K = \Z$: $N(a) = |a|$;
        \item $K = F[x]$: $N(f) = \deg f$;
        \item $K = \Z[i]$ --- кольцо гауссовых чисел
        \\$N(a + bi) = a^2 + b^2$.
    \end{enumerate}
\end{examples}
\begin{exercise}
    $\Z[x]$ --- целостное, но не евклидово кольцо.
\end{exercise}
\begin{proof}
    Покажем, что $\Z[x]$ --- целостное кольцо. Очевидно, что $\Z[x]$ --- коммутативное ассоциативное кольцо с единицей, то есть остаётся показать, что в нём нет делителей нуля: если $P(x), Q(x)\neq 0$, то, взяв произведение членов с максимальной степенью $x$ в обоих многочленах, получим одночлен, степень которого больше, чем у всех других в произведении. Значит, коэффициент при нём останется ненулевым, то есть $ab \neq 0$.\\
    Доказательство неевклидовости $\Z[x]$ проведём после следующей теоремы.
\end{proof}
\begin{reminder}
    Пусть $K$ --- евклидово кольцо. Тогда $\exists\ \textup{НОД}(a, b) = d$ и $\exists u, v \in K: d = ua + vb$. 
\end{reminder}
\begin{theorem}
    Всякий идеал евклидова кольца $K$ является главным.
\end{theorem}
\begin{proof}
    Рассмотрим произвольный идеал $I$ кольца $K$.\\
    Если $I = \{0\}$, то $I = (0)$. Иначе рассмотрим наименьший по норме элемент в $I$ --- обозначим его $u$. Докажем, что $I = (u)$:\\
    Пусть $a \in I$ --- произвольный элемент. Разделим $a$ на $u$ с остатком: $\exists q, r \in K: a = uq + r$, причём если $r \neq 0$, то $N(r) < N(u)$. При этом $r = a - uq \in I$, а значит, $r = 0$ из предположения, что $u$ --- наименьший по норме элемент в $I$. Значит, $\forall a \in I \ \exists q \in K: a = uq$, а значит, $I = (u)$. 
\end{proof}
\begin{consequense}
    Любое евклидово кольцо является кольцом главных идеалов.
\end{consequense}
\begin{examples}\tab
    \begin{enumerate}
        \item $F[x]$ --- кольцо главных идеалов;
        \item $F[x, y]$ --- не кольцо главных идеалов. Покажем это:\\
        Рассмотрим идеал $I = (x, y)$. Предположим, что $I = (f)$ --- тогда $\exists q_1, q_2 \in F[x, y]: x = fq_1, y = fq_2$. Из равенства многочленов $x = fq_1$ имеем, что либо $f \sim 1$, либо $f \sim x$. Рассмотрим эти случаи:
        \begin{itemize}
            \item $f \sim 1 \Longrightarrow 1 = ax + by$ --- очевидно невозможно (не можем получить ненулевую константу);
            \item $f \sim x \Longrightarrow f = cx \Longrightarrow y = cxq_2$ --- невозможно.
        \end{itemize}
        Значит, $I$ --- не главный идеал.
        \item Докажем, что $\Z[x]$ --- не кольцо главных идеалов. Рассмотрим $I = (2, x)$. Предположим, что $I = (f)$ --- тогда $\exists q_1, q_2 \in \Z[x]: 2 = fq_1, x = fq_2$. Из первого равенства имеем, что либо $f \sim 1$, либо $f \sim 2$:
        \begin{itemize}
            \item $f \sim 1 \Longrightarrow \pm 1 = 2a + xb$ --- невозможно, т.к. коэффициенты целые;
            \item $f \sim 2 \Longrightarrow x = \pm2 \cdot q_2$ --- невозможно по тем же соображениям.
        \end{itemize}
        Значит, $\Z[x]$ --- не кольцо главных идеалов, а отсюда и не евклидово кольцо.
    \end{enumerate}
\end{examples}
\begin{definition}
    Пусть $K$ --- целостное кольцо.\\
    Элемент $p \in K$ называется простым, если
    \begin{enumerate}
        \item $p \neq 0$;
        \item $p$ --- необратимый;
        \item если $p = ab$, то либо $a$, либо $b$ --- обратимый элемент.
    \end{enumerate}
\end{definition}
\begin{examples}\tab
    \begin{enumerate}
        \item $K = \Z$: простые элементы --- $\pm p$, где $p$ --- простое число;
        \item $K = F[x]$: простые элементы --- неприводимые многочлены. 
    \end{enumerate}    
\end{examples}
\begin{reminder}
    Любой элемент евклидова кольца раскладывается в произведение простых элементов этого кольца, причём это разложение единственно с точностью до домножения множителей на обратимые элементы и их порядка.
\end{reminder}
\begin{remark}
    Аналогичное утверждение верно для всех колец главных идеалов, однако в данном курсе оно рассматриваться не будет.
\end{remark}
\begin{reminder}
    $\Z/(n) = \Z/n\Z = \Z_n$ --- поле $\Longleftrightarrow n$ --- простое.
\end{reminder}
Теперь можем доказать гораздо более общее утверждение:
\begin{theorem}
    Пусть $K$ --- евклидово кольцо, $u \in K$. Тогда $K/(u)$ --- поле $\Longleftrightarrow u$ --- простой элемент. 
\end{theorem}
\begin{proof}
    $\\ \Longrightarrow: \ \ $Пусть $K/(u)$ --- поле. Допустим, что $u$ --- не простой. Тогда возможны случаи:
    \begin{enumerate}
        \item $u = 0 \Longrightarrow K/(u) = K$ --- не поле по определению евклидова кольца;
        \item $u$ обратим $\Longrightarrow 1 \in (u) \Longrightarrow K/(u) = K/K = \{0\}$ --- не поле;
        \item $u = ab$, где $a, b$ --- необратимы. Тогда покажем, что $a + (u) \neq (u)$: иначе
        \[a \in (u) \Longrightarrow a = ud = abd \Longrightarrow a(bd-1) = 0 \Longrightarrow bd = 1 \Longrightarrow b\text{ --- обратим}\]
        Аналогично $b + (u) \neq (u)$. Тогда из $u = ab$ следует, что $(a + (u))(b + (u)) = (u)$, то есть в поле $K/(u)$ элементы $a + (u)$ и $b + (u)$ --- делители нуля --- противоречие.
    \end{enumerate}
    Все случаи невозможны, а значит, $u$ не может не быть простым.\\
    $\Longleftarrow: \ \ $ Пусть $u$ --- простой элемент. $K/(u)$ --- коммутативное ассоциативное кольцо с единицей из соответствующих свойств $K$, причём единица в нём --- $1 + (u)$ \\
    ($\neq (u)$, так как иначе $1 \in (u) \Longrightarrow 1 = ua$ --- противоречие с необратимостью $u$)\\
    Пусть $a$ --- произвольный ненулевой элемент $K/(u)$. Тогда:
    \[a + (u) \neq (u) \Longrightarrow a \notin (u) \Longrightarrow u \nmid a \overset{u \text{-простой}}{\Longrightarrow} \textup{НОД}(u, a) = 1 \Longrightarrow\]\[\Longrightarrow \exists x, y \in K: xa + yu = 1 \text{ в } K \Longrightarrow (x + (u))(a + (u)) = 1 + (u) \text{ в } K/(u)\]
    Значит, $x + (u)$ --- обратный к $a + (u)$. Отсюда любой ненулевой элемент $K/(u)$ обратим, а тогда $K/(u)$ --- поле.
\end{proof}
\begin{consequense}
    Пусть $K = F[x]$ ($F$ --- поле), $f \in F[x]$. Тогда $F[x]/(f)$ --- поле $\Longleftrightarrow f$ --- неприводимый многочлен.
\end{consequense}
\begin{example}
    $\R[x] / (x^2 + 1)$ --- поле.
\end{example}
\subsection{Расширения полей}
\begin{subtheorem}
    Множество $\{a + (f) \ | \ a \in F\}$ образует подкольцо в кольце $F[x]/(f)$, где $F$ --- поле.\\
    В частности, если $f$ неприводим, то это подкольцо --- подполе поля $F[x]/(f)$.\\
    Более того, это подполе изоморфно полю $F$ (изоморфизм $a + (f) \mapsto a$).
\end{subtheorem}
\begin{proof}
    Все пункты определения подкольца (подполя) очевидно следуют из аналогичных утверждений для поля $F$.
\end{proof}
\begin{remark}
    В дальнейшем такое подполе и поле $F$ будут отождествляться.
\end{remark}
\begin{definition}
    Если $L$ --- подполе поля $K$, то $K$ называется расширением $L$.
\end{definition}
В таком случае $K$ можно рассматривать как векторное пространство над $L$.
\begin{examples}\tab
    \begin{enumerate}
        \item $\CC$ --- расширение $\R$;
        \item если $f \in F[x]$ --- неприводимый, то $F(x)/(f)$ --- расширение поля $F$.
    \end{enumerate}    
\end{examples}
\subsubsection{Конечные расширения полей}
\begin{definition}
    Расширение $K$ поля $L$ называется конечным, если $K$ --- конечномерное векторное пространство над $L$ (т.е. $\dim_L K < \infty$).\\
    В этом случае $\dim_L K$ называется степенью расширения.
\end{definition}
\begin{example}
    $\dim_\R \CC = 2$ --- базис $\{1, i\}$.
\end{example}
\begin{subtheorem}
    $\\$Пусть $F$ --- поле, $f \in F[x]$ --- неприводимый многочлен, $\deg f = n$. Тогда элементы $1 + (f), x + (f),...,x^{n-1} + (f)$ --- базис $F[x]/(f)$ как векторного пространства над $F$, то есть $F[x]/(f)$ --- расширение поля $F$ степени $n$.
\end{subtheorem}
\begin{proof}
    Рассмотрим произвольный многочлен $g \in F[x]$ и разделим его на $f$ с остатком:
    \[g(x) = f(x)q(x) + r(x), \  \left[\begin{array}{l}
        r = 0\\
        \deg r < \deg f
    \end{array}\right.\]
    Если $r = 0$, то $g(x) \in (f) \Longrightarrow g + (f) = 0$.\\
    Иначе (далее обозначим $(f)$ как $I$): 
    \[g(x) + I = r(x) + I = c_0 + c_1x + ... + c_{n-1}x^{n-1} + I =\]
    \[= (c_0 + I)(1 + I) + (c_1 + I)(x + I) + ... + (c_{n-1} + I)(x^{n-1} + I) =\]
    \[\text{(отождествление)  } = c_0(1 + I) + c_1(x + I) + ... + c_{n-1}(x^{n-1} + I)\] 
    --- отсюда система $\{1 + (f), x + (f),...,x^{n-1} + (f)\}$ порождает $F[x]/(f)$.\\
    Покажем линейную независимость:
    \[\lambda_0(1+I) + \lambda_1(x+I) + ... + \lambda_{n-1}x^{n-1} + I = I \Longrightarrow \lambda_0 + \lambda_1x + ...+\lambda_{n-1}x^{n-1} \in I\] 
    Тогда $f \mid (\lambda_0 + ...+\lambda_{n-1}x^{n-1})$, но $\deg f > n-1$. Значит, такое возможно только в случае, когда все $\lambda_i$ равны нулю, что и означает линейную независимость.
\end{proof}

Далее введём обозначения $\alpha = x + I, \alpha^k = x^k + I$. Тогда по утверждению $\{1,\alpha, ..., \alpha^{n-1}\}$ --- базис $F[x]/I$. Посмотрим на значение $f(\alpha)$:
\[f = a_0 + a_1x + ... + a_nx^n \Longrightarrow f(\alpha) = a_0 + a_1\alpha + ... + a_n\alpha^n + I =\]
\[= a_0 + a_1(x + I) + ... + a_n(x^n + I) + I = a_0 + a_1x + ... + a_nx^n + I = I\]
то есть в отождествлении $f(\alpha) = 0$, и $\alpha$ --- корень $f$ в $F[x]/(f)$.\\
При этом в $F$ у многочлена $f$ не было корней, так как он неприводим над $F$.
\begin{definition}
    Данный переход от поля $F$ к расширению $F[x]/(f)$ ($f$ --- неприводимый) называется присоединением к полю $F$ корня $\alpha$ многочлена $f$.
\end{definition}
\begin{example}
    Пусть $F = \R, f = x^2 + 1$ --- неприводимый над $\R$.\\
    $\R[x]/(f)$ --- поле, $\{1, \alpha\} = \{1 + (f), x + (f)\}$ --- его базис как векторного пространства над $\R$. Тогда любой элемент этого пространства представим в виде
    \[a + bx + (f) = a(1 + (f)) + b(x + (f)) = a + b\alpha\]
    При этом $f(\alpha) = 0 \Longrightarrow \alpha^2 + 1 = 0 \Longrightarrow \alpha^2 = -1$. Значит, получили поле, в котором все элементы представляются в виде $a + b\alpha$, где $\alpha^2 = -1$ --- это в точности поле комплексных чисел. Значит, $\R[x]/(x^2+1) \simeq \CC$.
\end{example}
\begin{example}
    Построим поле, состоящее из 4 элементов:\\
    Пусть $F = \Z_2, f(x) = x^2 + x + 1 \in \Z_2[x]$ --- неприводимый над $\Z_2$. Тогда $L = \Z_2[x]/(f)$ --- поле, причём (вновь обозначим $(f) = I$, $x+I = \alpha$):
    \[\forall g(x) \in L: \ g(x) = q(x)(x^2+x+1) + (a + bx) \Longrightarrow \{g(x) + I\} =\]
    \[= \{a + bx + I \ | \ a, b \in \Z_2\} = \{0+I, 1+I, x+I, x+1+I\} = \{0, 1, \alpha, \alpha + 1\}\]
    Полученное поле $L$ --- векторное пространство над $Z_2$ размерности $\deg f = 2$ (базис $\{1, \alpha\}$)\\
    Таблицу сложения просто построить, рассматривая $L$ как пространство над $\Z_2$:
    $$\begin{tabular}{c|c|c|c|c}
        + & 0 & 1 & $\alpha$ & $\alpha + 1$ \\ \hline
        0 & 0 & 1 & $\alpha$ & $\alpha + 1$ \\ \hline
        1 & 1 & 0 & $\alpha + 1$ & $\alpha$ \\ \hline
        $\alpha$ & $\alpha$ & $\alpha + 1$ & 0 & 1 \\ \hline
        $\alpha + 1$ & $\alpha + 1$ & $\alpha$ & 1 & 0
    \end{tabular}$$
    Умножение в $L$ зависит от $\alpha$: знаем, что $f(\alpha) = 0\Longrightarrow \alpha^2 + \alpha + 1 = 0$, т.е.:
    \[\alpha^2 = \alpha + 1; \ \ \alpha(\alpha + 1) = \alpha^2 + \alpha = 1; (\alpha + 1)^2 = \alpha^2 + 1 = \alpha\]
    Отсюда таблица умножения для $F$ имеет следующий вид:
    $$\begin{tabular}{c|c|c|c|c}
        $\times$ & 0 & 1 & $\alpha$ & $\alpha + 1$ \\ \hline
        0 & 0 & 0 & 0 & 0 \\ \hline
        1 & \ \ \ 0 \ \ \ & 1 & $\alpha$ & $\alpha + 1$ \\ \hline
        $\alpha$ & 0 & $\alpha$ & $\alpha + 1$ & 1 \\ \hline
        $\alpha + 1$ & 0 & $\alpha + 1$ & 1 & $\alpha$
    \end{tabular}$$
\end{example}
\begin{theorem}(о башне расширений)\\
    Пусть $F$ --- поле, $L$ --- конечное расширение $F$, $M$ --- конечное расширение $L$.\\
    Тогда $M$ --- конечное расширение $F$, причём $\dim_FM = \dim_LM \cdot \dim_FL$.
\end{theorem}
\begin{proof}
    Из конечности расширений можем выбрать:\\
    $\{e_1,...,e_n\}$ --- базис $L$ как векторного пространства над $F$ \ ($\dim_FL = n$);\\
    $\{g_1,...,g_m\}$ --- базис $M$ как векторного пространства над $L$ \ ($\dim_LM = m$).\\
    Докажем, что $\E = \{e_ig_i \ | \ i = \overline{1,n}, j = \overline{1,m}\}$ --- базис $M$ как векторного пространства над $F$:
    \begin{enumerate}
        \item Докажем, что $\E$ - порождающая система $M$ как пространства над $F$:
        \[\forall x \in M: x = \sum \limits_{j=1}^m\lambda_jg_j, \ \lambda_j \in L; \ \ \ \forall j: \ \lambda_j = \sum \limits_{i=1}^n\mu_{ij}e_i,\ \mu_{ij} \in F \Longrightarrow\]
        \[\Longrightarrow x = \sum\limits_j(\sum \limits_i \mu_{ij}e_i)g_j = \sum\limits_{i, j}\mu_{ij}e_ig_j\]
        \item Докажем линейную независимость:
    \end{enumerate}
    \[\sum \limits_{i,j}\mu_{ij}e_ig_j = 0\Longrightarrow \sum\limits_j(\sum \limits_i \mu_{ij}e_i)g_j = 0 \overset{1}{\Longrightarrow} \forall j: \ \sum \limits_i \mu_{ij}e_i = 0 \overset{2}{\Longrightarrow} \forall i,j: \ \mu_{ij} = 0\]
    (1, 2 -- т.к. $\{g_j\}$ является базисом $M$ над $L$, а $\{e_i\}$ -- базисом $L$ над $F$)\\
    Значит, $M$ --- конечное расширение размерности $mn$, что и требовалось.
\end{proof}
\subsubsection{Алгебраические расширения полей}
\begin{definition}
    Пусть $L$ --- расширение поля $F$.\\
    Элемент $a \in L$ называется алгебраическим над $F$, если $\exists h \in F[x], h \neq 0$ такой, что $h(a) = 0$. В противном случае $a$ называется трансцендентным над $F$.
\end{definition}
\begin{examples}\tab
    \begin{enumerate}
        \item $L = \CC, F = \Q$ --- привычные алгебраические и трансцендентные числа:
        \begin{itemize}
            \item $i, \sqrt{2}$ --- алгебраические ($i$ --- корень $x^2 + 1$, $\sqrt{2}$ --- корень $x^2 - 2$)
            \item $e, \pi$ --- трансцендентные.
        \end{itemize}
        \item Если $L$ --- расширение $F$, то $\forall a\in F$ --- алгебраический над $F$ (корень $x-a$) 
    \end{enumerate}    
\end{examples}
\begin{definition}
    Расширение $L$ поля $F$ называется алгебраическим, если любой элемент $a \in L$ --- алгебраический над $F$.
\end{definition}
\begin{example}
    $\CC$ --- алгебраическое расширение $\R$.
\end{example}
\begin{subtheorem}
    Конечное расширение $L$ поля $F$ является его алгебраическим расширением.
\end{subtheorem}
\begin{proof}
    Пусть $\dim_FL = n$. Для произвольного $a \in L$ рассмотрим элементы $\{1, a, a^2, a^3,...,a^n\}$ поля $L$ --- всего их $n+1$, а отсюда по ОЛЛЗ они линейно зависимы над $F$, т.е.
    \[\exists c_i \in F: c_0\cdot 1 + c_1 \cdot a + ... + c_na^n = 0 \Longrightarrow a \text{ -- корень }h(x) = c_0 + c_1x + ... + c_nx^n \in F[x]\]
    Отсюда любой $a \in L$ --- алгебраический над $F$, то есть $L$ --- алгебраическое расширение $F$.
\end{proof}
\begin{definition}
    Пусть $L$ --- расширение поля $F$, $a \in L$ --- произвольный алгебраический элемент. Минимальным многочленом элемента $a$ называется многочлен $\mu_a \in F[x]$ наименьшей степени такой, что $\mu_a \neq 0, \mu_a(a) = 0$.
\end{definition}
\begin{properties}(минимального многочлена)
    \begin{enumerate}
        \item $\mu_a$ --- неприводимый над $F$;
        \item Для любого $h \in F[x]$ такого, что $h(a) = 0$, верно $\mu_a \mid h$;
        \item Минимальный многочлен единственный с точностью до домножения на ненулевой элемент $\F$ (обратимый элемент $F[x]$)
    \end{enumerate}
\end{properties}
\begin{proof}\tab
    \begin{enumerate}
        \item Так как $\mu_a$ имеет корень $a$ и при этом не тождественно равен нулю, $\mu_a$ --- не константа, то есть необратим в $F[x]$. Также:
        \[\mu_a(x) = p(x)q(x) \Longrightarrow 0 = \mu_a(a) = p(a)q(a)\]
        $p(a)$ и $q(a)$ --- элементы поля $L$, а раз в поле нет делителей нуля,
        \[p(a)q(a) = 0 \Longrightarrow \left[\begin{array}{l} p(a) = 0\\q(a) = 0\end{array}\right.\]
        Тогда хотя бы один из многочленов $p(x)$ и $q(x)$ является константой --- иначе у одного из них есть корень $a$ и степень меньше степени $\mu_a$, что противоречит определению минимального многочлена.\\
        Значит, $\mu_a$ неприводим в $F[x]$; 
        \item Разделим $h$ на $\mu_a$ с остатком:
        \[h = \mu_a(x) \cdot q(x) + r(x), \left[\begin{array}{l} r = 0\\ \deg r < \deg f \end{array}\right.\]
        При этом
        \[0 = h(a) = \mu_a(a) \cdot q(a) + r(a) = r(a)\]
        а тогда по определению минимального многочлена $r(x) = 0$. Значит, $\mu_a \mid h$;
        \item Очевидно следует из пункта 2 --- если $\mu_1$ и $\mu_2$ минимальны для $a$, то
        \[\mu_1(a) = 0 \Longrightarrow \mu_2 \mid \mu_1; \ \mu_2(a) = 0 \Longrightarrow \mu_1 \mid \mu_2\]
        а отсюда $\mu_1 = c\mu_2$, где $c$ обратим в $F[x]$, то есть является элементом $F$.
    \end{enumerate}
\end{proof}
\begin{definition}
    Степень минимального многочлена $\mu_a \in F[x]$ для алгебраического элемента $a \in L$ называется степенью $a$.
\end{definition}
\begin{example}
    $\Q \subset \CC, a = \sqrt{2}: \ \mu_a(x) = x^2 - 2$, степень $\sqrt{2}$ равна $2$.
\end{example}
\begin{definition}
    Пусть $S$ --- подкольцо кольца $T$, $a_1,...,a_n \in T$. Кольцом, порождённым элементами $a_1,...,a_n$ над $S$, называется наименьшее подкольцо кольца $T$, содержащее $S$ и элементы $a_1,...,a_n$. Обозначается $S[a_1,...,a_n]$.
\end{definition}
\begin{subtheorem}
    $S[a_1,...,a_n] = \{f(a_1,...,a_n) \ | \ f \in S[x_1,...,x_n]\}$
\end{subtheorem}
\begin{proof}
    Очевидно, что $S_0 = \{f(a_1,...,a_n) \ | \ f \in S[x_1,...,x_n]\}$ --- подкольцо кольца $T$, содержащее $S$ и $a_1,...,a_n$. При этом любое подкольцо $T$, содержащее $S$ и $a_1,...,a_n$, обязано содержать все элементы вида $f(a_1,...,a_n)$ в силу замкнутости относительно сложения и умножения, а значит, $S_0$ --- наименьшее подкольцо $T$ с такими свойствами.
\end{proof}
\begin{examples}\tab
    \begin{enumerate}
        \item $\Q \subset \R: \ \ \Q[\sqrt{2}] = \{a + b\sqrt{2}\ |\ a, b\in \Q\}$;
        \item $F \subset L = F[x]/(f),\ \alpha = x + (f): \ \ F[\alpha] = L$. 
    \end{enumerate} 
\end{examples}
\begin{theoremnum}
    Пусть $L$ --- расширение поля $F$. Тогда:
    \begin{enumerate}
        \item $a \in L$ --- алгебраический над $\F \Longleftrightarrow F[a]$ --- конечномерное векторное пространство над $F$;
        \item если $a \in L$ --- алгебраический над $F$, то $F[a]$ --- поле, изоморфное $F[x]/(\mu_a)$ (в частности, $\dim_F F[a] = \dim F[x]/(\mu_a) = \deg \mu_a$).
    \end{enumerate}
\end{theoremnum}
\begin{proof} $F[u] = \{g(u) \ | \ g \in F[x]\}$
    \begin{enumerate}
        \item $\Longleftarrow: \ \ $ Пусть $F[a]$ --- $n$-мерное векторное пространство над $F$. Тогда аналогично рассуждениям о алгебраичности конечного расширения с помощью рассмотрения системы векторов $\{1, a, ..., a^n\}$, линейно зависимой над $F$, найдём ненулевой многочлен с корнем $a$;\\
        $\Longrightarrow: \ \ $ Пусть $a$ --- алгебраический элемент поля $F$, то есть
        \[\exists h(x) = c_0 + c_1x + ... + c_nx^n \in F[x], \ h \neq 0, h(a) = c_0 + c_1a + ... + c_na^n = 0\]
        Отсюда $a^n = \tilde{c}_0 + \tilde{c}_1a + ... + \tilde{c}_{n-1}a^{n-1}$, где $\tilde{c}_i \in F$.\\
        Покажем, что $F[a]$ порождается множеством $\{1,a,...,a^{n-1}\}$:\\
        Пусть $x \in F[a]$. Пусть $x \in F[a] \Longrightarrow x = \sum \limits_{j=0}^m \lambda_ja^j$. Индукция по $m$:\\
        База: $m < n$ --- уже имеем разложение по нужным элементам;\\
        Шаг: Заметим, что
        \[a^m = a^{m-n}a^n = a^{m-n}(\tilde{c}_0 + \tilde{c}_1a + ... + \tilde{c}_{n-1}a^{n-1}) = \tilde{c}_0a^{m-n} + ... + \tilde{c}_{n-1}a^{m-1}\]
        Значит, в разложении $x = \sum \limits_{j=0}^m \lambda_ja^j$ можно представить слагаемое $\lambda_ma^m$ как сумму меньших степеней $a$, а тогда после приведения подобных применимо предположение индукции. Отсюда $F[a]$ конечно порождается как векторное пространство над $F$, а значит, имеет конечный базис.
        \item Рассмотрим гомоморфизм колец $\phi: F[x] \rightarrow L$, заданный по правилу $f(x) \mapsto f(a)$. Тогда $\textup{Im }\phi = F[a], \textup{Ker }\phi = (\mu_a)$, и по теореме о гомоморфизме колец $F[a] \simeq F[x]/(\mu_a)$.\\ Притом $\mu_a$ --- неприводимый над $F$, то есть $F[x]/(\mu_a)$ --- поле, а отсюда и $F[a]$ --- поле.
    \end{enumerate}
\end{proof}
\begin{example}
    $\Q[\sqrt{2}] \simeq \Q[x]/(x^2-2)$ --- поле.
\end{example}
\subsubsection{Конечнопорождённые расширения полей}
\begin{definition}
    Пусть $L$ --- расширение поля $F$, $a_1,...,a_m \in L$. Подполем, порождённым элементами $a_1,...,a_m$ над $F$, называется наименьшее подполе $L$, содержащее $a_1,...,a_m$ и $F$. Обозначается $F(a_1,...,a_m)$.
\end{definition}
\begin{subtheorem}
    \[F(a_1,...,a_m...) = \left\{\frac{f(a_1,...,a_m)}{g(a_1,...,a_m)}\left|\right. m \in \N \ f, g \in F[x_1,...,x_m], \ g(a_1,...,a_n) \neq 0\right\}\]
\end{subtheorem}
\begin{proof}
    Аналогично выражению подкольца, порождённого элементами кольца (обратимость элементов в поле даёт возможность делить).
\end{proof}
\begin{definition}
    Поле $L$ называется конечнопорождённым расширением поля $F$, если $\exists a_1,...,a_m \in L:\ L = F(a_1,...,a_m)$.
\end{definition}
\begin{subtheorem} (Эффект уничтожения иррациональных знаменателей)\\
    Если $a \in L$ --- алгебраический элемент над $F$, то $F[a] = F(a)$. 
\end{subtheorem}
\begin{proof}
    Следует из п. 2 теоремы 1 ($F[a]$ --- поле).
\end{proof}
\begin{example}
    $\Q(\sqrt{2}) = \Q[\sqrt{2}]: \ \frac{1}{1 + \sqrt{2}}= \sqrt{2} - 1$
\end{example}
\begin{theoremnum}
    Следующие условия эквивиалентны:
    \begin{enumerate}
        \item $L$ --- конечное расширение поля $F$;
        \item $L$ --- конечнопорождённое алгебраическое расширение поля $F$;
        \item $L$ порождается конечным числом элементов, алгебраических над $F$.
    \end{enumerate}
\end{theoremnum}
\begin{proof}
    $\\ 1 \Longrightarrow 2: $ Алгебраичность любого конечного расширения уже доказывалась.\\
    Пусть $\{e_1,...,e_n\}$ --- базис векторного пространства $L$ над $F$.\\
    Тогда $L = F[e_1,...,e_n] = F(e_1,...,e_n)$, т.е. $L$ - конечнопорождённое;
    $\\ 2 \Longrightarrow 3: $ Из конечнопорождённости $L = F(a_1,...,a_n)$. При этом $a_1,...,a_n$ --- алгебраические над $F$ из алгебраичности расширения $L$, то есть $L$ порождается конечным числом элементов, алгебраических над $F$;
    $\\ 3 \Longrightarrow 1: $ Пусть $L = F(a_1,...,a_n)$, где $a_i$ --- алгебраические. Рассмотрим цепочку
    \[F \subset F(a_1) \subset F(a_1, a_2) \subset ... \subset F(a_1,...,a_n)\]
    и обозначим $L_k = F(a_1,...,a_k) = F(a_1,...,a_{k-1})(a_k) = L_{k-1}(a_k)$.\\
    Тогда $L_{k+1}$ --- конечное расширение $L_k$: так как $a_{k+1} \in L_{k+1}$ является алгебраическим над $F$, он является алгебраическим и над $L_k$, а отсюда
    \[L_{k+1} = L_{k}(a_{k+1}) = L_{k}[a_{k+1}] \ - \ \text{конечномерное в.п. над $L_k$ по теореме 1}\]
    Значит, все последовательные расширения в цепочке конечные, а тогда по теореме о башне расширений $L$ --- конечное расширение $F$.
\end{proof}
\subsubsection{Алгебраическое замыкание}
\begin{theorem}
    Пусть $L$ --- расширение поля $F$.\\
    Обозначим $\overline{F} = \{a \in L \ | \ a - \text{алгебраический над } F\}$. Тогда:
    \begin{enumerate}
        \item $F$ --- подполе $L$;
        \item если $b\in L$ --- алгебраический над $\overline{F}$, то $b$ --- алгебраический над $F$\\
        (то есть $b \in \overline{F}$)
    \end{enumerate}
\end{theorem}
\begin{proof}\tab
    \begin{enumerate}
        \item $\forall a, b \in \overline{F}$ рассмотрим $F(a, b)$ --- подполе, порождённое элементами $a, b$, алгебраическими над $F$. По теореме 2 $F(a, b)$ --- алгебраическое расширение $F$, а отсюда $F(a, b) \subset \overline{F}$. Тогда $\forall a, b \in \overline{F}$:
        \begin{itemize}
            \item $a + b \in \overline{F}, \ 0 \in \overline{F},\ -a \in \overline{F} \Longrightarrow \overline{F}$ --- аддитивная подгруппа;
            \item $ab \in \overline{F} \Longrightarrow \overline{F}$ --- подкольцо;
            \item $1 \in \overline{F}, \ a^{-1} \in \overline{F} \Longrightarrow \overline{F}$ --- подполе.
        \end{itemize}
        \item Пусть $b \in L$ --- алгебраический над $\overline{F}$, т.е. \[\exists h \in \overline{F}[x] \ (h = c_0 + c_1x + ... + c_kx^k, c_i \in \overline{F}): \ h \neq 0,\ h(b) = 0\] 
        Рассмотрим в $L$ подполе $\tilde{F} = F(c_0,c_1,...,c_k)$. Так как $c_i \in \overline{F}$, $c_i$ --- алгебраические над $F$. Тогда $\tilde{F}$ порождается над $F$ конечным числом алгебраических элементов, а отсюда по теореме 2 $\tilde{F}$ --- конечное расширение $F$.\\
        При этом $h \in \tilde{F}[x]$, а значит, $b$ является алгебраическим над $\tilde{F}$. Тогда по теореме 2 $\tilde{F}(b)$ --- конечное расширение $\tilde{F}$, а отсюда по теореме о башне расширений $\tilde{F}(b)$ --- конечное расширение $F$. Так как любое конечное расширение является алгебраическим, $\tilde{F}(b)$ --- алгебраическое расширение $F$, а отсюда $b$ является алгебраическим над $F$.
    \end{enumerate}    
\end{proof}
\begin{definition}
    $\overline{F}$ называется алгебраическим замыканием поля $F$ в поле $L$.
\end{definition}
\begin{remark}
    $\overline{F}$ не обязано быть алгебраически замкнутым, так как поле $L$ может быть не алгебраически замкнуто. Например, алгебраическое замыкание поля $\Q$ в поле $\R$ не алгебраически замкнуто, так как многочлен $x^2 + 1$ не имеет корней в нём.
\end{remark}
\begin{example}
    Пусть $\overline{\Q}$ --- алгебраическое замыкание $\Q$ в $\CC$.\\
    ($\overline{\Q}$ называется полем всех алгебраических чисел)
    \begin{exercise}\tab
        \begin{enumerate}
            \item Доказать, что $\overline{\Q}$ алгебраически замкнуто;
            \item Доказать, что любое конечное расширение $\Q$ --- подполе $\overline{\Q}$.
        \end{enumerate}        
    \end{exercise}
    \begin{proof}\tab
        \begin{enumerate}
            \item Рассмотрим произвольный многочлен $f \in \overline{\Q}[x]$ положительной степени. Это многочлен с комплексными коэффициентами, а так как $\CC$ алгебраически замкнуто, $f$ имеет корень $a \in \CC$. Тогда $a$ --- алгебраический над $\overline{\Q}$, и по пункту 2 предыдущей теоремы $a \in \overline{\Q}$. Поэтому $f$ имеет корень в $\overline{\Q}$, то есть $\overline{\Q}$ алгебраически замкнуто;
            \item Любое конечное расширение $F$ поля $\Q$ является алгебраическим расширением $\Q$. Рассмотрим произвольный $b \in F$ --- он является алгебраическим над $\Q$, а так как любой корень многочлена с рациональными коэффициентами является комплексным числом, $b \in \CC$. Тогда по определению $b \in \overline{\Q}$ ($b \in \CC$ и $b$ --- алгебраический над $\Q$), то есть $F \subseteq \overline{\Q}$. При этом все операции введены как операции над $\CC$, то есть $F$ --- подполе $\overline{\Q}$.
        \end{enumerate}        
    \end{proof}
\end{example}
\subsection{Поле разложения многочлена}
\begin{definition}
    Пусть $F$ --- поле, $f \in F[x]$ --- произвольный многочлен.\\
    Расширение $L$ поля $F$ называется полем разложения многочлена $f$, если 
    \begin{enumerate}
        \item $f$ можно разложить на линейные множители над $L$;
        \item $L$ порождается над $F$ корнями многочлена $f$.
    \end{enumerate} 
    То есть $L$ --- наименьшее поле, содержащее $F$ и все корни многочлена $f$.
\end{definition}
\begin{examples}\tab
    \begin{enumerate}
        \item $f(x) = x - 5,\ F = \R \Longrightarrow L = \R$;
        \item $f(x) = x^2 + 1,\ F = \R \Longrightarrow L = R[x]/(x^2+1)\simeq \CC$;
        \item $f(x) = x^3 - 2,\ F = \Q$.\\
        Поле $L_1 = \Q[x]/(x^3-2) \simeq \Q(\sqrt[3]{2})$ --- не поле разложения $f$, т.к.
        \[f(x) = (x - \sqrt[3]{2})(x^2+ \sqrt[3]{2}x + \sqrt[3]{4})\]
        где корни $x^2+ \sqrt[3]{2}x + \sqrt[3]{4}$ комплексные.\\
        Поэтому поле разложения $f$ должно иметь вид $L = L_1[x]/(x^2+ \sqrt[3]{2}x + \sqrt[3]{4})$.
    \end{enumerate}    
\end{examples}
\begin{lemma}
    Пусть $P, \tilde{P}$ --- поля, $h(x) =  c_0 + c_1x + ... + c_nx^n$ --- неприводимый многочлен из $P[x]$, $P(\alpha)$ --- поле, полученное из $P$ присоединением корня $\alpha$ многочлена $h$. Тогда если $\phi: P \rightarrow \tilde{P}$ --- гомоморфизм полей, то количество способов продолжить его до гомоморфизма $\psi: P(\alpha) \rightarrow \tilde{P}$ совпадает с числом корней $\tilde{h}$ в $\tilde{P}$, где $\tilde{h}(x) = \tilde{c}_0 + \tilde{c}_1x + ... + \tilde{c}_nx^n$, $\tilde{c}_i = \phi(c_i)$.
\end{lemma}
\begin{proof}
    $P(\alpha) = P[\alpha] = \{a_0 + a_1\alpha + ... + a_k\alpha^k \ | \ a_i \in P\}$.\\
    Если $\exists \psi: P(\alpha) \rightarrow \tilde{P}$, то
    \begin{equation}
    \psi(a_0 + a_1\alpha + ... + a_k\alpha^k) = \phi(a_0) + \phi(a_1)\beta + ... + \phi(a_k)\beta^k, \text{ где } \beta = \psi(\alpha)
    \end{equation}
    Применим это к $h(\alpha)=0$:
    \[0 = \psi(0) = \psi(h(\alpha)) = \psi(c_0 + c_1\alpha + ... + c_n\alpha^n) =  \tilde{c}_0 + \tilde{c}_1\beta + ... + \tilde{c}_n\beta^n\]
    Поэтому для любого продолжения $\psi$ элемент $\beta = \psi(\alpha)$ --- корень $\tilde{h}(x)$.\\
    С другой стороны, если $\beta$ --- произвольный корень $\tilde{h}(x)$, то правило (1) задаёт корректный гомоморфизм $\psi: P(\alpha) \rightarrow \tilde{P}$, причём все такие гомоморфизмы различны (различны их значения на $\alpha$). Значит, искомых гомоморфизмов столько же, сколько и корней у $\tilde{h}(x)$ в $\tilde{P}$.
\end{proof}
\begin{theorem}
    Поле разложения любого $f \in F[x]$ существует и единственно с точностью до изоморфизма, тождественного на $F$.
\end{theorem}
\begin{proof}
    Построим одно из полей разложения $L$ многочлена $f$ c помощью цепочки расширений
    \[F = L_0 \subseteq L_1 \subseteq ... \subseteq L_s = L, \ L_{i+1} = L_{i}(\alpha_{i+1}), \text{ где } \alpha_1,...,\alpha_s - \text{корни } f\]
    Разложим $f$ на неприводимые множители и докажем существование нужной цепочки индукцией по $S$ --- сумме степеней нелинейных многочленов в разложении $f$ над $L_i$:\\
    База: $S = 0\Longrightarrow$ все множители линейны --- $L_i = L$, цепочка завершена;\\
    Шаг: Пусть в разложении есть неприводимый множитель $f_i$ степени $\geqslant 2$ --- тогда $L_{i+1} = L_{i}[x]/(f_i) = L_i(\alpha_{i+1})$, где $\alpha_{i+1}$ --- корень $f_i$. В $L_{i+1}$ у многочлена $f(x)$ сохранились все имевшиеся корни и добавился хотя бы один новый корень $\alpha_{i+1}$ --- значит, в $L_{i+1}$ сумма степеней нелинейных многочленов в разложении $f$ уменьшилась хотя бы на 1, то есть к $L_{i+1}$ применимо предположение индукции. \\
    Осталось заметить, что по построению $L$ порождается над $F$ корнями $f$, не принадлежащими $F$, что то же самое, что и всеми корнями $f$. Значит, $L$ --- поле разложения $f$.

    Теперь докажем единственность. Пусть $\tilde{L}$ --- произвольное поле разложения многочлена $f$ (из опр. $F \subseteq \tilde{L}$). Рассмотрим последовательность гомоморфизмов 
    \[\phi_i : L_i \rightarrow \tilde{L}\text{ таких, что } \phi_0 = \id, \ \phi_{i+1}|_{L_i} = \phi_i\]
    По лемме при присоединении корня неприводимого $f_i$ хотя бы одно продолжение $\phi_{i+1}$ существует тогда и только тогда, когда $\phi_i(f_i)$ имеет корень в $\tilde{L}$.\\
    Для многочленов знаем, что $f_i \mid f$ над $L_i \Longrightarrow \phi_i(f_i) \mid \phi_i(f)$ над $\tilde{L}$, а $\phi_i(f) = f$, так как все $\phi_i$ тождественны на $f$. При этом $f$ раскладывается на линейные множители над $\tilde{L}$ --- значит, у $\phi_i(f_i)$ не может не быть корней над $L$, и расширение возможно.\\
    Так построим $\phi_s: L \rightarrow \tilde{L}$ --- гомоморфизм полей, тождественный на $F$.\\
    Ядро $\textup{Ker }\phi_s$ --- идеал в поле $L$, то есть оно равно либо $\{0\}$, либо $L$ --- второе невозможно, так как $\phi_s$ тождественный на $F$, а в $F$ есть $1 \neq 0$.\\
    Образ $\textup{Im }\phi_s$ --- подполе $\tilde{L}$, содержащее все корни многочлена $f$, а тогда из минимальности поля разложения совпадающее с $\tilde{L}$. Значит, $\phi_s$ --- биективный гомоморфизм полей, т.е. изоморфизм полей, тождественный на $F$.
\end{proof}
\subsection{Конечные поля}
\begin{definition}
    Наименьшее натуральное число $n$ такое, что $\overmat{n}{1 + ... + 1} = 0$ в $F$, называется характеристикой поля $F$. Если таких натуральных $n$ не существует, то характеристика поля $F$ равна 0. Обозначается как $\textup{char } F$.
\end{definition}
\begin{examples}
    $\textup{char } \R = 0,\ \textup{char } \Z_3 = 3$.
\end{examples}
\begin{subtheoremnum}
    Если $n = \textup{char } F > 0$, то $n$ --- простое.
\end{subtheoremnum}
\begin{proof}
    Доказывалось в курсе первого семестра.
\end{proof}
\begin{subtheoremnum}
    Пусть $F$ --- поле. Тогда
    \begin{enumerate}
        \item если $\textup{char } F = p > 0$, то $\exists M$ --- подполе $F$, изоморфное $\Z_p$;
        \item если $\textup{char } F = 0$, то $\exists M$ --- подполе $F$, изоморфное $\Q$; 
    \end{enumerate}
\end{subtheoremnum}
\begin{proof}\tab
    \begin{enumerate}
        \item $M = \{0, 1, 1+1, 1+1+1,...,\overmat{p-1}{1+...+1}\} \simeq \Z_p$ --- подполе $F$;
        \item $K = \{0, 1, (-1), 1+1, (-1) + (-1), ...\} \simeq \Z$ --- подкольцо $F$.\\
        Значит, в $F$ есть подполе $M = \{\frac{a}{b} \ | \ a, b \in K, b \neq 0\} \simeq \Q$.
    \end{enumerate}    
\end{proof}
\begin{subtheoremnum}
    Пусть $F$ --- поле, $\textup{char } F = p > 0$, $\phi: F \rightarrow F$ --- отображение, заданное по правилу $\phi: x \mapsto x^p$. Тогда $\phi$ --- инъективный гомоморфизм, причём если $|F| < \infty$, то $\phi$ --- изоморфизм.
\end{subtheoremnum}
\begin{proof}
    Очевидно, что $\phi(ab) = (ab)^p = a^pb^p = \phi(a)\phi(b)$.
    \[\phi(a+b) = (a+b)^p = \sum_{i=0}^p C_p^k a^kb^{p-k} = a^p + b^p = \phi(a) + \phi(b)\]
    (так как $C_p^k = \frac{p!}{k!(p-k)!}$, и при $k \neq 0, p$ только числитель кратен $p$, т.е. $p \mid C_p^k$)\\
    Также $\phi(a) = 0 \Longleftrightarrow a^p = 0 \Longleftrightarrow a = 0$ (в поле нет делителей нуля).\\
    Отсюда $\phi$ --- инъективный гомоморфизм.\\
    При $|F| < \infty$ он сюръективен, то есть является изоморфизмом.
\end{proof}
\begin{definition}
    $\phi$ из утверждения 3 называется эндоморфизмом Фробениуса (автоморфизмом Фробениуса для конечного $F$).
\end{definition}
\begin{subtheoremnum}
    $F$ --- конечное поле $\Longrightarrow |F| = p^k$, где $p = \textup{char } F$, $k \in \N$.
\end{subtheoremnum}
\begin{proof}
    По утверждению 2 в поле $F$ есть подполе $M \simeq \Z_p$.\\
    Тогда $F$ можно рассматривать как векторное пространство над $M$. Из конечности поля $F$ конечна его размерность как векторного пространства над $M$. Пусть она равна $k$ --- тогда $|F| = p^k$.\\
    (каждый элемент однозначно задаётся $k$ координатами из $\Z_p$)
\end{proof}
\begin{lemma}
    Пусть $\phi$ --- автоморфизм поля $F$. Тогда множество $S$ неподвижных точек $\phi$ --- подполе в $F$. 
\end{lemma}
\begin{proof}
    Если $a, b$ неподвижны относительно $\phi$, то из определения изоморфизма очевидно, что и неподвижны и точки $a + b,\ -a,\ ab, \ a^{-1}$. Также $\phi(0) = 0$ по свойству гомоморфизмов групп, $\phi(1) = 1$ по аналогичным рассуждениям для мультипликативной группы поля. Значит, $S$ --- подполе $F$.
\end{proof}
\begin{theorem}
    Для любого простого $p$ и любого $n \in \N$ существует единственное с точностью до изоморфизма поле $F$ из $p^n$ элементов.
\end{theorem}
\begin{proof}
    Построим поле порядка $p^n$: рассмотрим поле $F$ разложения многочлена $f(x) = x^{p^n} - x$ над $\Z_p$. Его производная $f' = p^nx^{p^n-1} - 1 = -1$, то есть у $f$ нет кратных корней. Значит, в $F$ есть $p^n$ различных корней $f$.\\
    Рассмотрим подмножество $S \subseteq F$ этих корней: 
    \[S = \{a \in F \ | \ a^{p^n} = a\} \ \  (|S| = p^n)\] 
    --- это множество неподвижных точек автоморфизма $\phi^n$, где $\phi$ --- автоморфизм Фробениуса. По лемме множество неподвижных точек автоморфизма поля является подполем, и при этом $S$ содержит все корни $f$ --- из минимальности поля разложения $S = F \Longrightarrow |F| = p^n$.\\
    Единственность: пусть $\tilde{F}$ --- произвольное поле из $p^n$ элементов. Тогда
    \[|\tilde{F}^*| = p^n - 1 \Longrightarrow \forall a \in \tilde{F}^*: \ a^{p^n - 1} = 1 \Longrightarrow \forall a \in F :\ a^{p^n} - a = 0\]
    то есть $\tilde{F}$ состоит из корней многочлена $f(x) = x^{p^n} - x$, которых ровно $p^n$. Значит, $\tilde{F}$ --- поле разложения $f$ над $\Z_p$, то есть $\tilde{F} \simeq F$. 
\end{proof}
\begin{definition}
    Поле из $p^n$ элементов обозначается $\F_{p^n}$.
\end{definition}
\begin{consequense}
    Для любого простого $p$ и любого $n \in \N$ существует неприводимый многочлен степени $n$ над $\Z_p$.
\end{consequense}
\begin{proof}
    Рассмотрим поле $F$ из $p^n$ элементов. В конце раздела \ref{th2} доказывалось, что мультипликативная группа конечного поля --- циклическая $\Longrightarrow F^* = \langle \alpha \rangle$. Тогда $F$ порождается над $\Z_p$ элементом $\alpha$, то есть $F = \Z_p(\alpha) = \Z_p[x]/(\mu_\alpha)$, где $\deg \mu_\alpha = \dim_{\Z_p}F = n$, и при этом $\mu_\alpha$ неприводим по свойству минимального многочлена. Значит, $\mu_\alpha$ --- искомый многочлен.  
\end{proof}
\begin{exercise}
    Доказать. что в $\F_{p^n}$ есть подполе, изоморфное $\F_{p^m} \Longleftrightarrow m \mid n$.
\end{exercise}
\begin{proof}
    $\\ \Longrightarrow: \ \ $ Если в $F_1 \simeq \F_{p^n}$ есть подполе $F_2 \simeq \F_{p^m}$, то $F_1$ можно рассматривать как векторное пространство над $F_2$. Тогда $|F_1| = |F_2|^k$, где $k$ --- размерность $F_1$ как пространства над $F_2$, то есть $p^n = p^{km} \Longrightarrow m \mid n$;\\
    $\Longleftarrow: \ \ $ Заметим, что если $n = km$, то многочлен $x^{p^n} - x$ делится на $x^{p^m} - x$:
    \[p^n - 1 = (p^m)^k - 1^k = (p^m-1)(p^{m(k-1)} + p^{m(k-2)} + ... + p^m + 1) = (p^m-1)Q \Longrightarrow\]
    \[\Longrightarrow x^{p^n} - x = x(x^{p^n-1} - 1) = x(x^{(p^m-1)Q} - 1^Q) = x(x^{p^m - 1} - 1)P(x) = (x^{p^m} - x)P(x)\]
    Таким образом, все корни многочлена $x^{p^m} - x$ являются корнями $x^{p^n} - x$, то есть в $\F_{p^n}$ как в поле разложения $x^{p^n} - x$ над $\Z_p$ есть все $p^m$ корней многочлена $x^{p^m} - x$, которые образуют поле, изоморфное $\F_{p^m}$, относительно тех же операций. Значит, в $\F_{p^n}$ есть подполе, изоморфное $\F_{p^m}$
\end{proof}
\setcounter{thcount}{0}
\setcounter{concount}{0}
\setcounter{subthcount}{0}
\setcounter{lemcount}{0}
\newpage