\section{Кольца и поля}
\begin{definition}
    Кольцо - множество $K$, на котором введены две бинарные операции - сложение и умножение - удовлетворяющие следующим условиям:
    \begin{enumerate}
        \item $(K, +)$ - абелева группа;
        \item $\forall a, b, c \in K: \ (b + c)a = ba + ca$ и $a(b+c) = ab + ac$ (дистрибутивность) 
    \end{enumerate}
\end{definition}
\begin{definition}
    Кольцо называется коммутативным (ассоциативным), если в нём умножение коммутативно (ассоциативно).
\end{definition}
\begin{definition}
    Кольцо $K$ называется кольцом с единицей, если \[\exists 1 \in K: \forall a \in K \ a\cdot 1 = 1 \cdot a = a\]
    Элемент 1 называется единицей.
\end{definition}
\begin{definition}
    Элемент кольца $K$ с единицей называется обратимым, если
    \[\exists b \in K: ab = ba = 1\]
    Элемент $b$ называется обратным к $a$ и обозначается $a^{-1}$.
\end{definition}
\begin{definition}
    Поле - коммутативное ассоциативное кольцо с единицей, в котором любой ненулевой элемент обратим.
\end{definition}
\begin{examples}\tab
    \begin{enumerate}
        \item $\Z, \R[x]$ - ассоциативное, коммутативное, с единицей;\\
        $M_n(\R)$ - некоммутативное, ассоциативное, с единицей;\\
        $(V^3, +, \times)$ - некоммутативное, неассоциативное, без единицы ($\times$ - векторное произведение);\\
        $2\Z$ - ассоциативное, коммутативное, без единицы;
        \item $\R, \Q, \CC$ - поле;\\
        $\Z_n$ - поле $\Longleftrightarrow n$ - простое;\\
        $\R(x)$ (рациональные дроби над $\R$) - поле.
    \end{enumerate}    
\end{examples}
\begin{definition}
    Если $a, b \in K$ такие, что $a, b \neq 0$ и $ab = 0$, то $a, b$ называются делителями нуля ($a$ - левый делитель нуля, $b$ - правый делитель нуля)
\end{definition}
\begin{examples} Делители нуля в кольцах:
    \begin{itemize}
        \item $\Z_6: 2, 3, 4$ - все делители нуля;
        \item $M_2(\R)$: например, $\begin{pmatrix} 1&0 \\ 0&0 \end{pmatrix}\cdot \begin{pmatrix} 0&0 \\ 0&1 \end{pmatrix} = \begin{pmatrix} 0&0 \\ 0&0 \end{pmatrix}$
    \end{itemize}
\end{examples}
\begin{subtheorem}
    Если $a$ - обратимый элемент ассоциативного кольца $K$ с единицей, то обратный элемент к $a$ единственный.
\end{subtheorem}
\begin{proof}
    Пусть $b$ и $c$ - обратные к $a$. Тогда:
    \[b = b \cdot 1 = b(ac) = (ba)c = 1 \cdot c = c\]
\end{proof}
\begin{remark}
    Далее все рассматриваемые кольца - ассоциативные (свойства неассоциативных колец не такие общие и не будут рассматриваться в данном курсе).
\end{remark}
\begin{subtheorem}
    Если элемент $a$ кольца $K$ обратим, то $a$ - не делитель нуля. 
\end{subtheorem}
\begin{proof}
    От противного: пусть $a$ обратим и $a$ - делитель нуля. Тогда:
    \[\exists b \neq 0 : ab = 0 \Longrightarrow a^{-1}ab = a\cdot 0 \Longrightarrow b = 0 \text{ - противоречие.}\]
\end{proof}
\begin{consequense}
    В поле нет делителей нуля.
\end{consequense}
\begin{definition}
    Подмножество $L$ кольца $K$ называется подкольцом, если
    \begin{enumerate}
        \item $(L, +)$ - подгруппа аддитивной группы кольца $(K, +)$;
        \item $\forall a, b \in L: \ ab \in L$.
    \end{enumerate}
\end{definition}
\begin{subtheorem}
    Любое подкольцо $L$ кольца $K$ является кольцом относительно операций кольца $K$.  
\end{subtheorem}
\begin{example}
    $2\Z \subset \Z \subset \Q$.
\end{example}
\begin{definition}
    Подмножество $L$ поля $K$ называется подполем, если
    \begin{enumerate}
        \item $(L, +, \cdot)$ - подкольцо кольца $(K, +, \cdot)$;
        \item $1 \in L$;
        \item $\forall a \in L: \ a^{-1} \in L$.
    \end{enumerate}
\end{definition}
\begin{subtheorem}
    Любое подполе $L$ поля $K$ является полем относительно операций поля $K$.  
\end{subtheorem}
\begin{examples}\tab
    \begin{enumerate}
        \item $\Z$ - подкольцо $\Q$, но не подполе $\Q$;
        \item $\Q \subset \R \subset \CC$.
    \end{enumerate}    
\end{examples}
\subsection{Идеалы колец и факторкольца}
\begin{definition}
    Подмножество $L$ кольца $K$ называется левым (правым) идеалом, если:
    \begin{enumerate}
        \item $(L, +)$ - подгруппа $(K, +)$;
        \item $\forall a \in K, x \in L: ax \in L \ (xa \in L)$
    \end{enumerate}
\end{definition}
\begin{remark}
    Левый идеал замкнут относительно умножения на элементы кольца слева, правый - относительно умножения справа.
\end{remark}
\begin{definition}
    Подмножество $L$ кольца $K$ называется (двусторонним) идеалом, если:
    \begin{enumerate}
        \item $(L, +)$ - подгруппа $(K, +)$;
        \item $\forall a \in K, x \in L: ax, xa \in L$
    \end{enumerate}
\end{definition}
\begin{subtheorem}
    (Левый, правый, двусторонний) идеал кольца $K$ - подкольцо $K$.
\end{subtheorem}
\begin{proof}
    Очевидно из определения.
\end{proof}
\begin{remark}
    Идеалы в кольцах можно считать аналогом нормальных подгрупп в группах.
\end{remark}
\begin{example}
    Пусть $K = \Z$. Тогда любое подкольцо $K$ имеет вид $H = m\Z$ - любое такое подкольцо является идеалом в $\Z$
\end{example}
\begin{definition}
    В любом кольце $K$ есть идеалы $\{0\}, K$ - они называются тривиальными идеалами.
\end{definition}
\begin{subtheorem}
    В поле нет нетривиальных идеалов.
\end{subtheorem}
\begin{proof}
    Пусть $\F$ - поле. Пусть $I \subset \F$ - идеал, причём $I \neq 0$. Тогда $\exists x \in I: x \neq 0$. Так как $\F$ - поле, $\exists x^{-1} \in \F$, а отсюда $1 = x^{-1}x \in I$ (т.к. $x \in I$).\\
    Тогда для любого $a \in \F$: $a = a\cdot 1 \in I \ (\text{т.к. } 1 \in \F)\Longrightarrow I = \F$.
\end{proof}

Пусть $K$ - кольцо, $I$ - идеал $K$. Рассмотрим множество
\[K/I = \{a + I \ | \ a \in K\}\]
и введём на нём операции:
\begin{enumerate}
    \item Сложение: $(a + I) + (b + I) = a + b + I$;
    \item Умножение: $(a + I) \cdot (b + I) = ab + I$.
\end{enumerate}
\begin{subtheorem}
    Данные операции корректны (не зависят от представителей классов).
\end{subtheorem}
\begin{proof}\tab
    \begin{enumerate}
        \item Сложение корректно из корректности сложения смежных классов в группах, так как $(I, +)$ - нормальная подгруппа $(K, +)$;
        \item Пусть $a + I = \tilde{a} + I$, $b + I = \tilde{b} + I$. Докажем, что $ab + I = \tilde{a}\tilde{b} + I$:\\ $\tilde{a} = a + y_a$, $\tilde{b} = b + y_b$, где $y_a, y_b \in I$. Тогда:
        \[\forall x \in \tilde{a}\tilde{b} + I: \ x = \tilde{a}\tilde{b} + y, y \in I\]
        \[x =  \tilde{a}\tilde{b} + y = (a + y_a)(b + y_b) + y = ab + \undermat{\in I}{y_ab + ay_b + y_ay_b + y} = ab + I\]
    \end{enumerate}    
\end{proof}
\begin{subtheorem}
    Множество $K/I$ с введёнными операциями - кольцо, причём если $K$ ассоциативно (коммутативно), то $K/I$ ассоциативно (коммутативно).
\end{subtheorem}
\begin{proof}
    Проверим определение кольца:
    \begin{itemize}
        \item $(K/I, +)$ - группа по сложению (как факторгруппа $K/I$);
        \item $(a + I) + (b + I) = a + b + I = b + a + I = (b + I) + (a + I)$ - коммутативность сложения;
        \item $(a + I)((b + I) + (c + I)) = (a + I)(b + c + I) = a(b + c) + I = ab + ac + I = \\ = (ab + I) + (ac + I) = (a+I)(b+I) + (a+I)(c+I)$;
        \item $((b + I) + (c + I))(a + I) = (b + c + I)(a + I) = (b + c)a + I = ba + ca + I = \\ = (ba + I) + (ca + I) = (b+I)(a+I) + (c+I)(a+I)$;
    \end{itemize}
    При этом:
    \begin{itemize}
        \item $K$ ассоциативно $\Longrightarrow (a+I)((b+I)(c+I)) = (a+I)(bc+I) = (a(bc) + I) = ((ab)c + I) = (ab+I)(c+I) = ((a+I)(b+I))(c+I) \Longrightarrow K/I$ ассоциативно;
        \item $K$ коммутативно $\Longrightarrow (a+I)(b+I) = ab+ I = ba + I = (b+I)(a+I) \Longrightarrow K/I$ коммутативно.
    \end{itemize} 
\end{proof}
\begin{definition}
    Данное кольцо $K/I$ называется факторкольцом $K$ по идеалу $I$.
\end{definition}
\subsection{Гомоморфизмы колец}
\begin{definition}
    Пусть $K, \tilde{K}$ - кольца. Отображение $\phi: K \rightarrow \tilde{K}$ называется гомоморфизмом колец, если:
    \begin{enumerate}
        \item $\forall a, b \in K: \ \phi(a+b) = \phi(a)+\phi(b)$;
        \item $\forall a, b \in K: \ \phi(ab) = \phi(a)\phi(b)$;
    \end{enumerate}
\end{definition}
\begin{definition}
    Изоморфизм колец - биективный гомоморфизм колец. 
\end{definition}
\begin{definition}
    Кольца $K, \tilde{K}$ называются изоморфными, если существует изоморфизм $\phi: K \rightarrow \tilde{K}$. Обозначается $K \simeq \tilde{K}$.
\end{definition}
\begin{definition}
    Пусть $\phi: K \rightarrow \tilde{K}$ - гомоморфизм.\\
    Множество $\textup{Ker } \phi = \{a \in K \ | \ \phi(a) = 0\}$ называется ядром $\phi$.\\
    Множество $\textup{Im } \phi = \{b \in \tilde{K} \ | \ \exists a \in K: \phi(a) = b\}$ называется образом $\phi$.
\end{definition}
\begin{subtheorem}
    Пусть $\phi: K \rightarrow \tilde{K}$ - гомоморфизм. Тогда:
    \begin{enumerate}
        \item $\textup{Ker } \phi$ - идеал $K$;
        \item $\textup{Im } \phi$ - подкольцо $\tilde{K}$.
    \end{enumerate}
\end{subtheorem}
\begin{proof}\tab
    \begin{enumerate}
        \item Проверим определение идеала:
        \begin{itemize}
            \item $\textup{Ker } \phi$ - подгруппа $(K, +)$, так как $\phi$ - гомоморфизм $(K, +) \rightarrow (\tilde{K}, +)$;
            \item $\forall a \in \textup{Ker }\phi, b \in K: \ \phi(ab) = \phi(a)\phi(b) = 0; \ \phi(ba) = \phi(b)\phi(a) = 0$. Значит, $ab, ba \in \textup{Ker } \phi$.
        \end{itemize}
        Отсюда $\textup{Ker } \phi$ - идеал кольца $K$.
        \item Проверим определение подкольца:
        \begin{itemize}
            \item $\textup{Im } \phi$ - подгруппа $(\tilde{K}, +)$, так как $\phi$ - гомоморфизм $(K, +) \rightarrow (\tilde{K}, +)$;
            \item $\forall a, b \in \textup{Im } \phi: \exists x_a, x_b \in K: \phi(x_a) = a, \phi(x_b) = b$. Тогда $\phi(x_ax_b) = ab$, то есть $ab \in \textup{Im }\phi$.
        \end{itemize}
        Отсюда $\textup{Im } \phi$ - подкольцо $\tilde{K}$.
    \end{enumerate}    
\end{proof}
\begin{subtheorem}
    Пусть $K$ - кольцо, $I$ - идеал $K$. Тогда $\exists$ гомоморфизм колец $\phi: K \rightarrow K/I$ такой, что $\textup{Ker } \phi = I, \textup{Im } \phi = K/I$.
\end{subtheorem}
\begin{proof}
    Подойдёт гомоморфизм $\pi: a \mapsto a + I$.
\end{proof}
\begin{definition}
    Приведённый выше гомоморфизм $\pi: K \rightarrow K/I$ называется каноническим (естественным, натуральным) гомоморфизмом колец $K$ и $K/I$.
\end{definition}
\begin{theorem} (О гомоморфизме колец)\\
    Пусть $K, \tilde{K}$ - кольца, $\phi: K \rightarrow \tilde{K}$ - гомоморфизм колец.\\
    Тогда $K / \textup{Ker }\phi \simeq \textup{Im }\phi$.
\end{theorem}
\begin{proof}
    $\textup{Ker }\phi$ - идеал кольца $K$, то есть факторкольцо $K/\textup{Ker }\phi$ определено. Рассмотрим отображение 
    \[\psi: K/\textup{Ker }\phi \rightarrow \textup{Im }\phi \ \ a + \textup{Ker } \phi \mapsto \phi(a)\]
    В доказательстве теоремы о гомоморфизме групп было доказано, что это отображение корректно, биективно и сохраняет сложение. Проверим сохранение умножения:
    \[\psi((a+\textup{Ker }\phi)(b + \textup{Ker } \phi)) = \psi(ab+\textup{Ker }\phi) = \phi(ab) =\]
    \[= \phi(a)\phi(b) = \psi(a + \textup{Ker }\phi)\psi(b+\textup{Ker }\phi)\]
    Значит, $\psi$ - изоморфизм колец. 
\end{proof}
\subsection{Главный идеал}
Пусть $K$ - коммутативное ассоциативное кольцо с единицей, $S \subset K$ - произвольное подмножество. Рассмотрим множество \[(S) = \{\sum \limits_{i=0}^ka_is_i \ |\ a_i \in K, s_i \in S\}\]
\begin{subtheorem}\tab
    \begin{enumerate}
        \item $(S)$ - двусторонний идеал;
        \item $(S)$ - наименьший двусторонний идеал, содержащий $S$.
    \end{enumerate}    
\end{subtheorem}
\begin{proof}\tab
    \begin{enumerate}
        \item Мы поверим, но проверим:
        \begin{itemize}
            \item $\forall x, y \in (S):\ x + y = \sum \limits_{i=0}^kx_is_i + \sum \limits_{i=0}^ky_is_i = \sum \limits_{i=0}^k(x_i + y_i)s_i \in (S)$;
            \item $0 = \sum \limits_{i=0}^k0\cdot s_i \in (S)$;
            \item $\forall x \in (S): \ -x = \sum \limits_{i=0}^kx_is_i = \sum \limits_{i=0}^k(-x_i)s_i \in (S)$;
            \item $\forall x \in (S), a \in K: \ xa = ax = a\sum \limits_{i=0}^kx_is_i = \sum \limits_{i=0}^kax_is_i \in (S)$.
        \end{itemize}
        Значит, $(S)$ - двусторонний идеал.
        \item Пусть $I$ - двусторонний идеал кольца $K$ такой, что $S \subset K$. Тогда из определения идеала $\forall a \in K, s \in S: \ as \in I$, а так как идеал - подкольцо, любая сумма вида $\sum \limits_{i=0}^ka_is_i$, где $a_i \in K, s_i \in S$, лежит в $I$. Значит, $(S) \subseteq I$.
    \end{enumerate}
\end{proof}
\begin{definition}
    Если $I = (S)$, то говорят, что $I$ порождается множеством $S$.\\
    Если при этом $|S| = 1$, то $I$ называется главным идеалом.\\
    Иными словами, $I$ - главный идеал кольца $K$, если $\exists u \in K: \ \forall x \in I: \ x = ua$ для некоторого $a \in K$.
\end{definition}
\begin{examples}\tab
    \begin{enumerate}
        \item $K = \Z$: $(m) = m\Z$;
        \item $K = \F[x]$: $(x+1) = \{(x+1)f \ |\ f \in \F[x]\}$.
    \end{enumerate}    
\end{examples}
\begin{definition}
    Коммутативное ассоциативное кольцо с единицей $1 \neq 0$, в котором любой идеал является главным, называется кольцом главных идеалов.
\end{definition}
\begin{example}
    $\Z$ - кольцо главных идеалов.
\end{example}
\begin{definition}
    Коммутативное ассоциативное кольцо с единицей $1 \neq 0$, в котором нет делителей нуля, называется целостным кольцом.
\end{definition}
\begin{examples}
    $\Z, \F[x]$ - целостные кольца.
\end{examples}
Все напоминания данной главы из 1 семестра, их доказательства см. \href{https://github.com/Viacheslavik122333/Halgebra1sem/blob/main/lecture.pdf}{здесь}.
\begin{reminder}
    Если $K$ - целостное кольцо, то в $K$ определены понятия $a\mid b$, $\textup{НОД}(a, b)$.\\
    \textup{НОД} бывает не определён, но если он существует - определён однозначно с точностью до ассоциированности (умножения на обратимые элементы). 
\end{reminder}
\begin{definition}
    Целостное кольцо $K$, не являющееся полем, называется евклидовым кольцом, если $\exists$ функция $N: K\setminus{0} \rightarrow \Z_+$ (она называется нормой) такая, что:
    \begin{enumerate}
        \item $\forall a, b \in K: N(ab) > N(a)$;
        \item $\forall a, b \in K, b \neq 0 \exists q, r \in K: \ a = bq + r$, где $\left[\begin{array}{l}
        r = 0\\
        N(r) < N(b)
    \end{array}\right.$
    \end{enumerate}
\end{definition}
\begin{examples}\tab
    \begin{enumerate}
        \item $K = \Z$: $N(a) = |a|$;
        \item $K = \F[x]$: $N(f) = \deg f$;
        \item $K = \Z[i]$ - кольцо гауссовых чисел
        \\$N(a + bi) = a^2 + b^2$.
    \end{enumerate}
\end{examples}
\begin{exercise}
    $\Z[x]$ - целостное, но не евклидово кольцо.
\end{exercise}
\begin{proof}
    Покажем, что $\Z[x]$ - целостное кольцо. Очевидно, что $\Z[x]$ - коммутативное ассоциативное кольцо с единицей, то есть остаётся показать, что в нём нет делителей нуля: если $P(x), Q(x)\neq 0$, то, взяв произведение членов с максимальной степенью $x$ в обоих многочленах, получим одночлен, степень которого больше, чем у всех других в произведении. Значит, коэффициент при нём останется ненулевым, то есть $ab \neq 0$.\\
    Доказательство неевклидовости $\Z[x]$ проведём после следующей теоремы.
\end{proof}
\begin{reminder}
    Пусть $K$ - евклидово кольцо. Тогда $\exists\ \textup{НОД}(a, b) = d$ и $\exists u, v \in K: d = ua + vb$. 
\end{reminder}
\begin{theorem}
    Всякий идеал евклидова кольца $K$ является главным.
\end{theorem}
\begin{proof}
    Рассмотрим произвольный идеал $I$ кольца $K$.\\
    Если $I = \{0\}$, то $I = (0)$. Иначе рассмотрим наименьший по норме элемент в $I$ - обозначим его $u$. Докажем, что $I = (u)$:\\
    Пусть $a \in I$ - произвольный элемент. Разделим $a$ на $u$ с остатком: $\exists q, r \in K: a = uq + r$, причём если $r \neq 0$, то $N(r) < N(u)$. При этом $r = a - uq \in I$, а значит $r = 0$ из предположения, что $u$ - наименьший по норме элемент в $I$. Значит, $\forall a \in I \ \exists q \in K: a = uq$, а значит $I = (u)$. 
\end{proof}
\begin{consequense}
    Любое евклидово кольцо является кольцом главных идеалов.
\end{consequense}
\begin{examples}\tab
    \begin{enumerate}
        \item $\F[x]$ - кольцо главных идеалов;
        \item $\F[x, y]$ - не кольцо главных идеалов. Покажем это:\\
        Рассмотрим идеал $I = (x, y)$. Предположим, что $I = (f)$ - тогда $\exists q_1, q_2 \in \F[x, y]: x = fq_1, y = fq_2$. Из равенства многочленов $x = fq_1$ имеем, что либо $f \sim 1$, либо $f \sim x$. Рассмотрим эти случаи:
        \begin{itemize}
            \item $f \sim 1 \Longrightarrow 1 = ax + by$ - очевидно невозможно (не можем получить ненулевую константу);
            \item $f \sim x \Longrightarrow f = cx \Longrightarrow y = cxq_2$ - невозможно.
        \end{itemize}
        Значит, $I$ - не главный идеал.
        \item Докажем, что $\Z[x]$ - не кольцо главных идеалов. Рассмотрим $I = (2, x)$. Предположим, что $I = (f)$ - тогда $\exists q_1, q_2 \in \Z[x]: 2 = fq_1, x = fq_2$. Из первого равенства имеем, что либо $f \sim 1$, либо $f \sim 2$:
        \begin{itemize}
            \item $f \sim 1 \Longrightarrow \pm 1 = 2a + xb$ - невозможно, т.к. коэффициенты целые;
            \item $f \sim 2 \Longrightarrow x = \pm2 \cdot q_2$ - невозможно по тем же соображениям.
        \end{itemize}
        Значит, $\Z[x]$ - не кольцо главных идеалов, а отсюда и не евклидово кольцо.
    \end{enumerate}
\end{examples}
\begin{definition}
    Пусть $K$ - целостное кольцо.\\
    Элемент $p \in K$ называется простым, если
    \begin{enumerate}
        \item $p \neq 0$;
        \item $p$ - необратимый;
        \item если $p = ab$, то либо $a$, либо $b$ - обратимый элемент.
    \end{enumerate}
\end{definition}
\begin{examples}\tab
    \begin{enumerate}
        \item $K = \Z$: простые элементы - $\pm p$, где $p$ - простое число;
        \item $K = \F[x]$: простые элементы - неприводимые многочлены. 
    \end{enumerate}    
\end{examples}
\begin{reminder}
    Любой элемент евклидова кольца раскладывается в произведение простых элементов этого кольца, причём это разложение единственно с точностью до домножения множителей на обратимые элементы и их порядка.
\end{reminder}
\begin{remark}
    Аналогичное утверждение верно для всех колец главных идеалов, однако в данном курсе оно рассматриваться не будет.
\end{remark}
\begin{reminder}
    $\Z/(n) = \Z/n\Z = \Z_n$ - поле $\Longleftrightarrow n$ - простое.
\end{reminder}
Теперь можем доказать гораздо более общее утверждение:
\begin{theorem}
    Пусть $K$ - евклидово кольцо, $u \in K$. Тогда $K/(u)$ - поле $\Longleftrightarrow u$ - простой элемент. 
\end{theorem}
\begin{proof}
    $\\ \Longrightarrow: \ \ $Пусть $K/(u)$ - поле. Допустим, что $u$ - не простой. Тогда возможны случаи:
    \begin{enumerate}
        \item $u = 0 \Longrightarrow K/(u) = K$ - не поле по определению евклидова кольца;
        \item $u$ обратим $\Longrightarrow 1 \in (u) \Longrightarrow K/(u) = K/K = \{0\}$ - не поле;
        \item $u = ab$, где $a, b$ - необратимы. Тогда покажем, что $a + (u) \neq (u)$: иначе
        \[a \in (u) \Longrightarrow a = ud = abd \Longrightarrow a(bd-1) = 0 \Longrightarrow bd = 1 \Longrightarrow b\text{ - обратим}\]
        Аналогично $b + (u) \neq (u)$. Тогда из $u = ab$ следует, что $(a + (u))(b + (u)) = (u)$, то есть в поле $K/(u)$ элементы $a + (u)$ и $b + (u)$ - делители нуля - противоречие.
    \end{enumerate}
    Все случаи невозможны, а значит, $u$ не может не быть простым.\\
    $\Longleftarrow: \ \ $ Пусть $u$ - простой элемент. $K/(u)$ - коммутативное ассоциативное кольцо с единицей из соответвующих свойств $K$, причём единица в нём - $1 + (u)$ \\
    ($\neq (u)$, так как иначе $1 \in (u) \Longrightarrow 1 = ua$ - противоречие с необратимостью $u$)\\
    Пусть $a$ - произвольный ненулевой элемент $K/(u)$. Тогда:
    \[a + (u) \neq (u) \Longrightarrow a \notin (u) \Longrightarrow u \nmid a \overset{u \text{-простой}}{\Longrightarrow} \textup{НОД}(u, a) = 1 \Longrightarrow\]\[\Longrightarrow \exists x, y \in K: xa + yu = 1 \text{ в } K \Longrightarrow (x + (u))(a + (u)) = 1 + (u) \text{ в } K/(u)\]
    Значит, $x + (u)$ - обратный к $a + (u)$. Отсюда любой ненулевой элемент $K/(u)$ обратим, а тогда $K/(u)$ - поле.
\end{proof}
\begin{consequense}
    Пусть $K = \F[x]$ ($\F$ - поле), $f \in \F[x]$. Тогда $\F[x]/(f)$ - поле $\Longleftrightarrow f$ - неприводимых многочлен.
\end{consequense}
\begin{example}
    $\R[x] / (x^2 + 1)$ - поле.
\end{example}
\begin{subtheorem}
    Множество $\{a + (f) \ | \ a \in \F\}$ образует подкольцо в кольце $\F[x]/(f)$, где $\F$ - поле.\\
    В частности, если $f$ неприводим, то это подкольцо - подполе поля $\F[x]/(f)$.\\
    Более того, это подполе изоморфно полю $\F$ (изоморфизм $a + (f) \mapsto a$).
\end{subtheorem}
\begin{proof}
    Все пункты определения подкольца (подполя) очевидно следуют из аналогичных утверждений для поля $\F$.
\end{proof}
\begin{remark}
    В дальнейшем такое подполе и поле $\F$ будут отождествляться.
\end{remark}
\begin{definition}
    Если $L$ - подполе поля $K$, то $K$ называется расширением $L$.
\end{definition}
В таком случае $K$ можно рассматривать как векторное пространство над $L$.
\begin{examples}\tab
    \begin{enumerate}
        \item $\CC$ - расширение $\R$;
        \item если $f \in \F[x]$ - неприводимый, то $\F(x)/(f)$ - расширение поля $\F$.
    \end{enumerate}    
\end{examples}
\begin{definition}
    Расширение $K$ поля $L$ называется конечномерным, если $K$ - конечномерное векторное пространство над $L$ (т.е. $\dim_L K < \infty$).\\
    В этом случае $\dim_L K$ называется степенью расширения.
\end{definition}
\begin{example}
    $\dim_\R \CC = 2$ - базис $\{1, i\}$.
\end{example}
\begin{subtheorem}
    $\\$Пусть $\F$ - поле, $f \in \F[x]$ - неприводимый многочлен, $\deg f = n$. Тогда элементы $1 + (f), x + (f),...,x^{n-1} + (f)$ - базис $\F[x]/(f)$ как векторного пространства над $\F$, то есть $\F[x]/(f)$ - расширение поля $\F$ степени $n$.
\end{subtheorem}
\begin{proof}
    Рассмотрим произвольный многочлен $g \in \F[x]$ и разделим его на $f$ с остатком:
    \[g(x) = f(x)q(x) + r(x), \  \left[\begin{array}{l}
        r = 0\\
        \deg r < \deg f
    \end{array}\right.\]
    Если $r = 0$, то $g(x) \in (f) \Longrightarrow g + (f) = 0$.\\
    Иначе (далее обозначим $(f)$ как $I$): 
    \[g(x) + I = r(x) + I = c_0 + c_1x + ... + c_{n-1}x^{n-1} + I =\]
    \[= (c_0 + I)(1 + I) + (c_1 + I)(x + I) + ... + (c_{n-1} + I)(x^{n-1} + I) =\]
    \[\text{(отождествление)  } = c_0(1 + I) + c_1(x + I) + ... + c_{n-1}(x^{n-1} + I)\] 
    - отсюда система $\{1 + (f), x + (f),...,x^{n-1} + (f)\}$ порождает $\F[x]/(f)$.\\
    Покажем линейную независимость:
    \[\lambda_0(1+I) + \lambda_1(x+I) + ... + \lambda_{n-1}x^{n-1} + I = I \Longrightarrow \lambda_0 + \lambda_1x + ...+\lambda_{n-1}x^{n-1} \in I\] 
    Тогда $f \mid (\lambda_0 + ...+\lambda_{n-1}x^{n-1})$, но $\deg f > n-1$. Значит, такое возможно только в случае, когда все $\lambda_i$ равны нулю, что и означает линейную независимость.
\end{proof}

Далее введём обозначения $\alpha = x + I, \alpha^k = x^k + I$. Тогда по утверждению $\{1,\alpha, ..., \alpha^{n-1}\}$ - базис $\F[x]/I$. Посмотрим на значение $f(\alpha)$
\[f = a_0 + a_1x + ... + a_nx^n \Longrightarrow f(\alpha) = a_0 + a_1\alpha + ... + a_n\alpha^n + I =\]
\[= a_0 + a_1(x + I) + ... + a_n(x^n + I) + I = a_0 + a_1x + ... + a_nx^n + I = I\]
то есть в отождествлении $f(\alpha) = 0$, и $\alpha$ - корень $f$ в $\F[x]/(f)$.\\
При этом в $\F$ у многочлена $f$ не было корней, так как он неприводим над $\F$.
\begin{definition}
    Данный переход от поля $\F$ к расширению $\F[x]/(f)$ ($f$ - неприводимый) называется присоединением к полю $\F$ корня $\alpha$ многочлена $f$.
\end{definition}
\begin{example}
    Пусть $\F = \R, f = x^2 + 1$ - неприводимый над $\R$.\\
    $\R[x]/(f)$ - поле, $\{1, \alpha\} = \{1 + (f), x + (f)\}$ - его базис как векторного пространства над $\R$. Тогда любой элемент этого пространства представим в виде
    \[a + bx + (f) = a(1 + (f)) + b(x + (f)) = a + b\alpha\]
    При этом $f(\alpha) = 0 \Longrightarrow \alpha^2 + 1 = 0 \Longrightarrow \alpha^2 = -1$. Значит, получили поле, в котором все элементы представляются в виде $a + b\alpha$, где $\alpha^2 = -1$ - это в точности поле комплексных чисел. Значит, $\R[x]/(x^2+1) \simeq \CC$.
\end{example}