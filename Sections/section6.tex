\section{Теоремы Силова}
Пусть $G$ - конечная группа, $|G| = p^s \cdot m$, где $p$ - простое, $(p, m) = 1$.
\begin{definition}
    Подгруппа $H \leq G$ называется силовской $p$-подгруппой, если $|H| = p^s$.
\end{definition}
\begin{remark}
    Несложно видеть, что определение корректно: если $H$ - силовская $p$-подгруппа, то $H$ - $p$-подгруппа; более того, это доказано в \hyperref[ex:ex1]{упражнении} п. 4.4
\end{remark}
\begin{theoremnum}(Первая теорема Силова - о существовании)\\
    Силовская $p$-подгруппа существует.   
\end{theoremnum}
\begin{remark}
    Напомним, что более общее утверждение $k \mid |G| \Longrightarrow \exists H \leq G: |H| = k$ неверно - в $A_4$ нет подгруппы порядка 6.
\end{remark}
\begin{theoremnum}(Вторая теорема Силова - о сопряжённости)\\
    Любая $p$-подгруппа лежит в некоторой силовской $p$-подгруппе.\\
    Все силовские $p$-подгруппы сопряжены.
\end{theoremnum}
\begin{theoremnum}(Третья теорема Силова - о количестве)\\
    Пусть $N_p$ - число силовских $p$-подгрупп в $G$. Тогда $\begin{cases}
        N_p \equiv 1(\textup{mod } p) \\
        N_p \mid m
    \end{cases}$
\end{theoremnum}
\begin{examples}\tab
    \begin{enumerate}
        \item $G = S_3, |G| = 6 = 2\cdot 3$. Силовские 2-подгруппы: $\langle (12) \rangle, \langle (13) \rangle, \langle (23) \rangle$.
        \item $G = S_4, |G| = 24 = 2^3\cdot 3$. Найдём силовские 2-подгруппы:\\
        Доказывалось, что $S_4 \simeq \textup{Sym}^+ K$ - группа вращений куба. Можем рассмотреть сечение куба плоскостью, параллельной некоторой паре противоположных граней - вращения, оставляющие квадрат сечения на месте, образуют подгруппу, очевидно изоморфную $D_4$ (по определению $D_4$). Такая подгруппа будет иметь порядок 8, и таких подгрупп будет 3 - столько же, сколько пар противоположных граней - по $III$ теореме Силова это все силовские $p$-подгруппы в $G$.
    \end{enumerate}
\end{examples}
\subsection{I теорема Силова}
    Пусть $G$ - группа, $|G| = p^sm$, где $p$ - простое, $(p, m) = 1$. Тогда $\exists$ силовская $p$-подгруппа в $G$.
\begin{proof}
    Рассмотрим случаи:
    \begin{enumerate}
        \item $G$ - абелева $\Longrightarrow G \simeq \langle a_1 \rangle_{p_1^{s_1}} \times ... \times \langle a_k \rangle_{p_k^{s_k}}$. Без ограничения общности $p_1 = ... = p_t = p,\ p_{t+1},...,p_k \neq p$. Тогда $H \simeq \langle a_1 \rangle_{p^{s_1}} \times ... \times \langle a_t \rangle_{p^{s_t}}$ - искомая силовская $p$-подгруппа: очевидно, что $H$ является $p$-подгруппой, а также $p^sm = |G| = |H|\cdot|G/H|$, где $p \nmid |G/H| \Longrightarrow p^s \mid |H| \Longrightarrow |H| = p^s$.
        \item Общий случай ($G$ - неабелева). Индукция по $|G|$:\\
        База: $n = 1$ - очевидно;\\
        Шаг: Пусть $G = Z(G) \sqcup x_1^G \sqcup ... \sqcup x_k^G$ - разложение $G$ на классы сопряжённости, где $x_i \notin Z(G)$, то есть $|x_i^G| > 1$. Вновь рассмотрим случаи:
        \begin{enumerate}
            \item $\exists i = \overline{1,...,k}: p \nmid |x_i^G|$. Знаем, что $|C(x_i)| = \frac{|G|}{|x_i^G|}$. По предположению индукции в $C(x_i) \ \exists$ силовская $p$-подгруппа $H \Longrightarrow |H| = p^s$ (так как степень вхождения $p$ в порядок группы не уменьшилась), т.е. $H$ - силовская $p$-подгруппа и для $G$;
            \item $\forall i = \overline{1,...,k}: p \mid |x_i^G|$. Тогда $p \mid Z(G) \Longrightarrow |Z(G)| = p^{s_0}m_0 \ ((p, m_0) = 1)$. Так как $Z(G)$ - абелева, по 1 случаю $\exists$ силовская $p$-подгруппа $S_0 \leq Z(G), \ |S_0| = p^{s_0}$.\\
            По свойству центра $S_0 \leq Z(G) \Longrightarrow S_0 \unlhd G$ - можем рассмотреть $G/S_0$.
            Так как $|G/S_0| < |G|$, по предположению индукции $\exists$ силовская $p$-подгруппа $S \leq G/S_0$. $|G/S_0| = p^{s-s_0}m \Longrightarrow |S| = p^{s-s_0}$\\
            Рассмотрим натуральный гомоморфизм $\pi: G \rightarrow G/S_0$, и $\tilde{S} = \pi^{-1}(S)$ - полный прообраз $S$ при этом гомоморфизме.\\
            $S_0 \subset \tilde{S}$, так как $\forall s_0 \in S_0: \pi(s_0) = eS_0$, причём $S_0 \unlhd G \Longrightarrow S_0 \unlhd \tilde{S}$, т.е. можем рассмотреть ограничение $\pi|_{\tilde{S}}: \tilde{S} \rightarrow \tilde{S}/S_0$. $\pi|_{\tilde{S}}$ - натуральный гомоморфизм с ядром $S_0$ и образом $\pi(\tilde{S}) = S$.\\
            Натуральный гомоморфизм сюръективен, а отсюда по теореме о гомоморфизме $|\tilde{S}| = |S_0|\cdot|S| = p^{s_0} \cdot p^{s-s_0} = p^s \Longrightarrow \tilde{S}$ - искомая силовская $p$-подгруппа $G$. 
        \end{enumerate}
    \end{enumerate}
\end{proof}
\subsection{II теорема Силова}
Пусть $G$ - группа, $|G| = p^sm$, где $p$ - простое, $(p, m) = 1$.\\
Тогда любая $p$-подгруппа группы $G$ лежит в некоторой силовской $p$-подгруппе.\\
Все силовские $p$-подгруппы группы $G$ сопряжены.
\begin{proof}
    Пусть $|G| = p^sm$, где $p$ - простое, $(p, m) = 1$.\\
    По $I$ теореме Силова $\exists$ силовская $p$-подгруппа $S \leq G$.
    Рассмотрим $H \leq G$ - произвольную нетривиальную $p$-подгруппу (случай $H= \{e\}$ очевиден).\\
    Рассмотрим множество $X = \{g_1S,...,g_mS\}$ смежных классов $G$ по $S$ и действие $H \acts X$, заданное по правилу $\alpha(h)g_iS = hg_iS$.\\
    $|\textup{Orb}(g_iS)| \mid |H| \Longrightarrow \left[\begin{array}{l}
        |\textup{Orb}(g_iS)| = 1\\
        p \mid |\textup{Orb}(g_iS)|
    \end{array}\right.$.\\
    Предположим, что $\forall i = \overline{1,...,m}: p \mid |\textup{Orb}(g_iS)|$. Тогда $p \mid \sum \limits_i |\textup{Orb}(g_iS)| = |X|$. Однако $|X| = m$ - взаимно просто с $p$. Противоречие.\\
    Отсюда $\exists i = \overline{1,...,m}: |\textup{Orb}(g_iS)| = 1$, т.е. точка $g_iS$ неподвижна при $H \acts X$. Значит, $\forall h \in H\  hg_iS = g_iS \Longrightarrow h \in g_iSg_i^{-1} \Longrightarrow H \leq g_iSg_i^{-1}$. Так как $|g_iSg_i^{-1}| = |S|$, $g_iSg_i^{-1}$ - силовская $p$-подгруппа, т.е. $H$ лежит в силовской $p$-подгруппе $G$.

    Заметим, что в доказательстве выше подгруппа $S$ зафиксирована.\\
    Если рассмотреть $H$ - произвольную силовскую $p$-подгруппу $G$, то $|H| = p^s$. Так как $H \leq g_iSg_i^{-1}$, $|g_iSg_i^{-1}| = p^s \Longrightarrow H = g_iSg_i^{-1}$ - любая силовская $p$-подгруппа сопряжена с $S$. Значит, все силовские $p$-подгруппы сопряжены.
\end{proof}
\begin{consequense}
    Пусть $|G| < \infty$. Тогда $G$ - $p$-группа $\Longleftrightarrow |G| = p^s (s\in \N)$. 
\end{consequense}
\begin{proof}
    $ \\ \Longleftarrow$ - доказано ранее;\\
    $\Longrightarrow: \ \ $ Пусть $|G| = p^sm, (p, m) = 1$. По $I$ теореме Силова $\exists$ силовская $p$-подгруппа в $G$ (порядка $p^s$), а по $II$ теореме Силова $G$ как своя $p$-подгруппа содержится в своей силовской $p$-подгруппе. Значит, $|G| \mid p^s$, а отсюда $|G| = p^s$.
\end{proof}
\subsection{Нормализатор. III теорема Силова}
Пусть $G$ - группа, $H \leq G, X = \{gHg^{-1} \ | \ g \in G\}$.\\
Рассмотрим действие $G \acts X: \alpha(\tilde{g})(gHg^{-1}) = \tilde{g}(gHg^{-1})\tilde{g}^{-1}$\\
Для точки $H \in X: \ \textup{Orb}(H) = X, \ \textup{St}(H) = \{\tilde{g} \in G \ | \ \tilde{g}H\tilde{g}^{-1} = H\} \leq G$
\begin{definition}
    Стабилизатор $H$ относительно этого действия называется нормализатором группы $H$. Обозначается $N_G(H)$. 
\end{definition}
\begin{subtheoremnum}
    Если $|G| < \infty$, то $|G| = |X| \cdot |N_G(H)|$, где $X$ - число подгрупп, сопряжённых с $H$. В частности, $|X| = |G : N_G(H)|$.
\end{subtheoremnum}
\begin{proof}
    Очевидно следует из утверждения $|\textup{Orb}(x)| = \frac{|G|}{|\textup{St}(x)|}$.
\end{proof}
\begin{subtheoremnum}
    $N_G(H)$ - наибольшая (по включению) подгруппа $G$, содержащая $H$ как нормальную подгруппу.
\end{subtheoremnum}
\begin{proof}
    Из определения $N_G(H)$ очевидно, что $H \unlhd N_G(H)$.\\
    Пусть $H \unlhd K \leq G$. Тогда $\forall g \in K\  gHg^{-1} = H \Longrightarrow g \in N_G(H)$.\\
    Отсюда $K \leq N_G(H)$.
\end{proof}
\textbf{$III$ теорема Силова}\\
Пусть $G$ - группа, $|G| = p^sm$, где $p$ - простое, $(p, m) = 1$.\\
Пусть $N_p$ - число силовских $p$-подгрупп в $G$. Тогда $N_p \equiv 1(\textup{mod } p), \ N_p \mid m$.
\begin{proof}
    $\\$Пусть $S$ - произвольная силовская $p$-подгруппа $G$ (хотя бы одна существует по $III$ теореме Силова). Рассмотрим $X = \{gSg^{-1} \ | \ g \in G\}$. По $II$ теореме Силова все силовские $p$-подгруппы $G$ сопряжены, а также порядок любой подгруппы вида $gSg^{-1}$ равен $|S|$, т.е. $gSg^{-1}$ - также силовкая $p$-подгруппа. Отсюда $X$ - множество всех силовских подгрупп $G$.\\
    $|X| = N_p \Longrightarrow$ по утверждению 1 получаем $N_p \mid |G|$. Осталось показать, что $N_p \equiv 1(\textup{mod } p)$ (если это так, то $N_p \mid |G| = p^sm \Longrightarrow N_p \mid m$).\\
    Рассмотрим действие $S \acts X$ сопряжениями. Очевидно, $S$ - неподвижная точка относительно него. Также $N_p = |X| = \sum \limits_{i=1}^k |\textup{Orb}(x_i)|$. При этом \[|\textup{Orb}(x_i)| \mid |S| = p^s \Longrightarrow \left[\begin{array}{l}
        |\textup{Orb}(x_i)| = 1\\
        p \mid |\textup{Orb}(x_i)|
    \end{array}\right.\]
    Значит, достаточно показать, что $S$ - единственная неподвижная точка относительно данного движения (тогда $|X| = \sum \limits_{i=1}^k |\textup{Orb}(x_i)| \equiv |\textup{Orb}(S)| = 1 (\textup{mod } p)$)\\
    Допустим, что $\tilde{S}$ - неподвижная точка $\Longrightarrow \forall g \in S \ g \tilde{S}g^{-1} = \tilde{S}$. \\
    Рассмотрим нормализатор $N_G(\tilde{S})$. Знаем, что $\tilde{S} \subseteq N_G(\tilde{S})$, а из неподвижности точки $\tilde{S}$ имеем $S \subseteq N_G(\tilde{S})$. Также  $N_G(\tilde{S}) \leq G$, то есть степень вхождения $p$ в $|N_G(\tilde{S})|$ также равна $s$. Значит, $S, \tilde{S}$ - силовские $p$-подгруппы в $N_G(\tilde{S})$. Тогда по $II$ теореме Силова $S$ и $\tilde{S}$ сопряжены в $N_G(\tilde{S})$, т.е. $S = g\tilde{S}g^{-1}, g \in N_G(\tilde{S})$, а тогда по определению нормализатора $S = \tilde{S}$. Значит, $S$ - единственная неподвижная точка.  
\end{proof}
\begin{consequense}
    Пусть $G$ - группа, $|G| = p^sm$, где $p$ - простое, $(p, m) = 1$.\\
    Тогда силовская $p$-подгруппа в $G$ единственна $\Longleftrightarrow$ эта подгруппа нормальна. 
\end{consequense}
\begin{proof}
    $\\\Longleftarrow: \ $ Пусть $S \unlhd G$ - силовская $p$-подгруппа. По $II$ теореме Силова все силовские $p$-подгруппы сопряжены с $S$, а из нормальности совпадают с $S$.\\
    $\Longrightarrow: \ $ Если $S$ - единственная, то $\forall g \in G: gSg^{-1} = S$\\
    (используется, что из конечности $G$: $gSg^{-1} \subseteq S \Rightarrow gSg^{-1} = S$). 
\end{proof}
\begin{exercise}
    Доказать, что любая группа порядка 15 циклическая.
\end{exercise}
\begin{proof}
    Пусть $G$ - группа порядка 15. По I теореме Силова в ней есть силовские подгруппы порядка 3 и порядка 5. Притом по III теореме Силова: 
    \[N_3 \equiv 1(\textup{mod } 3), \ N_3 \mid 5 \Longrightarrow N_3 = 1\]
    \[N_5 \equiv 1(\textup{mod } 5), \ N_5 \mid 3 \Longrightarrow N_5 = 1\]
    Таким образом, в $G$ есть по одной силовской подгруппе порядка 3 и 5, а по следствию из III теоремы Силова они обе нормальны в $G$. Так как их порядки простые, обе эти подгруппы циклические, т.е. изоморфны $\Z_3$ и $\Z_5$ соответственно.\\
    Остаётся заметить, что эти подгруппы пересекаются тривиально (у остальных элементов разные порядки), т.е. некоторая подгруппа $G$ раскладывается в их прямое произведение, а так как $15 = 3 \cdot 5$, эта подгруппа - вся $G$. Отсюда $G \simeq \Z_3 \times \Z_5 \simeq \Z_{15}$ - циклическая.
\end{proof}
\setcounter{thcount}{0}
\setcounter{concount}{0}
\setcounter{subthcount}{0}
\setcounter{lemcount}{0}
\newpage