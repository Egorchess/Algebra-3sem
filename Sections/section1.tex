\section{Группы}
\subsection{Основные понятия}
\begin{definition}
    Пусть $G$ - множество. Бинарной операцией на $G$ называется отображение $*: G \times G \rightarrow G$.
\end{definition}
\begin{definition}
    Множество $G$ с бинарной операцией $*$ называется группой, если выполнены следующие аксиомы:
    \begin{enumerate}
        \item $\forall a, b, c \in G \ \ a * (b * c) = (a * b) * c$;
        \item $\exists e \in G: \forall a \in G \ \ a * e = e * a = a$;
        \item $\forall a \in G \ \exists b \in G: a * b = b * a = e$
    \end{enumerate}
\end{definition}
Различные формы записи группы:
\begin{enumerate}
    \item Мультипликативная форма (терминология):\\
    Операция - $"\cdot"$ (умножение);\\
    Нейтральный элемент - единичный $(1)$;\\
    Элемент из аксиомы 3 - обратный ($a^{-1}$ для $a \in G$);
    \item Аддитивная форма (терминология):\\
    Операция - $"+"$ (сложение);\\
    Нейтральный элемент - нулевой $(0)$;\\
    Элемент из аксиомы 3 - противоположный ($-a$ для $a \in G$);
\end{enumerate}
\begin{definition}
    Если $G$ - группа и $\forall a, b \in G \ a \cdot b = b \cdot a$, то $G$ - абелева (коммутативная) группа. 
\end{definition}
\begin{remark}
    Обычно для обозначения абелевых групп будем использовать аддитивную форму записи, для иных - мультипликативную.
\end{remark}
\begin{subtheorem}[Простейшие свойства групп] \tab
    \begin{enumerate}
        \item Единичный элемент единственный;
        \item $\forall a \in G$ обратный к $a$ элемент единственный;
        \item $(ab)^{-1} = b^{-1}a^{-1}$;
        \item Если $a, b \in G$, то решение уравнения $ax = b \ (xa = b)$ единственно.
    \end{enumerate}
\end{subtheorem}
\begin{proof} \tab
    \begin{enumerate}
        \item (От противного) Допустим, что $\exists \ e_1,e_2 \in A$ - единичные. Тогда $e_1 = e_1 * e_2 = e_2$ по определению единичного элемента.
        \item Допустим $\exists \ b_1,b_2$ - обратные к $a$ элементы: $b_1 \not = b_2$ \\
        В силу ассоциативности:
        $$b_1 * (a * b_2) = (b_1 * a) * b_2$$
        $$b_1 * e = e * b_2$$ 
        $$b_1 = b_2$$ 
        \item $abb^{-1}a^{-1} = aea^{-1} = e$;\\
        $b^{-1}a^{-1}ab = b^{-1}eb = e \Longrightarrow (ab)^{-1} = b^{-1}a^{-1}$
        \item $ax = b \Longleftrightarrow a^{-1}ax = a^{-1}b \Longleftrightarrow x = a^{-1}b$;\\
        $xa = b \Longleftrightarrow xaa^{-1} = ba^{-1} \Longleftrightarrow x = ba^{-1}$; 
    \end{enumerate} 
\end{proof}
\begin{definition}
    Мощность множества $G$ называется порядком группы $G$.\\ Обозначается $|G|$.\\
    Если $|G| < \infty$, то группа называется конечной, иначе бесконечной.
\end{definition}
\begin{examples} \tab
    \begin{enumerate}
        \item $(\Z, +), (\Z_n, +)$;
        \item $GL_n(F)$ - группа невырожденных матриц порядка $n$ с коэффициентами из поля $F$;
        \item Пусть $\Omega$ - множество. Преобразованиями $\Omega$ назовём биекции $f: \Omega \rightarrow \Omega$.\\
        $S(\Omega)$ - множество всех преобразований $\Omega$ - образует группу относительно композиции.\\
        Если $\Omega = \{1,...,n\}$, то $S(n) = S_n$ - группа подстановок.
        \item Если $G = \{a_1,...,a_n\}$ - конечная группа, то её можно задать с помощью таблицы умножения (таблицы Кэли).\\
        Например, для $Z_2 = \{0,1\}$:
        $$\begin{tabular}{c|c|c}
        \null & 0 & 1\\ \hline
        0 & 0 & 1 \\ \hline
        1 & 1 & 0
        \end{tabular}$$ 
        \item Группа кватернионов: $Q_8 = \{\pm 1, \pm i, \pm j, \pm k\}$\\
        Таблица Кэли для кватернионов:
        $$\begin{tabular}{c|c|c|c|c}
        \null & 1 & i & j & k \\ \hline
        1 & 1 & i & j & k \\ \hline
        i & i & -1 & k & -j \\ \hline
        j & j & -k & -1 & i \\ \hline
        k & k & j & i & -1
        \end{tabular}$$ 
    \end{enumerate}
\end{examples}
\begin{definition}
    Подмножество $H \subseteq G$ называется подгруппой группы $G$, если:
    \begin{enumerate}
        \item $\forall a, b \in H \ ab \in H$;
        \item $\forall a \in H \ a^{-1} \in H$;
        \item $1 \in H$ (можно заменить на $H \neq \varnothing$)
    \end{enumerate}
    Обозначается $H \leqslant G$.
\end{definition}
\begin{subtheorem}
    Подгруппа $H$ группы $G$ является группой относительно бинарной операции группы $G$.
\end{subtheorem}
\begin{examples} \tab
    \begin{enumerate}
        \item $\Z \leqslant \Q \leqslant \R \leqslant \CC$ ($\N \nleqslant \Z$, т.к. не группа);
        \item $GL_n(F) \geqslant SL_n(F) = \{A \in GL_n(F)| \det A = 1\}$ - унимодулярная группа.
        \item $GL_n(F) \geqslant O_n(F) \geqslant SO_n(F)$ ($O_n(F)$ - ортогональная группа, $SO_n(F)$ - специальная ортогональная группа);
        \item $GL_n(F) \geqslant$ группа строго треугольных матриц.
    \end{enumerate}
\end{examples}
\begin{definition}
    Любая подгруппа группы $S(\Omega)$ называется группой преобразований множества $\Omega$.
\end{definition}
\begin{examples} \tab
    \begin{enumerate}
        \item $GL(V) \ (\leqslant S(V))$ - группа всех невырожденных линейных операторов векторного пространства $V$;
        \item $Aff(\A)$ - группа всех невырожденных аффинных преобразований аффинного пространства $\A$;
        \item $\E^2$ - аффинно-евклидово двумерное пространство.\\
        $\textup{Isom} \ \E^2$ - группа изометрий (движений) на $\E^2$.\\
        $\textup{Isom} \ \E^2 \geqslant O_2 \geqslant SO_2$, где $O_2$ - группа движений, сохраняющих точку $O$, $SO_2$ - группа поворотов вокруг точки $O$.
        \item $T \subseteq \E^2$ - некоторая фигура.\\
        $\textup{Sym}\ T = \{f \in \textup{Isom} \ \E^2 \ | \ f(T) = T\}$ - группа симметрий фигуры $T$.
        \begin{itemize}
            \item Если $T$ - окружность с центром в точке $O$, то $\textup{Sym}\ T = O_2$;
            \item Если $T$ - правильный $n$-угольник с центром в точке $O$, то $\textup{Sym}\ T = D_n$ - группа Диэдра.\\
            $|D_n| = 2n$ - $n$ поворотов и $n$ симметрий. 
        \end{itemize}
    \end{enumerate}
\end{examples}
\begin{definition}
    Пусть $(G_1, *, e_1), (G_2, \circ ,e_2)$ - группы. Отображение $\phi: G_1 \rightarrow G_2$ - изоморфизм, если
    \begin{enumerate}
        \item $\phi$ - биекция;
        \item $\forall a, b \in G_1 \ \phi(a * b) = \phi(a) * \phi(b)$
    \end{enumerate}
    Если между $G_1$ и $G_2$ существует изоморфизм, то $G_1$ и $G_2$ называются изоморфными. Обозначается $G_1 \cong  G_2$.
\end{definition}
\begin{example}
    $D_3 \cong S_3$.
\end{example}
\begin{proof}
    $D_3$ - группа движений, переводящая равносторонний треугольник в себя. Если пронумеровать вершины изначального треугольника, то каждый элемент группы $D_3$ будет соответствовать подстановке, переводящей старый порядок вершин в новый. Определение изоморфизма проверяется очевидно.
\end{proof}
\begin{subtheorem}
    Изоморфность групп - отношение эквивалентности на множестве групп.
\end{subtheorem}
\begin{subtheorem}[Свойства изоморфизмов] \tab
    \begin{enumerate}
        \item $\phi(e_1) = e_2$;
        \item $\phi(a^{-1}) = (\phi(a))^{-1}$;
        \item $G_1 \cong  G_2 \Longrightarrow |G_1| = |G_2|$.
        \begin{remark}
            Обратное утверждение неверно (например, $S_3 \ncong \Z_6$).
        \end{remark}
    \end{enumerate}
\end{subtheorem}
\begin{example}
    $SO_2 \cong (U, \cdot)$, где $U = \{z \in \CC : |z| = 1\}$.
\end{example}
\begin{definition}
    Пусть $(G, \cdot, e)$ - группа, $k \in \Z, g \in G$.\\
    Мультипликативный термин - элемент $g$ в степени $k$:
    $$g^k =
    \begin{cases} \undermat{k}{g \cdot g \cdot ... \cdot g}, k > 0 \\
        \\
        \undermat{-k}{g^{-1} \cdot g^{-1} \cdot ... \cdot g^{-1}}, k < 0 \\
        \\
        e, k = 0
    \end{cases}$$
\end{definition}
\begin{definition}
    Пусть $(G, +, e)$ - группа, $k \in \Z, g \in G$.\\
    Аддитивный термин - кратное элемента $g$:
    $$kg =
    \begin{cases} \undermat{k}{g + g + ... + g}, k > 0 \\
        \\
        \undermat{-k}{(-g) + (-g) + ... + (-g)}, k < 0 \\
        \\
        e, k = 0
    \end{cases}$$
\end{definition}
\begin{subtheorem}[Свойства ($k, m \in \Z, g \in G$)] \tab
    \begin{enumerate}
        \item $g^k \cdot g^m = g^{k+m}$;
        \item $(g^k)^m = g^{km}$;
        \item $(g^k)^{-1} = g^{-k}$. 
    \end{enumerate}    
\end{subtheorem}
\begin{subtheorem}
    Множество всех элементов $g^k$, где $k \in \Z$, $g \in G$, образует подгруппу в $G$. Обозначается $\langle g \rangle = \{e, g, g^{-1}, g^2, g^{-2}, ...\}$ .
\end{subtheorem}
\begin{definition}
    $\langle g \rangle$ - циклическая подгруппа. порождённая элементом $g$.
\end{definition}
\begin{examples} \tab
    \begin{enumerate}
        \item $G = \Z: \langle 2 \rangle = 2\Z$ - чётные целые числа;
        \item $G = \Z_6: \langle 2 \rangle = \{0, 2, 4\}$;
        \item $G = \CC: \langle i \rangle = \{\pm 1, \pm i\}$ 
    \end{enumerate}
\end{examples}
Пусть $(G, \cdot, e)$ - группа, $g \in G$.
Если $\forall k, m \in \Z: k \neq m \Longrightarrow g^k \neq g^m$, то $\langle g \rangle$ - бесконечная (элемент $g$ имеет бесконечный порядок).\\
Если $\exists k, m \in \Z: k \neq m, g^k = g^m \Longrightarrow g^{k-m} = e \Longrightarrow$ существует наименьшее $n \in \N$ такое, что $g^n = e$ (элемент $g$ имеет порядок $n$)  
\begin{definition}
    Порядком элемента $g \in G$ называется наименьшее натуральное число $n$ такое, что $g^n = e$, если такое существует. Иначе говорят, что элемент $g$ имеет бесконечный порядок. Обозначается $\textup{ord} \ g$.
\end{definition}
\begin{examples} \tab
    \begin{enumerate}
        \item $G = \Z: \textup{ord } 2 = \infty$;
        \item $G = \Z_{12}: \textup{ord } 2 = 6$;
        \item $G = \CC^*: \textup{ord } 2 = \infty$\\
        ($\CC^*$ - мультипликативная группа поля, $\CC\setminus\{0\}$ относительно умножения).
    \end{enumerate}
\end{examples}
\begin{subtheoremnum}[Свойства элементов конечного порядка]\tab
    \begin{enumerate}
        \item $g^m = e \Longleftrightarrow \textup{ord } g \ | \ m$;
        \item $g^m = g^l \Longleftrightarrow k \equiv l (\textup{mod ord } g)$
    \end{enumerate}
\end{subtheoremnum}
\begin{proof} \tab
    \begin{enumerate}
        \item Разделим $m$ на $n = \textup{ord } g$ с остатком: $m = nq + r$, где $0 \leqslant r < n$. Тогда:
        $$e = g^m = (g^n)^q \cdot g^r = g^r \Longrightarrow r = 0$$
        так как $r < n$, где $n$ - минимальное натуральное число такое, что $g^n = 0$.
        \item Следует из 1. 
    \end{enumerate}
\end{proof}
\begin{consequense}
    $\textup{ord } g = |\langle g \rangle|$
\end{consequense}
\begin{proof}
    Если $\textup{ord } g = \infty: \ \forall k \neq l \ g^k \neq g^l \Longrightarrow$ подгруппа $\langle g \rangle = \{e, g^{\pm 1}, g^{\pm 2},...\}$ бесконечна.\\
    Если $\textup{ord } g = n: \langle g \rangle = \{e, g^1, ... g^{n-1}\}$ - все эти элементы различны из пункта 2 утверждения, а других нет по определению порядка.
\end{proof}
\begin{examples} \tab
    \begin{enumerate}
        \item $i \in \CC^*$ - $\textup{ord } i = 4$;
        \item $\sigma \in S_n$:\\
        Если $\sigma = (i_1,...,i_k)$ - цикл длины $k$, то $\textup{ord } \sigma = k$.\\
        Так как любая подстановка раскладывается в произведение независимых циклов и независимые циклы коммутируют, если $\sigma = \tau_1...\tau_n$, где $\tau_i$ - независимые циклы, то верно: $\textup{ord } \sigma = $ НОК $\{|\tau_1|,...,|\tau_n|\}$.\\
        Например, $\sigma = (23)(145) \Longrightarrow \textup{ord } \sigma = 6$.
    \end{enumerate}
\end{examples}
\begin{subtheoremnum}
    Пусть $n = \textup{ord } g$. Тогда $\textup{ord } g^k = \frac{n}{\text{НОД}(n, k)}$.
\end{subtheoremnum}
\begin{proof}
    Пусть $\textup{ord } g^k = m$. Из утверждения 1: $g^{mk} = e \Longleftrightarrow n | mk$, откуда $\frac{n}{\text{НОД}(n, k)} | m$, т.е. $m \geqslant \frac{n}{\text{НОД}(n, k)}$. Очевидно, что при $m = \frac{n}{\text{НОД}(n, k)} \ n | mk$. 
\end{proof}
\begin{definition}
    Множество $S \subseteq G$ называется порождающим множеством для группы $G$, если $\forall g \in G \ \exists s_1,...,s_k \in S: \ g = s_1^{\epsilon_1}...s_k^{\epsilon_k}$, где $\epsilon_i = \pm 1$ ($s_i$ не обязательно различны).\\
    При этом говорят, что $G$ порождается множеством $S$.\\
    Если $\exists$ конечное множество $S$ такое, что $S$ порождает $G$, то $G$ называется конечно порождённой, и бесконечно порождённой иначе.\\
    Обозначается $\langle S \rangle = \{s_1^{\epsilon_1}...s_k^{\epsilon_k} | \epsilon_i = \pm 1\}$ - группа, порождённая $S$.
\end{definition}
\begin{examples} \tab
    \begin{enumerate}
        \item $S_n = \langle \text{все транспозиции} \rangle$;
        \item $GL_n(F) = \langle \text{все элементарные матрицы} \rangle$
        \item $Q_8 = \langle i, j \rangle$;
        \item $D_n = \langle \alpha, s\rangle$, где $\alpha$ - поворот на $\frac{2\pi}{n}$, а $s$ - любая из симметрий.
        \item Группа Клейна: $H = \{\textup{id}, a=(12)(34), b=(13)(24), c=(14)(23)\} \leqslant S_4$\\
        Это группа симметрий прямоугольника, не являющегося квадратом: $a, c$ - симметрии относительно средних линий, $b$ - поворот на $\pi$ вокруг центра. 
        Таблица Кэли для группы Клейна:
        $$\begin{tabular}{c|c|c|c|c}
        \null & e & a & b & c \\ \hline
        e & e & a & b & c \\ \hline
        a & a & e & c & b \\ \hline
        b & b & c & e & a \\ \hline
        c & c & b & a & e
        \end{tabular}$$ 
        Отсюда $\{e, a, b, c\} = \langle a, b \rangle$.
        \item $\Q$ - бесконечно порождённая.
    \end{enumerate}
\end{examples}
\subsection{Циклические группы}
\begin{definition}
    Группа $G$ называется циклической, если $G$ порождается одним элементом, т.е. $\exists g \in G: \forall h \in G \ \exists k \in \Z: h = g^k$. Элемент $g$ также называется образующим элементом группы $G$. 
\end{definition}
\begin{examples}\tab
    \begin{enumerate}
        \item $\Z = \langle 1 \rangle = \langle -1 \rangle$, $\Z_n = \langle 1 \rangle$;
        \item $U_n$ - множество всех комплексных корней степени $n$ из 1.\\
        $U_n$ - группа относительно умножения, причём $U_n = \langle \cos \frac{2\pi}{n} + i \sin \frac{2\pi}{n} \rangle$.
    \end{enumerate}    
\end{examples}
\begin{subtheoremnum}
    Если $G = \langle g \rangle$, то $|G| = \textup{ord } g$.
\end{subtheoremnum}
\begin{proof}
    Очевидно из определения порождающего множества.
\end{proof}
\begin{remark}
    Далее циклическую группу порядка $n$ обозначаем $\langle g \rangle_n$
\end{remark}
\begin{subtheoremnum}
    Пусть $G = \langle g \rangle_n$. Тогда $G = \langle g^k \rangle \Longleftrightarrow \textup{НОД}(k, n) = 1$.
\end{subtheoremnum}
\begin{proof}
    Из утверждения 3 $|G| = \textup{ord } g$. Тогда:
    $$G = \langle g^k \rangle \Longleftrightarrow \textup{ord } g^k = \frac{n}{\text{НОД}(n, k)} = n \Longleftrightarrow \text{НОД}(n, k) = 1$$ 
\end{proof}
\begin{theoremnum}[Классификация циклических групп] \tab
    \begin{enumerate}
        \item Если циклическая группа $G$ бесконечна, то $G \cong \Z$;
        \item Если циклическая группа $G$ конечна и имеет порядок $n$, то $G \cong \Z_n$.
    \end{enumerate}
\end{theoremnum}
\begin{proof}\tab
    \begin{enumerate}
        \item Пусть $\textup{ord } g = \infty, \forall h \in g \ \exists k \in \Z: h = g^k$\\
        Рассмотрим отображение $\phi: G \rightarrow \Z$ такого вида: $\phi: g^k \rightarrowtail k$. Очевидно, что $\phi$ - сюръекция (в $k\in \Z$ перешёл $g^k \in G$).\\
        $\phi(g^k) = \phi(g^m) \Longrightarrow k = m \Longrightarrow g^k = g^m$ - отсюда $\phi$ - инъекция.\\
        Проверим сохранение операции:
        \[\phi(g^k \cdot g^m) = \phi(g^{k+m}) = k+m = \phi(g^k) + \phi(g^m)\]
        Отсюда $\phi$ - изоморфизм.
        \item Пусть $\textup{ord } g = n$. Рассмотрим отображение $\phi: \Z_n \rightarrow G$ такого вида: $\phi: k \rightarrowtail g^k$. Очевидно, что $\phi$ - сюръекция (в $g^k\in G$ перешёл $k \in \Z_n$).\\
        $k \equiv m (\textup{mod } n) \Longleftrightarrow g^k = g^m$ - отсюда $\phi$ - инъекция.\\
        Сохранение операции - аналогично пункту 1.\\ 
        Отсюда $\phi$ - изоморфизм.
    \end{enumerate}
\end{proof}
\begin{consequense}
    Если $G_1, G_2$ - циклические группы, то $G_1 \cong G_2 \Longleftrightarrow |G_1| = |G_2|$.
\end{consequense}
\begin{proof} \tab\\
    $\Longrightarrow : \ \ $ верно всегда;\\
    $\Longleftarrow : \ \ $ из теоремы: если $G_1$ бесконечна, то $G_1 \cong \Z \cong G_2$, иначе $G_1 \cong \Z_n \cong G_2$, где $n = |G_1| = |G_2|$.
\end{proof}
\begin{theoremnum} \tab
    \begin{enumerate}
        \item Любая подгруппа циклической группы является циклической.
        \item Подгруппы циклической группы $G$ порядка $n$ находятся во взаимном соответствии с делителями $n$, т.е.
        \[\forall H \leqslant G \ |H| \ |\  n \text{ и } \forall d | n \ \exists! \ H \leqslant G : |H| = d\]
        \item Подгруппы группы $\Z$ исчерпываются группами $k\Z =  \langle k \rangle$, где $k \in \N \cup \{0\}$.
    \end{enumerate}
\end{theoremnum}
\begin{proof} \tab
    \begin{enumerate}
        \item Пусть $G = \langle g \rangle, H \leqslant G$. Если $H = \{e\}$, то $H = \langle e \rangle$.\\
        При $H \neq \{e\} : \forall h \in H \ \exists k \in \Z : h = g^k$. Так как $g^k \in H \Longrightarrow g^{-k} \in H$ и в $H$ есть элемент, отличный от $e$, $\exists$ наименьшее $k \in \N: g^k \in H$.\\
        Докажем, что $H = \langle g^k \rangle$. Рассмотрим произвольный $g^m \in H$. Разделим $m$ на $k$ с остатком: $m = kq + r, 0 \leqslant r < k$. Тогда:
        \[g^m = (g^k)^q\cdot g^r \Longrightarrow g^r = (g^k)^{-q}\cdot g^m \Longrightarrow r = 0, \text{ т.к. k - наименьшее} \in \N\]
        \item $G =  \langle g \rangle_n, H \leqslant G \Longrightarrow_{(1)} H =  \langle g^k \rangle$.\\
        Так как $g^n = e \in H$, то в силу рассуждений пункта 1 при $m = n$ получаем $k | n \Longrightarrow n = kq$.\\
        Отсюда $H = \{e, g^k, g^{2k},...,g^{(q-1)k}\} \Longrightarrow |H| = q$, где $q | n$.\\
        Обратно, $\forall d | n \ \exists! H =  \langle g^{\frac{n}{d}} \rangle$ (в силу описания выше других подгрупп такого порядка нет).
        \item Из пункта 1 в аддитивной форме получаем, что $H \leqslant \Z =  \langle 1 \rangle \Longrightarrow H =  \langle k\cdot 1 \rangle$
    \end{enumerate} 
\end{proof}
\begin{consequense}
    В циклической группе простого порядка существуют ровно две подгруппы - тривиальная и сама группа. 
\end{consequense}
\begin{examples}\tab
    \begin{enumerate}
        \item $H \leqslant \Z_5 \Longrightarrow H = \{0\}, H = \Z_5$;
        \item $H \leqslant \Z_6 \Longrightarrow H = \{0\}, H = \langle 2 \rangle, H = \langle 3 \rangle, H = \Z_6$. 
    \end{enumerate}
\end{examples}
\subsection{Смежные классы}
\begin{definition}
    Пусть $(G, \cdot, e)$ - произвольная группа, $H \leqslant G, g \in G$.\\
    Рассмотрим множества:
    \[gH = \{gh | h \in H\} \text{ - левый смежный класс G по H с представителем g}\]
    \[Hg = \{hg | h \in H\} \text{ - правый смежный класс G по H с представителем g}\]
\end{definition}
\begin{subtheorem}[Свойства смежных классов] \tab
    \begin{enumerate}
        \item $\forall a \in G \ a \in aH$;
        \item если $a \in bH$, то $bH = aH$; в частности, любые два смежных класса либо не пересекаются, либо совпадают.
        \item $aH = bH \Longleftrightarrow b^{-1}a \in H$;\\
        (Верны аналогичные утверждения для правых смежных классов)
    \end{enumerate}
\end{subtheorem}
\begin{proof} \tab
    \begin{enumerate}
        \item Очевидно;
        \item $a \in bH \Longrightarrow \exists h \in H: a = bh \Longrightarrow \forall \tilde{h} \in H \ a\tilde{h} = bh\tilde{h} \in bH \Longrightarrow aH \subseteq bH$.
        Аналогично $bH \subseteq aH \Longrightarrow aH = bH$.
        \item $\Longrightarrow: \ aH = bH \Longrightarrow a \in bH (a \in aH) \Longrightarrow \exists h \in H: a = bh \Longrightarrow b^{-1}a = h \in H$\\
        $\Longleftarrow: \ b^{-1}a = h \in H \Longrightarrow a = bh \Longrightarrow aH = bH$ по пункту 2.
    \end{enumerate}
\end{proof}