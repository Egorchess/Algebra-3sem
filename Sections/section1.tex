\section{Группы}
\subsection{Основные понятия}
\begin{definition}
    Пусть $G$ - множество. Бинарной операцией на $G$ называется отображение $*: G \times G \rightarrow G$.
\end{definition}
\begin{definition}
    Множество $G$ с бинарной операцией $*$ называется группой, если выполнены следующие аксиомы:
    \begin{enumerate}
        \item $\forall a, b, c \in G \ \ a * (b * c) = (a * b) * c$;
        \item $\exists e \in G: \forall a \in G \ \ a * e = e * a = a$;
        \item $\forall a \in G \ \exists b \in G: a * b = b * a = e$
    \end{enumerate}
\end{definition}
Различные формы записи группы:
\begin{enumerate}
    \item Мультипликативная форма (терминология):\\
    Операция - $"\cdot"$ (умножение);\\
    Нейтральный элемент - единичный $(1)$;\\
    Элемент из аксиомы 3 - обратный ($a^{-1}$ для $a \in G$);
    \item Аддитивная форма (терминология):\\
    Операция - $"+"$ (сложение);\\
    Нейтральный элемент - нулевой $(0)$;\\
    Элемент из аксиомы 3 - противоположный ($-a$ для $a \in G$);
\end{enumerate}
\begin{definition}
    Если $G$ - группа и $\forall a, b \in G \ a \cdot b = b \cdot a$, то $G$ - абелева (коммутативная) группа. 
\end{definition}
\begin{remark}
    Обычно для обозначения абелевых групп будем использовать аддитивную форму записи, для иных - мультипликативную.
\end{remark}
\begin{subtheorem}[Простейшие свойства групп] \tab
    \begin{enumerate}
        \item Единичный элемент единственный;
        \item $\forall a \in G$ обратный к $a$ элемент единственный;
        \item $(ab)^{-1} = b^{-1}a^{-1}$;
        \item Если $a, b \in G$, то решение уравнения $ax = b \ (xa = b)$ единственно.
    \end{enumerate}
\end{subtheorem}
\begin{proof} \tab
    \begin{enumerate}
        \item (От противного) Допустим, что $\exists \ e_1,e_2 \in A$ - единичные. Тогда $e_1 = e_1 * e_2 = e_2$ по определению единичного элемента.
        \item Допустим $\exists \ b_1,b_2$ - обратные к $a$ элементы: $b_1 \not = b_2$ \\
        В силу ассоциативности:
        $$b_1 * (a * b_2) = (b_1 * a) * b_2$$
        $$b_1 * e = e * b_2$$ 
        $$b_1 = b_2$$ 
        \item $abb^{-1}a^{-1} = aea^{-1} = e$;\\
        $b^{-1}a^{-1}ab = b^{-1}eb = e \Longrightarrow (ab)^{-1} = b^{-1}a^{-1}$
        \item $ax = b \Longleftrightarrow a^{-1}ax = a^{-1}b \Longleftrightarrow x = a^{-1}b$;\\
        $xa = b \Longleftrightarrow xaa^{-1} = ba^{-1} \Longleftrightarrow x = ba^{-1}$; 
    \end{enumerate} 
\end{proof}
\begin{definition}
    Мощность множества $G$ называется порядком группы $G$.\\ Обозначается $|G|$.\\
    Если $|G| < \infty$, то группа называется конечной, иначе бесконечной.
\end{definition}
\begin{examples} \tab
    \begin{enumerate}
        \item $(\Z, +), (\Z_n, +)$;
        \item $GL_n(F)$ - группа невырожденных матриц порядка $n$ с коэффициентами из поля $F$;
        \item Пусть $\Omega$ - множество. Преобразованиями $\Omega$ назовём биекции $f: \Omega \rightarrow \Omega$.\\
        $S(\Omega)$ - множество всех преобразований $\Omega$ - образует группу относительно композиции.\\
        Если $\Omega = \{1,...,n\}$, то $S(n) = S_n$ - группа подстановок.
        \item Если $G = \{a_1,...,a_n\}$ - конечная группа, то её можно задать с помощью таблицы умножения (таблицы Кэли).\\
        Например, для $Z_2 = \{0,1\}$:
        $$\begin{tabular}{c|c|c}
        \null & 0 & 1\\ \hline
        0 & 0 & 1 \\ \hline
        1 & 1 & 0
        \end{tabular}$$ 
        \item Группа кватернионов: $Q_8 = \{\pm 1, \pm i, \pm j, \pm k\}$\\
        Таблица Кэли для кватернионов:
        $$\begin{tabular}{c|c|c|c|c}
        \null & 1 & i & j & k \\ \hline
        1 & 1 & i & j & k \\ \hline
        i & i & -1 & k & -j \\ \hline
        j & j & -k & -1 & i \\ \hline
        k & k & j & i & -1
        \end{tabular}$$ 
    \end{enumerate}
\end{examples}
\begin{definition}
    Подмножество $H \subseteq G$ называется подгруппой группы $G$, если:
    \begin{enumerate}
        \item $\forall a, b \in H \ ab \in H$;
        \item $\forall a \in H \ a^{-1} \in H$;
        \item $1 \in H$ (можно заменить на $H \neq \varnothing$)
    \end{enumerate}
    Обозначается $H \leq G$.
\end{definition}
\begin{subtheorem}
    Подгруппа $H$ группы $G$ является группой относительно бинарной операции группы $G$.
\end{subtheorem}
\begin{examples} \tab
    \begin{enumerate}
        \item $\Z \leq \Q \leq \R \leq \CC$ ($\N \nleq \Z$, т.к. не группа);
        \item $GL_n(F) \geq SL_n(F) = \{A \in GL_n(F)| \det A = 1\}$ - унимодулярная группа.
        \item $GL_n(F) \geq O_n(F) \geq SO_n(F)$ ($O_n(F)$ - ортогональная группа, $SO_n(F)$ - специальная ортогональная группа);
        \item $GL_n(F) \geq$ группа строго треугольных матриц.
    \end{enumerate}
\end{examples}
\begin{definition}
    Любая подгруппа группы $S(\Omega)$ называется группой преобразований множества $\Omega$.
\end{definition}
\begin{examples} \tab
    \begin{enumerate}
        \item $GL(V) \ (\leq S(V))$ - группа всех невырожденных линейных операторов векторного пространства $V$;
        \item $Aff(\A)$ - группа всех невырожденных аффинных преобразований аффинного пространства $\A$;
        \item $\E^2$ - аффинно-евклидово двумерное пространство.\\
        $\textup{Isom} \ \E^2$ - группа изометрий (движений) на $\E^2$.\\
        $\textup{Isom} \ \E^2 \geq O_2 \geq SO_2$, где $O_2$ - группа движений, сохраняющих точку $O$, $SO_2$ - группа поворотов вокруг точки $O$.
        \item $T \subseteq \E^2$ - некоторая фигура.\\
        $\textup{Sym}\ T = \{f \in \textup{Isom} \ \E^2 \ | \ f(T) = T\}$ - группа симметрий фигуры $T$.
        \begin{itemize}
            \item Если $T$ - окружность с центром в точке $O$, то $\textup{Sym}\ T = O_2$;
            \item Если $T$ - правильный $n$-угольник с центром в точке $O$, то $\textup{Sym}\ T = D_n$ - группа Диэдра.\\
            $|D_n| = 2n$ - $n$ поворотов и $n$ симметрий. 
        \end{itemize}
    \end{enumerate}
\end{examples}
\begin{definition}
    Пусть $(G_1, *, e_1), (G_2, \circ ,e_2)$ - группы. Отображение $\phi: G_1 \rightarrow G_2$ - изоморфизм, если
    \begin{enumerate}
        \item $\phi$ - биекция;
        \item $\forall a, b \in G_1 \ \phi(a * b) = \phi(a) * \phi(b)$
    \end{enumerate}
    Если между $G_1$ и $G_2$ существует изоморфизм, то $G_1$ и $G_2$ называются изоморфными. Обозначается $G_1 \simeq  G_2$.
\end{definition}
\begin{example}
    $D_3 \simeq S_3$.
\end{example}
\begin{proof}
    $D_3$ - группа движений, переводящая равносторонний треугольник в себя. Если пронумеровать вершины изначального треугольника, то каждый элемент группы $D_3$ будет соответствовать подстановке, переводящей старый порядок вершин в новый. Определение изоморфизма проверяется очевидно.
\end{proof}
\begin{subtheorem}
    Изоморфность групп - отношение эквивалентности на множестве групп.
\end{subtheorem}
\begin{subtheorem}[Свойства изоморфизмов] \tab
    \begin{enumerate}
        \item $\phi(e_1) = e_2$;
        \item $\phi(a^{-1}) = (\phi(a))^{-1}$;
        \item $G_1 \simeq  G_2 \Longrightarrow |G_1| = |G_2|$.
        \begin{remark}
            Обратное утверждение неверно (например, $S_3 \ncong \Z_6$).
        \end{remark}
    \end{enumerate}
\end{subtheorem}
\begin{example}
    $SO_2 \simeq (U, \cdot)$, где $U = \{z \in \CC : |z| = 1\}$.
\end{example}
\begin{definition}
    Пусть $(G, \cdot, e)$ - группа, $k \in \Z, g \in G$.\\
    Мультипликативный термин - элемент $g$ в степени $k$:
    $$g^k =
    \begin{cases} \undermat{k}{g \cdot g \cdot ... \cdot g}, k > 0 \\
        \\
        \undermat{-k}{g^{-1} \cdot g^{-1} \cdot ... \cdot g^{-1}}, k < 0 \\
        \\
        e, k = 0
    \end{cases}$$
\end{definition}
\begin{definition}
    Пусть $(G, +, e)$ - группа, $k \in \Z, g \in G$.\\
    Аддитивный термин - кратное элемента $g$:
    $$kg =
    \begin{cases} \undermat{k}{g + g + ... + g}, k > 0 \\
        \\
        \undermat{-k}{(-g) + (-g) + ... + (-g)}, k < 0 \\
        \\
        e, k = 0
    \end{cases}$$
\end{definition}
\begin{subtheorem}[Свойства ($k, m \in \Z, g \in G$)] \tab
    \begin{enumerate}
        \item $g^k \cdot g^m = g^{k+m}$;
        \item $(g^k)^m = g^{km}$;
        \item $(g^k)^{-1} = g^{-k}$. 
    \end{enumerate}    
\end{subtheorem}
\begin{subtheorem}
    Множество всех элементов $g^k$, где $k \in \Z$, $g \in G$, образует подгруппу в $G$. Обозначается $\langle g \rangle = \{e, g, g^{-1}, g^2, g^{-2}, ...\}$ .
\end{subtheorem}
\begin{definition}
    $\langle g \rangle$ - циклическая подгруппа. порождённая элементом $g$.
\end{definition}
\begin{examples} \tab
    \begin{enumerate}
        \item $G = \Z: \langle 2 \rangle = 2\Z$ - чётные целые числа;
        \item $G = \Z_6: \langle 2 \rangle = \{0, 2, 4\}$;
        \item $G = \CC: \langle i \rangle = \{\pm 1, \pm i\}$ 
    \end{enumerate}
\end{examples}
Пусть $(G, \cdot, e)$ - группа, $g \in G$.
Если $\forall k, m \in \Z: k \neq m \Longrightarrow g^k \neq g^m$, то $\langle g \rangle$ - бесконечная (элемент $g$ имеет бесконечный порядок).\\
Если $\exists k, m \in \Z: k \neq m, g^k = g^m \Longrightarrow g^{k-m} = e \Longrightarrow$ существует наименьшее $n \in \N$ такое, что $g^n = e$ (элемент $g$ имеет порядок $n$)  
\begin{definition}
    Порядком элемента $g \in G$ называется наименьшее натуральное число $n$ такое, что $g^n = e$, если такое существует. Иначе говорят, что элемент $g$ имеет бесконечный порядок. Обозначается $\textup{ord} \ g$.
\end{definition}
\begin{examples} \tab
    \begin{enumerate}
        \item $G = \Z: \textup{ord } 2 = \infty$;
        \item $G = \Z_{12}: \textup{ord } 2 = 6$;
        \item $G = \CC^*: \textup{ord } 2 = \infty$\\
        ($\CC^*$ - мультипликативная группа поля, $\CC\setminus\{0\}$ относительно умножения).
    \end{enumerate}
\end{examples}
\begin{subtheoremnum}[Свойства элементов конечного порядка]\tab
    \begin{enumerate}
        \item $g^m = e \Longleftrightarrow \textup{ord } g \ | \ m$;
        \item $g^m = g^l \Longleftrightarrow k \equiv l (\textup{mod ord } g)$
    \end{enumerate}
\end{subtheoremnum}
\begin{proof} \tab
    \begin{enumerate}
        \item Разделим $m$ на $n = \textup{ord } g$ с остатком: $m = nq + r$, где $0 \leqslant r < n$. Тогда:
        $$e = g^m = (g^n)^q \cdot g^r = g^r \Longrightarrow r = 0$$
        так как $r < n$, где $n$ - минимальное натуральное число такое, что $g^n = 0$.
        \item Следует из 1. 
    \end{enumerate}
\end{proof}
\begin{consequense}
    $\textup{ord } g = |\langle g \rangle|$
\end{consequense}
\begin{proof}
    Если $\textup{ord } g = \infty: \ \forall k \neq l \ g^k \neq g^l \Longrightarrow$ подгруппа $\langle g \rangle = \{e, g^{\pm 1}, g^{\pm 2},...\}$ бесконечна.\\
    Если $\textup{ord } g = n: \langle g \rangle = \{e, g^1, ... g^{n-1}\}$ - все эти элементы различны из пункта 2 утверждения, а других нет по определению порядка.
\end{proof}
\begin{examples} \tab
    \begin{enumerate}
        \item $i \in \CC^*$ - $\textup{ord } i = 4$;
        \item $\sigma \in S_n$:\\
        Если $\sigma = (i_1,...,i_k)$ - цикл длины $k$, то $\textup{ord } \sigma = k$.\\
        Так как любая подстановка раскладывается в произведение независимых циклов и независимые циклы коммутируют, если $\sigma = \tau_1...\tau_n$, где $\tau_i$ - независимые циклы, то верно: $\textup{ord } \sigma = $ НОК $\{|\tau_1|,...,|\tau_n|\}$.\\
        Например, $\sigma = (23)(145) \Longrightarrow \textup{ord } \sigma = 6$.
    \end{enumerate}
\end{examples}
\begin{subtheoremnum}
    Пусть $n = \textup{ord } g$. Тогда $\textup{ord } g^k = \frac{n}{\text{НОД}(n, k)}$.
\end{subtheoremnum}
\begin{proof}
    Пусть $\textup{ord } g^k = m$. Из утверждения 1: $g^{mk} = e \Longleftrightarrow n | mk$, откуда $\frac{n}{\text{НОД}(n, k)} | m$, т.е. $m \geqslant \frac{n}{\text{НОД}(n, k)}$. Очевидно, что при $m = \frac{n}{\text{НОД}(n, k)} \ n | mk$. 
\end{proof}
\begin{definition}
    Множество $S \subseteq G$ называется порождающим множеством для группы $G$, если $\forall g \in G \ \exists s_1,...,s_k \in S: \ g = s_1^{\epsilon_1}...s_k^{\epsilon_k}$, где $\epsilon_i = \pm 1$ ($s_i$ не обязательно различны).\\
    При этом говорят, что $G$ порождается множеством $S$.\\
    Если $\exists$ конечное множество $S$ такое, что $S$ порождает $G$, то $G$ называется конечно порождённой, и бесконечно порождённой иначе.\\
    Обозначается $\langle S \rangle = \{s_1^{\epsilon_1}...s_k^{\epsilon_k} | \epsilon_i = \pm 1\}$ - группа, порождённая $S$.
\end{definition}
\begin{examples} \tab
    \begin{enumerate}
        \item $S_n = \langle \text{все транспозиции} \rangle$;
        \item $GL_n(F) = \langle \text{все элементарные матрицы} \rangle$
        \item $Q_8 = \langle i, j \rangle$;
        \item $D_n = \langle \alpha, s\rangle$, где $\alpha$ - поворот на $\frac{2\pi}{n}$, а $s$ - любая из симметрий.
        \item Группа Клейна: $H = \{\textup{id}, a=(12)(34), b=(13)(24), c=(14)(23)\} \leq S_4$\\
        Это группа симметрий прямоугольника, не являющегося квадратом: $a, c$ - симметрии относительно средних линий, $b$ - поворот на $\pi$ вокруг центра. 
        Таблица Кэли для группы Клейна:
        $$\begin{tabular}{c|c|c|c|c}
        \null & e & a & b & c \\ \hline
        e & e & a & b & c \\ \hline
        a & a & e & c & b \\ \hline
        b & b & c & e & a \\ \hline
        c & c & b & a & e
        \end{tabular}$$ 
        Отсюда $\{e, a, b, c\} = \langle a, b \rangle$.
        \item $\Q$ - бесконечно порождённая.
    \end{enumerate}
\end{examples}
\subsection{Циклические группы}
\begin{definition}
    Группа $G$ называется циклической, если $G$ порождается одним элементом, т.е. $\exists g \in G: \forall h \in G \ \exists k \in \Z: h = g^k$. Элемент $g$ также называется образующим элементом группы $G$. 
\end{definition}
\begin{examples}\tab
    \begin{enumerate}
        \item $\Z = \langle 1 \rangle = \langle -1 \rangle$, $\Z_n = \langle 1 \rangle$;
        \item $U_n$ - множество всех комплексных корней степени $n$ из 1.\\
        $U_n$ - группа относительно умножения, причём $U_n = \langle \cos \frac{2\pi}{n} + i \sin \frac{2\pi}{n} \rangle$.
    \end{enumerate}    
\end{examples}
\begin{subtheoremnum}
    Если $G = \langle g \rangle$, то $|G| = \textup{ord } g$.
\end{subtheoremnum}
\begin{proof}
    Очевидно из определения порождающего множества.
\end{proof}
\begin{remark}
    Для групп конечного порядка, очевидно, выполняется и обратное утверждение: если $\textup{ord } g = |G| < \infty$, то $G = \langle g \rangle$. 
\end{remark}
    Далее циклическую группу порядка $n$ будем обозначать $\langle g \rangle_n$.
\begin{subtheoremnum}
    Пусть $G = \langle g \rangle_n$. Тогда $G = \langle g^k \rangle \Longleftrightarrow \textup{НОД}(k, n) = 1$.
\end{subtheoremnum}
\begin{proof}
    Из утверждения 3 $|G| = \textup{ord } g$. Тогда:
    $$G = \langle g^k \rangle \Longleftrightarrow \textup{ord } g^k = \frac{n}{\text{НОД}(n, k)} = n \Longleftrightarrow \text{НОД}(n, k) = 1$$ 
\end{proof}
\begin{theoremnum}[Классификация циклических групп] \tab
    \begin{enumerate}
        \item Если циклическая группа $G$ бесконечна, то $G \simeq \Z$;
        \item Если циклическая группа $G$ конечна и имеет порядок $n$, то $G \simeq \Z_n$.
    \end{enumerate}
\end{theoremnum}
\begin{proof}\tab
    \begin{enumerate}
        \item Пусть $\textup{ord } g = \infty, \forall h \in g \ \exists k \in \Z: h = g^k$\\
        Рассмотрим отображение $\phi: G \rightarrow \Z$ такого вида: $\phi: g^k \mapsto k$. Очевидно, что $\phi$ - сюръекция (в $k\in \Z$ перешёл $g^k \in G$).\\
        $\phi(g^k) = \phi(g^m) \Longrightarrow k = m \Longrightarrow g^k = g^m$ - отсюда $\phi$ - инъекция.\\
        Проверим сохранение операции:
        \[\phi(g^k \cdot g^m) = \phi(g^{k+m}) = k+m = \phi(g^k) + \phi(g^m)\]
        Отсюда $\phi$ - изоморфизм.
        \item Пусть $\textup{ord } g = n$. Рассмотрим отображение $\phi: \Z_n \rightarrow G$ такого вида: $\phi: k \mapsto g^k$. Очевидно, что $\phi$ - сюръекция (в $g^k\in G$ перешёл $k \in \Z_n$).\\
        $k \equiv m (\textup{mod } n) \Longleftrightarrow g^k = g^m$ - отсюда $\phi$ - инъекция.\\
        Сохранение операции - аналогично пункту 1.\\ 
        Отсюда $\phi$ - изоморфизм.
    \end{enumerate}
\end{proof}
\begin{consequense}
    Если $G_1, G_2$ - циклические группы, то $G_1 \simeq G_2 \Longleftrightarrow |G_1| = |G_2|$.
\end{consequense}
\begin{proof} \tab\\
    $\Longrightarrow : \ \ $ верно всегда;\\
    $\Longleftarrow : \ \ $ из теоремы: если $G_1$ бесконечна, то $G_1 \simeq \Z \simeq G_2$, иначе $G_1 \simeq \Z_n \simeq G_2$, где $n = |G_1| = |G_2|$.
\end{proof}
\begin{theoremnum} \tab
    \begin{enumerate}
        \item Любая подгруппа циклической группы является циклической.
        \item Подгруппы циклической группы $G$ порядка $n$ находятся во взаимно однозначном соответствии с делителями $n$, т.е.
        \[\forall H \leq G \ |H| \ |\  n \text{ и } \forall d | n \ \exists! \ H \leq G : |H| = d\]
        \item Подгруппы группы $\Z$ исчерпываются группами $k\Z =  \langle k \rangle$, где $k \in \N \cup \{0\}$.
    \end{enumerate}
\end{theoremnum}
\begin{proof} \tab
    \begin{enumerate}
        \item Пусть $G = \langle g \rangle, H \leq G$. Если $H = \{e\}$, то $H = \langle e \rangle$.\\
        При $H \neq \{e\} : \forall h \in H \ \exists k \in \Z : h = g^k$. Так как $g^k \in H \Longrightarrow g^{-k} \in H$ и в $H$ есть элемент, отличный от $e$, $\exists$ наименьшее $k \in \N: g^k \in H$.\\
        Докажем, что $H = \langle g^k \rangle$. Рассмотрим произвольный $g^m \in H$. Разделим $m$ на $k$ с остатком: $m = kq + r, 0 \leqslant r < k$. Тогда:
        \[g^m = (g^k)^q\cdot g^r \Longrightarrow g^r = (g^k)^{-q}\cdot g^m \]
        то есть $g^r \in H$, а в силу того, что $k$ - наименьшее натуральное число такое, что $g^k \in H$, имеем $r = 0$. Значит, $g^m = (g^k)^q$, а отсюда $H = \langle g^k \rangle$.
        \item $G =  \langle g \rangle_n, H \leq G \overset{1}{\Longrightarrow} H =  \langle g^k \rangle$.\\
        Так как $g^n = e \in H$, то в силу рассуждений пункта 1 при $m = n$ получаем $k | n \Longrightarrow n = kq$.\\
        Отсюда $H = \{e, g^k, g^{2k},...,g^{(q-1)k}\} \Longrightarrow |H| = q$, где $q | n$.\\
        Обратно, $\forall d | n \ \exists! H =  \langle g^{\frac{n}{d}} \rangle$ (в силу описания выше других подгрупп такого порядка нет).
        \item Из пункта 1 в аддитивной форме получаем, что $H \leq \Z =  \langle 1 \rangle \Longrightarrow H =  \langle k\cdot 1 \rangle$
    \end{enumerate} 
\end{proof}
\begin{consequense}
    В циклической группе простого порядка существуют ровно две подгруппы - тривиальная и сама группа. 
\end{consequense}
\begin{examples}\tab
    \begin{enumerate}
        \item $H \leq \Z_5 \Longrightarrow H = \{0\}, H = \Z_5$;
        \item $H \leq \Z_6 \Longrightarrow H = \{0\}, H = \langle 2 \rangle, H = \langle 3 \rangle, H = \Z_6$. 
    \end{enumerate}
\end{examples}
\subsection{Смежные классы}
\begin{definition}
    Пусть $(G, \cdot, e)$ - произвольная группа, $H \leq G, g \in G$.\\
    Рассмотрим множества:
    \[gH = \{gh | h \in H\} \text{ - левый смежный класс G по H с представителем g}\]
    \[Hg = \{hg | h \in H\} \text{ - правый смежный класс G по H с представителем g}\]
\end{definition}
\begin{subtheorem}[Свойства смежных классов] \tab
    \begin{enumerate}
        \item $\forall a \in G \ a \in aH$;
        \item если $a \in bH$, то $bH = aH$; в частности, любые два смежных класса либо не пересекаются, либо совпадают.
        \item $aH = bH \Longleftrightarrow b^{-1}a \in H$;\\
        (Верны аналогичные утверждения для правых смежных классов)
    \end{enumerate}
\end{subtheorem}
\begin{proof} \tab
    \begin{enumerate}
        \item Очевидно;
        \item $a \in bH \Longrightarrow \exists h \in H: a = bh \Longrightarrow \forall \tilde{h} \in H \ a\tilde{h} = bh\tilde{h} \in bH \Longrightarrow aH \subseteq bH$.
        Аналогично $bH \subseteq aH \Longrightarrow aH = bH$.
        \item $\Longrightarrow: \ aH = bH \Longrightarrow a \in bH (a \in aH) \Longrightarrow \exists h \in H: a = bh \Longrightarrow b^{-1}a = h \in H$\\
        $\Longleftarrow: \ b^{-1}a = h \in H \Longrightarrow a = bh \Longrightarrow aH = bH$ по пункту 2.
    \end{enumerate}
\end{proof}
\begin{subtheorem}
    Отношение $a\equiv b \pmod{H} \Leftrightarrow b^{-1}a\in H$ является отношением эквивалентности, причём классы эквивалентности совпадают с левыми смежными классами (аналогично $ab^{-1} \in H$ для правых).
\end{subtheorem}
\begin{proof}\tab
    \begin{itemize}
        \item Рефлексивность: $a^{-1}a = e \in H \Longrightarrow a \equiv a \pmod{H}$;
        \item Симметричность: $a \equiv b \pmod{H} \Rightarrow b^{-1}a \in H \Rightarrow a^{-1}b = (b^{-1}a)^{-1} \in H \Rightarrow b \equiv a \pmod{H}$;
        \item Транзитивность: $a \equiv b, b \equiv c \pmod{H} \Longrightarrow c^{-1}b, b^{-1}a \in H \Longrightarrow c^{-1}b \cdot b^{-1}a = c^{-1}a \in H \Longrightarrow a \equiv c \pmod{H}$. 
    \end{itemize}
    Совпадение классов эквивалентности с левыми смежными классами следует из пункта 3 предыдущего утверждения. 
\end{proof}
\begin{subtheorem}
    Если $G$ - абелева, то $\forall a\in G: aH=Ha$. \\
    (В общем случае данное утверждение неверно).
\end{subtheorem}
\begin{proof}
    $\forall a \in G: \ \{ah: h \in H\} = \{ha: h \in H\} \Longrightarrow aH = Ha$.
\end{proof}
\begin{examples} \tab
    \begin{enumerate}
        \item $H = \langle (12) \rangle \leq S_3 \ \ (H = \{id, (12)\}),\ g=(13)$.\\
        $(13)(12) = (123); \ (12)(13) = (132)$.\\
        Тогда $\{(13), (123)\}=gH\not=Hg=\{(13), (132)\}$.
        \item $H = 3\Z \leq \Z$.
        Смежные классы - $3\Z, 1 + 3\Z, 2 + 3\Z$.
        \item $H = \R \leq \CC$.
        Смежные классы - $a+bi+\R=bi+\R$.
    \end{enumerate}
\end{examples}
\begin{subtheorem}
    Множество $\{aH : a\in G\}$ находится во взаимно однозначном соответствии с множеством $\{Ha: a\in G\}$.
\end{subtheorem}
\begin{proof}
    $gH \leftrightarrow Hg^{-1}: x=gh\in gH \leftrightarrow x^{-1}=h^{-1}g^{-1}\in Hg^{-1}$.
\end{proof}
\begin{consequense}
    $|\{aH: a\in G\}|=|\{Ha: a\in G\}|$
\end{consequense}
\begin{definition}
    Мощность множества левых смежных классов группы $G$ по подгруппе $H$ называется индексом $H$ в $G$. Обозначение: $|G: H|$
\end{definition}
\begin{example}
    $|\Z: 3\Z|=3$,\ т.к. смежные классы - $\{3\Z,\ 1+3\Z,\ 2+3\Z\}$.
\end{example}
\begin{theorem} (Теорема Лагранжа)\\
    Пусть $G$ - конечная группа, $H\leq G$. Тогда $|G|=|H|\cdot|G:H|$.
\end{theorem}
\begin{proof}
    Так как $|G|<\infty$, то $|H|<\infty$, т.е. $H=\{h_1,\dots,h_k\}$.\\
    $\forall g\in G,\ gH=\{gh_1,\dots,gh_k\}$, причем $gh_i=gh_j \Rightarrow h_i=h_j \Rightarrow |gH|=|H|$. Отсюда, если $|G:H| = n$:
    \[G=\bigsqcup \limits_{i=1}^{n} a_iH \Longrightarrow |G| = \sum \limits_{i=1}^{n} |a_iH| = |G:H|\cdot |H|\]
\end{proof}
\begin{consequensenum}
    Если $G$ - конечная группа, $H\leq G$, то $|H|\ |\ |G|$.\\
    (Обратное утверждение неверно).
\end{consequensenum}
\begin{exercise}
    Пусть $G = A_4$ (группа чётных перестановок).\\
    $|A_4|=\frac{4!}{2}=12$. Докажем, что в $A_4$ нет подгруппы порядка 6.\\
    Предположим, что $H \leq A_4$ и $|H| = 6$. $A_4$ состоит из элемента $id$, 3 элементов вида $(ab)(cd)$ и восьми элементов вида $(abc)$. Значит, $H$ содержит хотя бы один элемент вида $(abc)$ (с точностью до перенумерования - (123)). Тогда $H$ содержит и $(123)^{-1} = (132)$. Также знаем, что группа чётного порядка содержит элемент порядка 2 (иначе в группе все элементы, кроме $e$, разбиваются на пары обратных, и элементов нечётное число), поэтому $H$ содержит $\sigma = (**)(**)$.\\
    Рассмотрим $\omega = \sigma(123)\sigma^{-1} = (\sigma(1), \sigma(2), \sigma(3))$ (это равенство легко проверить, подставив в него $\sigma(1),...,\sigma(4)$). Очевидно, что это цикл длины 3, не оставляющий на месте 4 (т.к. $\sigma$ не оставляет на месте 4). Значит, $\omega$ и $\omega^{-1}$ принадлежат $H$ и не совпадают с предыдущими элементами (и друг с другом), т.е. 
    \[H = \{id, (123), (132), \sigma, \omega, \omega^{-1}\}\]
    Осталось перебрать возможные значения $\sigma$:
    \begin{itemize}
        \item $\sigma = (12)(34) \Longrightarrow (123)(12)(34)(132) = (14)(23) \notin H$;
        \item $\sigma = (13)(24) \Longrightarrow (123)(13)(24)(132) = (12)(34) \notin H$;
        \item $\sigma = (14)(23) \Longrightarrow (123)(14)(23)(132) = (13)(24) \notin H$;
    \end{itemize}
    Отсюда таких $H$ не существует.
\end{exercise}  
\begin{consequensenum}
    Если $G$ - конечная группа, то $\forall g\in G: \textup{ord }g\ |\ |G|$
\end{consequensenum}
\begin{proof}
    $\textup{ord } g =|\langle g \rangle|\ |\ |G|$.
\end{proof}
\begin{consequensenum}
    Если $G$ - конечная группа порядка $n$, то $\forall g\in G: g^n=e$ в $G$.
\end{consequensenum}
\begin{proof}
    По следствию 2: $n=\textup{ord}g\cdot k \Rightarrow g^n=g^{(\textup{ord} g)\cdot k} = e^k = e$.
\end{proof}
\begin{example}
    Пусть $G=\Z_p^*$, $p$ - простое, $|\Z_p^*| = p-1$. По следствию 3:\\
    $\forall a\in \Z_p^*: a^{p-1}=1$ в $\Z_p^*$, отсюда $\forall a\in \Z,\ p \nmid a: a^{p-1}\equiv 1\pmod{p}$ - малая теорема Ферма.
\end{example}
\begin{consequensenum}
    Любая группа $G$ простого порядка $p$ является циклической.
\end{consequensenum}
\begin{proof}
    $\forall a\in G,\ a \neq e: \textup{ord }a \neq 1,\ \textup{ord }a \mid |G|=p \Rightarrow \textup{ord }a = |G| \Rightarrow G=\langle a \rangle$.
\end{proof}
\begin{exercise}
    Доказать, что с точностью до изоморфизма существует ровно две группы порядка 4 - $\Z_4$ и $V_4$.
\end{exercise}
\begin{proof}
    Пусть $G$ - группа порядка 4. Заметим, что по следствию 2 порядок неединичного элемента в $G$ может быть равен либо 2, либо 4. Если в $G$ есть элемент порядка 4, то $G$ циклическая, а тогда по теореме о классификации циклических групп $G \simeq \Z_4$.\\
    Пусть $G = \{e, a, b, c\}, \textup{ord } a = \textup{ord } b = \textup{ord } c = 2$. Посмотрим, чему может быть равно $ab$:
    \begin{itemize}
        \item $ab = e \Longrightarrow aab = a \Longrightarrow b = a$ - противоречие;
        \item $ab = a \Longrightarrow aab = aa \Longrightarrow b = e$ - противоречие;
        \item $ab = b \Longrightarrow abb = bb \Longrightarrow a = e$ - противоречие.
    \end{itemize}
    Отсюда $ab = c$ - аналогично произведение любых двух различных неединичных элементов равно третьему. Отсюда таблица Кэли для $G$ имеет вид
    $$\begin{tabular}{c|c|c|c|c}
        \null & e & a & b & c \\ \hline
        e & e & a & b & c \\ \hline
        a & a & e & c & b \\ \hline
        b & b & c & e & a \\ \hline
        c & c & b & a & e
        \end{tabular}$$
    откуда видно, что $G \simeq V_4$.
\end{proof}
\begin{exercise}
    Доказать, что если в группе $G$ все неединичные элементы имеют порядок 2, то $G$ - абелева.
\end{exercise}
\begin{proof}
    $\textup{ord } a = 2 \Longrightarrow a = a^{-1} \Longrightarrow \forall a, b \in G: ab = (ab)^{-1} = b^{-1}a^{-1} = ba$.
\end{proof}
\begin{example}
    $H = \langle (12) \rangle \leq S_3,\ g=(13) \Rightarrow gH \neq Hg$
\end{example}
\begin{definition}
    Подгруппа $H$ группы $G$ называется нормальной, если 
    \[\forall g\in G: gH=Hg \Longleftrightarrow \forall g\in G: gHg^{-1}=H \Longleftrightarrow\]
    \[\Longleftrightarrow \forall g\in G: gHg^{-1}\subseteq H \Longleftrightarrow \forall g\in G,\ \forall h\in H: ghg^{-1}\in H\]
    Обозначение: $H\unlhd G$. 
\end{definition} 
\begin{proof}[Эквивалентность определений:]\tab
    \begin{itemize}
        \item $1 \Longleftrightarrow 2$ - очевидно;
        \item $2 \Longleftrightarrow 3$:\\
        $\Longleftarrow: \ \ gHg^{-1} \subseteq H \Leftrightarrow H \subseteq g^{-1}Hg$ - из условия на всевозможные $g$ получаем равенство;\\
        $\Longrightarrow$ - очевидно;
        \item $3 \Longleftrightarrow 4$ - из определения смежного класса.
    \end{itemize}
\end{proof}
\begin{examples} \tab
    \begin{enumerate}
        \item $A_n \unlhd S_n$, так как $\forall \sigma\in S_n,\ \forall \tau\in A_n: \sigma\tau\sigma^{-1}\in A_n$.
        \item $SL_n(\R)\unlhd GL_N(\R)$, так как $\forall A\in GL_n(\R),\ \forall B\in SL_n(\R): \det{(ABA^{-1})}=\det{B}=1 \Rightarrow ABA^{-1}\in SL_n(\R)$.
    \end{enumerate}
\end{examples}
\begin{subtheorem}
    В абелевой группе любая подгруппа является нормальной.
\end{subtheorem}
\begin{exercise}
    Докажите, что если $|G:H|=2$, то $H\unlhd G$ для произвольной группы $G$ и произвольной подгруппы $H\leq G$.
\end{exercise}
\begin{proof}
    Если $|G:H|=2$, то $G$ разбивается на два непересекающихся левых (правых) смежных класса по $H$. Очевидно, что один из этих классов  в обоих случаях - сама подгруппа $H$. Тогда $\forall g \in G \setminus H$ группа $G$ разбивается на левые смежные классы $H$ и $gH$, а также на правые смежные классы $H$ и $Hg$, откуда $gH = Hg$. Также очевидно, что $\forall h \in H: hH = H = Hh$. Значит, $\forall g \in G: gH = Hg \Longrightarrow H \unlhd G$.
\end{proof}

\subsection{Факторгруппа}
\begin{subtheorem}
    Пусть $G$ - группа, $H\unlhd G$. Тогда множество всех смежных классов $G$ по $H : G/H = \{eH, aH, ...\}$ образует группу относительно операции $aH \cdot bH = abH$.
\end{subtheorem}
\begin{proof}\tab
    \begin{enumerate}
        \item Проверим корректность операции, т.е. $\begin{cases}
            aH = \tilde{a}H\\
            bH = \tilde{b}H
        \end{cases} \Longrightarrow abH = \tilde{a}\tilde{b}H$.\\
        Действительно, если $\begin{cases}
            a = \tilde{a}h_a\\
            b = \tilde{b}h_b
        \end{cases}$ из равенства смежных классов, то:
        \[\forall x \in abH \Longrightarrow \exists h\in H: x = abh = \tilde{a}h_a\tilde{b}h_bh = \tilde{a}\tilde{b}h'h_bh \in \tilde{a}\tilde{b}H\]
        ($H \unlhd G \Longrightarrow Hb = bH \Longrightarrow \exists h' \in H: h_a\tilde{b} = \tilde{b}h'$)
        \item Проверим, что это группа:
        \begin{itemize}
            \item Ассоциативность:
            \[aH(bH\cdot cH) = aH(bcH) = a(bc)H = (ab)cH = (abH)cH = (aH\cdot bH)cH\]
            \item Нейтральный элемент:
            \[eH = H: aH \cdot eH = aeH = aH = eaH = eH\cdot aH\]
            \item Обратный элемент:
            \[\forall \ aH \ \exists \ a^{-1}H: aH \cdot a^{-1}H = eH = a^{-1}H \cdot aH\]
        \end{itemize}
    \end{enumerate}
\end{proof}
\begin{definition}
    Группа $G/H$ называется факторгруппой $G$ по $H$.
\end{definition}
\begin{remark}
    Если $H \ntrianglelefteq G$, то операция $aH\cdot bH = abH$ некорректна:
    \[\langle(12)\rangle \leq S_3: \ (13)H = (132)H, (23)H = (123)H;\]
    \[(13)(23)H = (132)H \neq H = (123)(123)H\]
\end{remark}
\begin{examples}\tab
    \begin{enumerate}
        \item $\Z / 3\Z \simeq \Z_3 = \{0, 1, 2\}$;
        \item $S_n \unlhd A_n, S_n/A_n \simeq \Z_2$ (по чётности);
        \item $\R \unlhd \CC, \CC / \R \simeq \R \ (bi + \R \mapsto b)$. 
    \end{enumerate}
\end{examples}

\setcounter{thcount}{0}
\setcounter{concount}{0}
\setcounter{subthcount}{0}

\subsection{Гомоморфизмы групп}
\begin{definition}
    Пусть $(G, \cdot, e), (\tilde{G}, \cdot, \tilde{e})$ - группы. Отображение $\phi: G \rightarrow \tilde{G}$ называется гомоморфизмом групп $G$ и $\tilde{G}$, если $\forall a, b, \in G \ \phi(a\cdot b) = \phi(a)\cdot\phi(b)$.
\end{definition}
\begin{remark}
    В частности, изоморфизм - биективный гомоморфизм.
\end{remark}
\begin{subtheorem}[Свойства гомоморфизмов]\tab
    \begin{enumerate}
        \item $\phi(e) = \tilde{e}$;
        \item $\phi(a^{-1}) = (\phi(a))^{-1}$
    \end{enumerate}
\end{subtheorem}
\begin{definition}
    Множество $\textup{Im } \phi = \{b \in \tilde{G} \ | \ \exists a \in G: \phi(a) = b\}$ - образ гомоморфизма.
    Множество $\textup{Ker } \phi = \{a \in G \ | \ \phi(a) = \tilde{e}\}$ - ядро гомоморфизма.
\end{definition}
\begin{subtheoremnum}\tab
    \begin{enumerate}
        \item $\textup{Im } \phi \leq \tilde{G}$;
        \item $\textup{Ker } \phi \unlhd G$.
    \end{enumerate}
\end{subtheoremnum}
\begin{proof}\tab
    \begin{enumerate}
        \item $\textup{Im }\phi \subseteq \tilde{G}$
        \begin{itemize}
            \item $x, y \in \textup{Im }\phi \Rightarrow \exists a, b \in G: x = \phi(a), y = \phi(b) \Longrightarrow xy = \phi(a)\phi(b) = \phi(ab) \in \textup{Im }\phi$;
            \item $\tilde{e} = \phi(e) \in \textup{Im }\phi$;
            \item $\forall x \in \textup{Im }\phi \ \exists a \in G: \phi(a) = x \Longrightarrow x^{-1} = (\phi(a))^{-1}= \phi(a^{-1}) \in \textup{Im }\phi$
        \end{itemize}
        Отсюда $\textup{Im } \phi \leq \tilde{G}$.
        \item $\textup{Ker }\phi \subseteq G$
        \begin{itemize}
            \item $\forall a, b \in \textup{Ker }\phi: \phi(a) = \phi(b) = \tilde{e} \Longrightarrow \phi(ab) = \phi(a)\phi(b) = \tilde{e} \Longrightarrow \newline ab \in \textup{Ker }\phi$;
            \item $\tilde{e} = \phi{e} \Longrightarrow e \in \textup{Ker }\phi$;
            \item $\forall a \in \textup{Ker }\phi \Rightarrow \phi(a^{-1}) = (\phi(a))^{-1} = \tilde{e}^{-1} = \tilde{e}  \Longrightarrow a^{-1} \in \textup{Ker }\phi$
        \end{itemize}
        Отсюда $\textup{Ker } \phi \leq G$.\\
        $\phi(ghg^{-1}) = \phi(g)\phi(h)\phi(g)^{-1} = \phi(g)\phi(g)^{-1} = \tilde{e} \Rightarrow ghg^{-1} \in \textup{Ker }\phi \Longrightarrow \textup{Ker }\phi \unlhd G$.
    \end{enumerate}
\end{proof}
\begin{subtheoremnum}
    $\phi(a) = \phi(b) \Longleftrightarrow a \textup{Ker }\phi = b \textup{Ker }\phi$.\\
    В частности, $\phi$ инъективно $\Longleftrightarrow \textup{Ker }\phi = \{e\}$. 
\end{subtheoremnum}
\begin{proof}
    \[\phi(a) = \phi(b) \Longleftrightarrow \phi(a)\phi(b)^{-1} = \tilde{e} \Longleftrightarrow \phi(ab^{-1}) = \tilde{e} \Longleftrightarrow\]
    \[ab^{-1} \in \textup{Ker }\phi \Longleftrightarrow a \textup{Ker }\phi = b \textup{Ker }\phi\]
\end{proof}
\begin{example}
    $\phi: GL_n(\R) \rightarrow \R^* : \phi(A) = \det A$.\\
    $\textup{Ker }\phi = SL_n(\R), \textup{Im }\phi = \R^* \Longrightarrow R^* \simeq GL_n(\R) / SL_n(\R)$.
\end{example}
\begin{theorem}[О гомоморфизме]
    Пусть $G, \tilde{G}$ - группы, $\phi: G \rightarrow \tilde{G}$ - гомоморфизм.\\
    Тогда $G / \textup{Ker }\phi \simeq \textup{Im } \phi$.
\end{theorem}
\begin{proof}
    Для начала заметим, что $\textup{Ker } \phi \unlhd G$, поэтому факторгруппа $G / \textup{Ker }\phi$ определена.\\
    Рассмотрим $\psi: \ g \textup{Ker }\phi \mapsto \phi(g)$:
    \begin{itemize}
        \item Корректность:\\
        По утверждению 2: $g_1 \textup{Ker }\phi = g_2 \textup{Ker }\phi \Longrightarrow \phi(g_1) = \phi(g_2)$;
        \item Биективность:\\
        Сюръективность: $\forall b \in \tilde{G} \ \exists a \in G: \phi(a) = b \Longrightarrow \psi(a \textup{Ker }\phi) = b$;\\
        Инъективность: по утверждению 2: $\psi(a \textup{Ker }\phi) = \psi(b \textup{Ker }\phi) \Longrightarrow \phi(g_1) = \phi(g_2) \Longrightarrow a \textup{Ker }\phi = b \textup{Ker }\phi$;
        \item Сохранение операции:
        \[\psi((g_1 \textup{Ker }\phi)(g_2 \textup{Ker }\phi)) = \psi(g_1g_2 \textup{Ker }\phi) = \phi(g_1g_2) =\]
        \[ = \phi(g_1)\phi(g_2) = \psi(g_1 \textup{Ker }\phi)\psi(g_2 \textup{Ker }\phi)\]
    \end{itemize}
    Отсюда $\psi: G/\textup{Ker }\phi \rightarrow \textup{Im }\phi$ - изоморфизм.
\end{proof}
\begin{example}
    Пусть $G = S_n, \tilde{G} = \R^*, \phi(\sigma) = \sgn \sigma$.\\
    Тогда из теоремы о гомоморфизме: 
    \[\textup{Im }\phi = \{\pm 1\}, \textup{Ker }\phi = A_n \Longrightarrow S_n / A_n \simeq \{\pm 1\} \simeq \Z_2\]
\end{example}
\begin{consequensenum}
    Гомоморфизм $\phi: G \rightarrow \tilde{G}$ - изоморфизм $\Longleftrightarrow \begin{cases}
        \textup{Ker }\phi = \{e\}\\
        \textup{Im }\phi = \tilde{G}
    \end{cases}$
\end{consequensenum}
\begin{proof}  
    $ \\ \Longrightarrow$ - очевидно из биективности;\\
    $\Longleftarrow$ - изоморфизм из теоремы совпадёт с $\phi$.  
\end{proof}
\begin{consequensenum}
    Если $|G| < \infty$, то $|G| = |\textup{Ker }\phi|\cdot|\textup{Im }\phi|$.
\end{consequensenum}
\begin{proof}
    $|G| = |G/\textup{Ker }\phi|\cdot|\textup{Ker }\phi| = |\textup{Im }\phi| \cdot |\textup{Ker }\phi|$.
\end{proof}
\begin{subtheorem}
    Пусть $G$ - группа, $H \unlhd G$. Тогда $\exists$ такая группа $\tilde{G}$, что $\exists$ сюръективный гомоморфизм $\pi: G \rightarrow \tilde{G}$, причём $\textup{Ker }\pi = H$.
\end{subtheorem}
\begin{proof}
    Подходят $\tilde{G} = G/H, \pi: g \mapsto gH$.
\end{proof}
\begin{definition}
    Приведённый выше гомоморфизм $\pi: G \mapsto G/H$ называется естественным (натуральным) гомоморфизмом из $G$ в $G/H$.
\end{definition}
\begin{definition}
    Эпиморфизм - сюръективный гомоморфизм.
\end{definition}
\begin{subtheorem}
    Пусть $\phi: G \rightarrow \tilde{\tilde{G}}$ - произвольный эпиморфизм с ядром $H$.\\
    Тогда $\exists$ изоморфизм $\psi: G/H \rightarrow \tilde{\tilde{G}}$ такой, что $\phi = \psi \circ \pi$, где $\pi$ - натуральный гомоморфизм из $G$ в $G/H$.
\end{subtheorem}
\begin{proof}
    По теореме о гомоморфизме $G/ \textup{Ker }\phi \simeq \textup{Im }\phi$.\\
    Так как $\phi$ - сюръекция, $\textup{Im }\phi =  \tilde{\tilde{G}}$, также по условию $\textup{Ker }\phi = H$. Тогда $\psi: G/H \rightarrow \tilde{\tilde{G}}$ - изоморфизм, заданный в доказательстве теоремы о гомоморфизме: $\psi: gH \mapsto \phi(g)$.\\
    Взяв этот изоморфизм, получим $\phi = \psi \circ \pi$ (так как $g \overset{\pi}{\mapsto} gH \overset{\tau}{\mapsto} \phi(g)$).
\end{proof}
\setcounter{thcount}{0}
\setcounter{concount}{0}
\setcounter{subthcount}{0}
\newpage