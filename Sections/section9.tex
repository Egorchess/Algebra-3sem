\section{Линейные представления}
Пусть $V$ - векторное пространство над полем $\F$, $GL(V)$ - группа обратимых линейных операторов над $V$, $G$ - произвольная группа.
\begin{definition}
    Произвольный гомоморфизм $\Phi: G \rightarrow GL(V)$ (действие $G \acts V$) называется линейным представлением группы $G$.\\
    $V$ называется пространством линейного представления, $\dim V$ - размерность (степень) линейного представления.
    Если $\F = \Q, \R, \CC$, то линейное представление называется рациональным, вещественным или комплексным соответственно.
\end{definition}
Из определения $\Phi(e) = I$ - тождественный оператор, $\Phi(g_1g_2) = \Phi(g_1)\Phi(g_2)$.
\begin{definition}
    $\forall g \in G \ \Phi(g)$ называется оператором линейного представления. Обозначается $\forall v \in V \ \ \Phi(g)v := \Phi(g)(v)$.
\end{definition}
\begin{definition}
    Если $\textup{Ker }\Phi = \{e\}$, то линейное представление называется точным. В этом случае $G \simeq \textup{Im }\Phi \leq GL(V)$.
\end{definition} 
\begin{definition}
    Если $G \leq GL(V)$, то тождественное линейное представление $\Phi = \textup{id}: G \rightarrow GL(V)$ называется тавтологическим линейным представлением $G$.
\end{definition}
\begin{examples}\tab
    \begin{enumerate}
        \item $\dim V = 1: GL(V) \simeq \F^*$, то есть $\Phi: G \rightarrow GL(V) \simeq \F^{*}$;
        \begin{enumerate}
            \item $\F = \R, G = \R:$ $\Phi(t) = e^t$;
            \item $\F = \R, G = S_n:$ $\Phi(\sigma) = \sgn \sigma$;
            \item $G = GL_n(\F):$ $\Phi(A) = \det A$. 
        \end{enumerate}
        \item $V = M_n(\CC), G = \R$. Если зафиксировать матрицу $B \in M_n(\CC)$, то определено линейное представление $\Phi(t) = e^{tB}$.
        \item Пусть задано линейное представление $\Phi: G \rightarrow GL(V)$ и задан гомоморфизм $\Psi: H \rightarrow G$. Тогда $\tilde{\Phi} = \Phi \circ \Psi: H \rightarrow GL(V)$ - линейное представление группы $H$.
        \item Пусть $X$ - некоторое множество, задано $G \acts X$.\\
        Рассмотрим векторное пространство функций $\mathcal{F}(X, \F) = \{f: X \rightarrow \F\} = V$\\
        Тогда $\Phi: G \rightarrow GL(V)$: $\forall g \in G \ \Phi(g)f = \tilde{f}$, где $\tilde{f}(x) = f(gx)$ - линейное представление группы $G$:
        \[\forall g_1, g_2\in G : \Phi(g_1g_2)f(x) = f(g_1g_2x) = \Phi(g_1)f(g_2x) = \Phi(g_1)\Phi(g_2)f(x)\]
    \end{enumerate}
\end{examples}

\subsection{Матричные представления группы}
\begin{definition}
    Произвольный гомоморфизм $\Phi: G \rightarrow GL_n(\F)$ называется матричным представлением группы $G$ размерности $n$ над полем $\F$.
\end{definition}
Заметим, что линейные и матричные представления связаны между собой:
\begin{enumerate}
    \item Если задано матричное представление $G$, то есть гомоморфизм $\Phi: G \rightarrow GL_n(\F)$, то $GL_n(\F) \underset{\psi}{\simeq} GL(\F^n) \Longrightarrow \tilde{\Phi} = \psi \circ \Phi: G \rightarrow GL(\F^n)$ - линейное представление $G$.
    \item Если задано $n$-мерное линейное представление $G$, то есть гомоморфизм $\Phi: G \rightarrow GL(V) : \forall g \in G \ g \mapsto \Phi(g) = \phi_g$ - линейный оператор. Если фиксировать базис $\E = \{e_1,...,e_n\}$, то $\phi_g \leftrightarrow A_g$ - матрица $\phi_g$ в базисе $\E$. Отсюда $GL(V) \underset{\tilde{\psi}}{\simeq} GL_n(\E)$ - получим матричное представление группы $G$.
\end{enumerate}
Поэтому при фиксированном базисе $V$ имеет место взаимно однозначное соответствие между линейными представлениями $G \rightarrow GL(V)$ и матричными представлениями $G \rightarrow GL_n(\F)$, где $n = \dim V$.
\begin{reminder}
    Если $\phi: V \rightarrow V$ - линейный оператор, $A$ и $\tilde{A}$ - его матрицы в базисах $\E$ и $\tilde{\E}$ соответственно, то $\tilde{A} = C^{-1}AC$, где $C$ - матрица перехода от $\E$ к $\tilde{\E}$.  
\end{reminder}
Здесь и далее: если вдруг не вспоминается - см. \href{https://github.com/Viacheslavik122333/Linear-algebra/blob/main/linal.pdf}{конспект} линейной алгебры.
\begin{definition}
    Матричные представления $\Phi_1, \Phi_2$ группы $G$ размерности $n$ над полем $\F$ называются эквивалентными (изоморфными, подобными), если $\exists C \in GL_n(\F): \forall g \in G: \ \Phi_2(g) = C^{-1}\Phi_1(g)C$. Обозначается $\Phi_1 \approx \Phi_2$.
\end{definition}
\begin{remark}
    Эквивалентные матричные представления группы $G$ соответствуют одному и тому же линейному представлению $G$ относительно разных базисов.
\end{remark}
\begin{definition}
    Пусть $V_1, V_2$ - векторные пространства над полем $\F$. Линейные представления $\Phi_1: G \rightarrow GL(V_1)$ и $\Phi_2: G \rightarrow GL(V_2)$ называются эквивалентными (изоморфными, подобными), если $\exists \phi: V_1 \rightarrow V_2$ - изоморфизм такой, что $\forall g \in G: \ \Phi_2 = \phi^{-1}\circ\Phi_1\circ\phi$. Обозначается $\Phi_1 \approx \Phi_2$.
\end{definition}
\begin{remark}
    Если $\Phi_1 \approx \Phi_2$ и $\E_1= \{e_1,...,e_n\}$ - базис $V_1$, то в базисе $\E_2 = \{\phi(e_1),...,\phi(e_n)\}$ (где $\phi$ - изоморфизм $V_1$ и $V_2$ из определения $\Phi_1 \approx \Phi_2$) для любого $g$ матрица линейного оператора $\Phi_2(g)$ равна матрице оператора $\Phi_1(g)$ в базисе $\E_1$.
\end{remark}
\subsection{Приводимость линейных представлений}
\begin{reminder}
    Пусть $\phi: V \rightarrow V$ - линейный оператор. Подпространство $U \subseteq V$ называется инвариантным относительно $\phi$, если $\forall u \in U: \ \phi(u) \in U$.
\end{reminder}
\begin{definition}
    Пусть $\Phi: G \rightarrow GL(V)$ - линейное представление группы $G$. Подпространство $U \subseteq V$ называется инвариантным относительно $\Phi$, если $\forall g \in G$ $U$ инвариантно относительно оператора $\Phi(G)$, то есть $\forall g \in G, u \in U: \ \Phi(g)u \in U$.
\end{definition}
\begin{remark}
    Подпространства $\{0\}, V$, очевидно, всегда инвариантны - они называются тривиальными инвариантными подпространствами.
\end{remark}
\begin{subtheorem}
    Сумма и пересечение инвариантных подпространств - инвариантное подпространство (как для оператора, так и для линейного представления).
\end{subtheorem}
\begin{proof}
    Очевидно из определений инвариантности.
\end{proof}
\begin{reminder}
    Если $U \subseteq V$ - инвариантное подпространство относительно линейного оператора $\phi$ и $\{e_1,...,e_n\}$ - базис $V$ такой, что $\{e_1,..., e_m\}$ - базис $U$, то матрица линейного оператора в базисе $\{e_1,...,e_n\}$ имеет вид $A = \begin{pmatrix} A_u&\vline&* \\ \hline 0&\vline &* \end{pmatrix}$, где $A_u$ - матрица $\phi|_U$ (ограничения на инвариантное подпространство).
\end{reminder}
\begin{subtheorem}
    Если $U \subseteq V$ - инвариантное подпространство относительно линейного представления $\Phi: G \rightarrow GL(V)$ и $\{e_1,...,e_n\}$ - базис $V$ такой, что $\{e_1,..., e_m\}$ - базис $U$, то $\forall g \in G$ матрица линейного оператора $\Phi(g)$ в базисе $\{e_1,...,e_n\}$ имеет вид $A_g = \begin{pmatrix} A_{g,u}&\vline&* \\ \hline 0&\vline &* \end{pmatrix}$.
\end{subtheorem}
\begin{definition}
    Если $U \subseteq V$ - инвариантное подпространство относительно линейного $\Phi: G \rightarrow GL(V)$, то ограничением линейного представления $\Phi$ на $U$ называется линейное представление $\Phi|_U : G \rightarrow GL(U)$ такое, что $\forall g \in G: \ \Phi|_U(g) = \Phi(g)|_U$
\end{definition}
\begin{definition}
    Линейное представление $\Phi: G \rightarrow GL(V)$ называется неприводимым, если:
    \begin{enumerate}
        \item $V \neq \{0\}$;
        \item $\Phi$ не имеет нетривиальных инвариантных подпространств.
    \end{enumerate}
    В противном случае $\Phi$ называется приводимым.
\end{definition}
\begin{examples}\tab
    \begin{enumerate}
        \item Любое одномерное линейное представление неприводимо;
        \item Пусть $G = \R, V = \E^2$ - двумерное евклидово пространство; \\
        $\Phi: G \rightarrow GL(\E^2)$ такое, что $\Phi(g)$ в ортонормированном базисе $\{e_1, e_2\}$ имеет матрицу $\begin{pmatrix} \cos g&-\sin g\\ \sin g &\cos g \end{pmatrix}$ (т.е. $\Phi(g)$ - поворот евклидова пространства).
        Над $\R$ данное линейное представление неприводимо, так как не имеет одномерных инвариантных подпространств (геометрически очевидно).\\
        Однако над $\CC$ нетривиальные инвариантные подпространства есть: они будут собственными подпространствами $\langle e_1 + ie_2 \rangle$, $\langle e_1 - ie_2 \rangle$ - поэтому $\Phi$ приводимо над $\CC$.
        \item Пусть $G = \langle a \rangle_k$, $V = \E^2$ - двумерное евклидово пространство; \\
        $\Phi: G \rightarrow GL(\E^2)$ такое, что $\Phi(a^m)$ в ортонормированном базисе $\{e_1, e_2\}$ - поворот на $\frac{2\pi m}{k}$. Абсолютно аналогично примеру (2) доказывается, что $\Phi$ неприводимо над $\R$ и приводимо над $\CC$;
        \item Пусть $G = D_n (n \geqslant 3)$, $V = \E^2$ - двумерное евклидово пространство; \\
        По определению $D_n = \textup{Sym } N \subset O_2 \subset GL(\E^2)$ (где $N$ - правильный $n$-угольник) - можем задать тавтологическое линейное представления $\Phi$.\\
        $\Phi$ неприводимо над $\R$, так как нет одномерных инвариантных подпространств относительно поворотов.\\
        Также $\Phi$ неприводимо над $\CC$, так как все одномерные инвариантные подпространства относительно поворотов - это $\langle e_1 + ie_2 \rangle$, $\langle e_1 - ie_2 \rangle$, которые не инвариантны относительно симметрий.
        \item Пусть $G = S_4$, $V = \E^3$ - трёхмерное евклидово пространство; \\
        Ранее доказывали, что $S_4 \simeq \textup{Sym}^+ K \subset O_3 \subset GL(\E^3)$ (где $K$ - куб) - задали линейное представление $\Phi$.\\
        $\Phi$ неприводимо над $\R$, так как одномерных инвариантных подпространств не может быть из геометрических соображений, а двумерных не может быть, так как если $U$ - инвариантное, то $U^\perp$ инвариантно из ортогональности $\Phi(g)$ для всех $g$, а $U^\perp$ одномерно.\\
        Также $\Phi$ неприводимо над $\CC$: любое одномерное инвариантное подпространство над $\CC$ соответствует двумерному инвариантному подпространству над $\R$ (либо одномерному, если оно полностью вещественное) - значит, таковых нет; двумерных инвариантных подпространств не может быть из рассуждений об ортогональности (как выше).
        \item Пусть $G = S_4$, $V = \E^3$ - трёхмерное евклидово пространство; \\
        Ранее доказывали, что $S_4 \simeq \textup{Sym } T \subset O_3 \subset GL(\E^3)$ (где $T$ - правильный тетраэдр)- задали линейное представление $\Phi$.\\
        Абсолютно аналогично примеру (5) доказывается, что $\Phi$ неприводимо над $\R$ и над $\CC$.
        \item Пусть $G = S_n$, $V$ - векторное пространство размерности $n$ над полем $\F$ ($\textup{char } \F = 0$). Зададим \textit{мономиальное} линейное представление $S_n$: зафиксируем базис $\E = \{e_1,...,e_n\}$ в $V$ и определим $M: G \rightarrow GL(V)$ так, что $\forall \sigma \in G: \ M(\sigma)e_i = e_{\sigma(i)}$ (очевидно, что для любой подстановки такой оператор существует и единственный)\\
        Заметим, что $M$ приводимо - оно имеет одномерное инвариантное подпространство $U = \langle e_1 + ... + e_n \rangle$ и $n-1$-мерное инвариантное подпространство $W = \{x = \sum x_ie_i \ | \ \sum x_i = 0\}$ (оно $n-1$-мерно, т.к. любой его элемент однозначно задаётся первыми $n-1$ координатами).\\Более того, докажем, что $V = U \oplus W$:
        \begin{itemize}
            \item $\dim V = \dim U + \dim W$;
            \item Если $x \in U \cap W$, то $x = k(e_1 + ... + e_n)$, причём сумма его координат равна нулю, т.е. $k = 0 \Longrightarrow x = 0$. Значит, $U \cap W = \{0\}$.
        \end{itemize}
        Однако ограничение $M|_W = M_W$ неприводимо - докажем это:\\
        Пусть $\tilde{U} \subset W$ - нетривиальное инвариантное подпространство.\\
        Тогда $\exists x \in \tilde{U}, x \neq 0$. Так как $x \in W$, $\sum x_i = 0 \Longrightarrow \exists i, j: x_i \neq x_j$. Рассмотрим $(ij) \in S_n$:
        \[M(ij)x \in \tilde{U} \Longrightarrow x - M(ij)x = (x_i - x_j)(e_i - e_j) \in \tilde{U} \Longrightarrow e_i - e_j \in \tilde{U}\]
        Отсюда из инвариантности $\tilde{U}$ $\forall \sigma \in S_n: e_{\sigma(i)} - e_{\sigma(j)} \in \tilde{U} \Longrightarrow \forall k, m:\ e_k - e_m \in \tilde{U} \Longrightarrow \tilde{U} = W$ (т.к. $W = \langle e_1 - e_2, ... , e_1 - e_n \rangle $) 
    \end{enumerate}    
\end{examples}
\begin{theorem}(Лемма Шура).\\
    Пусть $\F$ - алгебраически замкнутое поле, $V$ - векторное пространство над $\F$, $G$ - произвольная группа, $\Phi: G \rightarrow GL(V)$ - неприводимое линейное представление. Тогда если $\phi: V \rightarrow V$ - линейный оператор такой, что $\forall g \in G: \phi \circ \Phi(g) = \Phi(g) \circ \phi$, то $\phi$ - скалярный оператор (т.е. $\phi = \lambda I$).
\end{theorem}
\begin{proof}
    Так как $\F$ алгебраически замкнуто, $\phi$ имеет хотя бы одно собственное значение $\lambda$ и собственное подпространство $V_\lambda = \{v \in V\ |\ \phi(v) = \lambda v\}$. Докажем, что $V_\lambda$ - инвариантное подпространство, т.е. что $\forall g \in G, v \in V_\lambda: \ \Phi(g)v \in V_\lambda$, что равносильно $\phi(\Phi(g)v) = \lambda \Phi(g)v$:
    \[\phi(\Phi(g)v) = (\phi \circ \Phi(g))v = (\Phi(g) \circ \phi)v = \Phi(g)(\phi(v)) = \Phi(g)(\lambda v) = \lambda \Phi(g)v\] 
    Т.к. $V_\lambda \neq \{0\}$ и $\Phi$ неприводимо, $V_\lambda  = V \Longrightarrow \forall v \in V: \ \phi(v) = \lambda v \Longrightarrow \phi = \lambda I$. 
\end{proof}
\begin{consequense}
    Пусть $\F$ - алгебраически замкнутое поле, $V$ - векторное пространство над $\F$, $G$ - абелева группа, $\Phi: G \rightarrow GL(V)$ - неприводимое линейное представление. Тогда $\dim V = 1$, т.е. $\Phi$ - одномерное линейное представление.
\end{consequense}
\begin{proof}
    $\forall g, h \in G: \ gh = hg \Longrightarrow \Phi(gh) = \Phi(hg) \Longrightarrow \Phi(g)\Phi(h) = \Phi(h)\Phi(g)$. Тогда если обозначить $\Phi(h) = \phi$, то условия леммы Шура выполняются, а отсюда $\forall h: \ \Phi(h)$ - скалярный оператор. Но для скалярного оператора любое подпространство $V$ инвариантно, а значит любое подпространство $V$ инвариантно для $\Phi$. Тогда из неприводимости $\Phi$ любое нетривиальное подпространство $V$ совпадает с $V$, а отсюда $\dim V$ = 1.
\end{proof}
$\\$\textbf{Неприводимые лин. представления конечных абелевых групп над $\CC$}
Пусть $G$ - конечная абелева группа, $\F = \CC$. По следствию из леммы Шура любое неприводимое линейное представление $G$ имеет одномерное пространство представления, то есть $GL(V) \simeq \CC^*$.
\begin{subtheorem}
    Для конечной абелевой группы $G$ существует ровно $|G|$ различных комплексных неприводимых линейных представлений $G$.
\end{subtheorem}
\begin{proof}
    Опишем все гомоморфизмы $\Phi: G \rightarrow \CC^*$:\\
    Так как $G$ - конечная абелева, по основной теореме о конечнопорождённых абелевых группах $G \simeq \langle a_1 \rangle_{n_1} \times ... \times \langle a_k \rangle_{n_k}$. Пусть $\Phi(a_i) = c_i$. Тогда $c_i^{n_i} = \Phi(a_i^{n_i}) = \Phi(e) = 1$, то есть $c_i$ - комплексный корень степени $n_i$ из единицы. Так как $a_1,...,a_k$ порождают $G$, очевидно, что гомоморфизм однозначно задаётся выбором $c_1,...,c_k$. Способов выбрать $c_i$ ровно $n_i$ (количество комплексных корней степени $n_i$ из единицы) - отсюда гомоморфизмов $n_1 \cdot ... \cdot n_k = |G|$. 
\end{proof}
\begin{example}
    $V_4 = \langle a \rangle_2 \times \langle b \rangle_2 \Longrightarrow \Phi(a) = \pm 1, \Phi(b) = \pm 1$.
    $$\begin{tabular}{c|c|c|c|c}
        \null & e & a & b & ab \\ \hline
        $\Phi_1 = I$ & 1 & 1 & 1 & 1 \\ \hline
        $\Phi_2$ & 1 & -1 & 1 & -1 \\ \hline
        $\Phi_3$ & 1 & 1 & -1 & -1 \\ \hline
        $\Phi_4$ & 1 & -1 & -1 & 1
        \end{tabular}$$ 
\end{example}
$\\$\textbf{Одномерные комплексные лин. представления произвольной группы}
Пусть $G$ - произвольная группа.\\
Знаем, что коммутант $G' \unlhd G$ - подгруппа такая, что $G/G'$ абелева. Рассмотрим канонический гомоморфизм $\pi: G \rightarrow G/G'$:
\begin{subtheorem}
    Если $\Psi: G/G' \rightarrow \CC^*$ - одномерное комплексное линейное представление $G/G'$, то $\Phi = \psi \circ \pi: G \rightarrow \CC^*$ - одномерное комплексное линейное представление $G$.
\end{subtheorem}
\begin{proof}
    Очевидно (композиция гомоморфизмов - гомоморфизм).
\end{proof}
\begin{subtheorem}
    Пусть $G$ - произвольная группа, $\Phi: G \rightarrow \CC$ - произвольное одномерное комплексное линейное представление $G$. Тогда $\exists$ линейное представление $\Psi: G/G' \rightarrow \CC^*$ такое, что $\Phi = \Psi \circ \pi$.
\end{subtheorem}
\begin{proof}
    Заметим, что $\textup{Im } \Phi \leq \CC^* \Longrightarrow \textup{Im }\Phi$ - абелева. \\
    Из теоремы о гомоморфизме $\textup{Im }\Phi \simeq G/\textup{Ker }\Phi$, то есть $G/\textup{Ker }\Phi$ - абелева, а тогда $G' \subseteq \textup{Ker }\Phi$ (свойство коммутанта).\\
    Отсюда $\forall g \in G, h \in G': \ \Phi(gh) = \Phi(g)$, то есть значения $\Phi$ на всех элементах левого смежного класса $g$ по $G'$ совпадают. Поэтому корректно отображение $\Psi: G/G' \rightarrow \CC^*$, заданное по правилу $gG'\mapsto \Phi(g)$. Это гомоморфизм:
    \[\Psi(g_1H \cdot g_2G') = \Psi(g_1g_2G') = \Phi(g_1g_2) = \Phi(g_1)\Phi(g_2) = \Psi(g_1G')\Psi(g_2G')\] 
    При этом $\forall g \in G: \ g \overset{\pi}{\mapsto} gG' \overset{\Psi}{\mapsto} \Phi(g)$, то есть $\Phi = \Psi \circ \pi$.
\end{proof}
\begin{consequense}
    Если $G$ конечна, то одномерных комплексных линейных представлений $G$ ровно $|G/G'|$.
\end{consequense}
\begin{proof}
    Из двух предыдущих утверждений имеем взаимно однозначное соответствие между одномерными комплексными линейными представлениями $G$ и $G/G'$. Если $G$ конечна, то $G/G'$ - конечная абелева, а тогда она имеет ровно $|G/G'|$ представлений.
\end{proof}
\subsection{Сумма линейных представлений группы}
\begin{enumerate}
    \item \textbf{Внутренняя сумма линейных представлений}\\
    Пусть $\Phi: G \rightarrow GL(V)$ - линейное представление $G$, и пусть $V = U \oplus W$, где $U$ и $W$ инвариантны относительно $\Phi$. Тогда говорят, что $\Phi$ есть сумма (внутренняя) представлений $\Phi|_U$ и $\Phi|_W$.\\
    Заметим, что если выбрать базисы $\E_U = \{e_1,...,e_m\}$ в $U$, $\E_W = \{e_{m+1},...,e_n\}$ в $W$, то в базисе $\E_V = \{e_1,...,e_n\}$ пространства $V$ матрица оператора $\Phi(g)$ для любого $g \in G$ имеет вид $A_g = \begin{pmatrix} A_{U, g} &\vline&0 \\ \hline 0&\vline&A_{W, g} \end{pmatrix}$, где $A_{U, g}$ - матрица $\Phi|_U$ в базисе $\E_U$, а $A_{W, g}$ - матрица $\Phi|_W$ в базисе $\E_W$.
    \item \textbf{Внешняя сумма линейных представлений}\\
    Пусть $U, W$ - векторные пространства над полем $\F$, и пусть заданы линейные представления $\Psi_1: G \rightarrow GL(U)$ и $\Psi_2: G \rightarrow GL(W)$. Обозначим $V = U \oplus W$ - внешняя прямая сумма $U$ и $W$. Тогда линейное представление $\Phi: G \rightarrow GL(V)$, заданное по правилу $Phi(g)(u, w) = (\Psi_1(g)u, \Psi_2(g)w)$ (очевидно, что это гомоморфизм), называется суммой линейных представлений $\Psi_1, \Psi_2$ и обозначается $\Phi = \Psi_1 + \Psi_2$.\\
    Аналогично, если выбрать базисы $\E_U = \{e_1,...,e_m\}$ в $U$, $\E_W = \{e_{m+1},...,e_n\}$ в $W$, то в базисе $\E_V = \{(e_1, 0),...,(e_m, 0), (0, e_{m+1}),...,(0, e_n)\}$ пространства $V$ матрица оператора $\Phi(g)$ для любого $g \in G$ имеет вид $\begin{pmatrix} A_{\Psi_1} &\vline&0 \\ \hline 0&\vline&A_{\Psi_2} \end{pmatrix}$, где $A_{\Psi_1}$ - матрица $\Psi_1$ в базисе $\E_U$, а $A_{\Psi_2}$ - матрица $\Psi_2$ в базисе $\E_W$.
\end{enumerate}
\subsection{Вполне приводимые линейные представления}
\begin{definition}
    Линейное представление $\Phi: G \rightarrow GL(V)$ называется вполне приводимым, если для любого подпространства $U \subseteq V$, инвариантного относительно $\Phi$, существует такое подпространство $W \subseteq V$, инвариантное относительно $\Phi$, что $V = U \oplus W$.
\end{definition}
\begin{remark}
    Любое неприводимое линейное представление вполне приводимо - для него есть только инвариантные подпространства $V$ и $\{0\}$, причём $V \oplus \{0\} = V$.
\end{remark}
\begin{examples} (Напомним, что при фиксированном базисе $V$ линейные представления взаимно однозначно соответствуют матричным, где в соответствие каждому оператору поставлена его матрица).
    \begin{enumerate}
        \item $\Phi: \R \rightarrow GL_2(\CC) \ \ \Phi(t) = \begin{pmatrix} e^{it}&0 \\ 0&e^{-it} \end{pmatrix}$ - вполне приводимо.\\
        (поворот унитарного пространства - ортогональный оператор, то есть если $U$ инвариантно, то $U^\perp$ инвариантно, причём $V = U \oplus U^\perp$)
        \item $\Phi: \R \rightarrow GL_2(\CC) \ \ \Phi(t) = \begin{pmatrix} 1&t \\ 0&1 \end{pmatrix}$ - приводимо, но не вполне приводимо:\\
        Рассмотрим базис $\tilde{C} = \{e_1, e_2\}$ двумерного пространства линейного представления $V$, в котором записаны матрицы операторов.\\
        Заметим, что $\langle e_1 \rangle$ - инвариантное подпространство относительно $\Phi$. Однако никакое подпространство $W = \langle \mu e_1 + e_2 \rangle$ не инвариантно:
        \[\forall t > 0: \ \Phi(t)\begin{pmatrix} \mu \\ 1 \end{pmatrix} = \begin{pmatrix} 1&t \\ 0&1 \end{pmatrix}\begin{pmatrix} \mu \\ 1 \end{pmatrix} = \begin{pmatrix} \mu + t \\ 1 \end{pmatrix} \neq \lambda \begin{pmatrix} \mu \\ 1 \end{pmatrix}\]
    \end{enumerate}    
\end{examples}
\setcounter{lemcount}{0}
\begin{definition}
    Пусть $\Phi: G \rightarrow GL(V)$ - линейное представление $G$. Линейное представление $\tilde{\Phi}$ называется подпредставлением $\Phi$, если $\exists \tilde{V} \subseteq V$ - такое подпространство, инвариантное относительно $\Phi$, что $\tilde{\Phi} = \Phi|_{\tilde{V}} : G \rightarrow GL(\tilde{V})$.
\end{definition}
\begin{lemmanum}
    Любое подпредставление вполне приводимого представления вполне приводимо.
\end{lemmanum}
\begin{proof}
    Пусть $\Phi: G \rightarrow GL(V)$ - вполне приводимое линейное представление, $\tilde{\Phi}: G \rightarrow GL(\tilde{V})$ - его подпредставление ($\tilde{V}$ - подпространство $V$). Рассмотрим произольное подпространство $U \subseteq \tilde{V}$, инвариантное относительно $\tilde{\Phi}$. $U$ является подпространством и для $V$, а тогда $\exists$ инвариантное относительно $\Phi$ подпространство $W \subseteq V$ такое, что $V = U \oplus W$ (т.к. $\Phi$ вполне приводимо). Тогда если обозначить $\tilde{W} = W \cap \tilde{V}$, то $\tilde{W}$ - подпространство $\tilde{V}$, инвариантное относительно $\tilde{\Phi}$.\\
    Осталось показать, что $\tilde{V} = U \oplus \tilde{W}$, то есть что $\forall x \in \tilde{V}$ единственным образом раскладывается как $x = u + \tilde{w}, u \in U, \tilde{w} \in \tilde{W}$. Действительно, из $V = U \oplus W$ знаем, что $x$ как элемент $V$ единственно раскладывается в сумму $x = u + w$, где $u \in U, w \in W$, однако $w = x - u \in \tilde{V}$, то есть $w \in \tilde{V} \cap W = \tilde{W}$. Значит, такое разложение существует и единственно, что и требовалось.
\end{proof}
\begin{lemmanum}
    Пусть $\Phi: G \rightarrow GL(V)$ - вполне приводимое линейное представление. Тогда $\Phi$ раскладывается в сумму неприводимых линейных представлений (возможно, одного).
\end{lemmanum}
\begin{proof}
    Индукция по $n = \dim V$:\\
    База: $n = 1 \Longrightarrow \Phi$ неприводимо;\\
    Шаг: Пусть $V_1$ - минимальное ненулевое инвариантное подпространство линейного представления $\Phi$. Так как $\Phi$ вполне приводимо, $\exists$ инвариантное дополнение $W: V = V_1 \oplus W \Longrightarrow \Phi = \Phi_{V_1} + \Phi_W$. При этом $\Phi_{V_1}$ неприводимо из минимальности $V_1$ (нет нетривиальных инвариантных подпространств меньшей размерности), а $\Phi_W$ по лемме 1 вполне приводимо, и притом меньшей размерности - раскладывается в искомую сумму по предположению индукции. Значит, $\Phi$ также раскладывается в искомую сумму.  
\end{proof}
\begin{example}
    Если $\dim V = n > 1$ и $\Phi: G \rightarrow GL(n)$ такое, что $\Phi(g) = I \ \forall g \in G$, то все подпространства $V$ являются инвариантными для $\Phi$, откуда $\Phi$ вполне приводимо. Тогда $\Phi$ раскладывается в сумму одномерных представлений, причём не единственным образом - в зависимости от выбора базиса:
    \[\E = \{e_1,...,e_n\} \Longrightarrow V = \langle e_1 \rangle \oplus ... \oplus \langle e_n \rangle \Longrightarrow \Phi = \Phi_1 + ... + \Phi_n;\]
    \[\tilde{\E} = \{\tilde{e}_1,...,\tilde{e}_n\} \Longrightarrow V = \langle \tilde{e}_1 \rangle \oplus ... \oplus \langle \tilde{e}_n \rangle \Longrightarrow \Phi = \tilde{\Phi}_1 + ... + \tilde{\Phi}_n;\]
\end{example}
\begin{lemmanum}
    Пусть $V = V_1 + ... + V_k$, где $V_i$ - минимальные ненулевые подпространства, инвариантные относительно линейного представления $\Phi: G \rightarrow GL(V)$. Тогда $\Phi$ вполне приводимо, причём для произвольного инвариантного подпространства $U \subseteq V$ существует инвариантное дополнение вида $W = \sum \limits_{i \in I} V_i$ для некоторого $I \subseteq \{1,...,k\}$. 
\end{lemmanum}
\begin{proof}
    Обозначим $V_I = \sum \limits_{i \in I} V_i$, где $I \subseteq \{1,...,k\}$. Очевидно, что $V_I$ - инвариантное подпространство (как сумма инвариантных). Рассмотрим произвольное инвариантное $U \subseteq V$ и возьмём $I = \{i \ |\ U \cap V_i = \{0\}\}$. Докажем, что $V = U \oplus V_I$ (по построению $U \cap V_I = \{0\}$, поэтому достаточно $V = U + V_I$):\\
    Рассмотрим $j \notin I$. Тогда $V_{I \cup \{j\}} = V_I + V_j$, а также $U \cap V_{I \cup \{j\}} \neq 0 \Longrightarrow \exists u \in U : u = \sum \limits_{i \in I} v_i + v_j$, причём $v_j \neq 0$. Тогда $v_j  = u - \sum \limits_{i \in I} v_i \in U + V_I$, то есть $V_j \cap (U + V_i) \neq \{0\}$. При этом пересечение инвариантных подпространств инвариантно, то есть $V_j \cap (U + V_i)$ - инвариантное подпространство в $V_j$. Тогда из минимальности $V_j$ это подпространство совпадает с $V_j$, а значит $V_j \subseteq U + V_I$.\\
    Проведя такое рассуждение для всех $j \in \{1,...,k\} \setminus I$, получим $V \subseteq U + V_I$, а отсюда уже $V = U \oplus V_i$.
\end{proof}