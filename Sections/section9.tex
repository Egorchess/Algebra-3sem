\section{Линейные представления}
Пусть $V$ --- векторное пространство над полем $\F$, $GL(V)$ --- группа обратимых линейных операторов над $V$, $G$ --- произвольная группа.
\begin{definition}
    Произвольный гомоморфизм $\Phi: G \rightarrow GL(V)$ (действие $G \acts V$) называется линейным представлением группы $G$.\\
    $V$ называется пространством линейного представления, $\dim V$ --- размерность (степень) линейного представления.
    Если $\F = \Q, \R, \CC$, то линейное представление называется рациональным, вещественным или комплексным соответственно.
\end{definition}
Из определения $\Phi(e) = I$ --- тождественный оператор, $\Phi(g_1g_2) = \Phi(g_1)\Phi(g_2)$.
\begin{definition}
    $\forall g \in G \ \Phi(g)$ называется оператором линейного представления. Обозначается $\forall v \in V \ \ \Phi(g)v := \Phi(g)(v)$.
\end{definition}
\begin{definition}
    Если $\textup{Ker }\Phi = \{e\}$, то линейное представление называется точным. В этом случае $G \simeq \textup{Im }\Phi \leq GL(V)$.
\end{definition} 
\begin{definition}
    Если $G \leq GL(V)$, то тождественное линейное представление $\Phi = \textup{id}: G \rightarrow GL(V)$ называется тавтологическим линейным представлением $G$.
\end{definition}
\begin{examples}\tab
    \begin{enumerate}
        \item $\dim V = 1: GL(V) \simeq \F^*$, то есть $\Phi: G \rightarrow GL(V) \simeq \F^{*}$;
        \begin{enumerate}
            \item $\F = \R, G = \R:$ $\Phi(t) = e^t$;
            \item $\F = \R, G = S_n:$ $\Phi(\sigma) = \sgn \sigma$;
            \item $G = GL_n(\F):$ $\Phi(A) = \det A$. 
        \end{enumerate}
        \item $V = M_n(\CC), G = \R$. Если зафиксировать матрицу $B \in M_n(\CC)$, то определено линейное представление $\Phi(t) = e^{tB}$.
        \item Пусть задано линейное представление $\Phi: G \rightarrow GL(V)$ и задан гомоморфизм $\Psi: H \rightarrow G$. Тогда $\tilde{\Phi} = \Phi \circ \Psi: H \rightarrow GL(V)$ --- линейное представление группы $H$.
        \item Пусть $X$ --- некоторое множество, задано $G \acts X$.\\
        Рассмотрим векторное пространство функций $\mathcal{F}(X, \F) = \{f: X \rightarrow \F\} = V$\\
        Тогда $\Phi: G \rightarrow GL(V)$: $\forall g \in G \ \Phi(g)f = \tilde{f}$, где $\tilde{f}(x) = f(gx)$ --- линейное представление группы $G$:
        \[\forall g_1, g_2\in G : \Phi(g_1g_2)f(x) = f(g_1g_2x) = \Phi(g_1)f(g_2x) = \Phi(g_1)\Phi(g_2)f(x)\]
    \end{enumerate}
\end{examples}

\subsection{Матричные представления группы}
\begin{definition}
    Произвольный гомоморфизм $\Phi: G \rightarrow GL_n(\F)$ называется матричным представлением группы $G$ размерности $n$ над полем $\F$.
\end{definition}
Заметим, что линейные и матричные представления связаны между собой:
\begin{enumerate}
    \item Если задано матричное представление $G$, то есть гомоморфизм $\Phi: G \rightarrow GL_n(\F)$, то $GL_n(\F) \underset{\psi}{\simeq} GL(\F^n) \Longrightarrow \tilde{\Phi} = \psi \circ \Phi: G \rightarrow GL(\F^n)$ --- линейное представление $G$.
    \item Если задано $n$-мерное линейное представление $G$, то есть гомоморфизм $\Phi: G \rightarrow GL(V) : \forall g \in G \ g \mapsto \Phi(g) = \phi_g$ --- линейный оператор. Если фиксировать базис $\E = \{e_1,...,e_n\}$, то $\phi_g \leftrightarrow A_g$ --- матрица $\phi_g$ в базисе $\E$. Отсюда $GL(V) \underset{\tilde{\psi}}{\simeq} GL_n(\F)$ --- получим матричное представление группы $G$.
\end{enumerate}
Поэтому при фиксированном базисе $V$ имеет место взаимно однозначное соответствие между линейными представлениями $G \rightarrow GL(V)$ и матричными представлениями $G \rightarrow GL_n(\F)$, где $n = \dim V$.
\begin{remark}
    Далее зачастую $n$-мерное линейное представление $G \rightarrow GL(V)$ будет рассматриваться как матричное представление $G \rightarrow GL_n(\F)$, ему соответствующее.
\end{remark}
\begin{reminder}
    Если $\phi: V \rightarrow V$ --- линейный оператор, $A$ и $\tilde{A}$ --- его матрицы в базисах $\E$ и $\tilde{\E}$ соответственно, то $\tilde{A} = C^{-1}AC$, где $C$ --- матрица перехода от $\E$ к $\tilde{\E}$.  
\end{reminder}
Здесь и далее: если вдруг не вспоминается --- см. конспект линейной алгебры (\cite{linal}).
\begin{definition}
    Матричные представления $\Phi_1, \Phi_2$ группы $G$ размерности $n$ над полем $\F$ называются эквивалентными (изоморфными, подобными), если $\exists C \in GL_n(\F): \forall g \in G: \ \Phi_1(g) = C^{-1}\Phi_2(g)C$. Обозначается $\Phi_1 \approx \Phi_2$.
\end{definition}
\begin{remark}
    Эквивалентные матричные представления группы $G$ соответствуют одному и тому же линейному представлению $G$ относительно разных базисов.
\end{remark}
\begin{definition}
    Пусть $V_1, V_2$ --- векторные пространства над полем $\F$. Линейные представления $\Phi_1: G \rightarrow GL(V_1)$ и $\Phi_2: G \rightarrow GL(V_2)$ называются эквивалентными (изоморфными, подобными), если $\exists \phi: V_1 \rightarrow V_2$ --- изоморфизм такой, что $\forall g \in G: \ \Phi_1 = \phi^{-1}\circ\Phi_2\circ\phi$. Обозначается $\Phi_1 \approx \Phi_2$.
\end{definition}
\begin{remark}
    Если $\Phi_1 \approx \Phi_2$ и $\E_1= \{e_1,...,e_n\}$ --- базис $V_1$, то в базисе $\E_2 = \{\phi(e_1),...,\phi(e_n)\}$ (где $\phi$ --- изоморфизм $V_1$ и $V_2$ из определения $\Phi_1 \approx \Phi_2$) для любого $g$ матрица линейного оператора $\Phi_2(g)$ равна матрице оператора $\Phi_1(g)$ в базисе $\E_1$.
\end{remark}
\subsection{Приводимость линейных представлений}
\begin{reminder}
    Пусть $\phi: V \rightarrow V$ --- линейный оператор. Подпространство $U \subseteq V$ называется инвариантным относительно $\phi$, если $\forall u \in U: \ \phi(u) \in U$.
\end{reminder}
\begin{definition}
    Пусть $\Phi: G \rightarrow GL(V)$ --- линейное представление группы $G$. Подпространство $U \subseteq V$ называется инвариантным относительно $\Phi$, если $\forall g \in G$ $U$ инвариантно относительно оператора $\Phi(g)$, то есть $\forall g \in G, u \in U: \ \Phi(g)u \in U$.
\end{definition}
\begin{remark}
    Подпространства $\{0\}, V$, очевидно, всегда инвариантны --- они называются тривиальными инвариантными подпространствами.
\end{remark}
\begin{subtheorem}
    Сумма и пересечение инвариантных подпространств --- инвариантное подпространство (как для оператора, так и для линейного представления).
\end{subtheorem}
\begin{proof}
    Очевидно из определений инвариантности.
\end{proof}
\begin{reminder}
    Если $U \subseteq V$ --- инвариантное подпространство относительно линейного оператора $\phi$ и $\{e_1,...,e_n\}$ --- базис $V$ такой, что $\{e_1,..., e_m\}$ --- базис $U$, то матрица линейного оператора в базисе $\{e_1,...,e_n\}$ имеет вид $A = \begin{pmatrix} A_u&\vline&* \\ \hline 0&\vline &* \end{pmatrix}$, где $A_u$ --- матрица $\phi|_U$ (ограничения на инвариантное подпространство).
\end{reminder}
\begin{subtheorem}
    Если $U \subseteq V$ --- инвариантное подпространство относительно линейного представления $\Phi: G \rightarrow GL(V)$ и $\{e_1,...,e_n\}$ --- базис $V$ такой, что $\{e_1,..., e_m\}$ --- базис $U$, то $\forall g \in G$ матрица линейного оператора $\Phi(g)$ в базисе $\{e_1,...,e_n\}$ имеет вид $A_g = \begin{pmatrix} A_{g,u}&\vline&* \\ \hline 0&\vline &* \end{pmatrix}$.
\end{subtheorem}
\begin{definition}
    Если $U \subseteq V$ --- инвариантное подпространство относительно линейного представления $\Phi: G \rightarrow GL(V)$, то ограничением линейного представления $\Phi$ на $U$ называется линейное представление $\Phi|_U : G \rightarrow GL(U)$ такое, что $\forall g \in G: \ \Phi|_U(g) = \Phi(g)|_U$
\end{definition}
\begin{definition}
    Линейное представление $\Phi: G \rightarrow GL(V)$ называется неприводимым, если:
    \begin{enumerate}
        \item $V \neq \{0\}$;
        \item $\Phi$ не имеет нетривиальных инвариантных подпространств.
    \end{enumerate}
    В противном случае $\Phi$ называется приводимым.
\end{definition}
\begin{examples}\tab
    \begin{enumerate}
        \item Любое одномерное линейное представление неприводимо;
        \item Пусть $G = \R, V = \E^2$ --- двумерное евклидово пространство; \\
        $\Phi: G \rightarrow GL(\E^2)$ такое, что $\Phi(g)$ в ортонормированном базисе $\{e_1, e_2\}$ имеет матрицу $\begin{pmatrix} \cos g&-\sin g\\ \sin g &\cos g \end{pmatrix}$ (т.е. $\Phi(g)$ --- поворот евклидова пространства).
        Над $\R$ данное линейное представление неприводимо, так как не имеет одномерных инвариантных подпространств (геометрически очевидно).\\
        Однако над $\CC$ нетривиальные инвариантные подпространства есть: они будут собственными подпространствами $\langle e_1 + ie_2 \rangle$, $\langle e_1 - ie_2 \rangle$ --- поэтому $\Phi$ приводимо над $\CC$.
        \item Пусть $G = \langle a \rangle_k$, $V = \E^2$ --- двумерное евклидово пространство; \\
        $\Phi: G \rightarrow GL(\E^2)$ такое, что $\Phi(a^m)$ в ортонормированном базисе $\{e_1, e_2\}$ --- поворот на $\frac{2\pi m}{k}$. Абсолютно аналогично примеру (2) доказывается, что $\Phi$ неприводимо над $\R$ и приводимо над $\CC$;
        \item Пусть $G = D_n (n \geqslant 3)$, $V = \E^2$ --- двумерное евклидово пространство; \\
        По определению $D_n = \textup{Sym } N \subset O_2 \subset GL(\E^2)$ (где $N$ --- правильный $n$-угольник) --- можем задать тавтологическое линейное представление $\Phi$.\\
        $\Phi$ неприводимо над $\R$, так как нет одномерных инвариантных подпространств относительно поворотов.\\
        Также $\Phi$ неприводимо над $\CC$, так как все одномерные инвариантные подпространства относительно поворотов --- это $\langle e_1 + ie_2 \rangle$, $\langle e_1 - ie_2 \rangle$, которые не инвариантны относительно симметрий.
        \item Пусть $G = S_4$, $V = \E^3$ --- трёхмерное евклидово пространство; \\
        Ранее доказывали, что $S_4 \simeq \textup{Sym}^+ K \subset O_3 \subset GL(\E^3)$ (где $K$ --- куб) --- задали линейное представление $\Phi$.\\
        $\Phi$ неприводимо над $\R$, так как одномерных инвариантных подпространств не может быть из геометрических соображений, а двумерных не может быть, так как если $U$ --- инвариантное, то $U^\perp$ инвариантно из ортогональности $\Phi(g)$ для всех $g$, а $U^\perp$ одномерно.\\
        Также $\Phi$ неприводимо над $\CC$: любое одномерное инвариантное подпространство над $\CC$ соответствует двумерному инвариантному подпространству над $\R$ (либо одномерному, если оно полностью вещественное) --- значит, таковых нет; двумерных инвариантных подпространств не может быть из рассуждений об ортогональности (как выше).
        \item Пусть $G = S_4$, $V = \E^3$ --- трёхмерное евклидово пространство; \\
        Ранее доказывали, что $S_4 \simeq \textup{Sym } T \subset O_3 \subset GL(\E^3)$ (где $T$ --- правильный тетраэдр)- задали линейное представление $\Phi$.\\
        Абсолютно аналогично примеру (5) доказывается, что $\Phi$ неприводимо над $\R$ и над $\CC$. При этом данное представление неэквивалентно предыдущему, так как в образе данного представления есть несобственные движения, а в образе предыдущего --- только собственные;  
        \item Пусть $G = S_n$, $V$ --- векторное пространство размерности $n$ над полем $\F$ ($\textup{char } \F = 0$). Зададим \textit{мономиальное} линейное представление $S_n$: зафиксируем базис $\E = \{e_1,...,e_n\}$ в $V$ и определим $M: G \rightarrow GL(V)$ так, что $\forall \sigma \in G: \ M(\sigma)e_i = e_{\sigma(i)}$ (очевидно, что для любой подстановки такой оператор существует и единственный)\\
        Заметим, что $M$ приводимо --- оно имеет одномерное инвариантное подпространство $U = \langle e_1 + ... + e_n \rangle$ и $n-1$-мерное инвариантное подпространство $W = \{x = \sum \limits_i x_ie_i \ | \ \sum x_i = 0\}$ (оно $(n-1)$-мерно, т.к. любой его элемент однозначно задаётся первыми $n-1$ координатами).\\Более того, докажем, что $V = U \oplus W$:
        \begin{itemize}
            \item $\dim V = \dim U + \dim W$;
            \item Если $x \in U \cap W$, то $x = k(e_1 + ... + e_n)$, причём сумма его координат равна нулю, т.е. $k = 0 \Longrightarrow x = 0$. Значит, $U \cap W = \{0\}$.
        \end{itemize}
        Однако ограничение $M|_W = M_W$ неприводимо --- докажем это:\\
        Пусть $\tilde{U} \subset W$ --- нетривиальное инвариантное подпространство.\\
        Тогда $\exists x \in \tilde{U}, x \neq 0$. Так как $x \in W$, $\sum x_i = 0 \Longrightarrow \exists i, j: x_i \neq x_j$. Рассмотрим $(ij) \in S_n$:
        \[M(ij)x \in \tilde{U} \Longrightarrow x - M(ij)x = (x_i - x_j)(e_i - e_j) \in \tilde{U} \Longrightarrow e_i - e_j \in \tilde{U}\]
        Отсюда из инвариантности $\tilde{U}$ $\forall \sigma \in S_n: e_{\sigma(i)} - e_{\sigma(j)} \in \tilde{U} \Longrightarrow \forall k, m:\ e_k - e_m \in \tilde{U} \Longrightarrow \tilde{U} = W$ (т.к. $W = \langle e_1 - e_2, ... , e_1 - e_n \rangle $) 
    \end{enumerate}    
\end{examples}
\begin{theorem}(Лемма Шура).\\
    Пусть $\F$ --- алгебраически замкнутое поле, $V$ --- векторное пространство над $\F$, $G$ --- произвольная группа, $\Phi: G \rightarrow GL(V)$ --- неприводимое линейное представление. Тогда если $\phi: V \rightarrow V$ --- линейный оператор такой, что $\forall g \in G: \phi \circ \Phi(g) = \Phi(g) \circ \phi$, то $\phi$ --- скалярный оператор (т.е. $\phi = \lambda I$).
\end{theorem}
\begin{proof}
    Так как $\F$ алгебраически замкнуто, $\phi$ имеет хотя бы одно собственное значение $\lambda$ и собственное подпространство $V_\lambda = \{v \in V\ |\ \phi(v) = \lambda v\}$. Докажем, что $V_\lambda$ --- инвариантное подпространство, т.е. что $\forall g \in G, v \in V_\lambda: \ \Phi(g)v \in V_\lambda$, что равносильно $\phi(\Phi(g)v) = \lambda \Phi(g)v$:
    \[\phi(\Phi(g)v) = (\phi \circ \Phi(g))v = (\Phi(g) \circ \phi)v = \Phi(g)(\phi(v)) = \Phi(g)(\lambda v) = \lambda \Phi(g)v\] 
    Т.к. $V_\lambda \neq \{0\}$ и $\Phi$ неприводимо, $V_\lambda  = V \Longrightarrow \forall v \in V: \ \phi(v) = \lambda v \Longrightarrow \phi = \lambda I$. 
\end{proof}
\begin{consequense}
    Пусть $\F$ --- алгебраически замкнутое поле, $V$ --- векторное пространство над $\F$, $G$ --- абелева группа, $\Phi: G \rightarrow GL(V)$ --- неприводимое линейное представление. Тогда $\dim V = 1$, т.е. $\Phi$ --- одномерное линейное представление.
\end{consequense}
\begin{proof}
    $\forall g, h \in G: \ gh = hg \Longrightarrow \Phi(gh) = \Phi(hg) \Longrightarrow \Phi(g)\Phi(h) = \Phi(h)\Phi(g)$. Тогда если обозначить $\Phi(h) = \phi$, то условия леммы Шура выполняются, а отсюда $\forall h: \ \Phi(h)$ --- скалярный оператор. Но для скалярного оператора любое подпространство $V$ инвариантно, а значит, любое подпространство $V$ инвариантно для $\Phi$. Тогда из неприводимости $\Phi$ любое нетривиальное подпространство $V$ совпадает с $V$, а отсюда $\dim V$ = 1.
\end{proof}
\subsubsection{Неприводимые комплексные представления конечных абелевых групп}
Пусть $G$ --- конечная абелева группа, $\F = \CC$. По следствию из леммы Шура любое неприводимое линейное представление $G$ имеет одномерное пространство представления, то есть $GL(V) \simeq \CC^*$.
\begin{subtheorem}
    Для конечной абелевой группы $G$ существует ровно $|G|$ различных комплексных неприводимых линейных представлений $G$.
\end{subtheorem}
\begin{proof}
    Опишем все гомоморфизмы $\Phi: G \rightarrow \CC^*$:\\
    Так как $G$ --- конечная абелева, по основной теореме о конечнопорождённых абелевых группах $G \simeq \langle a_1 \rangle_{n_1} \times ... \times \langle a_k \rangle_{n_k}$. Пусть $\Phi(a_i) = c_i$. Тогда $c_i^{n_i} = \Phi(a_i^{n_i}) = \Phi(e) = 1$, то есть $c_i$ --- комплексный корень степени $n_i$ из единицы. Так как $a_1,...,a_k$ порождают $G$, очевидно, что гомоморфизм однозначно задаётся выбором $c_1,...,c_k$. Способов выбрать $c_i$ ровно $n_i$ (количество комплексных корней степени $n_i$ из единицы) --- отсюда гомоморфизмов $n_1 \cdot ... \cdot n_k = |G|$. 
\end{proof}
\begin{example}
    $V_4 = \langle a \rangle_2 \times \langle b \rangle_2 \Longrightarrow \Phi(a) = \pm 1, \Phi(b) = \pm 1$.
    $$\begin{tabular}{c|c|c|c|c}
        \null & e & a & b & ab \\ \hline
        $\Phi_1 = I$ & 1 & 1 & 1 & 1 \\ \hline
        $\Phi_2$ & 1 & -1 & 1 & -1 \\ \hline
        $\Phi_3$ & 1 & 1 & -1 & -1 \\ \hline
        $\Phi_4$ & 1 & -1 & -1 & 1
        \end{tabular}$$ 
\end{example}
\subsubsection{Одномерные комплексные представления группы}
Пусть $G$ --- произвольная группа.\\
Знаем, что коммутант $G' \unlhd G$ --- подгруппа такая, что $G/G'$ абелева. Рассмотрим канонический гомоморфизм $\pi: G \rightarrow G/G'$:
\begin{subtheorem}
    Если $\Psi: G/G' \rightarrow \CC^*$ --- одномерное комплексное линейное представление $G/G'$, то $\Phi = \psi \circ \pi: G \rightarrow \CC^*$ --- одномерное комплексное линейное представление $G$.
\end{subtheorem}
\begin{proof}
    Очевидно (композиция гомоморфизмов -- гомоморфизм).
\end{proof}
\begin{subtheorem}
    Пусть $G$ --- произвольная группа, $\Phi: G \rightarrow \CC^*$ --- произвольное одномерное комплексное линейное представление $G$. Тогда $\exists$ линейное представление $\Psi: G/G' \rightarrow \CC^*$ такое, что $\Phi = \Psi \circ \pi$.
\end{subtheorem}
\begin{proof}
    Заметим, что $\textup{Im } \Phi \leq \CC^* \Longrightarrow \textup{Im }\Phi$ --- абелева. \\
    Из теоремы о гомоморфизме $\textup{Im }\Phi \simeq G/\textup{Ker }\Phi$, то есть $G/\textup{Ker }\Phi$ --- абелева, а тогда $G' \subseteq \textup{Ker }\Phi$ (свойство коммутанта).\\
    Отсюда $\forall g \in G, h \in G': \ \Phi(gh) = \Phi(g)$, то есть значения $\Phi$ на всех элементах левого смежного класса $g$ по $G'$ совпадают. Поэтому корректно отображение $\Psi: G/G' \rightarrow \CC^*$, заданное по правилу $gG'\mapsto \Phi(g)$. Это гомоморфизм:
    \[\Psi(g_1G' \cdot g_2G') = \Psi(g_1g_2G') = \Phi(g_1g_2) = \Phi(g_1)\Phi(g_2) = \Psi(g_1G')\Psi(g_2G')\] 
    При этом $\forall g \in G: \ g \overset{\pi}{\mapsto} gG' \overset{\Psi}{\mapsto} \Phi(g)$, то есть $\Phi = \Psi \circ \pi$.
\end{proof}
\begin{consequense}
    Если $G$ конечна, то одномерных комплексных линейных представлений $G$ ровно $|G/G'|$.
\end{consequense}
\begin{proof}
    Из двух предыдущих утверждений имеем взаимно однозначное соответствие между одномерными комплексными линейными представлениями $G$ и $G/G'$. Если $G$ конечна, то $G/G'$ --- конечная абелева, а тогда она имеет ровно $|G/G'|$ представлений.
\end{proof}
\subsection{Вполне приводимые линейные представления}
\begin{definition} Сумма линейных представлений:
\begin{enumerate}
    \item \textbf{Внутренняя сумма линейных представлений}\\
    Пусть $\Phi: G \rightarrow GL(V)$ --- линейное представление $G$, и пусть $V = U \oplus W$, где $U$ и $W$ инвариантны относительно $\Phi$. Тогда говорят, что $\Phi$ есть сумма (внутренняя) представлений $\Phi|_U$ и $\Phi|_W$.\\
    Заметим, что если выбрать базисы $\E_U = \{e_1,...,e_m\}$ в $U$, $\E_W = \{e_{m+1},...,e_n\}$ в $W$, то в базисе $\E_V = \{e_1,...,e_n\}$ пространства $V$ матрица оператора $\Phi(g)$ для любого $g \in G$ имеет вид $A_g = \begin{pmatrix} A_{U, g} &\vline&0 \\ \hline 0&\vline&A_{W, g} \end{pmatrix}$, где $A_{U, g}$ --- матрица $\Phi|_U$ в базисе $\E_U$, а $A_{W, g}$ --- матрица $\Phi|_W$ в базисе $\E_W$.
    \item \textbf{Внешняя сумма линейных представлений}\\
    Пусть $U, W$ --- векторные пространства над полем $\F$, и пусть заданы линейные представления $\Psi_1: G \rightarrow GL(U)$ и $\Psi_2: G \rightarrow GL(W)$. Обозначим $V = U \oplus W$ --- внешняя прямая сумма $U$ и $W$. Тогда линейное представление $\Phi: G \rightarrow GL(V)$, заданное по правилу $\Phi(g)(u, w) = (\Psi_1(g)u, \Psi_2(g)w)$ (очевидно, что это гомоморфизм), называется (внешней) суммой линейных представлений $\Psi_1, \Psi_2$ и обозначается $\Phi = \Psi_1 + \Psi_2$.\\
    Аналогично, если выбрать базисы $\E_U = \{e_1,...,e_m\}$ в $U$, $\E_W = \{e_{m+1},...,e_n\}$ в $W$, то в базисе $\E_V = \{(e_1, 0),...,(e_m, 0), (0, e_{m+1}),...,(0, e_n)\}$ пространства $V$ матрица оператора $\Phi(g)$ для любого $g \in G$ имеет вид $\begin{pmatrix} A_{\Psi_1} &\vline&0 \\ \hline 0&\vline&A_{\Psi_2} \end{pmatrix}$, где $A_{\Psi_1}$ --- матрица $\Psi_1$ в базисе $\E_U$, а $A_{\Psi_2}$ --- матрица $\Psi_2$ в базисе $\E_W$.
\end{enumerate}
\end{definition}

\begin{definition}
    Линейное представление $\Phi: G \rightarrow GL(V)$ называется вполне приводимым, если для любого подпространства $U \subseteq V$, инвариантного относительно $\Phi$, существует такое подпространство $W \subseteq V$, инвариантное относительно $\Phi$, что $V = U \oplus W$.
\end{definition}
\begin{remark}
    Любое неприводимое линейное представление вполне приводимо --- для него инвариантные подпространства --- только $V$ и $\{0\}$, причём $V \oplus \{0\} = V$.
\end{remark}
\begin{examples} (Напомним, что при фиксированном базисе $V$ линейные представления взаимно однозначно соответствуют матричным, где в соответствие каждому оператору поставлена его матрица).
    \begin{enumerate}
        \item $\Phi: \R \rightarrow GL_2(\CC) \ \ \Phi(t) = \begin{pmatrix} e^{it}&0 \\ 0&e^{-it} \end{pmatrix}$ --- вполне приводимо.\\
        (поворот унитарного пространства --- ортогональный оператор, то есть если $U$ инвариантно, то $U^\perp$ инвариантно, причём $V = U \oplus U^\perp$)
        \item $\Phi: \R \rightarrow GL_2(\CC) \ \ \Phi(t) = \begin{pmatrix} 1&t \\ 0&1 \end{pmatrix}$ --- приводимо, но не вполне приводимо:\\
        Рассмотрим базис $\tilde{C} = \{e_1, e_2\}$ двумерного пространства линейного представления $V$, в котором записаны матрицы операторов.\\
        Заметим, что $\langle e_1 \rangle$ --- инвариантное подпространство относительно $\Phi$. Однако никакое подпространство $W = \langle \mu e_1 + e_2 \rangle$ не инвариантно:
        \[\forall t > 0: \ \Phi(t)\begin{pmatrix} \mu \\ 1 \end{pmatrix} = \begin{pmatrix} 1&t \\ 0&1 \end{pmatrix}\begin{pmatrix} \mu \\ 1 \end{pmatrix} = \begin{pmatrix} \mu + t \\ 1 \end{pmatrix} \neq \lambda \begin{pmatrix} \mu \\ 1 \end{pmatrix}\]
    \end{enumerate}    
\end{examples}
\setcounter{lemcount}{0}
\begin{definition}
    Пусть $\Phi: G \rightarrow GL(V)$ --- линейное представление $G$. Линейное представление $\tilde{\Phi}$ называется подпредставлением $\Phi$, если $\exists \tilde{V} \subseteq V$ --- такое подпространство, инвариантное относительно $\Phi$, что $\tilde{\Phi} = \Phi|_{\tilde{V}} : G \rightarrow GL(\tilde{V})$.
\end{definition}
\begin{lemmanum}
    Любое подпредставление вполне приводимого представления вполне приводимо.
\end{lemmanum}
\begin{proof}
    Пусть $\Phi: G \rightarrow GL(V)$ --- вполне приводимое линейное представление, $\tilde{\Phi}: G \rightarrow GL(\tilde{V})$ --- его подпредставление ($\tilde{V}$ --- подпространство $V$). Рассмотрим произвольное подпространство $U \subseteq \tilde{V}$, инвариантное относительно $\tilde{\Phi}$. $U$ является подпространством и для $V$, а тогда $\exists$ инвариантное относительно $\Phi$ подпространство $W \subseteq V$ такое, что $V = U \oplus W$ (т.к. $\Phi$ вполне приводимо). Тогда если обозначить $\tilde{W} = W \cap \tilde{V}$, то $\tilde{W}$ --- подпространство $\tilde{V}$, инвариантное относительно $\tilde{\Phi}$.\\
    Осталось показать, что $\tilde{V} = U \oplus \tilde{W}$, то есть что $\forall x \in \tilde{V}$ единственным образом раскладывается как $x = u + \tilde{w}, u \in U, \tilde{w} \in \tilde{W}$. Действительно, из $V = U \oplus W$ знаем, что $x$ как элемент $V$ единственно раскладывается в сумму $x = u + w$, где $u \in U, w \in W$, однако $w = x - u \in \tilde{V}$, то есть $w \in \tilde{V} \cap W = \tilde{W}$. Значит, такое разложение существует и единственно, что и требовалось.
\end{proof}
\begin{lemmanum}
    Пусть $\Phi: G \rightarrow GL(V)$ --- вполне приводимое линейное представление. Тогда $\Phi$ раскладывается в сумму неприводимых линейных представлений (возможно, одного).
\end{lemmanum}
\begin{proof}
    Индукция по $n = \dim V$:\\
    База: $n = 1 \Longrightarrow \Phi$ неприводимо;\\
    Шаг: Пусть $V_1$ --- минимальное ненулевое инвариантное подпространство линейного представления $\Phi$. Так как $\Phi$ вполне приводимо, $\exists$ инвариантное дополнение $W: V = V_1 \oplus W \Longrightarrow \Phi = \Phi_{V_1} + \Phi_W$. При этом $\Phi_{V_1}$ неприводимо из минимальности $V_1$ (нет нетривиальных инвариантных подпространств меньшей размерности), а $\Phi_W$ по лемме 1 вполне приводимо, и притом меньшей размерности --- раскладывается в искомую сумму по предположению индукции. Значит, $\Phi$ также раскладывается в искомую сумму.  
\end{proof}
\begin{example}
    Если $\dim V = n > 1$ и $\Phi: G \rightarrow GL(V)$ такое, что $\Phi(g) = I \ \forall g \in G$, то все подпространства $V$ являются инвариантными для $\Phi$, откуда $\Phi$ вполне приводимо. Тогда $\Phi$ раскладывается в сумму одномерных представлений, причём не единственным образом --- в зависимости от выбора базиса:
    \[\E = \{e_1,...,e_n\} \Longrightarrow V = \langle e_1 \rangle \oplus ... \oplus \langle e_n \rangle \Longrightarrow \Phi = \Phi_1 + ... + \Phi_n;\]
    \[\tilde{\E} = \{\tilde{e}_1,...,\tilde{e}_n\} \Longrightarrow V = \langle \tilde{e}_1 \rangle \oplus ... \oplus \langle \tilde{e}_n \rangle \Longrightarrow \Phi = \tilde{\Phi}_1 + ... + \tilde{\Phi}_n;\]
\end{example}
\begin{lemmanum}
    Пусть $V = V_1 + ... + V_k$, где $V_i$ --- минимальные ненулевые подпространства, инвариантные относительно линейного представления $\Phi: G \rightarrow GL(V)$. Тогда $\Phi$ вполне приводимо, причём для произвольного инвариантного подпространства $U \subseteq V$ существует инвариантное дополнение вида $W = \sum \limits_{i \in I} V_i$ для некоторого $I \subseteq \{1,...,k\}$. 
\end{lemmanum}
\begin{proof}
    Обозначим $V_I = \sum \limits_{i \in I} V_i$, где $I \subseteq \{1,...,k\}$. Очевидно, что $V_I$ --- инвариантное подпространство (как сумма инвариантных). Рассмотрим произвольное инвариантное $U \subseteq V$ и возьмём $I = \{i \ |\ U \cap V_i = \{0\}\}$. Докажем, что $V = U \oplus V_I$ (по построению $U \cap V_I = \{0\}$, поэтому достаточно $V = U + V_I$):\\
    Рассмотрим $j \notin I$. Тогда $V_{I \cup \{j\}} = V_I + V_j$, а также $U \cap V_{I \cup \{j\}} \neq 0 \Longrightarrow \exists u \in U : u = \sum \limits_{i \in I} v_i + v_j$, причём $v_j \neq 0$. Тогда $v_j  = u - \sum \limits_{i \in I} v_i \in U + V_I$, то есть $V_j \cap (U + V_i) \neq \{0\}$. При этом пересечение инвариантных подпространств инвариантно, то есть $V_j \cap (U + V_i)$ --- инвариантное подпространство в $V_j$. Тогда из минимальности $V_j$ это подпространство совпадает с $V_j$, а значит, $V_j \subseteq U + V_I$.\\
    Проведя такое рассуждение для всех $j \in \{1,...,k\} \setminus I$, получим $V \subseteq U + V_I$, а отсюда уже $V = U \oplus V_i$.
\end{proof}
\begin{example}
    Пусть $\dim V = n, \textup{char } \F = 0$. Рассмотрим мономиальное линейное представление $M: S_n \rightarrow GL(V)$ для некоторого базиса $\E = \{e_1,...,e_n\} : \  M(\sigma)e_i = e_{\sigma(i)}$. Ранее доказывали, что относительно него инвариантны $U = \langle e_1 + ... + e_n \rangle$ и $W = \{x = \sum \limits_i x_ie_i \ | \ \sum x_i = 0\}$, причём $V = U \oplus W$. Отсюда по лемме 3 $M$ вполне приводимо.
\end{example}
\begin{theorem}(Машке)\\
    Пусть $G$ --- произвольная конечная группа, $\F$ --- поле, $\textup{char } \F \nmid |G|$ (в частности, верно при $\textup{char } \F = 0$), $V$ --- векторное пространство над $\F$. Тогда произвольное линейное представление $\Phi: G \rightarrow GL(V)$ вполне приводимо.
\end{theorem}
\begin{proof}
    Пусть $U$ --- произвольное подпространство $V$, инвариантное относительно $\Phi$. Докажем, что к $U$ существует инвариантное дополнение $W \subseteq V: \ V = U \oplus W$.\\
    Рассмотрим произвольное дополнение $U$ до $V$ --- подпространство $U'$ такое, что $V = U \oplus U'$. Обозначим за $\psi$ линейный оператор проектирования на $U'$ вдоль $U$, то есть $\psi: V \rightarrow V$ такой, что $\forall v: \ v = u + u' \Rightarrow \psi(v) = u'$.\\
    Обозначим очевидные свойства $\psi:$
    \begin{enumerate}[label=(\roman*)]
        \item $\forall u \in U: \ \psi(u) = 0$;
        \item $\forall v \in V: \ v - \psi(v) \in U$.
    \end{enumerate}
    Построим новый линейный оператор $\tilde{\psi}$ по правилу $\tilde{\psi} = \frac{1}{|G|}\sum \limits_{h \in G} \Phi(h)\circ \psi \circ \Phi(h^{-1})$ (из условия $\textup{char } \F \nmid |G|$ возможно деление на $|G|$, т.к. оно не соответствует $0_{\F}$) и докажем, что $W = \textup{Im } \tilde{\psi}$ подойдёт под условие:
    \begin{enumerate}
        \item Инвариантность $W$ относительно $\Phi$:
        \begin{enumerate}
            \item Докажем, что $\forall g \in G: \ \Phi(g)\tilde{\psi} = \tilde{\psi}\Phi(g)$:
            \begin{center}
                $\Phi(g)\tilde{\psi}\Phi(g^{-1}) = \Phi(g)\left(\frac{1}{|G|}\sum \limits_{h \in G} \Phi(h) \psi \Phi(h^{-1})\right)\Phi(g^{-1}) = \frac{1}{|G|}\sum \limits_{h \in G} \Phi(g)\Phi(h) \psi \Phi(h^{-1})\Phi(g^{-1}) = \frac{1}{|G|}\sum \limits_{h \in G} \Phi(gh) \psi \Phi((gh)^{-1}) = \frac{1}{|G|}\sum \limits_{t \in G} \Phi(t) \psi \Phi(t^{-1}) = \tilde{\psi}$
            \end{center}
            \item Докажем инвариантность $\textup{Im } \tilde{\psi}$ относительно $\Phi$:
            \begin{center}
                $\forall w \in \textup{Im } \tilde{\psi}: \exists v\in V: w = \tilde{\psi}(v)$
            \end{center}
            \begin{center}
                $\forall g \in G: \ \Phi(g)(w) = \Phi(g)(\tilde{\psi}(v)) = \tilde{\psi}(\Phi(g)(v)) \in \textup{Im } \tilde{\psi}$
            \end{center}
        \end{enumerate}
        Отсюда $W$ инвариантно относительно $\Phi$;
        \item Докажем, что $V = U + W$:\\
        Заметим, что $\forall v \in V: \ v = \frac{1}{|G|}\sum \limits_{h \in G} v = \frac{1}{|G|}\sum \limits_{h \in G} \Phi(h)\Phi(h^{^-1})v$. Тогда:
        \begin{center}
            $v - \tilde{\psi}(v) = \frac{1}{|G|}\sum \limits_{h \in G} \Phi(h)\Phi(h^{^-1})v - \frac{1}{|G|}\sum \limits_{h \in G} \Phi(h) \psi \Phi(h^{-1})v = \frac{1}{|G|}\sum \limits_{h \in G} \Phi(h) (\Phi(h^{-1})v - \psi(\Phi(h^{-1})v))$
        \end{center}
        При этом $\Phi(h^{-1})v - \psi(\Phi(h^{-1})v) \in U$ по свойству (ii), а в силу инвариантности $U$ $\Phi(h) (\Phi(h^{-1})v - \psi(\Phi(h^{-1})v)) \in U$ для любого $h \in G$. Значит, $\forall v \in V: \ v - \tilde{\psi}(v) \in U \Longrightarrow v = u + \tilde{\psi}(v)$. Отсюда $V = U + W$.
        \item $U \cap W = \{0\}$:
        \begin{enumerate}
            \item Заметим, что $\forall u \in U: \ \tilde{\psi}(u) = 0$:\\
            $\tilde{\psi}(u) = \frac{1}{|G|}\sum \limits_{h \in G} \Phi(h) \psi \Phi(h^{-1})u$. Из инвариантности $U$ $\Phi(h^{-1})u \in U$, а тогда $\psi(\Phi(h^{-1})u) = 0 \ \ \forall h \in G$, то есть все слагаемые равны 0.
            \item Докажем, что $\tilde{\psi}^2 = \tilde{\psi}:$\\
            $\forall v \in V$ по (2) имеем $v - \tilde{\psi}(v) \in U$, а тогда по (3а) $\tilde{\psi}(v - \tilde{\psi}(v)) = 0 \Longrightarrow \tilde{\psi}(v) - \tilde{\psi}^2(v) = 0 \Longrightarrow \tilde{\psi}(v) = \tilde{\psi}^2(v)$.
            \item Докажем, что $U \cap W = \{0\}$:\\
            Рассмотрим $v \in U \cap W$. Тогда $\tilde{\psi}(v) = 0$ по (3а), а также $\exists v':\ v = \tilde{\psi}(v')$, т.к. $v \in W = \textup{Im }\tilde{\psi}$. Отсюда $\tilde{\psi}^2(v') = \tilde{\psi}(v) = 0$, а тогда по (3b) $\tilde{\psi}(v') = 0$, то есть $v = 0$. Отсюда $U \cap W = \{0\}$.
        \end{enumerate}
    \end{enumerate} 
    Значит, для $\Phi$ выполняется определение вполне приводимости.
\end{proof}
\begin{consequensenum}
    Любое вещественное (комплексное) линейное представление конечной группы является вполне приводимым.
\end{consequensenum}
\begin{consequensenum}
    Любое комплексное линейное представление конечной абелевой группы раскладывается в сумму одномерных линейных представлений (то есть в $V$ существует базис $\E$ такой, что $\forall g \in G$ матрица оператора $\Phi(g)$ диагональна в $\E$).
\end{consequensenum}
\begin{example}
    Найдём число неэквивалентных двумерных комплексных линейных представлений $\Z_2$. $\Z_2 = \langle a \rangle_2$ --- конечная абелева группа, то есть по следствию $2$ любое комплексное линейное представление $\Phi$ представимо в виде суммы одномерных линейных представлений $\Phi_1 + \Phi_2$. Таким образом, матрица $\Phi(a)$ имеет вид $\begin{pmatrix} \Phi_1(a)&0 \\ 0&\Phi_2(a) \end{pmatrix}$,  а также $\Phi_i(a) = \pm 1$, т.к. $a^2 = e$. Значит, $\Phi(a) = \begin{pmatrix} \pm1&0 \\ 0&\pm1 \end{pmatrix}$ --- 4 различные матрицы. Осталось заметить, что линейные представления неэквивалентны, если ЖНФ матриц $\Phi(a)$ различны --- отсюда случаи $\begin{pmatrix} 1&0 \\ 0&-1 \end{pmatrix}$ и $\begin{pmatrix} -1&0 \\ 0&1 \end{pmatrix}$ эквивалентны, что оставляет нам 3 неэквивалентных (разные собственные значения) линейных представления.
\end{example}

\subsubsection{Ортогональные (унитарные) представления}
\setcounter{lemcount}{0}
\begin{definition}
    Вещественное (комплексное) линейное представление $\Phi: G \rightarrow GL(V)$ называется ортогональным (унитарным), если в $V$ можно ввести скалярное произведение так, что $\forall g \in G: \Phi(g)$ --- ортогональный (унитарный) оператор относительно этого скалярного произведения.
\end{definition}
\begin{lemma}
    Любое ортогональное (унитарное) линейное представление является вполне приводимым.
\end{lemma}
\begin{proof}
    Пусть $U$ --- произвольное подпространство $V$, инвариантное относительно $\Phi$. Тогда из ортогональности $\Phi(g)$ для всех $g \in G$ получаем, что $U^\perp$ также инвариантно относительно $\Phi$, причём $V = U \oplus U^{\perp}$. Значит, $\Phi$ вполне приводимо.
\end{proof}
\begin{theorem}
    Любое вещественное (комплексное) линейное представление конечной группы является ортогональным (унитарным).
\end{theorem}
\begin{proof}
    Рассмотрим случай $\F = \R$.\\
    Рассмотрим произвольную симметрическую положительно определённую билинейную функцию $\beta: V \times V \rightarrow \R$. Определим отображение \[(x|y) := \sum \limits_{h \in G} \beta(\Phi(h)x, \Phi(h)y)\]
    Докажем, что $(x|y)$ --- искомое скалярное произведение:
    \begin{enumerate}
        \item Симметричность и билинейность --- очевидны из симметричности и билинейности $\beta$;
        \item Положительная определённость --- докажем, что $\forall x \neq 0: \ (x|x) > 0$:\\
        По определению $(x|x) = \sum \limits_{h \in G} \beta(\Phi(h)x, \Phi(h)x)$. Из положительной определённости $\beta$ знаем, что $\beta(\Phi(h)x, \Phi(h)x) \geqslant 0$, причём $\beta(x, x) = 0 \Longleftrightarrow x = 0$. Остаётся заметить, что при $h = e: \ \beta(\Phi(h)x, \Phi(h)x) = \beta(x, x) > 0$, а значит, $(x|x) > 0$ (все слагаемые $\geqslant 0$, причём хотя бы одно $> 0$).
        \item Докажем, что $\forall g \in G: \Phi(g)$ ортогонально относительно $(x|y)$, то есть $(\Phi(g)x, \Phi(g)y) = (x|y)$:
        \[(\Phi(g)x, \Phi(g)y) = \sum \limits_{h \in G} \beta(\Phi(h)\Phi(g)x, \Phi(h)\Phi(g)y) = \sum \limits_{h \in G} \beta(\Phi(hg)x, \Phi(hg)y) =\]
        \[= \sum \limits_{t \in G} \beta(\Phi(t)x, \Phi(t)y) = (x|y)\]
    \end{enumerate} 
    Значит, $\Phi$ --- ортогональное линейное представление.\\
    Случай $\F = \CC$: рассмотрим произвольную эрмитову положительно определённую полуторалинейную функцию $\gamma: V \times V \rightarrow \CC$. Тогда аналогично определим отображение $(x|y)$ --- его свойства и унитарность $\Phi(g)$ проверяются аналогично свойствам $\beta$ и ортогональности $\Phi(g)$. 
\end{proof}
\subsection{Неприводимые линейные представления над \texorpdfstring{$\CC$}{полем комплексных чисел}}
Сформулируем две теоремы, описывающие поведение неприводимых комплексных представлений произвольной конечной группы. Они будут доказаны в разделе 9.5 (а именно \hyperlink{chtarget}{здесь}), а этот раздел посвящён их практическому применению. 
\begin{theoremnum}
    Пусть $G$ --- конечная группа, $r$ --- количество классов сопряжённости в $G$. Тогда существует ровно $r$ попарно неэквивалентных неприводимых комплексных линейных представлений $G$ над $\CC$.
\end{theoremnum}
\begin{theoremnum}
    Пусть $G$ --- конечная группа, $\Phi_1,...,\Phi_r$ --- все её попарно неэквивалентные неприводимые комплексные линейные представления, $n_1,...,n_r$ --- их размерности. Тогда $|G| = n_1^2 + ... + n_r^2$.
\end{theoremnum}
\begin{consequense}
    Пусть $G$ --- конечная группа. Тогда $G$ имеет только одномерные неприводимые комплексные линейные представления $\Longleftrightarrow G$ абелева.
\end{consequense}
\begin{proof}
    $ \\ \Longleftarrow$ --- было доказано как следствие леммы Шура;
    $ \\ \Longrightarrow$ --- по теореме 2: $|G| = \sum \limits_{i = 1}^r n_i = \sum \limits_{i = 1}^r 1 = r$ (где $r$ --- число классов сопряжённости по теореме 1), то есть все классы сопряжённости состоят из одного элемента. Значит, $\forall g_1, g_2 \in G: g_2g_1g_2^{-1} = g_1 \Longrightarrow g_2g_1 = g_1g_2$, то есть $G$ абелева. 
\end{proof}
Опишем неприводимые линейные представления над $\CC$ некоторых групп:
\begin{enumerate}
    \item $G = S_3$:
    \begin{enumerate}
        \item $\dim V = 1$: $|G/G'| = |S_3/A_3| = 2$\\
        Одномерные комплексные представления $G$ уже умеем классифицировать --- они соответствуют представлениям $G/G'$. В данном случае их два --- $\forall \sigma \in S_3: \Phi_1(\sigma) = I, \Phi_2(\sigma) = \sgn \sigma \cdot I$;
        \item $\dim V = 2$: Заметим, что $S_3 \simeq D_3 = \textup{Sym } \triangle \subset GL_2(\R) \subset GL_2(\CC)$. Так зададим двумерное линейное представление $\Phi_3$ --- в примерах раздела 2 данной главы доказана неприводимость такого представления. 
    \end{enumerate}
    Остаётся заметить, что $|S_3| = 6 = 1^2 + 1^2 + 2^2$, причём уже найдены одно двумерное и два неэквивалентных одномерных  неприводимых комплексных линейных представления. Отсюда по теореме 2 других представлений быть не может, то есть любое неприводимое комплексное линейное представление $S_3$ эквивалентно одному из представлений $\Phi_1, \Phi_2, \Phi_3$;
    \item $G = S_4:$
    \begin{enumerate}
        \item $\dim V = 1$: $|G/G'| = |S_4/A_4| = 2$\\
        Одномерные комплексные представления $S_4$, аналогично $S_3$, имеют вид $\forall \sigma \in S_3: \tilde{\Phi}_1(\sigma) = I, \tilde{\Phi}_2(\sigma) = \sgn \sigma \cdot I$;
        \item $\dim V = 2$: Для начала докажем, что $S_4/V_4 \simeq S_3$:\\
        Рассмотрим произвольный элемент $S_4/V_4$ --- это смежный класс вида $H = \sigma V_4$. Заметим, что в $V_4$ четыре элемента, причём все они переводят 4 в различные элементы --- значит, для всех четырёх $\tilde{\sigma} \in H$ значения $\tilde{\sigma}(4)$ различны, то есть $H$ содержит ровно один элемент $\sigma'$ такой, что $\sigma'(4) = 4$. Отсюда каждый элемент можно единственным образом домножить справа на элемент из $V_4$, чтобы результат оставлял 4 на месте.\\
        Рассмотрим отображение $\phi: S_4/V_4 \rightarrow S_3$, которое переводит каждый смежный класс в его элемент, оставляющий 4 на месте. Такое отображение биективно по соображениям выше, а также, очевидно, является гомоморфизмом ($\phi(\sigma V_4)\phi(\tau V_4) = \sigma' \tau' = \phi(\sigma\tau V_4)$, так как $\sigma'\tau' \in \sigma\tau V_4$ и оставляет 4 на месте). Значит, $\phi$ --- изоморфизм.\\
        Поэтому можем задать линейное представление $\tilde{\Phi}_3: S_4\rightarrow GL_2(\CC)$:
        \[S_4 \overset{\pi}{\longrightarrow} S_4/V_4 \overset{\phi}{\longrightarrow} S_3 \overset{\Phi_3}{\longrightarrow} GL_2(\CC)\] 
        где $\Phi_3$ --- линейное представление из предылущего пункта. Неприводимость $\tilde{\Phi}_3$ очевидно следует из неприводимости $\Phi_3$ (инвариантное подпространство для $\tilde{\Phi}_3$ было бы инвариантно и для $\Phi_3$).
        \item $\dim V = 3:$ В разделе 2 данной главы (примеры 5,6) приводились два трёхмерных линейных представления $S_4$:
        \[\tilde{\Phi}_4: \ S_4 \simeq \textup{Sym}^+ K \subset O_3 \subset GL(\E^3) \text{  (где K --- куб)}\]
        \[\tilde{\Phi}_5: \ S_4 \simeq \textup{Sym}^+ T \subset O_3 \subset GL(\E^3) \text{  (где T --- правильный тетраэдр)}\]
        Там же была доказана неэквивалентность и неприводимость этих представлений.
    \end{enumerate}
    Остаётся заметить, что $|S_4| = 24 = 1^2 + 1^2 + 2^2 + 3^2 + 3^2$, причём уже найдены два трёхмерных, одно двумерное и два неэквивалентных одномерных неприводимых комплексных линейных представления. Отсюда по теореме 2 других представлений быть не может, то есть любое неприводимое комплексное линейное представление $S_3$ эквивалентно одному из представлений $\tilde{\Phi}_1 - \tilde{\Phi}_5$.
\end{enumerate}
\begin{lemma}
    Пусть $V_1, V_2$ --- векторные пространства над $\CC$, и $\Phi_1: G \rightarrow GL(V_1)$, $\Phi_2: G \rightarrow GL(V_2)$ --- неприводимые линейные представления произвольной группы $G$.\\
    Пусть $\exists \phi: V_1 \rightarrow V_2$ --- линейное отображение такое, что $\forall g \in G: \Phi_2(g) \circ \phi = \phi \circ \Phi_1(g)$. Тогда:
    \begin{enumerate}
        \item Если $\Phi_1 \not \approx \Phi_2$, то $\phi = 0$;
        \item Если $\Phi_1 \approx \Phi_2$, то либо $\phi = 0$, либо $\phi$ --- изоморфизм;
        \item Если $V_1 = V_2$ и $\Phi_1 = \Phi_2$, то $\phi = \lambda I$ для некоторого $\lambda \in \CC$.
    \end{enumerate}
\end{lemma}
\begin{proof}
    Пункт 3 --- частный случай леммы Шура для $\CC$;\\
    \tab $2)$ Предположим, что $\phi \neq 0$, то есть $\textup{Ker } \phi \neq V_1, \textup{Im } \phi \neq \{0\}$.
    \[\forall x \in \textup{Ker } \phi: \ \phi(\Phi_1(g)x) = \Phi_2(g)\phi(x) = 0 \Longrightarrow \Phi_1(g)x \in \textup{Ker } \phi;\]
    \[\forall y \in \textup{Im } \phi \ (y = \phi(x)): \ \Phi_2(g)y = \Phi_2(g)\phi(x) = \phi(\Phi_1(g)x) \Longrightarrow \Phi_2(g)y \in \textup{Im } \phi\]
    Отсюда $\textup{Ker } \phi$ --- инвариантное пространство для $\Phi_1$, а $\textup{Im } \phi$ --- для $\Phi_2$. Из их неприводимости и нетривиальности $\phi$ следует $\textup{Ker } \phi = \{0\}$, $\textup{Im }\phi = V_2$, то есть $\phi$ --- изоморфизм.\\
    \tab $1)$ Аналогично пункту 2, но если $\phi$ --- изоморфизм, то $\Phi_1 \approx \Phi_2$, что противоречит условию.
\end{proof}
\setcounter{concount}{0}
\begin{consequensenum}
    Пусть $V_1, V_2$ --- векторные пространства над $\CC$, и $\Phi_1: G \rightarrow GL(V_1)$, $\Phi_2: G \rightarrow GL(V_2)$ --- неприводимые линейные представления конечной группы $G$, $\psi: V_1 \rightarrow V_2$ --- произвольное линейное отображение.\\
    Рассмотрим "усреднённое" \ линейное отображение:
    \[\tilde{\psi} = \frac{1}{|G|}\sum \limits_{g \in G} \Phi_2(g)\circ\psi\circ\Phi_1(g^{-1})\]
    Тогда:
    \begin{enumerate}
        \item Если $\Phi_1 \not \approx \Phi_2$, то $\tilde{\psi} = 0$;
        \item Если $V_1 = V_2$ и $\Phi_1 = \Phi_2$, то $\tilde{\psi} = \lambda I$, где $\lambda = \frac{\textup{tr } \psi}{\dim V_1}$.
    \end{enumerate}
\end{consequensenum}
\begin{proof}
    Докажем, что $\forall g \in G: \Phi_2(g) \circ \tilde{\psi} = \tilde{\psi} \circ \Phi_1(g)$:
    \[\Phi_2(g)\tilde{\psi}\Phi_1(g^{-1}) = \frac{1}{|G|}\sum \limits_{h \in G} \Phi_2(g)\Phi_2(h) \psi \Phi_1(h^{-1})\Phi_1(g^{-1}) =\]\[= \frac{1}{|G|}\sum \limits_{h \in G} \Phi_2(gh) \psi \Phi_1((gh)^{-1}) = \frac{1}{|G|}\sum \limits_{t \in G} \Phi_2(t) \psi \Phi_1(t^{-1}) = \tilde{\psi}\]
    Отсюда можем применить доказанную лемму.\\
    Осталось показать, что $\lambda = \frac{\textup{tr } \psi}{\dim V_1}$. Рассмотрим $\textup{tr } \tilde{\psi}$:\\
    С одной стороны, $\tilde{\psi} = \lambda I$, то есть $\textup{tr } \tilde{\psi} = \lambda \cdot \dim V_1$;\\
    С другой стороны, из аддитивности следа $\textup{tr } \tilde{\psi} = \frac{1}{|G|}\sum \limits_{g \in G} \textup{tr }(\Phi_1(g)\psi\Phi_1(g^{-1})) = \frac{1}{|G|}\sum \limits_{g \in G} \textup{tr } \psi = \textup{tr } \psi$. Значит, $\textup{tr } \psi = \textup{tr } \tilde{\psi} = \lambda \cdot \dim V_1$, то есть $\lambda = \frac{\textup{tr } \psi}{\dim V_1}$.
\end{proof}
\begin{consequensenum} (Следствие 1 в матричной форме)\\
    $\Phi_1: G \rightarrow GL(V_1)$, $\Phi_2: G \rightarrow GL(V_2)$ --- неприводимые комплексные линейные представления конечной группы $G$, $\dim V_1 = n_1$, $\dim V_2 = n_2$, $\E_1, \E_2$ --- некоторые базисы $V_1, V_2$ соответственно. Обозначим для всех $g \in G$ матрицу оператора $\Phi_1(g)$ в $\E_1$ как $A(g) = (a_{ij}(g))$, а матрицу $\Phi_2(g)$ в $\E_2$ --- как $B(g) = (b_{ij}(g))$\\
    Тогда $\forall i, j, i_0, j_0$:
     \begin{enumerate}
        \item Если $\Phi_1 \not \approx \Phi_2$, то $\frac{1}{|G|}\sum \limits_{g \in G} b_{ii_0}(g)a_{j_0j}(g^{-1}) = 0$;
        \item Если $V_1 = V_2$ и $\Phi_1 = \Phi_2$, то $\frac{1}{|G|}\sum \limits_{g \in G} b_{ii_0}(g)a_{j_0j}(g^{-1}) = \frac{\delta_{i_0}^{j_0}\delta_i^j}{\dim V_1}$.
    \end{enumerate}
\end{consequensenum}
\begin{proof}
    Для любых $i_0, j_0$ возьмём в качестве $\psi$ из следствия 1 линейное отображение, матрица $C$ которого в базисах $\E_1, \E_2$ равна $E_{i_0, j_0}$ (матричная единица). Тогда матрица линейного отображения $\tilde{\psi}$ в базисах $\E_1, \E_2$ имеет вид
    \[\tilde{C} = \frac{1}{|G|}\sum \limits_{g \in G} B(g)CA(g^{-1})\]
    При этом элемент на $ij$-ой позиции матрицы $B(g)CA(g^{-1})$ равен
    \[\sum \limits_{k = 1}^{n_2} b_{ik}(g)[CA(g^{-1})]_{kj} = \sum \limits_{k = 1}^{n_2} b_{ik}(g)\sum \limits_{l = 1}^{n_1} c_{kl}a_{lj}(g^{-1}) = \sum \limits_{k = 1}^{n_2} \sum \limits_{l = 1}^{n_1} b_{ik}(g)c_{kl}a_{lj}(g^{-1})\]
    Ненулевым будет только слагаемое при $k = i_0, l = j_0$, и оно равно $b_{ii_0}(g)a_{j_0j}(g^{-1})$.
    Значит, $\tilde{c}_{ij} = \frac{1}{|G|}\sum \limits_{g \in G} b_{ii_0}(g)a_{j_0j}(g^{-1})$. По следствию 1:
    \begin{enumerate}
        \item $\Phi_1 \not \approx \Phi_2 \Longrightarrow \tilde{\psi} = 0 \Longrightarrow \forall i, j: \ c_{ij} = \frac{1}{|G|}\sum \limits_{g \in G} b_{ii_0}(g)a_{j_0j}(g^{-1}) = 0$;
        \item $V_1 = V_2, \Phi_1 = \Phi_2 \Longrightarrow \tilde{\psi} = \lambda I \Longrightarrow \tilde{C} = \lambda E \Longrightarrow \tilde{c}_{ij} = \lambda \delta_i^j$.\\
        $\lambda = \frac{\textup{tr } \psi}{\dim V_1} = \frac{\textup{tr } C}{\dim V_1} = \frac{\delta_{i_0}^{j_0}}{\dim V_1} \Longrightarrow \tilde{c}_{ij} = \frac{\delta_{i_0}^{j_0}\delta_i^j}{\dim V_1}$.
    \end{enumerate}
\end{proof}
\subsection{Характеры комплексных линейных представлений}
\begin{definition}
    Пусть $\Phi$ --- комплексное линейное представление группы $G$.\\
    Отображение $\rchi_\Phi: G \rightarrow \CC$ такое, что $\forall g \in G: \ \rchi_\Phi(g) = \textup{tr } \Phi(g)$, называется характером линейного представления $\Phi$.
\end{definition}
\begin{remark}\tab
    \begin{enumerate}
        \item Если $\lambda_1(g),...,\lambda_n(g)$ --- все собственные значения $\Phi(g)$ с учётом кратности, то $\rchi_\Phi(g) = \lambda_1(g) + ... + \lambda_n(g)$ (в частности, характер не зависит от базиса);
        \item Если $\Phi_1 \approx \Phi_2$, то $\rchi_{\Phi_1} = \rchi_{\Phi_2}$.
    \end{enumerate}    
\end{remark}
\begin{properties}\tab
    \begin{enumerate}
        \item $\rchi_\Phi(e) = \dim V$;
        \item $\forall g, h \in G: \rchi_\Phi(hgh^{-1}) = \rchi_\Phi(g)$ (характер равен для всех элементов одного класса сопряжённости);
        \item Если $\Phi = \Phi_1 + \Phi_2$, то $\rchi_\Phi = \rchi_{\Phi_1} + \rchi_{\Phi_2}$;
        \item Если $\textup{ord } g < \infty$, то $\rchi_\Phi(g^{-1}) = \overline{\rchi_\Phi(g)}$.
    \end{enumerate}    
\end{properties}
\begin{proof}\tab
    \begin{enumerate}
        \item Очевидно ($\Phi(e) = I$, то есть его матрица единичная);
        \item В произвольном базисе $\E$: $A(hgh^{-1}) = A(h)A(g)(A(h))^{-1}$ --- след матрицы не меняется при сопряжении (так как не зависит от базиса);
        \item Следует из того, что в произвольном базисе $\E$: $A_{\Phi} = \begin{pmatrix} A_{\Phi_1} &\vline&0 \\ \hline 0&\vline&A_{\Phi_2} \end{pmatrix}$; 
        \item Пусть $\textup{ord } g = k$. Тогда $\textup{ord } \Phi(g) = m \mid k$ (т.к. $\Phi(g)^k = I$).\\
        Рассмотрим матрицу $A(g)$ в жордановом базисе. Так как $A(g)^m = E$, все собственные значения $\lambda_i(g)$ --- комплексные корни степени $m$ из единицы. Также из определения линейного представления $A(g^{-1}) = A(g)^{-1}$, то есть на диагонали $A(g^{-1})$ стоят $\lambda_i(g^{-1}) = \frac{\overline{\lambda_i(g)}}{|\lambda_i(g)|} = \overline{\lambda_i(g)}$, т.к. модуль корня из единицы равен 1. Тогда:
        \[\rchi_\Phi(g^{-1}) = \textup{tr} A(g^{-1}) = \sum \limits_i \lambda_i(g^{-1}) = \sum \limits_i \overline{\lambda_i(g)} = \overline{\sum \limits_i \lambda_i(g)} = \overline{\textup{tr} A(g)} = \overline{\rchi_\Phi(g)}\] 
    \end{enumerate}
\end{proof}
\begin{definition}
    Множество всех функций $f: G \rightarrow \CC$ будем обозначать как $\CC^G$.
\end{definition}
\begin{subtheorem}
    $\CC^G$ --- векторное пространство над $\CC$.
\end{subtheorem}
\begin{proof}
    Очевидно.
\end{proof}
\begin{subtheorem}
    Пусть $G$ --- конечная группа. Тогда функция \[(f_1, f_2) = \frac{1}{|G|}\sum \limits_{g \in G}f_1(g)\overline{f_2(g)}\] задаёт на $\CC^G$ скалярное произведение, т.е. $\CC^G$ с данной функцией --- эрмитово пространство.
\end{subtheorem}
\begin{proof}
    Проверим определение:
    \begin{enumerate}
        \item $\forall f \in \CC^G:\ (f, f) = \frac{1}{|G|}\sum \limits_{g \in G}f(g)\overline{f(g)} = \frac{1}{|G|}\sum \limits_{g \in G}|f(g)|^2 \geqslant 0$, причём\\
        $(f, f) = 0 \Longleftrightarrow \forall g \in G: |f(g)| = 0 \Longleftrightarrow f = 0$;
        \item $\overline{(f_1, f_2)} = \overline{\frac{1}{|G|}\sum \limits_{g \in G}f_1(g)\overline{f_2(g)}} = \frac{1}{|G|}\sum \limits_{g \in G}\overline{f_1(g)}f_2(g) = (f_2, f_1)$;
        \item $(\alpha f_1 + \beta f_2, f_3) = \frac{1}{|G|}\sum \limits_{g \in G}(\alpha f_1 + \beta f_2)(g)\overline{f_3(g)} = \\ = \frac{\alpha}{|G|}\sum \limits_{g \in G}f_1(g)\overline{f_3(g)} + \frac{\beta}{|G|}\sum \limits_{g \in G}f_2(g)\overline{f_3(g)} = \alpha(f_1, f_3) + \beta(f_2, f_3)$.
    \end{enumerate}
    Значит, $(f_1, f_2)$ задаёт на $\CC^G$ скалярное произведение.
\end{proof}
\begin{theorem}(Свойство ортогональности характеров)\\
    Пусть $\Phi_1, \Phi_2: G \rightarrow GL(V)$ --- неприводимые комплексные линейные представления конечной группы $G$. Тогда $(\rchi_{\Phi_1}, \rchi_{\Phi_2}) = \begin{cases} 1, \ \Phi_1 \approx \Phi_2 \\ 0, \ \Phi_1 \not \approx \Phi_2 \end{cases}$
\end{theorem}
\begin{proof}
    Пусть $\dim V = n$. Зафиксируем базис $\E$ пространства $V$ --- в нём $\Phi_1(g) \leftrightarrow A(g) = (a_{ij}(g))$, $\Phi_2(g) \leftrightarrow B(g) = (b_{ij}(g))$. Так как $G$ конечна, $\forall g\in G: \textup{ord } g < \infty$, то есть по свойству 4 $\overline{\rchi_{\Phi_1}(g)} = \rchi_{\Phi_1}(g^{-1})$. Тогда:
    \[(\rchi_{\Phi_2}, \rchi_{\Phi_1}) = \frac{1}{|G|}\sum \limits_{g \in G}\rchi_{\Phi_2}(g)\overline{\rchi_{\Phi_1}(g)} = \frac{1}{|G|}\sum \limits_{g \in G}\rchi_{\Phi_2}(g)\rchi_{\Phi_1}(g^{-1}) =\]\[=\frac{1}{|G|}\sum \limits_{g \in G}(\sum \limits_{i=1}^n b_{ii}(g))(\sum \limits_{j=1}^n a_{jj}(g^{-1})) = \sum \limits_{i,j = 1}^n \left(\frac{1}{|G|}\sum \limits_{g \in G}( b_{ii}(g) a_{jj}(g^{-1}))\right) = \]\[= \begin{cases} \sum \limits_{i,j = 1}^n \frac{\delta_i^j}{n}, \ \Phi_1 = \Phi_2 \\ 0, \ \Phi_1 \not \approx \Phi_2 \end{cases} = \begin{cases} 1, \ \Phi_1 = \Phi_2 \\ 0, \ \Phi_1 \not \approx \Phi_2 \end{cases}\]
    Остаётся заметить, что $\Phi_1 \approx \Phi_2 \Longrightarrow \rchi_{\Phi_1} = \rchi_{\Phi_2}$, т.е. $(\rchi_{\Phi_2}, \rchi_{\Phi_1}) = (\rchi_{\Phi_1}, \rchi_{\Phi_1}) = 1$. Теорема доказана.
\end{proof}
\begin{remark}
    Далее в разложениях линейного представления в сумму неприводимых будут группироваться эквивалентные слагаемые --- это записывается в виде
    \[\Phi = m_1\Phi_1 + ... + m_s\Phi_s \Longleftrightarrow \Phi = \undermat{m_1}{\Phi_1 + ... + \Phi_1} + ... + \undermat{m_s}{\Phi_s + ... + \Phi_s}\]
    $\\$В данной записи подразумевается, что $\Phi_i \not \approx \Phi_j$ при $i \neq j$.
\end{remark}
\begin{consequense}
    Пусть $\Phi = m_1\Phi_1 + ... + m_s\Phi_s$, где $\Phi_1,...,\Phi_s$ --- комплексные неприводимые линейные представления конечной группы $G$. Тогда:
    \begin{enumerate}
        \item $(\rchi_\Phi, \rchi_{\Phi_i}) = m_i$;
        \item $(\rchi_\Phi, \rchi_\Phi) = m_1^2 + ... + m_s^2$;
        \item Если $(\rchi_\Phi, \rchi_\Phi) = 1$, то $\Phi$ --- неприводимое.
    \end{enumerate}
\end{consequense}
\begin{proof}
    Заметим, что $\Phi = \sum \limits_i m_i\Phi_i \Longrightarrow \rchi_\Phi = \sum \limits_i m_i\rchi_{\Phi_i}$. При этом $\Phi_i \not \approx \Phi_j$ при $i \neq j$, а значит, $(\rchi_j, \rchi_i) = \delta_i^j$ из теоремы. Тогда:
    \begin{enumerate} 
        \item $(\rchi_\Phi, \rchi_{\Phi_i}) = (\sum \limits_j m_j\rchi_{\Phi_j}, \rchi_{\Phi_i}) = \sum \limits_j m_j(\rchi_j, \rchi_i) = m_i$;
        \item $(\rchi_\Phi, \rchi_\Phi) = (\sum \limits_i m_i\rchi_{\Phi_i}, \sum \limits_j m_j\rchi_{\Phi_j}) = \sum \limits_{i, j} m_im_j(\rchi_i, \rchi_j) = \sum \limits_i m_i^2$;
        \item Следует из пункта 2 --- если $(\rchi_\Phi, \rchi_\Phi) = 1$, то ненулевой коэффициент может быть только один, и он равен единице. Пусть $m_i = 1$ --- тогда $\Phi = \Phi_i$, т.е. $\Phi$ --- неприводимое. 
    \end{enumerate}
\end{proof}
\begin{consequense}
    Пусть $G$ --- конечная группа, $V$ --- векторное пространство над $\CC$, $\Phi$, $\Psi$ --- линейные представления группы $G$. Тогда если $\rchi_\Phi = \rchi_\Psi$, то $\Phi \approx \Psi$.
\end{consequense}
\begin{proof}
    По теореме Машке $\Phi$ и $\Psi$ вполне приводимы, то есть \begin{equation*}
        \Phi = m_1\Phi_1 + ... + m_s\Phi_s, \ \ \Psi = n_1 \Psi_1 + ... + n_t\Psi_t
    \end{equation*}
    При этом $\forall i: (\rchi_\Phi, \rchi_{\Phi_i}) = m_i$ из 1 пункта предыдущего следствия, то есть 
    \[(\rchi_\Psi, \rchi_{\Phi_i}) = m_i \Longrightarrow n_1(\rchi_{\Psi_1}, \rchi_{\Phi_i}) + ... + n_t(\rchi_{\Psi_t}, \rchi_{\Phi_i}) = m_i\]
    Так как эта сумма не равна нулю, среди слагаемых в разложении $\Psi$ найдётся $\Psi_j \approx \Phi_i$, а также $n_j = m_i$. Аналогичными рассуждениями для всех слагаемых обоих разложений получим, что каждое из них имеет эквивалентное в другом разложении, причём с тем же коэффициентом. Таким образом, с точностью до нумерации $\Psi = m_1\Psi_1 + ... + m_s\Psi_s$, где $\Psi_i \approx \Phi_i$. Отсюда очевидно, что $\Phi \approx \Psi$.
\end{proof}

\subsubsection{Пространство центральных функций}
\begin{definition}
    Функция $f: G \rightarrow \CC$ называется центральной, если она постоянна на классах сопряжённости $G$. Множество всех центральных функций для группы $G$ обозначается как $\rchi_\CC(G)$.
\end{definition}
\begin{example}
    Для произвольного комплексного линейного представления $\Phi$ гр. $G$ $\rchi_\Phi$ --- центральная функция.
\end{example}
\begin{subtheorem}
    $\rchi_\CC(G)$ --- подпространство в $\CC^G$.
\end{subtheorem}
\begin{proof}
    Очевидно, что $\forall \rchi_1, \rchi_2 \in \rchi_\CC(G), \lambda \in \CC$ функции $\rchi_1 + \rchi_2$ и $\lambda \rchi_1$ также являются центральными, так как постоянность на классах сопряжённости не нарушается. Также $0 \in \rchi_\CC(G)$, то есть определение подпространства выполнено.
\end{proof}
\begin{subtheorem}
    Если $r$ --- число классов сопряжённости в $G$, то $\dim \rchi_\CC(G) = r$.
\end{subtheorem}
\begin{proof}
    Если в группе $G$ есть ровно $r$ классов сопряжённости, то $G = x_1^G \sqcup ... \sqcup x_r^G$. В таком случае функции \[\Gamma_1,...,\Gamma_r,\text{ где }\Gamma_i = \begin{cases} 1,\ g \in x_i^G\\ 0,\ \text{иначе} \end{cases}\]
    очевидно, образуют базис в $\rchi_\CC(G)$.
\end{proof}
\begin{lemmanum}
    Пусть $G$ --- конечная группа, $V$ --- векторное пространство над $\CC$,\\
    $\Phi: G \rightarrow GL(V)$ --- неприводимое линейное представление, $\Gamma \in \rchi_\CC(G)$.\\
    Тогда оператор \[\Psi_{\Gamma, \Phi} = \sum \limits_{g \in G} \overline{\Gamma(g)}\Phi(g)\text{ равен }\lambda I,\text{ где }\lambda = |G|\frac{(\rchi_\Phi, \Gamma)}{\rchi_\Phi(e)}\]
\end{lemmanum}
\begin{proof}
    Докажем, что $\forall g \in G: \ \Psi_{\Gamma, \Phi} \circ \Phi(G) = \Phi(g) \circ \Psi_{\Gamma, \Phi}$:
    \[\Phi(g)\circ\Psi_{\Gamma, \Phi}\circ\Phi(g^{-1}) = \Phi(g)(\sum \limits_{h \in G} \overline{\Gamma(h)}\Phi(h))\Phi(g^{-1}) = \sum \limits_{g \in G} \overline{\Gamma(h)}\Phi(ghg^{-1}) =\]
    \[(\text{из центральности } \Gamma) \ \ = \sum \limits_{g \in G} \overline{\Gamma(ghg^{-1})}\Phi(ghg^{-1}) = \sum \limits_{\tilde{g} \in G} \overline{\Gamma(\tilde{g})}\Phi(\tilde{g}) = \Psi_{\Gamma, \Phi}\]
    Отсюда по лемме Шура $\Psi_{\Gamma, \Phi} = \lambda I$. Рассмотрим след $\Psi_{\Gamma, \Phi}$:
    \[\textup{tr } \Psi_{\Gamma, \Phi} = \textup{tr } \lambda I = \lambda \dim V = \lambda \rchi_\Phi(e)\]
    \[\textup{tr } \Psi_{\Gamma, \Phi} = \sum \limits_{g \in G} \overline{\Gamma(g)}\cdot\textup{tr } \Phi(g) = \sum \limits_{g \in G} \overline{\Gamma(g)}\rchi_\Phi(g) = |G|\cdot(\rchi_\Phi, \Gamma)\]
    Значит, $\lambda = \frac{\textup{tr } \Psi_{\Gamma, \Phi}}{\rchi_\Phi(e)} = |G|\frac{(\rchi_\Phi, \Gamma)}{\rchi_\Phi(e)}$.
\end{proof}
\begin{definition}
    Пусть $G = \{g_1,...,g_n\}$, $V$ --- векторное пространство над $\F$ такое, что $\dim V = n$, $\E = \{e_{g_1},...,e_{g_n}\}$ --- базис $V$. Линейное представление $\rho: G \rightarrow GL(V)$, заданное по правилу $\forall g \in G: \ \rho(g)e_{g_r} = e_{gg_r}$, называется регулярным представлением группы $G$ над полем $\F$.
\end{definition}
\begin{lemmanum}
    Пусть $G$ --- конечная группа, $\Phi_1,...,\Phi_k$ --- все попарно неэквивалентные комплексные линейные представления группы $G$, $\rchi_1,...,\rchi_k$ --- их характеры. Тогда $\rchi_1,...,\rchi_k$ --- ортонормированный базис в $\rchi_\CC(G)$.
\end{lemmanum}
\begin{proof}
    Так как $\Phi_1,...,\Phi_k$ неприводимы и попарно неэквивалентны, $(\rchi_i, \rchi_j) = \delta_i^j$, то есть $\rchi_1,...,\rchi_k$ линейно независимы как попарно ортогональные векторы. При этом $\dim V < \infty$ (равно числу классов сопряжённости в $G$), то есть $k < \infty$. Осталось доказать, что $\rchi_\CC(G) = \langle \rchi_1,...,\rchi_k \rangle$.

    Докажем от противного: предположим, что $\rchi_\CC(G) \neq \langle \rchi_1,...,\rchi_k \rangle$. Тогда \\
    $\langle \rchi_1,...,\rchi_k \rangle^\perp \neq \{0\}$, то есть $\exists \Gamma \in \langle \rchi_1,...,\rchi_k \rangle^\perp$, $\Gamma \neq 0$. Пусть $G = \{g_1,...,g_n\}$. Выберем произвольное векторное пространство $V$ размерности $n$ и произвольный базис $\E = \{e_{g_1},...,e_{g_n}\}$ в нём. Рассмотрим регулярное представление $\rho: G \rightarrow GL(V)$ над $\CC$. По теореме Машке оно вполне приводимо, то есть представимо в виде $\rho = m_1\tilde{\Phi}_1 + ... + m_s\tilde{\Phi}_s$, где $\tilde{\Phi}_i$ неприводимы и попарно неэквивалентны.\\
    Рассмотрим линейный оператор $\Psi_{\Gamma, \rho} = \sum \limits_{g \in G} \overline{\Gamma(g)}\rho(g)$ пр-ва $V$, как в лемме 1:
    \[\Psi_{\Gamma, \rho} = \sum \limits_{g \in G} \overline{\Gamma(g)}\rho(g) = \sum \limits_{g \in G} \overline{\Gamma(g)}(m_1\tilde{\Phi}_1 + ... + m_s\tilde{\Phi}_s)(g) = m_1\Psi_{\Gamma, \tilde{\Phi}_1} + ... + m_s\Psi_{\Gamma, \tilde{\Phi}_s}\]
    Из неприводимости $\tilde{\Phi}_i$ по лемме 1 $\Psi_{\Gamma, \tilde{\Phi}_i} = \lambda_i I$, где $\lambda_i = |G|\frac{(\rchi_{\tilde{\Phi}_i}, \Gamma)}{\rchi_{\tilde{\Phi}_i}(e)}$. Но при этом $\tilde{\Phi}_i$ --- неприводимое представление $G$, то есть оно эквивалентно одному из $\Phi_i$, а отсюда $\rchi_{\tilde{\Phi}_i} \in \langle \rchi_1,...,\rchi_k \rangle$. Значит, $(\rchi_{\tilde{\Phi}_i}, \Gamma) = 0 \Longrightarrow \lambda_i = 0 \Longrightarrow \Psi_{\Gamma, \rho} = 0$.\\
    С другой стороны,
    \[\Psi_{\Gamma, \rho}(e_{g_1}) = \sum \limits_{g \in G} \overline{\Gamma(g)}\rho(g)(e_{g_1}) = \sum \limits_{g \in G} \overline{\Gamma(g)}e_{gg_1}\]
    А так как $\Psi_{\Gamma, \rho} = 0$, имеем $\sum \limits_{g \in G} \overline{\Gamma(g)}e_{gg_1} = 0 \Longrightarrow \forall g \in G: \Gamma(g) = 0 \Longrightarrow \Gamma = 0$ --- противоречие. Значит, $\rchi_\CC(G) = \langle \rchi_1,...,\rchi_k \rangle$, и $\rchi_1,...,\rchi_k$ --- ортонормированный базис в $\rchi_\CC(G)$.
\end{proof}
\begin{consequense}
    \hypertarget{chtarget}{Пусть} $G$ --- конечная группа, $r$ --- количество классов сопряжённости в $G$. Тогда существует ровно $r$ попарно неэквивалентных неприводимых комплексных линейных представлений $G$ над $\CC$.
\end{consequense}
\begin{proof}
    Из леммы 2 количество попарно неэквивалентных комплексных линейных представлений $G$ равно $\dim \rchi_\CC(G)$, что равно количеству классов сопряжённости в $G$.
\end{proof}
\begin{consequense}
    Пусть $G$ --- конечная группа, $\Phi_1,...,\Phi_r$ --- все попарно неэквивалентные комплексные линейные представления группы $G$, $n_1,...,n_r$ --- их размерности. Тогда:
    \begin{enumerate}
        \item $\rho = n_1\Phi_1 + ... + n_r\Phi_r$, где $\rho$ --- регулярное представление группы $G$ над $\CC$;
        \item $|G| = n_1^2 + ... + n_r^2$
    \end{enumerate}
\end{consequense}
\begin{proof}\tab
    \begin{enumerate}
        \item По теореме Машке $\rho$ вполне приводимо, то есть $\rho = m_1\Phi_1 + ... + m_r\Phi_r$ ($m \geqslant 0$). Пусть $\rchi_\rho$ и $\rchi_{\Phi_i}$ --- характеры $\rho$ и $\Phi_i$ соответственно. Так как $\rchi_1,...,\rchi_r$ --- ортонормированный базис, $(\rchi_\rho, \rchi_{\Phi_i}) = m_i$. С другой стороны:
        \[(\rchi_\rho, \rchi_{\Phi_i}) = \frac{1}{|G|}\sum \limits_{g \in G}\rchi_\rho(g)\overline{\rchi_{\Phi_i}(g)}\]
        Посмотрим, какие значения принимает $\rchi_\rho$. Выберем векторное пространство $V$ размерности $n$ и базис $\E = \{e_{g_1},...,e_{g_n}\}$. Так как $\rho(g)e_{g_i} = e_{gg_i}$, в базисе $\E$ каждый столбец матрицы $A_\rho(g)$ содержит одну единицу и $n-1$ ноль. При этом если $g = 1$ в $G$, то $A_\rho(g) = E$, а иначе ни один базисный вектор не перейдёт в себя, то есть все элементы на главной диагонали $A_\rho(g)$ равны нулю. Отсюда:
        \[\rchi_\rho(g) = \begin{cases} |G|, \ g = 1\\ 0, \ g \neq 1 \end{cases}\]
        Значит,
        \[\frac{1}{|G|}\sum \limits_{g \in G}\rchi_\rho(g)\overline{\rchi_{\Phi_i}(g)} = \frac{1}{|G|}\cdot|G|\overline{\rchi_{\Phi_i}(1)} = n_i\]
        так как $\Phi_i(1) = I$
        \item Выразим $|G|$:
        \[|G| = \dim V = \textup{tr } \rho(1) = n_1 \textup{tr } \Phi_1 + ... + n_r \textup{tr } \Phi_r = n_1^2 + ... + n_r^2\]
    \end{enumerate}    
\end{proof}
\begin{example}
    Построим таблицу характеров неприводимых комплексных представлений $S_3$:
    Сами представления уже были найдены:
    \begin{itemize}
        \item $\Phi_1(\sigma) = I$;
        \item $\Phi_2(\sigma) = \sgn \sigma \cdot I$;
        \item $\Phi_3: S_3 \simeq D_3 = \textup{Sym } \triangle \subset GL_2(\CC)$
    \end{itemize}
    $$\begin{tabular}{c|c|c|c|}
        \text{Классы сопр.} & id & (12) & (123)\\ \hline
        $\rchi_{\Phi_1}$ & 1 & 1 & 1 \\ \hline
        $\rchi_{\Phi_2}$ & 1 & -1 & 1 \\ \hline 
        $\rchi_{\Phi_3}$ & 2 & 0 & -1 \\ \hline
    \end{tabular}$$
    Значения $\rchi_{\Phi_1}$ и $\rchi_{\Phi_2}$ ищутся очевидно. Опишем подробнее поиск значений $\rchi_{\Phi_3}$:
    \begin{itemize}
        \item $\rchi_{\Phi_3}(\id) = I$ --- в любом базисе матрица $\begin{pmatrix} 1&0 \\ 0&1 \end{pmatrix}$, след равен 2;
        \item $\rchi_{\Phi_3}((12))$ --- симметрия относительно прямой, содержащей вершину 3 и середину стороны 1-2. Так как характер не зависит от базиса, можем выбрать удобный нам --- в ортонормированном базисе, где первый базисный вектор параллелен оси симметрии, матрица примет вид $\begin{pmatrix} 1&0 \\ 0&-1 \end{pmatrix}$, след равен 0;
        \item $\rchi_{\Phi_3}((123))$ --- поворот на $\frac{2\pi}{3}$ относительно центра треугольнка. Матрица этого поворота имеет вид $\begin{pmatrix} \cos(\frac{2\pi}{3})&\sin(\frac{2\pi}{3}) \\ -\sin(\frac{2\pi}{3})&\cos(\frac{2\pi}{3}) \end{pmatrix} = \begin{pmatrix} -\frac{1}{2}&\frac{\sqrt{3}}{2} \\ -\frac{\sqrt{3}}{2}&-\frac{1}{2} \end{pmatrix}$, след равен -1.
    \end{itemize}
    Покажем с помощью характера, что $\Phi_3$ --- неприводимое представление:
    \[(\rchi_{\Phi_3}, \rchi_{\Phi_3}) = \frac{1}{|S_3|}\sum \limits_{g \in S_3} \rchi_{\Phi_3}(g)\overline{\rchi_{\Phi_3}(g)} = \frac{1}{6}(1 \cdot 2^2 + 3 \cdot 0 + 2 \cdot (-1)^2) = 1\]
\end{example}
\setcounter{thcount}{0}
\setcounter{concount}{0}
\setcounter{subthcount}{0}
\setcounter{lemcount}{0}
\newpage