\section{Линейные представления}
Пусть $V$ - векторное пространство над полем $\F$, $GL(V)$ - группа обратимых линейных операторов над $V$, $G$ - произвольная группа.
\begin{definition}
    Произвольный гомоморфизм $\Phi: G \rightarrow GL(V)$ (действие $G \acts V$) называется линейным представлением группы $G$.\\
    $V$ называется пространством линейного представления, $\dim V$ - размерность (степень) линейного представления.
    Если $\F = \Q, \R, \CC$, то линейное представление называется рациональным, вещественным или комплексным соответственно.
\end{definition}
Из определения $\Phi(e) = I$ - тождественный оператор, $\Phi(g_1g_2) = \Phi(g_1)\Phi(g_2)$.
\begin{definition}
    $\forall g \in G \ \Phi(g)$ называется оператором линейного представления. Обозначается $\forall v \in V \ \ \Phi(g)v := \Phi(g)(v)$.
\end{definition}
\begin{definition}
    Если $\textup{Ker }\Phi = \{e\}$, то линейное представление называется точным. В этом случае $G \simeq \textup{Im }\Phi \leq GL(V)$.
\end{definition} 
\begin{definition}
    Если $G \leq GL(V)$, то тождественное линейное представление $\Phi = \textup{id}: G \rightarrow GL(V)$ называется тавтологическим линейным представлением $G$.
\end{definition}
\begin{examples}\tab
    \begin{enumerate}
        \item $\dim V = 1: GL(V) \simeq \F^*$, то есть $\Phi: G \rightarrow GL(V) \simeq \F^{*}$;
        \begin{enumerate}
            \item $\F = \R, G = \R:$ $\Phi(t) = e^t$;
            \item $\F = \R, G = S_n:$ $\Phi(\sigma) = \sgn \sigma$;
            \item $G = GL_n(\F):$ $\Phi(A) = \det A$. 
        \end{enumerate}
        \item $V = M_n(\CC), G = \R$. Если зафиксировать матрицу $B \in M_n(\CC)$, то определено линейное представление $\Phi(t) = e^{tB}$.
        \item Пусть задано линейное представление $\Phi: G \rightarrow GL(V)$ и задан гомоморфизм $\Psi: H \rightarrow G$. Тогда $\tilde{\Phi} = \Phi \circ \Psi: H \rightarrow GL(V)$ - линейное представление группы $H$.
        \item Пусть $X$ - некоторое множество, задано $G \acts X$.\\
        Рассмотрим векторное пространство функций $\mathcal{F}(X, \F) = \{f: X \rightarrow \F\} = V$\\
        Тогда $\Phi: G \rightarrow GL(V)$: $\forall g \in G \ \Phi(g)f = \tilde{f}$, где $\tilde{f}(x) = f(gx)$ - линейное представление группы $G$:
        \[\forall g_1, g_2\in G : \Phi(g_1g_2)f(x) = f(g_1g_2x) = \Phi(g_1)f(g_2x) = \Phi(g_1)\Phi(g_2)f(x)\]
    \end{enumerate}
\end{examples}

\subsection{Матричные представления группы}
\begin{definition}
    Произвольный гомоморфизм $\Phi: G \rightarrow GL_n(\F)$ называется матричным представлением группы $G$ размерности $n$ над полем $\F$.
\end{definition}
Заметим, что линейные и матричные представления связаны между собой:
\begin{enumerate}
    \item Если задано матричное представление $G$, то есть гомоморфизм $\Phi: G \rightarrow GL_n(\F)$, то $GL_n(\F) \underset{\psi}{\simeq} GL(\F^n) \Longrightarrow \tilde{\Phi} = \psi \circ \Phi: G \rightarrow GL(\F^n)$ - линейное представление $G$.
    \item Если задано $n$-мерное линейное представление $G$, то есть гомоморфизм $\Phi: G \rightarrow GL(V) : \forall g \in G \ g \mapsto \Phi(g) = \phi_g$ - линейный оператор. Если фиксировать базис $\E = \{e_1,...,e_n\}$, то $\phi_g \leftrightarrow A_g$ - матрица $\phi_g$ в базисе $\E$. Отсюда $GL(V) \underset{\tilde{\psi}}{\simeq} GL_n(\E)$ - получим матричное представление группы $G$.
\end{enumerate}
