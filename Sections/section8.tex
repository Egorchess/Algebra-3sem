\section{Разрешимые и простые группы}
\subsection{Разрешимые группы}
\begin{definition}
    Кратный коммутант группы $G$:
    \[G^{(1)} = G; \ \ G^{(k+1)} = [G^{(k)}, G^{(k)}] = (G^{(k)})'\]
\end{definition}
Очевидно, что $G \geq G^{(1)} \geq G^{(2)} \geq ...$.
\begin{definition}
    Группа $G$ называется разрешимой, если $\exists m \in \N: G^{(m)} = \{e\}$.
\end{definition}
\begin{subtheorem}
    $G$ --- абелева $\Longrightarrow \ G$ --- разрешимая.
\end{subtheorem}
\begin{proof}
    $G$ --- абелева $\Longrightarrow \ G' = \{e\}$.
\end{proof}
\begin{subtheorem}\tab
    \begin{enumerate}
        \item $S_n$ --- разрешимая $\Longleftrightarrow \ n \leqslant 4$; 
        \item $A_n$ --- разрешимая $\Longleftrightarrow \ n \leqslant 4$.
    \end{enumerate}
\end{subtheorem}
\begin{proof}
    $S_n' = A_n$, поэтому $S_n$ --- разрешимая $\Longleftrightarrow$ $A_n$ --- разрешимая.\\
    $A_2 = \{\textup{id}\}, \ A_3 \simeq \Z_3$ --- абелева, $A_4' = V_4$ --- абелева $\Longrightarrow$ при $n \leqslant 4 \ A_n$ разрешима. 
    При $n \geq 5 \ A_n' = A_n$, то есть $A_n$ --- не разрешимая. 
\end{proof}
\begin{subtheorem}
    Пусть $\F$ --- поле, $|\F| > 3$. Тогда $GL_n(\F)$ и $SL_n(\F)$ не разрешимы. 
\end{subtheorem}
\begin{proof}
    $GL_n(\F)' = SL_n(\F)' = SL_n(\F)$.
\end{proof}
\begin{subtheorem}\tab
    \begin{enumerate}
        \item $G$ --- разрешимая, $H \leq G \Longrightarrow H$ --- разрешимая;
        \item $G$ --- разрешимая, $H \unlhd G \Longrightarrow G/H$ --- разрешимая;
        \item $H \unlhd G$, $H$ и $G/H$ --- разрешимые  $\Longrightarrow G$ --- разрешимая.
    \end{enumerate}
\end{subtheorem}
\begin{proof}\tab
    \begin{enumerate}
        \item Для начала заметим, что $H \leq G \Longrightarrow H' \leq G'$, так как любой коммутатор элементов из $H$ --- также коммутатор элементов из $G$. Значит, $H^{(m)} \leq G^{(m)}$.\\
        $G$ разрешима $\Longrightarrow \exists m: G^{(m)} = \{e\} \Longrightarrow H^{(m)} = \{e\} \Longrightarrow H$ разрешима.
        \item Рассмотрим натуральный гомоморфизм $\pi: G \rightarrow G/H$. Очевидно, что образ коммутатора при гомоморфизме --- коммутатор:
        \[\alpha([x, y]) = \alpha(xyx^{-1}y^{-1}) = \alpha(x)\alpha(y)\alpha(x)^{-1}\alpha(y)^{-1} = [\alpha(x), \alpha(y)]\]
        то есть $\pi(G') \subseteq (G/H)'$. При этом натуральный гомоморфизм сюръективен, а прообраз коммутатора --- также коммутатор (аналогично), то есть $\pi(G') = (G/H)'$. Аналогично $\pi(G^{(m)}) = (G/H)^{(m)}$.\\
        $G$ разрешима $\Longrightarrow \exists m: G^{(m)} = \{e\} \Longrightarrow \pi(G^{(m)}) = (G/H)^{(m)} = \{e\}$.
        \item $(G/H)$ разрешима $\Longrightarrow \exists k: (G/H)^{(k)} = \{e\} \Longrightarrow \pi(G^{(k)}) = \{e\} \Rightarrow G^{(k)} \subseteq H$.\\
        Также $H$ разрешима $\Longrightarrow \exists l: H^{(l)} = \{e\} \Longrightarrow (G^{(k)})^{(l)} = G^{(k+l)} = \{e\}$. Значит, $G$ разрешима.
    \end{enumerate}
\end{proof}
\begin{subtheorem}
    Группа $T_n(\F)$ невырожденных верхнетреугольных матриц порядка $n$ с коэффициентами из поля $\F$ разрешима.
\end{subtheorem}
\begin{proof}
    Индукция по $n$:\\
    База: $n = 1 \Longrightarrow T_n(\F) \simeq \F^*$ --- абелева, а значит, разрешима;\\
    Шаг: пусть $T_{n-1}(\F)$ разрешима. Рассмотрим гомоморфизм $\phi: T_{n} \rightarrow T_{n-1}$:
    \[\phi: \begin{pmatrix} a_{11}&&*&\vline&a_{1n} \\ &\ddots&&\vline&a_{2n} \\ 0&&a_{n-1 n-1} & \vline & a_{n-1 n} \\ \hline 0 & \cdots & 0 & \vline & a_{nn}  \end{pmatrix} \mapsto \begin{pmatrix} a_{11}&&* \\ &\ddots& \\ 0&&a_{n-1 n-1} \end{pmatrix}\]
    Этот гомоморфизм, очевидно, сюръективен, т.е по теореме о гомоморфизме $T_n/\textup{Ker }\phi \simeq T_{n-1}$. Так как $T_{n-1}$ разрешима по предположению индукции, по пункту 3 предыдущего утверждения достаточно доказать разрешимость группы
    \[\textup{Ker }\phi = \left\{\begin{pmatrix} &&&\vline&a_{1n} \\ &\text{\Huge{E}}&&\vline&a_{2n} \\ &&& \vline & a_{n-1 n} \\ \hline 0 & \cdots & 0 & \vline & a_{nn}  \end{pmatrix}: a_{in} \in \F, a_{nn}\neq 0\right\}\]
    Аналогично, рассмотрим гомоморфизм $\psi$: $\textup{Ker }\phi \rightarrow \F^*$:
    \[\psi: \begin{pmatrix} &&&\vline&a_{1n} \\ &\text{\Huge{E}}&&\vline&a_{2n} \\ &&& \vline & a_{n-1 n} \\ \hline 0 & \cdots & 0 & \vline & a_{nn}  \end{pmatrix} \mapsto a_{nn}\]
    Заметим, что $\textup{Im } \psi = \F^*$ --- абелева, а $\textup{Ker }\psi$ состоит из матриц $\textup{Ker } \phi$ с $a_{nn} = 1$.
    \[\begin{pmatrix} &&&\vline&a_{1n} \\ &\text{\Huge{E}}&&\vline&a_{2n} \\ &&& \vline & a_{n-1 n} \\ \hline 0 & \cdots & 0 & \vline & 1  \end{pmatrix} \cdot \begin{pmatrix} &&&\vline&b_{1n} \\ &\text{\Huge{E}}&&\vline&b_{2n} \\ &&& \vline & b_{n-1 n} \\ \hline 0 & \cdots & 0 & \vline & 1 \end{pmatrix} = \begin{pmatrix} &&&\vline&a_{1n} + b_{1n} \\ &\text{\Huge{E}}&&\vline&a_{2n} + b_{2n} \\ &&& \vline & a_{n-1 n} + b_{n-1 n} \\ \hline 0 & \cdots & 0 & \vline & 1 \end{pmatrix}\]
    Отсюда несложно видеть, что $\textup{Ker } \psi$ --- также абелева, то есть разрешимая группа. Значит, $\textup{Ker }\phi$ разрешима, а отсюда и $T_n(\F)$ --- разрешимая группа.
\end{proof}
\begin{subtheoremnum}
    Всякая конечная примарная группа $G$ разрешима.
\end{subtheoremnum}
\begin{proof}
    Пусть $p$ --- простое, для которого $G$ является $p$-группой.\\
    Индукция по $n = |G|$:\\
    База: $n = 1 \Longrightarrow G = \{e\}$ --- разрешима;\\
    Шаг: $G \neq \{e\} \Longrightarrow Z(G) \neq \{e\}$ (из примарности).
    Знаем, что $Z(G) \unlhd G$ --- рассмотрим $G / Z(G)$. Это также $p$-группа, причём порядка $\frac{|G|}{|Z(G)|}$, что меньше $n$. Значит, $G/Z(G)$ разрешима по предположению индукции, а также $Z(G)$ разрешима, так как абелева. Отсюда $G$ разрешима.
\end{proof}
\begin{subtheoremnum}
    Всякая группа $G$ порядка $pq$, где $p,q$ простые, разрешима.
\end{subtheoremnum}
\begin{proof}
    Случай $p = q$ очевиден из утверждения 1.\\
    Пусть $p \neq q$ --- без ограничения общности $p > q$.\\
    По I теореме Силова $\exists$ силовская $p$-подгруппа $H \leq G$.\\
    По III теореме Силова $\begin{cases}
        N_p \mid q\\ N_p \equiv 1 (\textup{mod }p)
    \end{cases} \Longrightarrow N_p = 1$\\(не может равняться $q$ в силу $q < p$)\\
    Тогда по следствию из II теоремы Силова единственная силовская $p$-подгруппа $H$ нормальна в $G$. Притом $|H| = p \Longrightarrow H \simeq \Z_p$ и $|G/H| = q \Longrightarrow G/H \simeq \Z_q$ --- абелевы. Значит, $H$ и $G/H$ разрешимы, а отсюда $G$ разрешима.
\end{proof}
\subsection{Простые группы}
\begin{definition}
    Подгруппа $H \leq G$ называется собственной, если $H \neq \{e\}, G$.
\end{definition}
\begin{definition}
    Группа $G$ называется простой, если $G \neq \{e\}$ и в $G$ нет собственных нормальных подгрупп.
\end{definition}
\setcounter{subthcount}{0}
\begin{subtheoremnum}
    Абелева группа $G$ --- простая $\Longleftrightarrow G \simeq \Z_p$, где $p$ --- простое.
\end{subtheoremnum}
\begin{proof}
    $ \\\Longleftarrow$ --- очевидно ($\Z_p$ --- циклическая, т.е. нет собственных подгрупп);
    $ \\\Longrightarrow$: Пусть $G$ --- абелева и простая группа. \\
    Тогда $G$ циклическая, так как $\forall g \neq e: \langle g \rangle \unlhd G$ (т.к. абелева) и $g \neq \{e\}$, т.е. $\langle g \rangle = G$. Теперь, если $G$ бесконечна, то $G \simeq \Z$, но $2\Z \vartriangleleft \Z$ --- противоречие, т.е. $G$ конечна. А если $|G|$ составное, то $G \simeq \Z_{mn}$, где $\langle m \rangle \vartriangleleft \Z_{mn} (m, n \neq 1)$. Значит, $|G|$ простое, т.е. $G \simeq \Z_p$. 
\end{proof}
\begin{subtheoremnum}
    Если $G$ --- разрешимая и простая, то $G \simeq \Z_p$, где $p$ --- простое.
\end{subtheoremnum}
\begin{proof}
    Так как $G$ разрешима, $G' \neq G$. Притом $G' \unlhd G$, а отсюда из простоты $G' = \{e\}$. Значит, $G$ --- абелева, а тогда $\simeq \Z_p$ из утверждения 1.
\end{proof}
\begin{remark}
    Таким образом, всякая простая группа либо изоморфна $\Z_p$, либо не абелева и не разрешима.
\end{remark}
\subsection{Значение простых групп}
\begin{definition}
    Субнормальной матрёшкой называется последовательность
    \[G = G_0 \geq G_1 \geq ... \geq G_m = \{e\}; \ \ G_{i+1} \unlhd G_i \ \forall i = \overline{0....m-1}\]
\end{definition}
\begin{example}
    $G = A_4, H = V_4, K = \langle (12)(34) \rangle$. Тогда $H \unlhd G$, $K \unlhd H$, то есть $G \geq H \geq K \geq \{\textup{id}\}$ --- субнормальная матрёшка.
\end{example}
\begin{theorem}
    Группа $G$ разрешима $\Longleftrightarrow G$ обладает субнормальной матрёшкой такой, что $G_i/G_{i+1}$ --- абелева $\forall i = \overline{0....m-1}$. 
\end{theorem}
\begin{proof}
    Без доказательства.
\end{proof}
\begin{definition}
    Композиционным рядом называется субнормальная матрёшка такая, что $\forall i = \overline{0....m-1}: G_i \neq G_{i+1}$ и $G_i/ G_{i+1}$ --- простая группа.
\end{definition}
\begin{subtheoremnum}
    Всякая конечная группа $G$ обладает композиционным рядом.
\end{subtheoremnum}
\begin{proof}
    Если $G$ --- простая, то $G \gneq \{e\}$ --- композиционный ряд.\\
    Если $G$ --- не простая, то $\exists$ собственная подгруппа $N \unlhd G$, т.е. $G \gneq N \gneq \{e\}$ --- субнормальная матрёшка. Будем уплотнять эту матрёшку следующим образом:

    Предположим, что в субнормальной матрёшке $G_0 \gneq ... \gneq G_m$ группа $G_i / G_{i+1}$ --- не простая. Тогда $\exists$ собственная $\tilde{N} \unlhd G_i / G_{i+1}$.\\
    Рассмотрим натуральный гомоморфизм $\pi: G_i \rightarrow G_i / G_{i+1}$. Тогда $\pi^{-1}(\tilde{N}) = \tilde{\tilde{N}}$ --- собственная нормальная подгруппа $G_i$, содержащая $G_{i+1}$, то есть в матрёшке кусок "$...\gneq G_i \gneq G_{i+1} \gneq ...$"\ заменяем на "$...\gneq G_i \gneq \tilde{\tilde{N}} \gneq G_{i+1} \gneq ...$".
    Очевидно, что процесс таких уплотнений конечен, так как количество членов матрёшки явно не превышает $|G|$ (порядок строго убывает). Значит, за конечное число уплотнений сможем построить композиционный ряд для $G$.
\end{proof}
\begin{theorem}(Жордана --- Гёльдера)\\
    Если группа $G$ обладает композиционным рядом, то набор факторгрупп в нём определён однозначно с точностью до перестановки. 
\end{theorem}
\begin{proof}
    Без доказательства.
\end{proof}
\begin{example}
    Пусть $G = \langle a \rangle_{12}$. Композиционные ряды:
    \[\langle a \rangle_{12} > \langle a^2 \rangle_6 > \langle a^4 \rangle_3 > \{e\} \ \ - \ \ \Z_2, \Z_2, \Z_3;\]
    \[\langle a \rangle_{12} > \langle a^2 \rangle_6 > \langle a^6 \rangle_2 > \{e\} \ \ - \ \ \Z_2, \Z_3, \Z_2;\]
    \[\langle a \rangle_{12} > \langle a^3 \rangle_4 > \langle a^6 \rangle_2 > \{e\} \ \ - \ \ \Z_3, \Z_2, \Z_2;\]
\end{example}
\begin{remark}
    Группа $G$ не задаётся однозначно набором простых факторов композиционного ряда: пусть набор факторов --- $\Z_2, \Z_2$, тогда возможны композиционные ряды $0 < \Z_2 < \Z_4$ и $0 < \Z_2 < \Z_2 \oplus \Z_2 \simeq V_4$.
\end{remark}
\subsection{Примеры простых групп}
\setcounter{lemcount}{0}
\begin{exercise}
    Если $|G| < 60$ и $|G|$ --- не простое, то $G$ --- не простая группа.
\end{exercise}
\begin{proof}(довольно объёмное и вряд ли пригодится)\\
    Всё в лучших традициях --- докажем несколько лемм:
    \begin{lemmanum}
        Всякая неабелева примарная группа не является простой.
    \end{lemmanum}
    \begin{proof}
        Очевидно из нетривиальности центра --- он не совпадает со всей группой из неабелевости, а значит является собственной нормальной подгруппой $G$ 
    \end{proof}
    \begin{lemmanum}
        Пусть $G$ --- неабелева группа, $|G| = p^lm$, где $p \nmid m$, $p^l \nmid (m-1)!$\\
        Тогда $G$ --- не простая группа.
    \end{lemmanum}
    \begin{proof}
        Случай $m = 1$ очевиден из леммы 1.\\
        Пусть $m > 1$, $S$ --- силовская $p$-подгруппа $G$. Рассмотрим действие $G \acts G/S$ левыми сдвигами ($G/S$ --- множество левых смежных классов). Таких смежных классов ровно $m$ (из теоремы Лагранжа), причём каждый элемент $g$ переводит разные смежные классы в разные --- отсюда каждый элемент $G$ соответствует некоторой подстановке из $S_m$, то есть определён гомоморфизм $\alpha: G \rightarrow S_m$. Если $G$ простая, то $\textup{Ker }\alpha = \{e\}$ либо $G$ --- второе, очевидно, невозможно (класс $g_1S$ в класс $g_2S$ переводит элемент $g_2g_1^{-1}$). Значит, $G \simeq \alpha(G) \leq S_m$ из теоремы о гомоморфизме. Отсюда $|G| \mid |S_m| \Longrightarrow p^lm \mid m! \Longrightarrow p^l \mid (m-1)!$ --- противоречие. 
    \end{proof}
    \begin{remark}
        Данная лемма очень сильна при решении некоторых упражнений --- например, из неё несложными рассуждениями следует непростота (а по индукции и разрешимость) неабелевых групп порядков $2p^k, 3p^k, 4p^k$ ($p$ --- простое).
    \end{remark}

    Остаётся лишь перебор случаев составных чисел $< 60$ --- целиком его несложно провести самому, поэтому здесь он приведён не будет в целях сохранения моего морального и физического благополучия. \\
    В результате под лемму 2 не попадут порядки $30, 40$ и $56$ --- разберём их:
    \begin{itemize}
        \item $|G| = 40 = 2^3 \cdot 5$: По $III$ теореме Силова в $G$ единственная подгруппа порядка 5 ($N_5 \mid 8, N_5 \equiv 1 (\textup{mod }5)$) --- она нормальна;
        \item $|G| = 30 = 2 \cdot 3 \cdot 5$: По $III$ теореме Силова число силовских $5$-подгрупп в $G$ либо 1, либо 6 ($N_5 \mid 6, N_5 \equiv 1 (\textup{mod }5)$). Если такая подгруппа единственна --- то она нормальна, и $G$ не простая. Аналогично силовских 3-подгрупп либо 1, либо 10 --- случай единственности очевиден. Остаётся заметить, что порядки самих силовских подгрупп простые, то есть эти подгруппы циклические --- значит, различные 3- и 5-подгруппы пересекаются тривиально. Значит, в $G$ есть $6 \cdot (5-1) = 24$ различных элементов порядка 5 и $10 \cdot (3-1) = 20$ различных элементов порядка 3, что невозможно для группы порядка 30.
        \item $|G| = 56 = 2^3 \cdot 7$: По $III$ теореме Силова силовских $7$-подгрупп в $G$ либо 1, либо 8 ($N_7 \mid 8, N_7 \equiv 1 (\textup{mod }7)$). Случай $N_7 = 1$ очевиден, а иначе по рассуждениям выше силовские $7$-подгруппы пересекаются тривиально, а значит в $G$ есть хотя бы $8 \cdot (7-1) = 48$ различных элементов порядка 7. При этом в $G$ есть силовская 2-подгруппа, которой принадлежат 8 элементов порядка $2^k$ --- либо она единственна, то есть нормальна, либо их больше одной, что невозможно в группе порядка $56 = 48 + 8$.
    \end{itemize} 
    Все случаи разобраны.
\end{proof}
\begin{theorem}
    Если $G$ --- простая и $|G| = 60$, то $G \simeq A_5$. 
\end{theorem}
\begin{proof}
    Без доказательства.
\end{proof}
\begin{example}
    $A_2 = \{\textup{id}\}, A_3 \simeq \Z_3$ --- простая, $A_4$ --- не простая ($V_4 \unlhd A_4$).
\end{example}
\begin{lemma}
    Пусть $n \geqslant 5, N \leq  A_n, N \neq \{\textup{id}\}, N \unlhd S_n$. Тогда $A_n = N$.
\end{lemma}
\begin{proof}
    Так как $N \neq \{\textup{id}\}$, то $\exists \sigma \in N, \sigma \neq \textup{id}$. Разложим $\sigma$ в независимые циклы: $\sigma = c_1c_2...c_k$. Рассмотрим случаи:
    \begin{enumerate}
        \item $\exists i$ такой, что длина $c_i \geqslant 3$, то можем считать, что $c_1 = (i_1...i_k), k \geqslant 3$. Так как $N \unlhd S_n$, $\forall \tau \in S_n: \tau\sigma\tau^{-1} \in N$. Рассмотрим $\tau = (i_1i_2)$:
        \[\tau\sigma\tau^{-1}\sigma^{-1} \in N; \tau\sigma\tau^{-1}\sigma^{-1} = \tau c_1 \tau^{-1} c_1^{-1} = (i_2i_1i_3...i_k)(i_k...i_1) = (i_1i_2i_3)\]
        ($\tau\sigma\tau^{-1}\sigma^{-1} = \tau c_1 \tau^{-1} c_1^{-1}$, так как остальные циклы в $\sigma$ независимы с $\tau$, т.е. коммутируют с $\tau$)\\
        Тогда в $N$ содержатся все тройные циклы --- $A_n = N$.
        \item Если же $\forall i$ длина $c_i$ равна 2, то $k$ --- чётно, т.е. $\sigma = (i_1i_2)(i_3i_4)c_3...c_k$.\\
        Тогда при аналогичных рассуждениях и $\tau = (i_2i_3)$:
        \[\tau\sigma\tau^{-1}\sigma^{-1} = (i_2i_3)(i_1i_2)(i_3i_4)(i_2i_3)(i_1i_2)(i_3i_4) =\]
        \[= (i_1i_3)(i_2i_4)(i_1i_2)(i_3i_4) = (i_1i_4)(i_2i_3) \in N\]
        Так как в $S_n$ все произведения пар независимых транспозиций сопряжены, все пары независимых транспозиций $\in N$.
        Так как $n \geqslant 5$, $A_n$ порождается парами независимых транспозиций, а значит, $N = A_n$.
    \end{enumerate}
\end{proof}
\begin{theorem}
    $A_n$ --- простая при $n \geqslant 5$.
\end{theorem}
\begin{proof}
    Рассмотрим произвольную нормальную подгруппу $N \unlhd A_n$. Если $N \unlhd S_n$, то по лемме 1 $N = A_n$. Иначе: $|S_n : A_n| = 2, S_n = A_n \sqcup (12)A_n$.\\
    Пусть $N$ не нормальна в $S_n$. Обозначим $N_1 = N, N_2 = (12)N_1(12)$.\\
    Если $N_1 = N_2$, то $N$ при сопряжении любой $\sigma \in S_n$ не изменится (для $\sigma \in A_n$ очевидно из $N \unlhd A_n$, для $\tau \in S_n$: $\tau = (12)\tau' \Longrightarrow \tau N \tau^{-1} = (12)\tau'N{\tau'}^{-1}(12) = (12)N(12) = N$), т.е. $N \unlhd S_n$ --- противоречие.\\
    Поэтому $N_1 \neq N_2$, причём $|N_1| = |N_2|$.\\
    Докажем, что $A_n = N_1 \times N_2$ (отсюда получим, что $|A_n| = |N|^2$):
    \begin{enumerate}
        \item $N_1 \unlhd A_n$ --- уже имеем;
        \item $N_2 \unlhd A_n$:
        \[\forall \sigma \in A_n: \sigma N_2\sigma^{-1} = \sigma(12)N_1(12)\sigma^{-1} = (12)\underset{ = \tilde{\sigma}}{(12)\sigma(12)}N_1\underset{=\tilde{\sigma}^{-1}}{(12)\sigma^{-1}(12)}(12) = \]\[= (12)\tilde{\sigma}N_1\tilde{\sigma}^{-1}(12) = (12)N_1(12) = N_2\]
        \item $N_1 \cap N_2 = \{\textup{id}\}$: Пусть $K = N_1 \cap N_2 \leq A_n$. Тогда $K \unlhd S_n$:
        \begin{itemize}
            \item $\forall \sigma \in A_n: \sigma K\sigma^{-1} \subseteq N_1, N_2$ из $N_1, N_2 \unlhd A_n$, то есть $\sigma K\sigma^{-1} \subseteq K$;
            \item $(12)K(12) \subseteq N_2$ из $K \subseteq N_1$, $(12)K(12) \subseteq N_1$ из $K \subseteq N_2 \Longrightarrow \\ \Longrightarrow (12)K(12) \subseteq K$
        \end{itemize}
        Значит, $K$ не изменится при сопряжении любой подстановкой из $S_n$, то есть $K \unlhd S_n$.
        Тогда по лемме 1 и $K \neq A_n$ имеем $K = \{\textup{id}\}$.
        \item $N_1N_2 = A_n$: Пусть $L = N_1N_2 \leq A_n$ Тогда $L \unlhd S_n$:
        \begin{itemize}
            \item $\forall \sigma \in A_n: \sigma L\sigma^{-1} = \sigma N_1N_2 \sigma^{-1} = \sigma N_1\sigma^{-1}\sigma N_2 \sigma^{-1} = N_1N_2$;
            \item $(12) L(12) = (12) N_1N_2 (12) = (12)N_1(12)(12) N_2 (12) = N_2N_1 = N_1N_2$
        \end{itemize}
        При этом $L \neq id$ --- по лемме 1 $L = A_n$.
    \end{enumerate}
    Теперь индукцией по $n \geqslant 5$ докажем, что $A_n$ --- простая.\\
    База: $n = 5$ --- $|A_5| = 60$ --- не точный квадрат, то есть невозможна ненормальность $N$ в $S_n$, а отсюда $A_n$ --- простая;\\
    Шаг: Пусть $A_{n-1}$ --- простая. Обозначим $A_n \geq H = \{\sigma \in A_n \ | \ \sigma(n) = n\} \simeq A_{n-1}$.\\
    Предположим, что $N_1 \neq N_2 \Longrightarrow A_n = N_1 \times N_2 \Longrightarrow H \leq N_1 \times N_2$. Тогда $H \cap N_2 \unlhd H$, т.к. $N_2 \unlhd A_n$ и $H \cap N_2 \subseteq H$.\\
    Так как $H$ --- простая, то $H \cap N_2$ равно либо $H$, либо $id$.\\
    Если $H \cap N_2 = H$, то $H \subseteq N_2 \Longrightarrow |H| \leqslant |N_2| = |N|$.\\
    Если $H \cap N_2 = \{id\}$, то рассмотрим гомоморфизм $\phi: A_n = N_1 \times N_2 \rightarrow N_1$. $\textup{Ker } \phi = N_2 \Longrightarrow H \simeq \phi(H) \leq N_1 \Longrightarrow |H| \leqslant |N_1| = |N|$.\\
    В каждом случае $|H| = |A_{n-1}| \leqslant |N|$. Тогда из предположения
    \[|A_n| = |N|^2 \geqslant |A_{n-1}|^2 \Longrightarrow \frac{n!}{2} \geqslant \frac{((n-1)!)^2}{4} \Longrightarrow 2n \geqslant (n-1)!\]
    Последнее неравенство, очевидно, неверно при $n \geqslant 5$.
\end{proof}
\begin{theorem}
    $SO_3$ --- простая.
\end{theorem}
\begin{proof}
    Без доказательства.
\end{proof}
\setcounter{thcount}{0}
\setcounter{concount}{0}
\setcounter{subthcount}{0}
\newpage
