\section{Заключение и источники}
\begin{subtheoremnum}
    Если вы дочитали этот конспект до конца и всё поняли --- вы круты!
\end{subtheoremnum}
\begin{subtheoremnum}
    Если вы дочитали этот конспект до конца и не всё поняли --- прошу не стестняться задавать мне вопросы, большинство опечаток и ошибок в моём понимании обнаруживаются именно так, да и помочь я всегда буду рад)
\end{subtheoremnum}
\begin{subtheoremnum}
    Если вы не дочитали этот конспект до конца --- да пребудет с вами удачный билет на экзамене)
\end{subtheoremnum}
\begin{proof}
    Остаётся читателю в качестве упражнения.
\end{proof}
$\\$Без следующих людей конспект не состоялся бы, а потому отдельно благодарю:
\begin{itemize}
    \item Куликову Ольгу Викторовну - за прекрасные лекции, изучение которых не мешало мне редко на них просыпаться;
    \item Техающую команду 208 группы:
    \begin{itemize}
        \item Кирилл Яковлев (\textit{мастер спорта по пупупу});
        \item Вячеслав Молчанов (\textit{знаток ангема});
        \item Егор Цыбулин (\textit{гений линала});
    \end{itemize}
    --- с вами рядом никогда не было ощущения, что я одинок в своих страданиях;
    \item Людей, без чьих конспектов я бы не выжил:
    \begin{itemize}
        \item Сергей Криворученко (209 гр.) --- за живые и подробные конспекты;
        \item Евгения Ковтун (212 гр.) --- за аккуратные и читаемые коспекты; 
    \end{itemize}
    --- я мог спать, зная, что эти герои проснутся;
    \item Всех, кто присылал (и будет присылать) мне вопросы, ошибки и опечатки --- благодаря вам конспект становится понятнее, правильнее и чище;
    \item И наконец, всех, кто нашёл в себе силы открыть алгебру --- благодаря вам моя работа имела смысл)
\end{itemize} 
На этом всё, всем удачи и до встречи в новых конспектах!
